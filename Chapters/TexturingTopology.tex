\chapter{How to map Textures to Surfaces or Conformal Mappings and harmonic Differential forms}
	
	This section will highlight how Topology, i.e. very general characterization of the type a surface is of, plays a huge role in what solutions exist for basic questions.

	Our goal will be to find good ways to map textures to surfaces. This task directly translates to finding a parametrization of a surface, as a parametrization is, like the assignment of texture positions, assigning (u,v) coordinates to Points on the Surface.
	
	For simple geometric forms we have a clear intuition on what a parametrization should look like. For example for the Torus or the Sphere you will have seen parametrizations so often you will think of a paramtrization in blink $(image)$. But there already is a fundamental difference between those two parametrizations. While with the Torus squares get mapped very nicely to to squares, for the ball the squares degenerate to triangles at the poles. What does that actually mean? Was the parametrization of the Sphere just chosen badly?
	
	
	\section{A closer look at coordinate functions}
	Lets have a closer look at the sphere. The parametrization here is a mapping of $[0,2\pi]\times[-\pi/2,\pi/2] \rightarrow S^2$ with $f(u,v) = (\cos(u)\cos(v), \sin(u)\cos(v), \sin(v))$. The coordinate lines displayed above are given by $f(u,c)$ or $f(c,v)$ for different constants $c$. Lets look at both coordinates separately.
	
	At the poles the $u$ coordinate line vanishes completely and the $v$ coordinate lines meet in one point; if we associate directions to the coordinate line we see that all lines are incoming or outgoing. ($image$)
	
	\subsection{Vector Fields and Coordinates}
	What we actually just did was not so much looking at coordinate lines but looking at vector fields namely the vector fields $\frac{\partial f}{\partial u}$ and $\frac{\partial f}{\partial v}$. The first one vanishes at the poles and the second one is not even defined at the poles! As it turns out the 'Topology' of a surface determines how many such 'singularities' a vector field must at least have.
	
	\section{Some Basic Topology}
	As just motivated the topology of a surface is the determining thing that decides if or if not vector fields and parametrization without singularities exist (and how many singularities there must be). Topology covers a great deal of topics and provides very general tools that can  be applied as soon as you have a notion of "continuity". And you are mostly interested in describing properties of topological objects that will not changed if the object is changed in a continuous revertible way.
	
	Here we restrict ourselves to differential topology and our objects are manifolds. THE example of 
	
	
	Topological properties and the Dimension of the solution space of harmonic functions.
	
	Conformal maps on Riemannian Manifolds.	
	\section{Problems with Texture Mappings Or a short introduction to topology}
	