\documentclass[draft]{scrbook}
\usepackage{amsmath}
\usepackage{amssymb}
\usepackage{graphicx}
\usepackage{color}
\usepackage{algorithmic}
\usepackage[english]{babel}

%\usepackage[latin1]{inputenc}
\usepackage{tikz}
\usepackage{longtable}
\usetikzlibrary{shapes,arrows}

\title{Discrete Differential Calculus for Academics with No Time}
\author{Peter Bertholet}

\newenvironment{definition}[1][]{\begin{trivlist}
\item[\hskip \labelsep {\bfseries Definition (#1)}]\begin{itshape}}{\end{itshape}\end{trivlist}}

\newenvironment{packed_enum}{
\begin{enumerate}
  \setlength{\itemsep}{1pt}
  \setlength{\parskip}{0pt}
  \setlength{\parsep}{0pt}
}{\end{enumerate}}
\newenvironment{packed_itemize}{
\begin{itemize}
  \setlength{\itemsep}{1pt}
  \setlength{\parskip}{0pt}
  \setlength{\parsep}{0pt}
}{\end{itemize}}


\newcommand{\note}[1]{\textcolor{red}{\textit{(#1)}}}
\newcommand{\abs}[1]{\left| #1 \right|}

\begin{document}

	\maketitle
	\tableofcontents
	
%	\begin{abstract}
%		Discrete Exterior Calculus...
%\end{abstract}
	
\chapter{Introduction}

The goal of this Thesis is to present an easy to understand introduction to Discrete External Calculus (DEC) without assuming more than some basic knowledge of linear Algebra and standard Calculus. After reading this tutorial like Thesis you should have enough theoretical background and enough practical knowledge to understand DEC and apply it to problems.

But before getting started, you need to get a glimpse at what Discrete External Calculus and what it is good for. DEC is, broadly speaking, a way to state and treat a large class of differential equations on meshes. 

You most probably have already worked with some differential Operators. Some of the most common of them being the usual partial derivative $\frac{\partial}{\partial x_i}$, the gradient $\nabla = (\frac{\partial}{\partial x_1},\frac{\partial}{\partial x_2},...,\frac{\partial}{\partial x_n})$ or the Laplace Operator $\Delta = \frac{\partial^2}{\partial x_1^2} + \frac{\partial^2}{\partial x_2^2} +... + \frac{\partial^2}{\partial x_n^2}$. External Calculus provides a framework to treat those operators in a very clean and unified way. And it allows you to define and use these Operators on curved surfaces without much work.

But why should you care about this? As you probably know these operators are extremely powerful tools to describe all kind of problems. Examples---
And (non-discrete) external Calculus gives you tools to state and treat these problems very neatly.

You now might say that okay, external calculus is probably a nice way to write down equations that else would seem more cluttered and okay, it might help you describe problems even on curved surfaces, but how does this help with actual practical applications and computations? And this is what Discrete External Calculus is about: it provides a consistent and straight forward way to adapt the operators for the use on meshes for computations. 

\begin{figure}[bht]
\begin{center}
\includegraphics[width=6cm]{Imgs/1_1_line.eps}
\includegraphics[width=6cm]{Imgs/1_1_spline.eps}
\end{center}
\caption{Left: the discrete mesh you have to work with. Right: Splines are used to associate a differentiable function to the discrete set of points; saying this is a good way to approximate the underlying function makes assumptions about the underlying sampled function}
\label{fig::1_1_linevsspline}
\end{figure}

Consider this very simple situation: you have a 1-D function on a discrete set of points and want to calculate its derivative (Fig. \ref{fig::1_1_linevsspline}). Here you already encounter a problem. Obviously mathematically it is less than clear what derivatives should be on the pointset. You have different options. One would be to assume the points represent a differentiable curve which has a closed formula (depending on the pointset) such that a derivative can be calculated at any point. Another would be to embrace the situation, take the non-differentiable piecewise linear line and accept that you know only average values of the derivative on each line segment. I.e. you would say the derivative is the set of differences $f_{i+1} - f_i$ associated to the line segments $i$.

Both possibilities have advantages and drawbacks. While the first option (which leads to Finite Element like methods (FEM)) provides you a value for the derivative everywhere, the assumption that the spline is a good approximation of the ''real'' curve might be wrong and introduce errors. The second option gives you only a discrete set of values but you do not make any assumptions about the underlying curve, as for any differentiable function $f$ taking the values $f_i$, the averaged value is
\[\int_{line\;segment\;i} f^\prime dx = f_{i+1} - f_i\] 
So the second option never gives a wrong result. 

Discrete External Calculus is based on the second option. It takes the position that differential operations like taking the derivative on a discrete surface gives a discrete set of values (which have a true meaning for any underlying continuous function) and that interpolation is a somewhat independent problem.

\begin{figure}
\begin{center}
\includegraphics[width=6cm]{Imgs/1_1_1dLap.eps}
\includegraphics[width=6cm]{Imgs/1_1_1dLapSmoothing.eps}
\end{center}
\caption{An example illustrating how the DEC Laplace operator conserves a geometric property of the continuous Laplacian: When the Laplace operator is applied to the coordinates of the surface it is defined on, it defines an area minimizing flow. The image depicts the initial line with the initial Laplacian vectors (left) and the deformation of the line according to the Laplacian flow, where the border is fixed.}
\label{fig::1_1_1dSmoothing} 
\end{figure}

But the great strength of DEC is yet another one. Differential operators mostly are of a very geometric nature and have geometric properties. The gradient of a function for example always points in the direction of the highest local ascent. External Calculus is very well suited to capture some key geometric properties of those operators. 

While the piecewise linear line considered above does not allow differential operators to give concise values at all (or any) points, the geometric properties of the operators often are still meaningful. 
You can define the Laplace Operator $\Delta$ on and for surfaces; for now just assume that this is done somehow. If you then have a function defined ON the curve, you can apply $\Delta$ to this function (if both the surface and the function are 'nice' enough).  Some special functions that are defined on the line are the functions that assign coordinates to every point on the line, i.e. for a line in 2d the functions $x(p), y(p)$. 
The Laplace Operate has many very important properties. One of its (maybe less essential but rather plastic) geometric properties is that, if applied to the coordinate functions, it defines an area minimizing flow, as shown in Fig. \ref{fig::1_1_1dSmoothing}. This means that the vectors returned by the Laplacian point in a direction such that, if you move all points according to them, the area of the surface gets minimized in an optimal way.

While we can not calculate the usual Laplacian on the piecewise linear curve of Fig. \ref{fig::1_1_1dSmoothing} (it would be undefined between every two line segments and 0 everywhere else) the geometric property of area minimization stays meaningful. Thus this property can be directly used to define a discrete Laplacian. This discrete Laplacian then not merely approximates the continuous Laplacian, it also preserves some of its geometric properties \emph{exactly}.

And this is the fundament of DEC: differential operators are not merely approximated; some geometric key features from the continuous counterparts are identified and DEC is then designed to respect them exactly on a piecewise linear mesh. And, as you will see, external calculus serves you some rather deep geometric relations of these operators on a golden tablet.

% Define block styles
%\tikzstyle{decision} = [diamond, draw, fill=blue!20, 
%   text width=4.5em, text badly centered, node distance=3cm, inner sep=0pt]
%\tikzstyle{block} = [rectangle, draw, fill=blue!10, text width=7em, text centered, rounded corners, minimum height=4em]
%\tikzstyle{line} = [draw, -latex']
%\tikzstyle{cloud} = [draw, ellipse,fill=red!20, node distance=3cm,
%    minimum height=2em]


%\begin{tikzpicture}[node distance = 4cm, auto]
% Place nodes
%\node [block] (step1) {Identify Geometric Key Property};
%\node [block, right of= step1] (step2) {Formulate Property for Meshes};
%\node [block, right of= step2] (step3) {Define Operator such that the property is preserved};
% Draw edges
%\path [line] (step1) -- (step2);
%\path [line] (step2) -- (step3);
%\end{tikzpicture}    



\section{A Short Tour of this Thesis}
Before you lie dozens of pages full of incredible fun and full of mysteries which will be unravelled before your eyes. But so you don't have to take my word for it: take the small tour of this Thesis. 

\textcolor{red}{(TODO/REDO this is conceptual. Idea: about one or two sentences per theme and a nice looking image where a keyfeature is depicted. Plus the Text should have like different modules so it is easy to skip parts if you are more or less familiar with them)}

To get started some basics are covered; if you are already familiar with them you can skip them without any loss.
To begin with the geometric objects we want to work with are presented: smooth surfaces and meshes. Or more generally manifolds and simplicial complexes. Manifolds are any curved hyper-surfaces that are inbedded in a higher dimensional space and simplicial complexes are the higher dimensional equivalents of triangle meshes, including for example tetrahedra meshes.

Then the basics of differential geometry are treated; we look at local properties of surfaces and various kinds of curvature. We then try to define the equivalent on triangle meshes, which will lead to defining a Laplacian on meshes. As an application meshes get smoothed.

In the next section  differential forms are introduced and explained. This is the heart of the matter and explains why the core concepts of discrete differential forms make sense. Differential forms being a concept you might not have met as such this chapter is a crucial one. It will lead to Stokes Theorem which encodes the geometry of the differential Operator and also forms the heart of DEC.
\[\int_{\partial\Omega} \phi = \int_{\Omega} d\phi\]

The rest are various aspects, applications and refinements, where we will paralelly deepen the theory about differential forms (i.e. External Calculus) to understand aspects of practical problems and derive further parts of the discrete exterior calculus to computationally solve these problems.


%Smooth manifolds (i.e. curved surfaces) are introduced so we can later define differential operators ON them. Important notions are tangential spaces; as all applications presented here will take place on meshes (i.e. 2d surfaces embedded in 3d space) some basic notions from differential geometry are repeated; if you are more or less familiar with these notions you can skip this part.


What needs to be introduced:
\begin{enumerate}
\item Some basics about Manifolds (Background of the continuous Objects the Theory is based on)
	\begin{enumerate}
		\item Manifolds and Tangential Spaces
		\item Tangential Spaces, 
		\item Orientations
		\item how manifolds get a differential structure i.e. how differential Operators like a derivative can be defined ON surfaces
	\end{enumerate}
\item Basics about Meshes (About the Objects we are making computations on) HANDS ON
	\begin{enumerate}
		\item Simplices
		\item Orientations, Border Operator and Wind Edge structure
		\item dual meshs (and their orientations) Either here or later.
	\end{enumerate}
\item Some basic Differential Geometry
	\begin{enumerate}
		\item Curves and Tangents
		\item curvatures, curvature tensor
		\item how manifolds get a differential structure i.e. how differential Operators like a derivative can be defined ON surfaces
	\end{enumerate}
\item Differential Geometry on the Mesh: HANDS ON
	\begin{enumerate}
		\item Discrete Mean curvature
		\item Other curvatures
	\end{enumerate}
\item Differential forms:
	\begin{enumerate}
		\item What they are and how they arise
		\item How they form a linear space
		\item Operations: Stokes Theorem and the differential Operator.
		\item Identifying the linear spaces with vector fields etc,
		\item Mapping the differential operator d to classical differential operators
		\item The Hodge Star and the codifferential Operator (maybe later)
		\item Thingy when something is a closed form. (these things will be introduced when needed)
		\item The Hodge Decomposition
		\item DeRham Komplex: topological constraints to harmonic fields.
	\end{enumerate}
\item Basic Properties of differential Operators: (good question where to put this in, its quite important)
	\begin{enumerate}
		\item Divergence
		\item Gradient
		\item Curl
		\item Laplace Operator
	\end{enumerate}
\item Discrete Differential Forms HANDS ON
	\begin{enumerate}
		\item What they represent (avg values)
		\item How they are introduced
		\item Derivation of discrete d 
		\item dual mesh mesh stuff
	\end{enumerate}
\item Applications
	\begin{enumerate}
		\item Differential Geometry: Curvatures to analyse the local structure of a mesh, Smoothing, remeshing
		\item Conformal Maps
		\item Vector Field Design
		\item Fluid Simulation
	\end{enumerate}
\end{enumerate}
Oh lord, what shall i do...? Write a short! summary for each topic (MAX one Paragraph each), place them in a good order,work them out in a second iteration. Explain math without proofs, write proofs where helpfull later.

The next part 
The first practical part describes one or two applications where notions from differential geometry are discretized
First of all the smooth mathematical counterparts are treated. How to describe them and properties. Then 

What is contained in this Thesis: (TODO/REDO after the rest is written)
\begin{enumerate}
\item An explanation of the geometric nature of differential operators
\item Some basic properties of Manifolds i.e. curved surfaces.
\item The basics of external Calculus and refresh your knowledge of differential Geometry 
\item See how Discrete External Calculus is developed and which key features it tries to conserve.
\item Application of DEC to a group of problems out of various fields.
\item Cookbook recipe ?
\end{enumerate}
What is not treated in this Thesis but should not be ignored:
\begin{enumerate}
\item Other standard methods like finite differences or finite element method, even thought DEC is closely related to those and one would benefit of a closer look at FEM and its error analysis.
\item An analysis how well DEC methods perform in contrast to these standard methods or any Error Analysis of FEM.
\end{enumerate}
\chapter{The Basic Objects: Meshes, Manifolds and Sparse Matrices}

So lets get started. In this chapter we will have a closer look at the basic objects we will deal with throughout this text.  We start with the description of so called ''manifolds'' which describe smooth surfaces and more generally smooth spaces embedded in higher dimensional spaces.
In the second section we have a closer look at the meshes (and more generally Simplicial Complexes) that we use in practice in the place of manifolds. The third section explains how to implement meshes and simplicial complexes in a way that is convenient for DEC purposes. 

The third section is the actual hands-on section, whereas the first two sections provide the background and the foundations of Exterior Calculus and Discrete Exterior Calculus.
\begin{figure}[ht]
\begin{center}
	\begin{longtable}{|p{4cm}|p{4cm}|p{4cm}|}
		\hline
		Smooth Theory& Discrete Theory& Implementation (Notes)\\
		\hline
		Smooth Manifolds (Surfaces)
		\begin{packed_itemize}
		\item[-] Maps and Coordinates
		\item[-] Tangential Space
		\item[-] Orientations
		\item[-] Bordered Manifolds
		\end{packed_itemize}
		&
		Discrete Manifolds
		\begin{packed_itemize}
		\item[-] Simplices and Simplicial Complexes
		\item[-] Discrete Manifolds
		\item[-] Orientations
		\item[-] Border Operator, Border Matrix
		\end{packed_itemize}
		&
		Meshes
		\begin{packed_itemize}
		\item[-] Winged Edge Structure
		\item[-] General Complexes
		\item[-] Sparse Matrices
		\item[-] Simple Geometric Operations
			\begin{packed_itemize}
				\item[-] Manifold Check
				\item[-] Border Computation
			\end{packed_itemize}
		\end{packed_itemize} \\
%			Smooth Surfaces & Meshes & General Meshes\\
%			-Maps and Coordinates & & \\
%			-Tangential Space & -Simplices / Simplicial Complexes & -Winged edge / Incidence\\
%			-Orientations & -Orientations& -Simple geometric operations:\\
%			-Functions on Surfaces & -Border Operator & --Orientation\\
%			-Derivative on Surfaces & & --Iterating over Neighborhoods\\
%			& & --Wellformedness \\
%			& & --Finding Border Components\\
%			& & Notes on Sparse Matrices\\
		\hline
	\end{longtable}
	\caption{Overview of the topics of this chapter}
\end{center}
\end{figure}

\section{Manifolds}
		This section is an introduction to general surfaces. The whole thesis is about things you can do on or with smooth (hyper) surfaces, so it is important to understand their mathematical model. While no application in this thesis actually exceeds 3 dimensions, Discrete Exterior Calculus (DEC) and Exterior Calculus provide tools for arbitrary dimensions, without any complications. So there is no reason to stick with two and three dimensional spaces. 
		
\subsection{Describing Manifolds}	
		You have already seen various manifolds in your life. Spheres are manifolds.any smooth curve without self intersections are 1 manifolds, or most things you would call 'surface' are manifolds (e.g.  Fig. \ref{fig::2_1_manifold} depicts a two dimensional manifold). Generally A $k$ dimensional manifold in $\mathbb R^n$ is simply a geometric object in $\mathbb R^n$  that locally looks like $\mathbb R^k$.  Formally this is done by assuring that at every point on the manifold we have a 'map' linking the manifold to $\mathbb R^k$.
		
For the sake of simplicity we assume that all functions and mappings are infinitely differentiable.
		
\begin{figure}
	\begin{center}
		\includegraphics[width=5cm]{imgs/2_1_manifold.eps}
		\caption{A simple 2-manifold}
		\label{fig::2_1_manifold}
	\end{center}
\end{figure}
		
\begin{definition}[Map] A $k$ dimensional map is a differentiable mapping 
\[\phi: U \subset \mathbb R^k \rightarrow \phi(U)\] 
\[\begin{pmatrix}
	u_1\\ \vdots \\ u_k
\end{pmatrix} \rightarrow \begin{pmatrix}x_1(u_1,...,u_k)\\x_2(u_1,...,u_k)\\ \vdots \\ \vdots \\ x_n(u_1,...,u_k)\end{pmatrix}\]
that is injective and whose Jakobimatrix has rank $k$ on all $U$, where $U$ is some open subset of $\mathbb R^2$ (image).
		
\end{definition} 

A map is exactly like the maps you have in an atlas; they link the surface of the world locally to a flat 2d surface. Now a $k$ manifold is an object where at any point you can find a map of its neighborhood:

\begin{definition}[Manifold] A subset $S\subset \mathbb R^n$ is a (regular) $k$ manifold, if for each point $p \in S$ there exist an open set $V\subset \mathbb R^k$ such that there is a map $\phi: U \rightarrow  V\cap S$.
\end{definition} 

	
\begin{figure}
	\begin{center}
		\includegraphics[width=12cm]{imgs/2_1_mapping.eps}
		\caption{An example for a 2d map $\phi$}
		\label{fig::2_1_mapping}
	\end{center}
\end{figure}

This definition using maps ensures that the surface locally has a structure that is equivalent to a patch in $\mathbb R^k$. Note that this definition does not allow borders to occur. \note{(Does that need explanation?)}

You can also say that a map $\phi: U \subset \mathbb R^k \rightarrow S$ assigns local coordinates to a surface. The tuple $u=(u1_,...,u_k)\in U$ are the local coordinates of the point $\phi(u)$. You then can speak of expressing some functions $f:S \rightarrow ?$ in the local coordinates given by $\phi$. All this means is that you consider $f \circ \phi : U \subset \mathbb R^k \rightarrow ?$ instead of $f$. %(Image : local coordinates..) 

\subsubsection*{A Note on Coordinates}
Note that there are two types of coordinates here. Lets look at a 2-Manifold. You can have the local coordinates given by and depending on the local parametrisations you chose (giving you (u,v) coordinates). Then you have the global coordinates of the surrounding space $\mathbb R^3$, assigning (x,y,z) coordinates to every point.

But in the end all coordinates are just descriptive tools; the geometric object we describe is assumed to exist independently of all these parameters. The 'tricky' part then is to get independence of parametrizations when we want to describe geometric properties of our manifolds. Even to get a definition of 'differentiability' on the manifold that is independent of the selected parametrization one must be careful with the definition.

In fact in later applications  we will play around with coordinates, (with surface smoothing in chapter ... the global coordinates, with conformal maps in chapter ... the local coordinates). Rather than taking coordinates as describing the manifolds we take them as something associated to the geometric object to work with them or solve for them.

External calculus (introduced in chapter...) will provide tools and results that do not use parametrizations.


	
\subsection{Tangential Spaces}		
Manifolds have tangential spaces.
For curves calculating tangents is easy. If you have a parametrization $\alpha(t): I \subset \mathbb R \to \mathbb R^n$, then $\alpha'(t) = (x_1'(t),x_2'(t),...,x_n'(t))$ is the direction of a tangent vector to the curve at the position $\alpha(t)$. 

\begin{figure}[tb]
\begin{center}
\includegraphics[width=10cm]{imgs/2_1_tangent.eps}
\end{center}
\caption{A parametrized curve, the vector $\alpha'$ and the tangential space $T_{\alpha(t_0)}$ at $t_0$}
\end{figure}

		While the length of $\alpha'(t)$ at the point $\alpha(t)$ depends on the parametrization $\alpha$ (for example $\alpha(2t)$ is a different parametrization of the curve $S$ described by $\alpha$ where the length doubles) the \textbf{tangential space} $ T_{\alpha(t)}S = span(\alpha'(t)) = \{x \in \mathbb R^3: x = c \alpha'(t), c \in \mathbb R\}$ does obviously only depend of the position $\alpha(t)$ on the curve. 
		
We can do the same for a $k$ dimensional manifold $M$. The tangential space $T_p M$ at a point $p$ is:

\begin{enumerate}
	\item The space that approximates the surface in the best way, locally at $p$.
	\item The space that contains the tangents of all curves on the surface that go through $p$.
	\item For a given parametrization $\phi: \mathbb R^k \to \mathbb R^n$, $\phi(u) = (\phi_1(u),...,\phi_n(u))$ the tangential plane $T_{\phi(u)}$ is given by
			\[span(\frac{\partial \phi} {\partial u_1},..., \frac{\partial \phi} {\partial u_k}) = span(\begin{pmatrix}
	\frac{\partial \phi_1} {\partial u_1} \\
	\frac{\partial \phi_2} {\partial u_1}\\
	\vdots\\
	\frac{\partial \phi_n} {\partial u_1}
\end{pmatrix},...,\begin{pmatrix}
	\frac{\partial \phi_1} {\partial u_1} \\
	\frac{\partial \phi_2} {\partial u_1}\\
	\vdots\\
	\frac{\partial \phi_n} {\partial u_1}
\end{pmatrix})\]
	One could also interpret this as the $span$ of the tangents of a set of special curves, the curves $\phi(t, u_2,...,u_k)$, $\phi(u_1, t,...,u_k)$ ,..., $\phi(u_1, ...,u_{k-1},t)$, where $u_1,..,uk$ are fixed values.
\end{enumerate}

\begin{figure}[tb]
\begin{center}
\includegraphics[width=12cm]{imgs/2_1_mapping_coords.eps}
\end{center}
\caption{A map $\phi$ of a 2 manifold $M$ is used to determine the tangential space $T_{p}M$ at some point $p$}
\label{fig::2_1_mapping_coords}
\end{figure}

The tangential spaces $T_p M$ are the vector spaces containing all tangential vectors at a point $p$. Every point on a manifold gets an associated vector space. While the vector space itself is depending solely on $p$ the choice of a basis of this vector space is open. Choosing the basis $\frac{\partial \phi}{\partial u_i}$ obviously is dependent on $\phi$. Note that, just as in Fig. \ref{fig::2_1_mapping_coords}, these vectors are in general not orthogonal and if we want to define and use a scalar product consistently in all tangential spaces we will have to take care.
\note{Reformulate last sentence, what do i want to say?}.

One special example of a manifold is $\mathbb R^k$ seen as a manifold parametrized by $\mathbb R^k$ with $\phi = id$. Then the tangential space $T_p \mathbb R^k$ at any point $p$ is again $\mathbb R^k$. But still: the vectors in one tangential space $T_p \mathbb R^k$ and the vectors of another tangential space $T_q \mathbb R^k$ can not be mixed; every point gets its own proper tangential space, not shared with any other point.
		
\subsection{Orientations}
We do not want to talk about just any Manifolds but only about a subgroup: orientable manifolds. Orientation is easy to define; it is to assign a sign to the volume you are treating. Either your volume is positive or negative. Which orientation is positive and which negative is arbitrary. 

For a vector space you can encode orientation in the ordering of basis vectors. Two ordered bases $v_1,...v_k$ and $w_1,...,w_k$ describe the same orientation if the matrix that describes the change of bases has a positive determinant (see Fig. \ref{fig::2_1_orientation2d}). The determinant actually measures signed volume spanned by a set of vectors.

\begin{figure}[hb]
\begin{center}
\includegraphics[width = 10cm]{imgs/2_1_orientation2d.eps}
\caption{Three bases orienting $\mathbb R^2$, the first two describe the same orientation, the third a different one }
\label{fig::2_1_orientation2d}
\end{center}
\end{figure}

In the last section we introduced tangential spaces and emphasized the point that every point gets its own proper tangential vector space. Tangential spaces of points that are very close together are very similar and it makes sense to ask if they have the same orientation.

%In the last section we introduced tangential spaces and 

We saw that parametrizations can provide bases on tangential spaces. One single parametrization actually induces consistent orientations for the tangential spaces of all points it hits. Now we say that a Manifold can be oriented if all tangential spaces can be oriented consistently.

\begin{definition}[Oriented Manifold] A manifold can be oriented if the exists a set of maps $A = \{\phi: U_\phi \to \phi(U_\phi) \subset M\}$ such that the maps describe the whole manifold and any two maps $\phi$, $\psi$ which describe a common patch $\psi(U_\psi) \cap \phi(U_\phi)$ result in the same orientations i.e. for the base change matrix $C$ from the base $D\phi$ to $D\psi$ has positive determinant
\[det(C) >0\]
A manifold is oriented if for all tangential spaces a consistent orientation has been chosen.

\end{definition}

For 2d manifolds in 3d space this is the same as saying that you should be able to consistently chose a surface normal, as depicted in Fig. \ref{fig::2_1_mobius}.

\begin{figure}[t]
\begin{center}
\includegraphics[width = 6cm]{imgs/2_1mobius.eps}
\caption{The Moebius strip, the pathological example of a not orientable manifold}
\label{fig::2_1_mobius}
\end{center}
\end{figure}

\subsection{Manifolds with Border}
Up to now we have only looked at manifolds without borders, like the surface of a sphere or a torus. But treating borders is a central part of DEC so we extend our notions now. While up to now we required that a manifold looks locally like $\mathbb R^k$, we now allow it to alternatively look like $\mathbb H^k$, where $\mathbb H^k$ denotes the $k$ dimensional half space $\mathbb H^k = \{x = (x_1,...,k_k)\in \mathbb R^k: x_k \geq 0\}$.

\begin{definition}[Bordered Manifold] $M \subset \mathbb R^n$ is a $k$ dimensional manifold with border, if for any point $p \in M$ either conditions hold:
\begin{enumerate}
\item There are two open sets $p \in V\subset \mathbb R^n$ and $0\in U \subset \mathbb R^k$ and a differentiable injective map $U\to V$ whose Jacobi matrix has rank $k$ such that 
\[h(U\cap \mathbb H^k) =  V\cap M\]
\item  There are two open sets $p \in V\subset \mathbb R^n$ and $U \subset \mathbb R^k$ and a differentiable injective map $U\to V$ whose Jacobi matrix has rank $k$ such that 
\[h(U) =  V\cap M\]
\end{enumerate}
\end{definition}

\note{This definition might not be rigorously correct.} This just makes sure that every point on bordered $k$ manifold either is inside the manifold or on a $k-1$ dimensional border. Actually the border of a $k$ manifold is a $k-1$ dimensional manifold without border. \note{(Image! Hk and bordermap)}.  We denote the border of a manifold $M$ by $\delta M$ and call $\delta$ the border operator.

While nothing changes for tangential spaces of non-border points, at border points $p$ two tangential spaces are present. One is the tangential space of the manifold $T_pM$ and $k$ dimensional, the other one is the $k-1$ dimensional tangential space of the border manifold $T_p \delta M$ (Fig. \ref{fig::2_1_borderManifold}, left) . The important thing is that an oriented manifold induces an orientation to its border. Remember that orientation is defined by the orientation of the tangential spaces, so inducing an orientation to the border means inducing an orientation in $T_p\delta M$ using the orientation of $T_p M$. This happens defining normals on the border.

\begin{figure}
\begin{center}
\includegraphics[width = 13cm]{imgs/2_1_borderedManifold_combined.eps}
\end{center}
\label{fig::2_1_borderManifold}
\caption{A bordered manifold with tangential spaces $T_pM$, $T_p\delta M$ and the border normal $N$ (left image). The right image depicts how an orientation of the manifold given by bases oriented like ($b_1$,$b_2$) induces an orientation to the border: $N$ and a vector following the border orientation build a basis oriented like $b_1,b_2$.}
\end{figure}

For any border point you can define a surface normal $N$. The surface normal $N$ is the vector in $T_p N$ with:
\begin{itemize}
\item $N$ is orthogonal to $T_p \delta N$
\item $N$ has length 1
\item $N$ points 'outside'
\end{itemize}
Pointing outside is defined formally using the map $h$ at the border; $Dh$ is a linear bijective map from $\mathbb R^k$ to $T_pM$, so $N$ can be pulled back to $\mathbb R^k$ and it points 'outside' if the $k$th component of $Dh^{-1} N$ is negative \note{(Image?)}.

We defined orientation by the enumeration of basis vectors. So if a basis $b_1,...,b_k$ gives the orientation of $T_pM$,
a basis $\widetilde{b_1},...,\widetilde{ b_{k-1}}$ of the tangential space of the border $T_p\delta M$ is oriented according to the manifold if prepending the normal $N$ to the basis $N,\widetilde{b_1},...,\widetilde{ b_{k-1}}$ has the same orientation as $b_1,...,b_k$. This is also shown in Fig. \ref{fig::2_1_borderManifold}.


\subsection{Functions and Derivatives on Manifolds}
\note{	Tangential Spaces and Differential structure. Maybe put it in next chapter; here its all about geometry, not about functions.}

\newpage
\section{Meshes and Simplicial Complexes}

In the last sections we had a look at the geometric objects exterior calculus will be defined on, i.e. smooth surfaces and manifolds. The next step is to introduce the discrete analogues we want to do computations with: triangle meshes, or more generally simplices and simplicial complexes. Simplices are for example points (0 dimensional), lines (1 dimensional) triangles (2 dimensional) and tetrahedra (3 dimensions). Simplicial complexes are 'meshes' made out of them.

\subsection{Simplices and Simplicial Complexes}

\begin{figure}[b]
\begin{center}
\includegraphics[width = 10cm]{Imgs/2_3_simplices.eps}
\end{center}
\caption{A 0-simplex (point) 1-simplex (line) 2-simplex (triangle) and 3-simplex (tetrahedra) }
\label{fig::2_3_simplices}
\end{figure}

A $k$ simplex is the most basic geometric object with a $k$ dimensional volume: the convex hull of $k+1$ points, as depicted in Fig. \ref{fig::2_3_simplices}. You also have to make sure that no point lies in the convex hull of the others, as else no $k$ dimensional volume is spanned, else you would call the simplex degenerated.

\begin{definition}[Simplex]: A non degenerated $k$ simplex is the convex hull of $k + 1$ points $p_1,...,p_{k+1}$, where the $k$ vectors $p_2 -p_1, p_3,-p_1, ..., p_{k+1} -p_1$ are linearly independent. It is represented as a tuple of its corner vertices $\{p_1,...,p_{k+1}\}$.
\end{definition}

Every simplex has faces of various dimensions: any combination of $l+1$ of its corner vertices forms an $l$ dimensional face. For example the tetrahedra has 4 2-dimensional faces (triangles), 6 1-dimensional faces (edges) and 4 0-dimensional faces (vertices) (see Fig. \ref{fig::2_3_simplices}). A 4-Simplex (living at least in $\mathbb R^4$) would have 5 tetrahedral faces and so on.

Out of this simplices one can make simplicial complexes, just as you can build meshes out of triangles. The restrictions are the usual: the interior of any two simplices should not overlap, and if the intersection of two simplices is not empty, the intersection has to be a face of both simplices. \note{(image)} A simplicial complex then is a list of simplices. 

Formally, when a $k$-simplex is listed, its faces are not yet in the list; this situation we don't want to have. If a simplicial complex contains a  $k$-simplex we demand that the faces of those $k$- simplices are contained too. This is not just a tedious technical detail; you will later see that we want to associate values to faces of simplices. In a triangle mesh for example we will therefore need to keep track not only of triangles and vertices but also of the edges; see also section \ref{sec::2_handsOnSimplicialComplexes}.

\begin{definition}[Simplicial Complex]
A simplicial  complex is a collection $\kappa$ of simplices, such that if a simplex is contained in $\kappa$, all its faces are too. Furthermore the intersection of any two simplices in  $\kappa$ is either empty or a common face.
\end{definition}

Lastly we do not want our discrete manifolds  to have the analogue of dangling triangles. To ensure this formally one has to make a restriction that is similar to the definition of manifolds. Just as we ensured that a (border less) $k$ Manifold locally looks like $\mathbb R^k$, i.e. every point has an open $k$ dimensional environment, we want to make sure our discrete manifold (with border) looks locally like either a $k$-dimensional ball or a $k$-dimensional half-ball. This is a technicality to get rid of dangling things (Fig. \ref{fig::2_2_dangling}).

\begin{figure}
\begin{center}
\includegraphics[width=13cm]{imgs/2_2_dangling.eps}
\caption{These are not discrete 2d manifolds: the first mesh has a dangling triangle, the second mesh has a 'wheel' and is not locally equivalent to a plane, the same holds for the third mesh}
\label{fig::2_2_dangling}
\end{center}
\end{figure}

\begin{definition}[Discrete Manifold]
A $k$-dimensional discrete Manifold is a simplicial complex where for every vertex in $\kappa$ the union of all incident simplices is equivalent to a $k$-dimensional ball or a $k$-dimensional half ball.
\end{definition} 
\note{(defs from the computational modeling paper)}


\subsection{Orientations of Simplicial Complexes}
As with manifolds we have to deal with orientations in discrete manifolds; as we will later see they will be quite of some practical importance and can be a notorious source of switched sign errors when implementing things using discrete exterior calculus. 

We can assign one of two orientations to any simplex of any dimension, meaning that the volume represented by the simplex should be considered as positive or negative. While we coded orientation before via the enumeration of basis vectors, for simplices we encode orientation via the enumeration of their corner vertices.  For edges it is the most intuitive what this means: we simply assign a direction to the edge $\{p_1,p_2\}$ by saying the first vertex listed is the start vertex of the edge. Note that for an edge or geometric object there is not a strict 'positive' or a 'negative' orientation; we can only say how something is oriented relative to something else. For example the edge $\{p_1,p_2\}$ is oriented negatively to the edge $\{p_2,p_1\}$; this is noted as
\[-\{p_1,p_2\} = \{p_2,p_1\}\]
So the orientation of a $k$-simplex depends on the way its corner vertices are enumerated. Two enumerations of corner vertices result in the same orientation if they are related by an even permutation. A permutation is called even, if it can be reproduced by switching pairs of vertices an even time. E.g.
\[\{a,b,c,d\} \sim \{c,a,b,d\}\]
\[\{a,b,c,d\} \rightarrow \{c,b,a,d\}\rightarrow \{c,a,b,d\}\]
by first switching $a$ and $c$ and then $a$ and $b$.
You can also use the determinant to determine the sign of the permutation; just calculate the determinant of the permutation matrix
\[\{a,b,c,d\} \rightarrow \{c,a,b,d\}\]
\[\begin{pmatrix}c\\a\\b\\d \end{pmatrix}=\begin{pmatrix} 0 & 0 & 1 &0 \\ 1 &0&0&0 \\ 0&1&0&0 \\ 0&0&0&1 \end{pmatrix}\begin{pmatrix} a\\b\\c\\d \end{pmatrix}\]
Or again you can use the simplex to induce a base to the affine space it is aligned to
\[p_1 -p_2,...,p_{k}-p_{k+1}\]
and two enumerations induce the same orientation if these bases have the same orientation. This also shows that looking at the orientation of a simplex by looking at the ordering of its corner vertices amounts to the same as orienting its volume by choosing a bases.

One exception are vertices or 0-simplices $\{v_0\}$, where orientation is not encodable in the enumeration of the vertex. We need to assign orientations to single points too and say that $-\{v_0\}$ is the negatively oriented version of $\{v_0\}$. Orientation is 'imprinted' on the point. The best way of thinking of orientation is that orientation adds a sign to volumes. A negatively oriented point is then a point whose $0$-dimensional volume is negative. The $0$-dimensional volume of any single point is defined to be either $1$ or $-1$ and the 0 dimensional volume of a point set is then $\#positive\; points - \#negative\; points$.

Anyway as long as you stick with calculations in $\mathbb R^3$ it stays pretty simple to determine if two orientations a simplex are the same, if you stick with triangle meshes it is trivial. Just make sure you always remember to respect orientations. In the implementation chapter \ref{sec::2_handsOnSimplicialComplexes} we will also come back to the question of how to compute relative orientations.


\subsection{Border Operations}
\label{sec::2_borderOrientation}

Just as with manifolds we have a border operator for discrete manifolds. And just like with manifolds an oriented discrete manifold induces an orientation to its border.

We first introduce some notation: we can respresent collections of simplices as $formal$ sums, as depicted in Fig. \ref{fig::2_2_formalsum}. The simplices are represented as tuples, a negaitve sign means a change of orientation. Simplices that are different from each other are not summed up (the sum is formal), only if two tuples describe the same simplex (but for orientation) the sum is taken. Particularly simplices of opposed orientation cancel out.
\begin{figure}
\begin{center}
\includegraphics[width = 12cm]{imgs/2_2_formalsum.eps}
\end{center}
\caption{Two sets of edges expressed as a formal sums that get summed up}
\label{fig::2_2_formalsum}
\end{figure}

The border of a single $k$ simplex is then the following formal sum
\[\delta\{v_0,v_1,...,v_k\} = \sum_{j=0}^k (-1)^j\{v_0,...,\widehat{v_j},...,v_k\}\]
where  the $\widehat{v_j}$ means omitting $v_j$. So this expresses that the border of the simplex is a set of $k-1$ simplices. The orientations they get are exactly the ones that the simplex induces. Note that prepending the omitted vertex $v_j$ to $(-1)^j\{v_0,...,\widehat{v_j},...,v_k\}$ exactly leads to a simplex with the orientation of $\{v_0,v_1,...,v_k\}$.

For example the border of a triangle $\{a,b,c\}$ then is \[\{b,c\} -\{a,c\} + \{a,b\} = \{a,b\} + \{b,c\} + \{c,a\},\] just as we want it to be.

But if we can take the border of single simplices, we can also take the border of a set of simplices and especially of discrete manifolds; it is simply the formal sum of the borders of the $k$ simplices the discrete $k$ manifold is made out of. As you see in the figure \ref{fig::2_2_borderUnoriented} this can go 'wrong', and it does so when the discrete manifold was not oriented. 

\begin{figure}
\begin{center}
\includegraphics[width = 7cm]{imgs/2_2_borderUnoriented.eps}
\end{center}
\label{fig::2_2_borderUnoriented}
\caption{The border operator that respects orientation only makes sense with oriented discrete manifolds. Orientation of faces is depicted by an arrow that says what orientation a simplex induces to its border}
\end{figure}

If all simplices occurring in a complex are enumerated and have a fixed orientation, the border operator can be expressed as a matrix, the incidence matrix. A set of $j$ simplices is represented as vector of integers of dimension $\#(j\;simplices)$. The $k$-th entry of this vector represents the number of times the $k$-th $j$-simplex occurs.
\begin{center}
\includegraphics[width=10cm]{imgs/2_2_complexEnumeration.eps}
\end{center}
Using this enumeration the two border matrices are
\[\delta_1 = \begin{pmatrix}
-1&-1&0 &0 & 0\\
0&1&1 &1 & 0\\
1&0&-1 &0&1\\
0&0&0&-1&-1\\
\end{pmatrix} \]
\[\delta_2 = \begin{pmatrix}
1 & -1 & 1 &0&0\\
0& 0& -1 & 1 & -1
\end{pmatrix}\]
We will always make the difference between the border operators; we add to the border operator of $j$-simplices a $j$ as subscript: $\delta_j$. The border of the line segment $e_0 -e_4 + e_3$ is then given by
\[\begin{pmatrix}
-1\\ 1\\ 0 \\0 
\end{pmatrix} = \delta_1\begin{pmatrix}
1\\0\\ 0\\1\\-1
\end{pmatrix}\]
i.e. $-v_0 + v_1$ saying that $v_0$ is the 'start' and $v_1$ the 'end' border of the line. 

The entry $(i,j)$ in the border matrix is the relative orientation of the two simplices concerned. For example $\delta_1(0,1) = -1$ because the vertex $v_0$ is oriented negatively relative to the edge $e_1$. Relative orientation means considering the border orientation induced by the edge $e_1$ and the orientation of $v_0$.

 \note{Think there is no need to say more, is there? Yes image simplex k=> simplex k-1,,,, via $\delta$}

\subsection{Oriented Discrete Manifold}
\label{sec::2_orientedDiscreteMF}
Lastly we can not only orient single simplices, but also a whole discrete manifold. You have seen that the orientation of borders is strongly linked with the orientation of the volume. For convenience we will define well orientedness of a discrete manifold using the border orientations.

Two $k$-simplices that share a $k-1$ dimensional face are oriented consistently exactly if the induced orientation of this face is opposed for both $k$ simplices, as in Fig. \ref{fig::2_2_borderUnoriented}.
A $k$ manifold is oriented if all $k$ simplices are oriented consistently. As you have seen above, if the manifold is not oriented, the border operation will not be as expected. So make sure that in all implementations your discrete manifolds are well oriented.



\newpage
\section{Implementation: Mesh and basic operations}
This is the hands-on part of this chapter. The implementation chapters provide a guideline of what you need to implement to get DEC and the later applications up and running. The components needed are described and some of the more tricky details are mentioned.

\subsection{A word on Sparse Matrices}
The point of DEC is to reformulate differential equations with sparse matrices. Therefore the implementation is somewhat centered around sparse matrices.

If you plan to implement your DEC framework you should start by looking for a sparse matrix solver. For all results in this thesis the sparse Solver from the Pardiso-Project of the University of Basel has been used as a black box solver. Unfortunately it is not free-ware but any other sparse solver will do as well.

Sparse matrices are matrices where most entries are zero. Instead of storing all $n\times m$ values of a matrix, you choose to only store the non-zero values and their indices. There are different ways to do this; the pardiso solver uses the so called Yale format.

The Yale format uses 3 vectors to describe an arbitrary $n\times m$ matrix $A$. The first vector $a$ stores all non-zero values of $A$, enumerated by row. The second vector $ja$ stores the column indices of the non-zero values, again enumerated by row. The third vector $ia$ stores for every row the index $i$, such that $a(i)$ and $ja(i)$ describe the value and the row of the first element in the row. Additionally one appends the number of values in $a$ (or $ja$) to $ia$.

For example 
\[\begin{pmatrix}
1 & 0 & 0 &3 \\
 0 & 0 & 0 &2 \\
 0 & 4&2&0
\end{pmatrix} \Rightarrow \begin{cases} a &= [1,3,2,4,2] \\ ia &= [0,2,3,5]  \\ ja &= [0,3,3,1,2]\end{cases}\]
Iterating over the values and indices of the $k$th row then amounts to
\begin{algorithmic}
\FOR{i = ia(k):ia(k+1)}
	\STATE out $\gets (k,ja(i))$   //the index pair
	\STATE out $\gets a(i)$  //of this value
\ENDFOR
\end{algorithmic}

Whatever implementation of sparse matrices you choose to use or implement, the usual basic operations need to be implemented:
\begin{itemize}
\item multiplication of matrices
\item transposing matrices
%\item iterate over the column indices and values of any row
\item adding matrices
\item inverting the elements of a matrix (replace the non zero values $a$ by $1/a$)
\item multiplication of vectors
\end{itemize}

\subsection{Implementing a Mesh for DEC}
\label{sec::2_handsOnSimplicialComplexes}
Most application in this thesis focus 2-complexes i.e. classical triangle meshes, you might not need any more general implementation so we treat general $k$ complexes separately in the next section. 

For DEC we need the complete geometric information of meshes; we explicitly keep lists of vertices, edges and faces, the full information about their incidence and border relations, as well as their assigned orientations. Edges are stored once, with an arbitrary chosen orientation.

For 2d meshes a winged edge structure is a convenient choice of representation. You could also choose to represent the mesh just by keeping lists of vertices, edges and faces plus the incidence matrices, as mentioned in the next section. \note{There might be better choices... what? References?}

\begin{figure}[tb]
\begin{center}
\includegraphics[width=8cm]{imgs/2_1_wingedEdge.eps}%	
\end{center}
\caption{The information stored on a winged structure edge}%
\label{figs::2_1_wingedEdge}%
\end{figure}

In a winged edge structure you have the following three objects:
\begin{figure}[h]
\begin{center}
\includegraphics[width = 11cm]{imgs/2_2_wingedEdge2.eps}
\caption{Impementation of a winged edge structure}
\end{center}
\end{figure}

With this information present it is easy to do things like iterating over the incident edges or faces of a vertex.

%\subsection{Implementing the Border Operator}
%\note{mention this here or include it in the border section?}


%\newpage
\subsection{Implementing k-Simplicial Complexes for DEC}

\begin{figure}%
	\begin{center}
	\includegraphics[width=10cm]{imgs/2_1_Complex.eps}%	
	\end{center}
	\caption{Implementation of a $k$ Complex; use tuple of ordered indices to characterize a simplex}%
	\label{fig::2_1_Complex}%
\end{figure}

Chances are you do not need simplicial complexes of higher dimensions other than tetrahedral meshes embedded in $\mathbb R^3$. 
	
Never the less one straight forward and for DEC suitable way to implement arbitrary $k$ complexes is to store lists of simplices and represent the incidence information explicitly as sparse matrices. The incidence matrices (border operator matrices) play a central role in DEC and need to be set up anyway.

An implementation of a $k$-complex then might look like this: the vertices (0-Simplices) are stored in a list and contain their positions. A single $j$-Simplex is then represented by a $j$ tuple of vertex indices. A $k$ complex consists then of $k+1$ simplex lists; for every dimension one list, as sketched in Figure \ref{fig::2_1_Complex}.

Setting up the border operator matrices $\delta_j$ for complexes of arbitrary dimensions is not completely trivial, as to compute relative orientations of simplices you need to find the sign of some permutation. It gets much easier if the index tuples describing the simplices are sorted i.e.
\[(i_1,i_2,...,i_j): i_1 < i_2 <...<i_j.\] 
With sorted indices we can directly use the definition of the border operator from Section \ref{sec::2_borderOrientation} to compute the relative orientation of a $j-1$ simplex $(v_0,...,\widehat{v_l},..., v_{j-1})$ lying on the border of a $j$ simplex $(v_0,...,\widehat{v_l},..., v_{j})$:
\[Orientation = (-1)^l\]

But while it is easier to compute relative orientations if the indices of your simplices are sorted, you loose the ability to store arbitrary simplex orientations using the ordering of vertices. For all but the $k$ simplices this does not matter, even for oriented discrete $k$-manifolds, as there is nor 'right' or 'wrong' orientation and all that matters is that you consistently use the same orientation all the time. But for the $k$ simplices in an oriented $k$ manifold you really need to be able to chose the orientation, so you have to keep track of the orientation independently in an additional variable (as is done in Figure \ref{fig::2_1_Complex}).

So when you resort the tuple of a $k$-simplex in a $k$-manifold you need to determine if an index tuple describes the same orientation as the sorted index tuple. This can be done using a so called inversion table. Lets say the tuple $(1,2...,n)$ is scramble to the tuple $(i_1,...,i_n)$. An inversion is then an index pair $(i_l,i_k)$, where $l<k$ but $i_l >i_k$, i.e. the order of $i_l,i_k$ was inverted. The number of inversions of a single index is then the number of indices left to it that are greater than the index. The relative orientation of a simplex represented by a scrambled tuple to a simplex with the sorted tuple is $(-1)^{\#inversions}$ where $\#inversions$ is the total number of inversions.

Example: 
\begin{eqnarray*}
3,2,5,4,1 \\
0,1,0,1,4
\end{eqnarray*}
The first line represents the permuted indices the lower the number of inversions of every index. The total number of inversion is 6 and the relative orientation of
$\{3,2,5,4,1\}$ to $\{1,2,3,4,5\}$ is $(-1)^6 = 1$ which would meant that both tuples represent simplices with the same orientation.

Setting up a $k$ complex  and all the border matrices $\delta_j$ might then look like this: start with the $k$ simplices; resort their indices if needed and adapt the stored orientation.
Then enumerate all occuring $k-1$ simplices (sort their indices, you do not need to adapt the orientation) and set up the border matrix $\delta_k$. Proceed with enumerating all $k-2$ simplices occuring as borders of $k-1$ simplices and the set up of the $\delta_{k-1}$ matrix and so on.

		
\subsection{Implementing Basic Mesh Operations}
\note{IMAGE of the bad teapot that has dangling triangles, an unexpected border and multiple components}
To get later applications working you will need most of the following tools. For the applications presented in this thesis they do not need to perform extremely fast, as these operations will occur only once when setting up an application.

\subsubsection{Setup a DEC mesh from a Wireframe Mesh Representation}
The usual representation of meshes as e.g. with .obj files is by just giving a list of vertex positions and a list of faces; you need to set up the winged edge structure or $k$ complex from these.
	
\subsubsection{Set Up Border Matrices}
If you chose a winged edge structure to implement 2-complexes, you additionally need to set up the border matrices $\delta_2$ and $\delta_1$. Store them directly with your mesh. These matrices play a central role in DEC; you could say that this whole thesis is about these matrices, so test them well. If you have an oriented discrete manifold, you can also use the relation $\delta_{k}\cdot \delta_{k+1} = 0$ (the border of the border of an oriented manifold is empty) to check their correctness. Implementing the other tools in this section will test these matrices further.

\subsubsection{Check if a DEC Mesh is a Discrete Manifold}
\begin{figure}[tb]
	\begin{center}
	\includegraphics[width=12cm]{imgs/2_3_danglingTeapot.eps}
	\end{center}
	\caption{A teapot mesh that on the first look seems to be a discrete borderless 2 manifold but turns out to be a mesh with border and dangling triangles, which makes it a non-manifold mesh and therefore not suited for some DEC applications}
\end{figure}

To avoid singular matrices and to eliminate the possibility that bugs occur due to the ill-formedness of a $k$-complex or mesh (as in Fig. \ref{fig::2_2_dangling} and \ref{fig::2_2_borderUnoriented}) it is handy to have such a test method. I.e. you should test for orientation errors and the connectedness to avoid dangling simplices.
The orientation needs to be checked for $k$ simplices only, which would be for example the faces for a 2D-mesh. 

That a $k$-complex is oriented can be checked by looking at $\delta_k$. Any column has to have either exactly one entry or two entries that sum up to zero. This checks exactly the condition we gave in \ref{sec::2_orientedDiscreteMF}: Any 1 simplex is either on the border (therefore being part of exactly one $k$ simplex) or between 2 $k$-simplices, having once positive and once negative orientation.

If you are sticking with the winged edge mesh, finding dangling faces is fairly simple; at every vertex iterate over all edges and make sure that exactly 2 or no edges have only one neighbour face.
\begin{figure}
	\begin{center}
	\includegraphics[width=7cm]{imgs/2_3_danglingTriangles.eps}
	\vspace{0.5cm}
	
	\includegraphics[width=10cm]{imgs/2_3_DanglingTetrahedra.eps}
	\end{center}
	\caption{Top: On the left all 2-simplices (triangles) can be reached indirectly by hopping from triangle to triangle where every hopping pair shares a 1-simpex (considering only triangles that neighbour the marked vertex). On the right the manifold property is violated. Bottom: two 3-simplices that share a common face (no dangling), that share an edge (dangling) and that share a vertex (dangling)}
	\label{fig::2_3_dangling}
\end{figure}
Detecting dangling $k$-simplices in a $k$-complex needs slightly more work. Formally dangling was prevented by asking that at any vertex, the incident $k$-simplices form either a ball or a half ball. This is equivalent to asking that any two $k$ simplices neighbouring some vertex $v_0$ are connected via $k$ simplices neighbouring $v_0$ that share $k-1$ faces, as depicted in Figure \ref{fig::2_3_dangling}.  So to check wellformedness at a vertex, take all neighbour $k$-simplices, choose one and put it on a stack. Pop the stack and push all $k$-simplices that share a $k-1$ face with the popped simplex. Keep on popping and pushing like this. If any $k$ simplex remains you have a dangling situation. \note{(Clear enough?)}
	
	
Note that these non-manifold detection algorithms will not detect non-manifoldness due to self intersections (like the most right situation in Fig. \ref{fig::2_2_dangling}), as such problems do not show in the incidence matrices. But for most applications it is enough that we have well-formed incidence matrices and do not care if such self intersections occur.
	
\subsubsection{Find Borders}
Given an oriented discrete $k$-manifold: find the $(k-1)$-complex that represents its border. Finding the border can be easily done by applying the border Operator $\delta_k$ to the $(1,1,1,1,1...)$ vector. The resulting vector then exactly represents the border manifold. This also tests the correctness of your border matrix and the correctness of the orientation of your discrete $k$-manifold.
	
\subsubsection{Get Connected Components of a set of K-Simplices}
It also comes in handy to be able to identify the different connected components of a mesh, as the different components a mesh often need to be treated independently from each other as separate objects.
	
Given a list of $k$-simplices which forms one or more discrete manifolds you should identify the different connected components. You can also apply this
to the border of a $k$ manifold with multiple borders to get the various border components as in Figure \ref{fig::2_3_bunnyBorder}.
	
\begin{figure}[t]
	\begin{center}
	\includegraphics[width=7cm]{imgs/2_3_bunny_borders.eps}
	\end{center}
	\caption{A mesh with multiple border components.}
	\label{fig::2_3_bunnyBorder}
\end{figure}

	

\input{Chapters/TwoDimensionalSurfaces.tex}

\chapter{Differential Structure and Differential Calculus on Manifolds}

\note{Rework this stupid chapter i never wanted to write because i dont want it to take any space at all because well, for one it just distracts from the actual goal of differential forms, but then differential forms can only be understood as a generalization of something if the something is known a priori which makes this stupid chapter kind of needed but i do not want it!!!!!!}
In the last chapter we already encountered a Laplacian defined on a 2D surface and considered derivatives of functions defined on a surface, especially when calculating the derivative of the surface normal field. We saw how the derivative $Df$ of a function is mapping between two tangential spaces. Differential Forms and External Calculus, which are the real topics of this thesis, are a generalization of classical calculus (gradients, curls etc) on manifolds.  So before we move on to the core topic of this Thesis we want to elaborate derivatives and other differential operators on manifolds just a bit more, to give you a better feeling of what we are generalizing later, such that you understand where the operators exactly 'live' and why we have calculus on manifolds.

This chapter is rather short and of a completely theoretical nature.

\section{Differentiability Revisited}
What we have seen up to here actually amounts to defining manifolds and providing a differential structure on them. With a differential structure manifolds become more than geometric objects that are measured and described. They become spaces on which you can define functions for which derivatives and other operations can be calculated directly ON the manifold. While all operations are defined by mapping everything back to $\mathbb R^n$, this differential structure exists on its own right.

The differential structure is 'glued' on a manifold using maps, as seen in the subsection Derivatives on Surfaces. Differentiability is then defined in the following way:

\begin{definition}[Differentiability] If $M^k$ and $N^l$ are two manifolds and $f: M^k \rightarrow N^l \subset \mathbb R^n$ is a continuous mapping we call $f$ differentiable if for every map $h:\mathbb R^k \rightarrow M^k$ the map $f \circ h :  \mathbb R^k \rightarrow \mathbb R^n$ is differentiable.
\end{definition}

As seen the differential of a function at some point is a linear mapping between two tangential spaces.
\[D_pf: T_p M^k \rightarrow T_{f(p)} N^l\]
A slightly more abstract way of viewing this is looking at $T M^k$ : the space of all tangential spaces, called the tangential bundle. Given a map $h:\mathbb R^k \rightarrow M^k$ we can get a local map of the tangential bundle 
\[h_* : \mathbb R^k \times \mathbb R^k \rightarrow T M^k\]
\[h_*(x,v) = (h(x), Dh\, v)\]
suggesting that the tangential bundle actually is $2k$-manifold. The differential of a function $f: M^k \rightarrow N^l$ then is a mapping between the two tangential bundles $T M^k$ and $T N^l$ which are a $2k$ and a $2l$ manifold.

For our first steps of differential calculus on manifolds we also need vector fields (which we will later generalize to Forms).  A vectorfield $\mathcal V$ on a manifold $M^k$ is simply the assignment of a vector from the tangential space to every point of $M^k$.
\begin{definition}[Vector Field]
A vector field is an assignment $\mathcal V: M^k \rightarrow T M^k$ with $\mathcal V(x) \in T_x M^k$.
\end{definition}
Using that the tangential bundle actually is a manifold with a differential structure we can ask from a vector field that it is smooth or differentiable.

We can then define a vectorfield in a coordinate/ map dependent way. We do this with a simple example taken from [Thomas Friedrich, Global Analysis].

\subsubsection{An example Vector Field} Taking $\mathbb R^2$ as a manifold parametrized by the identity map $\phi(u,v) = ID$, i.e. para-metrized with euclidean coordinates. The tangential spaces get the bases $\frac{\partial \phi}{\partial u}$ , $\frac{\partial \phi}{\partial v}$ which form simply the standard basis $(1,0), (0,1)$ at any point. We can use the (for now a bit alianating) notation $  \frac{\partial}{\partial u}$, $\frac{\partial}{\partial v}$ for the two bases vectors, dropping the $\phi$. We can the define a vector field
\[\mathcal V(u,v) = u \frac{\partial }{\partial v} - v \frac{\partial }{\partial u}.\]
Note that $\frac{\partial }{\partial v}$ really only is a strange notation for $\frac{\partial \phi}{\partial v} = (0,1)$ and the same for $\frac{\partial }{\partial u}$. We can now try to express the vector field $\mathcal V$ in a different map; in polar coordinates:
\[h(r,\omega) = (r \cos (\omega), r \sin (\omega))\]
The base of an arbitrary tangential space induced by this map is then
\[\frac{\partial}{\partial r} = (\cos(\omega), \sin(\omega)) \]
\[\frac{\partial}{\partial \omega} = (-r \sin(\omega), r\cos(\omega))\]
again using the fancy notation $\frac{\partial}{\partial r}$ to denote a vector. Expressed in euclidean coordinates given by $\phi$ these two vectors are
\[\frac{\partial}{\partial r} = \frac{1}{\abs{(u,v)}} ( u \frac{\partial }{\partial u} + v \frac{\partial }{\partial v} ) \]
\[\frac{\partial}{\partial \omega} = u \frac{\partial }{\partial v} - v \frac{\partial }{\partial u}\]
such that $\mathcal V$ expressed in polar coordinates is simply
\[\mathcal V = \frac{\partial}{\partial \omega}\]

\note{Image}

\section{Derivatives, Vectorfields and Differential Operators}
On manifolds you can calculate derivatives relative to a vector field. This is simply the directional derivative relative to the vector field's direction. Given a function $f$ we denote the derivative with respect to the vector field $\mathcal V$ as $\mathcal V (f)$. Geometrically we already did that, if $\alpha$ is a curve with $\alpha(0) = p$ and $\alpha'(0)= v = \mathcal V (p)$, then
\[\mathcal V (f) (p) = \frac{\partial }{\partial t} f \circ \alpha(t)\]
Now if $\mathcal V(u)$ is written in some map $\phi$ as $\sum_i v_i(u) \frac{\partial}{\partial u_i}$, again using the $\frac{\partial}{\partial u_i}$ as fancy notations of the induced base vectors the derivative with respect to the vector field becomes
\[\mathcal V (f) = \sum_i v_i(u)\frac{\partial(f \circ \phi)}{\partial u_i}\]
which motivates the 'strange' vector notation. \note{further reasoning needed?}

\section{Riemannian Metric}
From standard calculus we are used to that the gradient of a function $f:\mathbb R^k \rightarrow \mathbb R$ is a vector field with vectors pointing in the direction where $f$ has the largest increase. But in fact the gradient of a function $f$ is in fact only a linear mapping that approximates $f$ via $f(x + h) \approx f(x) + grad_x(f) (h)$ and the vector is merely a representation of the gradient. What actually happens here is that we represent the gradient using a vector AND a scalar product:
\[grad_x(f) (h) = \langle grad, h\rangle\]

To do the same on manifolds $M^k$ we need a scalar product on all tangential spaces. As we consider manifolds as objected embedded in a higher dimensional space $M^k \subset \mathbb R^n$ all tangential spaces are subspaces of the embedding space such that they inherit a scalar product. So if $\phi: \mathbb R^k \rightarrow M^k \subset \mathbb R^n$ is a local map inducing the local bases $\frac{\partial \phi}{\partial u_i}$, $i= 1...k$ to the tangential spaces and we have two vectors $v, w$ in some tangential space $T_p M^k$ expressed in the local bases as
\[v= v_1 \frac{\partial \phi}{\partial u_1} +...+ v_k\frac{\partial \phi}{\partial u_k} \]
\[w = w_1 \frac{\partial \phi}{\partial u_1} +...+ w_k\frac{\partial \phi}{\partial u_k}\]
the scalar product induced by the embedding space is
\[\langle v,w \rangle = \sum_{i,j = 1}^k v_iw_j\langle \frac{\partial \phi}{\partial u_i},\frac{\partial \phi}{\partial u_j}\rangle.\]
So if $v = (v_1,...,v_k)$ and $w = (w_1,...,w_k)$ in the map induced base, the induced scalar product is represented by the matrix
\[G= \begin{pmatrix}\langle \frac{\partial \phi}{\partial u_1},\frac{\partial \phi}{\partial u_1}\rangle &\cdots& \langle \frac{\partial \phi}{\partial u_1},\frac{\partial \phi}{\partial u_k}\rangle \\
\vdots &&\vdots\\
\langle \frac{\partial \phi}{\partial u_k},\frac{\partial \phi}{\partial u_1}\rangle &\cdots& \langle \frac{\partial \phi}{\partial u_k},\frac{\partial \phi}{\partial u_k}\rangle \end{pmatrix} = (D\phi)^T D\phi\]
This set of scalar products that is consistently defined for all tangential spaces $T_p M^k$ is the so called Riemannian metric.

Equipped with the Riemannian metric we can define a gradient and a gradient vector field for functions as well as divergience and a laplacian. \note{We will generalize these later but it is worth seeing them once for themselves before the generalization. TODO}

Note the Riemannian metric can also be used to measure volumes and angles. Angles are quite obvious; if we have two curves $\alpha$ and $\beta$ intersecting at some point $p$ with tangents $v$ and $w$, the angle between the curves is $\langle v,w\rangle$. The volume comes from the fact that $det(A A^T)$ equals the volume spanned by $A$'s row vectors squared. The determinant $det(G)$ then is the volume spanned by the column vectors of $D\phi$ squared and the $k$-dimensional volume of some subset $\phi(U) \subset M^k$ covered by a map $\phi$ is
\[vol(\phi(U))= \int_U \sqrt{det(G)}\;du_1,...,du_k\]
This measures 'absolute' orientation independent volume.



\section{Some Differential Operators}

The Goal of Discrete Differential Forms is to define discrete differential operators on Meshes, such that the retain some important geometric properties. 
You probably have seen Operators like divergence, gradients, rotation or the Laplace Operator. They arise very naturally in many settings. Most of the time you will have seen those operators written in standard Kartesian coordinates, like
\[\nabla = (\frac{ \partial}{\partial x_1},\frac{ \partial}{\partial x_2},\frac{ \partial}{\partial x_3})\]
where $x_1, x_2, x_3$ are the usual coordinates. If then $f$ is expressed in such coordinates, i.e. $f = f(x_1,x_2,x_3)$ it is obvious what $\nabla f$ should be. But what if $f$ is expressed in a different set of coordinates, e.g. $f(y_1,y_2,y_3)$? But for the coordinates $f$ would still describe the same function, so there should be an operation $\tilde{\nabla}$ that does the same to $f$ in these new coordinates as $\nabla$ did in the old coordinates.

Of course you can just formalize the change of coordinates and deduce what $\tilde{\nabla}$ is. We will try to give geometric coordinate free properties of these operators, which should enhance the understanding of them and which we will use to define discrete version of them, or that helps understand in what the discrete operators coincide with the continuous operators.

Differential forms unify these operators more from an algebraic than a geometric point of view. Thus a general Operator might have different (even if somehow related) geometric meanings, when applied to different objects.

\subsection{Divergence}

The divergence Operator is defined for vector valued functions. In Kartesian coordinates it is 
\[\nabla \cdot f = (\frac{ \partial}{\partial x_1},\frac{ \partial}{\partial x_2},\frac{ \partial}{\partial x_3})\cdot f = \frac{ \partial}{\partial x_1} f + \frac{ \partial}{\partial x_2}f + \frac{ \partial}{\partial x_3}f\]

But what does that mean? One way to look at divergence is with flows. Let $f$ be a vector field that describes the velocity (i.e. direction and speed) of a ''fluid'' at any position. For a closed Volume $V$ with surface $\partial V = S$ we then can calculate the netflow into the volume: how much flows out minus how much flows into this volume.

To determine this it is enough to look at the boundary of this volume and determine the flow through the boundary. This gives rise to this expression:

\[\int_{\partial V} f \cdot n ds\]
which measures the net flow, where $n$ are the surface normals at the given points. Now lets restrict $f$ to be a linear map $f(x) = A x$ with $A$ being a Matrix and $V$ be an axis aligned Volume with widths $h_1,h_2,h_3$ and base point $x_0,y_,z_0$, surface Areas $A_1, A_2, A_3$ and normals $n_1,n_2,n_3$. Then

\[\int_{\partial V} f \cdot n ds = \int_{\partial V} Ax \cdot n ds = \int_{\partial V} Ax \cdot n ds\]
\[= \int_{\partial A_x} A(x + h_xn_x) \cdot n_x ds - \int_{\partial A_x} A(x) \cdot n_x ds + ...\]
\[= \int_{\partial A_x} A(h_xn_x) \cdot n ds + ... = \sum_{i=1}^3 h_i n_i^TA n_i Area(A_i) = \sum_{i=1}^3 n_i^TA n_i Vol(V) = Tr(A) Vol(V)\]
So you see that the Trace of $A$ describes the netflow of a linear vector field. 

Back to the Divergence. Any function $f$ an locally be described with its Jacobi Matrix $Df$
\[Df = (\partial f_i / \partial f_j)_{i,j}\]
by 
\[f \approx f(x_0) + Df \cdot (x-x_0) \]
If we then look at local netflow at some point by letting shrink a Volume $V$ to the point  
\[\lim_{Vol(V) \rightarrow 0}\frac{\int_{\partial V} f \cdot n ds}{Vol(V)}\]
and approximate $f$ by its Jacobian, we get
\[\lim_{Vol(V) \rightarrow 0}\frac{\int_{\partial V} Df \cdot n ds}{Vol(V)} = Tr(Df)= \frac{ \partial}{\partial x_1} f + \frac{ \partial}{\partial x_2}f + \frac{ \partial}{\partial x_3}f\]
which actually is the divergence of $f$! This means the Divergence of a vectorvalued function $f$ is geometrically the local netflow.

\subsection{Gradient}
The gradient of a real valued function is fairly easy.

\subsection{Rotation}
Rotation is often denoted as $rot$ or $\nabla \times$ and is defined for vector valued functions and returns a vector valued function.  In Kartesian Coordinates this is
\[\nabla \times f = \left( \frac{ \partial f_3}{\partial x_2}- \frac{ \partial f_2}{\partial x_3}, \frac{ \partial f_1}{\partial x_3}- \frac{ \partial f_3}{\partial x_1}, \frac{ \partial f_2}{\partial x_1}- \frac{ \partial f_1}{\partial x_2}\right)\]
And what does this one do? The name already tells it; it measures the rotation of the vector field around some axis. One coordinate free way to describe it is
\[n\cdot rot(A) = \lim_{Area(F)\rightarrow 0 } \frac{\int_{\partial F} A \cdot dx}{Area(F)} \]
This maybe needs some explanation.... (image) That this extends the definition in cartesian coordinates directly follows from stokes Theorem, as the divergence Definition would have as well.


\subsection{Laplacian}
Deserves an own chapter.

Closedness and coclosedness of the laplace beltrami Operator. Riemann surfaces by Farkas and Kra..

The Laplacian is given by
\[\delta d + d \delta\]
and a form is said to be harmonic if
\[\delta d + d \delta f = 0\]
An important property is that $f$ is harmonic if and only if
\[df = 0\]
\[\delta f = 0\]
This is an extremely strong property and somewhat easier to understand geometrically. What is nice also is that it is very easy to show. We already know that
\[\langle \delta \omega,\phi\rangle = \langle \omega, d\phi\rangle\]
i.e. that $\delta$ and $d$ are adjoint. But then
\[\delta d + d \delta f = 0 \Rightarrow \langle \delta d + d \delta f, f\rangle = 0\]
and
\[0 = \langle \delta d + d \delta f, f\rangle = \langle \delta d f, f\rangle  + \langle d \delta f, f\rangle  = \langle d f, d f\rangle +\langle \delta f, \delta f\rangle\]
As the inner product of a form with itself is not negative both these terms have to be zero. And therefore
\[df = 0\]
\[\delta f = 0\]
But what does that mean geometrically? Well this depends on the form we are looking at. For Vector fields (1 Forms) in $\mathbb R^3$ this means they are divergence and rotation free. For a harmonic 2 Form (''Pseudo Vector Field'') this means.... for a 0 Form this means... constant???? wtf?
See e.g. %http://en.wikipedia.org/wiki/Hodge_dual#The_codifferential about the adjointness and http://en.wikipedia.org/wiki/De_Rham_cohomology about harmonic forms.
Fact is the adjointness is the root of my problem and HAS to be true only conditionally!. Or the innerproduct may be degenerated on open or infinite sets..???? I dont know

But consider an integral over Rn of a harmonic form ... its not defined. Over open Sets...? ...... domt lmpw. would expect it to be ok.
It might be that this adjointness only holds on compact or borderless manifolds or when 0 or constant on border. This here is a major hitch in my understanding...! 

Ahahahahaha! this is also mentioned on p 370 in my smart book (the yello one, you know which one.)
-As trace of the hesse, or derivative of the determinant of the ... or sth.


%\newpage
\chapter{Differential Forms}
\label{chap:diffforms}

\begin{figure}[h]%
\begin{center}
\includegraphics[height=3.5cm]{imgs/4_differentialform.eps}%	
\end{center}

\caption{Here a differential form $\omega$ is represented. A single differential form provides linear mappings $\omega_p$ at all points $p$ on a manifold; the mappings $\omega_p$ map the tangential spaces $T_pM$ to $\mathbb R$.}%
\label{fig:4_differentialform}%
\end{figure}
As seen in the last chapter, you can do differential calculus on smooth manifolds. To be able to define discrete calculus on discrete manifolds we first need a geometric understanding of calculus. We get to this understanding by generalizing functions to 'objects` that can be integrated over subsets of manifolds. These objects are \emph{differential forms}; one is schematically depicted in Figure \ref{fig:4_differentialform}. In a next step we will then define an exterior derivative $d$ for differential forms, whose geometry is easier to understand and can be used to define a discrete exterior derivative on discrete manifolds.
 The point why we are interested in differential forms and the exterior derivative $d$ is that $d$ unifies many differential operators, like divergence, gradient and curl.

%In the next two chapters we introduce forms, differential forms and exterior calculus. Differential forms are mathematical objects that allow us to treat many things in a unified way, such as real valued functions and vector fields. Differential forms can be integrated and differentiated, actually they are designed to behave well under an integral. The point why we are interested in differential forms is that they allow the definition of a differential operator $d$ that unifies many differential operators, like divergence, gradient and curl. Exterior calculus will unmask common properties of these differential operators and also their geometry, via Stokes theorem
%\[\int_{\delta \Omega} \omega = \int_{\Omega} d\omega,\]
%which directly connects the differential operator $d$ to the border operator $\delta$. It will be this relation that is exploited to define the differential operator $d$ for simplicial complexes, and by doing so preserve many important properties of these differential operators.

The basic theory for differential forms is split in two chapters. This chapter motivates differential forms, captures them more formally, covers integration of differential forms and relates them to standard calculus objects like real valued functions  and vector fields. It ends with the introduction of discrete differential forms.

In the next chapter, Chapter \ref{chap:EC}, we will then introduce the most important elements of exterior calculus and discrete exterior calculus, like the exterior derivative $d$, the operators $\partial$ and $\star$ and Stokes theorem, which describes the geometry of the exterior derivative.
%\begin{table}[h]
%	\begin{longtable}{|p{4.5cm}|p{4.5cm}|p{4.5cm}|}
%		\hline
%		Smooth Theory& Discrete Theory& Implementation (Notes)\\
%		\hline
%			Differential forms: \begin{itemize}
%			  \setlength{\itemsep}{1pt}
%			  \setlength{\parskip}{0pt}
%				\setlength{\parsep}{0pt}
%				\item[-]Diff form motivation
%				\item[-]Forms (multilinear mappings) and the dimension of $k$-form space 
%				\item[-]Differential forms 
%				\item[-]Riemann Integral of Diff forms 
%				\item[-]Interpretation of Diff forms in $\mathbb R^3$ 
%			\end{itemize}
%			&
%			\begin{packed_enum}
%				\item[-] Discrete forms
%				\item[-] Sampling forms
%			\end{packed_enum}
%			 & - none
%			 \\		
%		\hline
%	\end{longtable}
%	\end{table}

\section{Smooth Differential Forms}

We start introducing differential forms by motivating them as objects that fulfill all requirements to be useful under an integral in Section \ref{sec:dfmotivation}. In Section \ref{subsec::forms} we describe the  local structure of differential forms. At any point a differential form $\omega$ provides a \emph{form} $\omega_p$, a multilinear, antisymmetric mapping. These multilinear mappings form a vector space of a finite dimension. Using the so called wedge product $\wedge$, we describe simple bases for them in Section \ref{subsec:wedge}. Having understood the local structure of differential forms we give a clean definition of them in Section \ref{subsec:defDiffform} and finally use everything to relate the differential forms to the more intuitive well known objects from standard calculus in Section \ref{subsec:diffformsare}.

\subsection{The perfect thing to integrate}
\label{sec:dfmotivation}
\begin{figure}
	\begin{center}
	\includegraphics[width = 14cm]{imgs/5_1_riemann.eps}
	\end{center}
	\vspace{-1cm}
	\caption{To calculate the Riemann integral over a surface we select a grid and refine it. In the sums $f$ is evaluated at arbitrary positions in the patches.}
	\label{fig:5_1_riemsum}
\end{figure}

Differential forms arise very naturally when considering integrals. Suppose that we have a two dimensional surface $M$ and a function $f$ defined on the surface. We want to integrate $f$ over $M$ i.e. calculate the Riemann integral
\[\int_{M} f dA.\]
To compute the integral by brute force we can use Riemann sums as depicted in Figure \ref{fig:5_1_riemsum}: we choose a grid, sum up the areas of the parallelograms $s$ weighted by the  function value $f(s)$, and take the limit under grid refinement:
\[\lim_{diam(s\in grid)\rightarrow 0} \sum_{s \in grid} f(s)\cdot area(s) \]
We take a step back and consider what is essential for an 'object` to be integrated in this way.
Basically we can integrate anything that assigns values to areas $s$. Say $\omega_p$ assigns the value $\omega_p(s)$ to an area $s$ located at some point $p$, then its integral can be computed as:
\[\int_M \omega = \lim_{diam(s\in grid)\rightarrow 0} \sum_{s \in grid} \omega_{p\in s}(s)\]
Obviously $\omega_p$ has to follow some rules to be useful for integration. For one it should scale with the area of $s$. Assume a grid segment $s$ is spanned by two vectors $a_s$,$b_s$, then we can write $\omega_p(s)$ as $\omega_p(a_s,b_s)$, see Figure \ref{fig:5_linear}. For $\omega_p$ to be proportional to the area of $s$ we need it to be linear in both $a_s$ and $b_s$
\[\omega_p(\lambda a_s, b_s) =\lambda \omega_p(a_s,b_s)\]
\[\omega_p(a_s , \lambda b_s) =\lambda \omega_p(a_s,b_s)\]
Furthermore, $\omega$ should behave well when the parameters are swapped, as the vector pairs $(a_s,b_s)$ and $(b_s,a_s)$ span the same area, but for orientation. There are two possibilities that make sense: we can choose $\omega$ to be symmetric or to be antisymmetric
\[\omega(a_s,b_s) = \omega(b_s,a_s)\]
\[\omega(a_s,b_s) = - \omega(b_s,a_s).\]
Symmetry would mean that $\omega$ only depends on the absolute area of $s$. Antisymmetry means that $\omega$ respects the orientation of $s$. We choose antisymmetry, $\omega$ then is a \emph{differential form}. The first variant would lead to so called \emph{pseudo forms}. 

\begin{figure}%
\begin{center}
	\includegraphics[height = 3.5cm]{imgs/4_difformscaling.eps}%
\end{center}
\caption{Locally at some point $p$, a differential form 'measures` volumes and therefore has to be proportional to the volume measured. If the volumes are described by the vectors that span the volume, here $a_s$ and $b_s$, this means that $\omega$ should be linear in both $a_s$ and $b_s$.}%
\label{fig:5_linear}%
\end{figure}

Lastly we have to clarify what $a_s$ and $b_s$ are. The vectors $a_s$ and $b_s$ are bound to some position $p$. Also, if you look at the grid in Figure \ref{fig:5_1_riemsum}, you see that the grid elements nearly lie in the tangential spaces of the surface $M$. At least they do so in the limit. This gets us to the full definition of a differential form on a two dimensional surface: a differential form $\omega$ provides at any point $p$ on the surface $M$ a mapping $\omega_p$ that takes two vectors from the tangential space $T_pM$, is linear in both arguments and antisymmetric. I.e. for all $p \in M$
\[\omega_p: T_p M \times T_p M \to \mathbb R\]
\begin{align*}&\omega_p(\lambda a,b) = \lambda \omega_p(a,b) = \omega_p(a,\lambda b) &\text{bilinearity} \\
&\omega_p(a,b) = -\omega_p(b,a)  &\text{antisymmetry}\end{align*}
A differential form $\omega$ should also  change smoothly between neighboring $p$'s.


\subsection{Forms}
\label{subsec::forms}
We motivated that differential forms have a position $p$ and some number of tangential vectors from $T_pM$ as input variables. And for a fixed $p$ the differential form should be multilinear and antisymmetric.  Before we give a full definition of differential forms, we have a look at how they behave at single points. A multilinear antisymmetric mapping is called a form (without the word 'differential`). Forms form a finite dimensional vector space and allow the definition of a wedge product. This is important for us because this will let us describe differential forms with greater ease and let us relate differential forms to standard calculus objects.
\subsubsection{Definition}
A $k$-form on $\mathbb R^l$ is a multilinear antisymmetric mapping $\mathbb R^l \times ... \times \mathbb R^l \to \mathbb R$ which depends on $k$ vectors from $\mathbb R^l$: 
\begin{definition}[$k$-form]
 A $k$-form (not a differential $k$ form, mind you) on a $l$ dimensional space $\mathbb R^l$ is a mapping $\omega : \mathbb R^l \times \mathbb R^l\times \cdots \times  \mathbb R^l \rightarrow \mathbb R$ with the following properties:
\begin{enumerate}
\item $\omega(x_1,...,x_k)$ is linear in all $k$ parameters, meaning that \[\omega(x_1,..,\lambda a + b,..., x_k) = \lambda \omega(x_1,..,a,..., x_k) + \omega(x_1,.., b,..., x_k).\]
\item $\omega(x_1,...,x_k)$ is skew symmetric or antisymmetric, meaning that switching any two variables leads to a change of sign:
\[\omega(x_1,...,x_i,...,x_j,...,x_k) = - \omega(x_1,...,x_j,...,x_i,...,x_k)\]
\item In particular:
\[\omega(x_1,...,x_k) = 0 \;\text{ if }\;x_1,...,x_k \text{ are linearly dependent}\]
\end{enumerate}
\end{definition}

The first property makes sure that $\omega$ is proportional to the volume spanned by the input vectors, while the second ensures that $\omega$ respects the orientation of the input.
The third property follows from the first two.\footnote{From $\omega$'s antisymmetry follows $\omega(...,v,...,v,...) = -\omega(...,v,...,v,...)$, i.e. $\omega(...,v,...,v,...) = 0$. Then from the linearity directly follows  $\omega(x_1,...,x_{k-1}, \sum_{j=1}^{k-1} a_jx_j) = 0$} The space of all $k$-Forms on $\mathbb R^l$ is denoted by $\Lambda^k (\mathbb R^l)$ and is a vector space: if $\omega$ and $\nu$ are $k$ forms so are $\omega + \nu$ and any multiples $\lambda \omega$. A natural question is what dimension $\Lambda^k(\mathbb R^l)$ has and to find a suitable basis of this space.

\subsubsection{Basis and Dimension of $\Lambda^k(\mathbb R^l)$}
Given a $k$-form $\omega$ on $\mathbb R^l$ and a basis $e_1,...,e_l$ of $\mathbb R^l$, then $\omega(a_1,...,a_k)$ can be rewritten the following way: we express the $a_j$ explicitly as a sum of basis vectors
\[a_j = \sum_{i}a_j^ie_i\]
and using the linearity of forms we get
\[\omega(a_1,...,a_l) = \omega(\sum_{i}a_1^ie_i,...,\sum_{i}a_l^ie_i)\]
\[= \sum_{i_1,...,i_k \in\{1,...,l\}}a_1^{i_1}\cdot ... \cdot a_k^{i_k} \omega(e_{i_1},...,e_{i_k}).\]
We can reorder this sum such that all terms treating the same set of basis vectors are grouped together
\begin{equation}=\sum_{i_1<...<i_k}\left(\sum_{\sigma \in S^k} a_1^{i_{\sigma(1)}}\cdot ... \cdot a_k^{i_{\sigma(k)}} \omega(e_{i_{\sigma(1)}},...,e_{i_{\sigma(k)}})\right),\label{eq:diffformsum}\end{equation}
where the permutation group $S^k$ is used to express that the inner sum goes over all orderings of basis vectors. Because of the antisymmetry, reordering $e_{i_{\sigma(1)}},...e_{i_{\sigma(k)}}$ to $e_{i_1},...,e_{i_k}$ such that $i_1<...<i_k$, affects only the sign of $\omega(e_{i_1},...,e_{i_k})$:
\[\omega(e_{i_1},...,e_{i_k}) = sgn(\sigma)\omega(e_{i_{\sigma(1)}},...e_{i_{\sigma(k)}}).\]
Rewriting the sum \ref{eq:diffformsum} yields
\[\sum_{i_1<...<i_k}\left(\sum_{\sigma \in S^k} sgn(\sigma) a_1^{i_{\sigma(1)}}\cdot ... \cdot a_k^{i_{\sigma(k)}}\right) \omega(e_{i_1},...,e_{i_k})\]
\[ = \sum_{i_1<...<i_k} det_{i_1,...,i_k}(a_{i_1},...,a_{i_{k}}) \omega(e_{i_1},...,e_{i_k}),\]
where $det_{i_1,...,i_k}(a_{i_1},...,a_{i_{k}})$ is a sub determinant of the matrix formed by the vectors $a_1,...,a_k$ restricted to the lines $i_1,...,i_k$:
\[det_{i_1,...,i_k}(a_1,...,a_k) = det \begin{pmatrix}
a_1^{i_1} &a_2^{i_1} &...&a_k^{i_1} \\
\vdots & & & \vdots \\
a_1^{i_k} &a_2^{i_k} &...&a_k^{i_k} 
\end{pmatrix}\]
Put on one line we get

%\fbox{\parbox{\textwidth}{
\[\omega(a_1,...,a_k)= \sum_{i_1<...<i_k} \omega(e_{i_1},...,e_{i_k}) \cdot det_{i_1,...,i_k}(a_{i_1},...,a_{i_{k}})\]
%}}
We can read a few things out of this. For one, the $k$-form $\omega$ is determined uniquely by the values it assumes on $k$-tuples of basis vectors $e_{i_1},...,e_{i_k}$ with $i_1 <...< i_k$. And the $k$-forms
\[det_{i_1,...,i_k}(a_{1},...,a_{k})\]
which calculate $k$-subdeterminants of the input vectors, form a basis of $\Lambda^k(\mathbb R^l)$. From this follows directly that that the dimension of the  space of $k$-forms on $\mathbb R^l$ equals the number of ordered tuples $i_1<...<i_k$ of integers $i_1,...,i_k \in \{1,...,l\} $ i.e.
\[\dim (\Lambda^k(\mathbb R^l)) = \begin{pmatrix}
l \\
k
\end{pmatrix}\] 
In particular, the space of $k$ forms on $\mathbb R^l$ with $k>l$ is 0-dimensional, which means that there are no $k>l$-forms.

\subsection{The Wedge Product}
\label{subsec:wedge}
The wedge product for forms is a way to create higher order forms out of lower order forms, for example out of a $j$ form $\omega^j$ and a $k$ form $\nu^k$ you can make a $j+k$ form $\omega^j\wedge \nu^k$. The important points to understand in this section are that the wedge product can be used to create higher order forms and to simply describe a base of the space of $k$-Forms $\Lambda^k(\mathbb R^l)$. Furthermore, the wedge product is associative, distributive and has some symmetry. 

\subsubsection{Definition of the Wedge Product}
The wedge product is easy to define but not very intuitive. You directly define the wedge product as
\[\omega^j\wedge \nu^k (v_1,...,v_{l+k}) = \frac{1}{k!l!}\sum_{\sigma \in S^{k+l}} sgn(\sigma) \omega^j(v_{\sigma(1)},...,v_{sigma(j)})\nu^l(v_{\sigma(k+1)},...,v_{\sigma(k+l)}).\]
The wedge product has the following properties, as is easy to show and is done e.g. in \cite{globalAnalysis}. These algebraic rules are handy for calculations.
\emph{
\begin{enumerate}
\item Linearity in both arguments, i.e. $(\lambda\omega_1^k + \omega_2^k)\wedge\nu^l = \lambda(\omega_1^k \wedge\nu^l) + \omega_2^k \wedge \nu^l$ and the same for $\nu^l$
\item Associativity, i.e. $ (\omega^j \wedge \nu^k) \wedge \mu^l = \omega^j \wedge (\nu^k \wedge \mu^l)$
\item Symmetry: $\omega^k\wedge \nu^l = (-1)^{kl} \nu^l \wedge \omega^k$
\end{enumerate}
}

The wedge product is closely connected to determinants. For two arbitrary 1-forms $\omega^1$, $\nu^1$ we get
\[\omega^1\wedge\nu^1(a,b)= det \begin{pmatrix}
\omega(a) & \omega(b) \\
\nu(a) & \nu(b)
\end{pmatrix}\]
and wedging $k$ one forms $\omega_1^1,...\omega_k^1$ leads to
\[\omega_1^1\wedge\omega_2^1 \wedge...\wedge\omega^1_k(a_1,...,a_k):= det \begin{pmatrix}
\omega_1(a_1) &  ... & \omega_1(a_k) \\
\vdots & & \vdots \\
\omega_k(a_1) &... & \omega_k(a_k)
\end{pmatrix}\]

\subsubsection{A Basis with the Wedge Product}
The wedge product allows to elegantly describe a basis for the space of $k$-forms $\Lambda^k(\mathbb R^l)$, using a basis $e_1,...,e_l$ of $\mathbb R^l$. The space of 1-forms $\Lambda^1(\mathbb R^l)$ has dimension $l$ and is spanned by the special set of basis forms
\[de_i(a) := det_i(a) = a^i\]
i.e. the projection to the $i$th coordinate of $a$ with respect to the chosen base $e_1,...,e_l$.  If we apply the wedge product to the 'standard' basis 1-forms $de_1,..., de_l$ we get
\begin{eqnarray*}de_{i1}\wedge de_{i2} \wedge ... \wedge de_{ik}(a_1,...,a_k) &= &det \begin{pmatrix}
de_{i1}(a_1) &  ... & de_{i1}(a_k) \\
\vdots & & \vdots \\
de_{ik}(a_1) &... & de_{ik}(a_k)
\end{pmatrix} \end{eqnarray*}
which is $det_{i_1,...,i_k}(a_1,...,a_k)$ when $i_1 <...<i_k$. These are exactly the 'standard' basis forms for the space of $k$-forms from the Section \ref{subsec::forms}. This means that a basis of $\mathbb R^l$ induces a basis to the space of forms and any $k$-form can be written as a linear combination 
\[\omega^k = \sum_{i_1<...<i_k} w_{i_1,..,i_k} de_{i_1}\wedge...\wedge de_{i_k} \]
%Finally for a $k$ and an $l$ form $\omega^j$ and $\nu^k$ we define the wedge product by writing them in such a basis
%\[\omega^j = \sum_{i_1<...<i_j} w_{i_1,..,i_j} de_{i_1}\wedge...\wedge de_{i_j} \]
%\[\nu^k = \sum_{h_1<...<h_k} v_{h_1,..,h_k} de_{h_1}\wedge...\wedge de_{h_k} \]
%and then using the associativity and distributivity of the wedge product you can compute
%\begin{eqnarray*}
%\omega^j\wedge \nu^k &=& \sum_{i_1<...<i_j} w_{i_1,..,i_j} de_{i_1}\wedge...\wedge de_{i_j} \\
%& & \wedge \sum_{h_1<...<h_k} v_{h_1,..,h_k} de_{h_1}\wedge...\wedge de_{h_k} \\
%& =& \sum_{\begin{subarray}
%\{\{ i_1,...,i_j\}\cap \{h_1,...,h_k\} = \emptyset \\
%i_1 < ... < i_j,\;\; h_1 <...<h_k
%\end{subarray}} w_{i_1,..,i_j} \cdot v_{h_1,..,h_k} de_{i_1}\wedge...\wedge de_{i_j} \wedge de_{h_1}\wedge...\wedge de_{h_k}
%\end{eqnarray*}
Note that often $x_1,...,x_l$ or $x,y,z$ or similar is chosen to denote the base of $\mathbb R^l$ and the basis forms consequently are denoted by $dx_1,..., dx_l$ or $dx,dy,dz$, $dx \wedge dy$ and so on. 

\subsection{Differential Forms}
\label{subsec:defDiffform}
Now we can correctly define differential forms. They are exactly as motivated in the beginning of this chapter; a differential form assigns to each point $p$ of a manifold $M$ a form $\omega_p$ defined on the tangential space $T_pM$. With the wedge product we can now formulate that the differential form should vary smoothly between points. A local map 
\[\phi:  U \subset\mathbb R^l \to M\] 
induces a basis to all tangential spaces of points in $\phi(U)$, namely $\frac{\partial \phi}{\partial u_i}$. We can directly use this basis to induce a basis to the space of $k$-forms, as done in the last section. Henceforth we will use the following notation for the basis forms induced by a map $\phi$:
\[d u_i:= d\frac{\partial \phi}{\partial u_i}.\]

\begin{definition}[Differential Form]
A differential $k$-form $\omega^k$ is a mapping that assigns a $k$-form $\omega_p \in \Lambda^k(T_pM)$ to every point $p\in M$.

Given a local map $\phi: U \rightarrow M$ all $k$-forms $\omega_p$ with $p\in \phi(M)$ can be expressed in the  coordinates induced by $\phi$:
\[\omega_p = \sum_{i_1<...<i_k}\omega_{i_1,...,i_k}(p) du_{i_1}\wedge...\wedge du_{i_k}\]
for some realvalued functions $\omega_{i_1,...,i_k}(p)$. We then say that the differential form $\omega$ is $k$ times differentiable if expressed in local coordinates, the $\omega_{i_1,...,i_k}(p)$ are $k$ times differentiable. For simplicity sake we will always assume that $\omega$ is infinitely often differentiable, i.e. smooth.

\end{definition}

We will see examples and relate differential forms to more common things like vector fields in the next section. 
%Note that any operation defined for forms can point-wisely be defined for differential forms. For example the wedge product $\wedge$ for two differential forms $\omega^k$ and $\nu^l$ would be
%\[(\omega^k\wedge \nu^l)_p := \omega^k_p\wedge \nu^l_p\]


\subsection{Interpretation of Differential Forms in $\mathbb{R}^3$}


\label{subsec:diffformsare}
Differential forms are of high practical relevance because standard calculus objects like vector fields are just realizations of differential forms. Therefore, the theory about differential forms can be applied directly to a wide range of standard problems. In the following we focus on $\mathbb R^3$, or equivalently, 3-dimensional manifolds, and relate the differential forms to standard calculus objects. The relation is depicted in Figure \ref{fig:4_difformsAre}.

\subsubsection{Differential 0-Forms}
Differential 0-forms are real valued functions. By definition a differential 0-form assigns a 0-form, i.e. a constant, to every point $p$ on a manifold. This means a differential 0-form is simply a smooth function $\omega: M \to \mathbb R$. 

\subsubsection*{Differential 1-Forms}
Differential $1$-forms are equivalent to tangential vector fields. A 1-form is a linear mapping $\omega: \mathbb R^k \rightarrow \mathbb R$. Linear mappings to $\mathbb R$ can be represented as the scalar product of some vector $\omega^{\#} \in \mathbb R ^k$ with the input vector:
\[\omega(v) = \langle \omega^{\#}, v \rangle\]
This works just as well on manifolds with tangential spaces. Yet, there is an additional technical difficulty to mention: we need a scalar product on the tangential spaces. As we look only at manifolds embedded in a higher dimensional space $M \subset \mathbb R^n$, the most natural choice of a scalar product on the tangential spaces is the scalar product induced by the surrounding space. A scalar product that is consistently defined for all tangential spaces is called a \emph{Riemannian metric}.

Note that in principle you could use a different metric i.e. different scalar products. In that case $\omega$ would be described by a different vector $\omega^{\#}$. This is why the sharp operator is \emph{depending on a metric}. The reverse operation of making a 1-form out of a vector $v$ is usually denoted by the 'flat' operator $\flat$, i.e. $v^\flat$.

\subsubsection*{Differential n-1-Forms}
The space of 2-forms on tangential spaces $\Lambda^2(T_pM)$ on three dimensional manifolds $M$ has dimension \[\dim\Lambda^2(T_pM) =\begin{pmatrix}
3\\
2
\end{pmatrix} =3.\] Therefore a differential $2$-form can again be represented as a vector field. In the $\mathbb R^3$ with the standard basis and euclidean scalar product a basis of $\Lambda^2(\mathbb R^3)$ is given by $dy \wedge dz$, $dz \wedge dx$, $dx \wedge dy$. If $\widehat{w} = (w_1,w_2,w_3)$ is a vector field we define the related $2$-form as
\[\omega^2 = w_1 dy \wedge dz + w_2 dz \wedge dx + w_3 dx \wedge dy \]
 As $dy \wedge dz (a,b) = a_yb_z -a_zb_y$,  the 2-form $\omega^2(a,b)$ can be written as
\begin{align*}
\omega^2(a,b) &= (w_1 dy \wedge dz + w_2 dz \wedge dx + w_3 dx \wedge dy)(a,b) \\
&= w_1(a_2b_3 - b_2 a_3) + w_2(a_3b_1-b_3a_1) + w_3(a_1b_2 - a_2b_1) \\
&= \langle \widehat{w}, a \times b \rangle\end{align*} 
This means if we want a vector $\widehat{w} \in \mathbb R^3$ to act like a two form on two input vectors $a,b$ in $\mathbb R^3$, it amounts to taking the scalar product of $\widehat{\omega}$ and a vector normal to $a,b$ scaled by the area spanned by $a,b$. 
This can be done for $(n$-$1)$-forms on $n$-manifolds in general.

\begin{figure}%
\def\svgwidth{\columnwidth}
%\includegraphics[width=\columnwidth]{imgs/4_differentialforms.eps}%
\input{imgs/4_differentialforms.eps_tex}
\caption{Interpretation of differential forms as standard calculus objects. On a $n$-dimensional manifold $0$-forms and $n$-forms can be identified with real valued functions, $1$-forms and $(n$-$1)$-forms as tangential vector fields.}%
\label{fig:4_difformsAre}%
\end{figure}

\subsubsection*{Volume Forms}
Differential $3$-forms on $\mathbb R^3$ can again be represented as real valued functions. As can differential $n$-forms on $n$-dimensional manifolds in general.
There is a special differential $n$-form on an $n$ dimensional manifold $M^n$: the 'volume form'. The volume form measures the signed (orientation dependent) volume spanned by the input vectors. In $\mathbb R^n$ with the standard basis $e_1,...,e_n$ and the euclidean scalar product this is exactly the determinant:
\[d\mathbb R^n (v_1,..., v_n) = det(v_1,...,v_n) = de_1\wedge ...\wedge d e_n\]
The volume on some space $V$ is sometimes denoted as $dV$ or $dVol$.
 As the dimension of the space of $n$-forms on $T_pM^n$ is
\[\dim(\Lambda^n(\mathbb R^n)) = \begin{pmatrix}
	n\\n
\end{pmatrix}= 1,\]
every $n$-form is simply a multiple of the volume form $dV$. Therefore any differential $n$-form can be represented with a real valued function $f:M \rightarrow \mathbb R$ as
\[\omega^n = f \cdot dVol\]

%\subsubsection{Example: Volume Forms in Local Coordinates}
%As an example we express the volume form on a $k$-manifold $M^k$ in local coordinates given by a map $\phi$.
%The $k$-dimensional volume spanned by the vectors $\frac{\partial \phi}{\partial u_i}$ is $\sqrt{det((D\phi)^TD\phi)}$, therefore
%\[dT_pM (\frac{\partial \phi}{\partial u_1},...,\frac{\partial \phi}{\partial u_k}) = \sqrt{det((D\phi)^TD\phi)}.\]
%As any $k$-form is a multiple of the volume form, the $k$-form given by $\phi$ must be
%\[du_1\wedge...\wedge du_k = c \cdot dT_pM\]
%for some $c$. The $c$ can be computed by
%\[c = \frac{dT_pM(\frac{\partial \phi}{\partial u_1},...,\frac{\partial \phi}{\partial u_k})}{du_1\wedge...\wedge du_l (\frac{\partial \phi}{\partial u_1},...,\frac{\partial \phi}{\partial u_k})} = \frac{\sqrt{det((D\phi)^TD\phi)}}{det((D\phi)^TD\phi)}\]
%and therefore the volume form in local coordinates is
%\[dT_pM =\frac{1}{\sqrt{det((D\phi)^TD\phi)}}  du_1\wedge...\wedge du_k.\]

%\subsubsection{Differential $n$-Forms on $\mathbb R^n$}
%The one-dimensionality of the space of $n$-forms on $\mathbb R^n$ can also be used to get a very simple description of what happens to a $n$ differential form under a base change.
%If $A$ is a linear map $\mathbb R^n \to \mathbb R^n$ 
%\[det(Av_1,Av_2,...,Av_n) = det(A)\cdot det(v_1,...,v_n)\]
%and therefore if $\omega^n$ is a $n$ differential form on $\mathbb R^n$
%\[\omega^n(Av_1,...,Av_n) = det(A)\cdot \omega^n(v_1,...,v_n)\]
%as any $n$ form is simply a multiple of the volume form $\omega^n = c \cdot det$. We can play around with this a bit more such that for fixed $v_1,...,v_n$ building a matrix $V$ we get
%\[\omega^n(Av_1,...,Av_n) = det(A)\cdot \omega^n(v_1,...,v_n)= det(V) \cdot \omega^n(a_1,...,a_n)\]

%\subsubsection*{Summary}
%Images or tables that summarise what differential forms are on 2 dimensional and 3 dimensional manifolds. \note{Todo, as image, consistent with later images...}

%\begin{table}[h]
%\begin{longtable}{lccc}
%& & 2-Manifolds &\\
%Forms & $\Lambda^0(T_pM)$ & $\Lambda^1(T_pM)$ & $\Lambda^2(T_pM)$\\
%Dimension & 1 & 2 & 1 \\
%Differential Form  & $f:M\rightarrow \mathbb R$ & $\omega^{\sharp}: M \rightarrow TM$  & $f:M\rightarrow \mathbb R$ \\
% & $\omega_p =f(p)$ & $\omega_p(v) = \langle \omega^{\sharp} ,v\rangle$ & $\omega_p(v_1,v_2)=   f(p) \cdot dArea(v_1,v_2)$
%\end{longtable}
%\end{table}
%
%\begin{table}[h]
%\begin{longtable}{lcccccr}
% Forms & $\dim$ & Diff. Forms & Representation & & \\
% $\Lambda^0(T_pM)$ & 1 & 0-Forms & Function & $f:M\rightarrow \mathbb R$ \\%& $\omega_p =f(p)$ \\
% $\Lambda^1(T_pM)$ & 3 & 1-Forms & VField & $\omega^{\sharp}: M \rightarrow TM$  \\%& $\omega_p(v) = \langle \omega^{\sharp} ,v\rangle$\\
% $\Lambda^2(T_pM)$ & 3 & 2-Forms & VField & $\widehat{w}: M \rightarrow TM$ \\%& $\omega_p(v_1,v_2) = \langle \widehat{w} ,v_1\times v_2\rangle$ \\
% $\Lambda^3(T_pM)$ & 1 & 3-Forms & Function & $f:M\rightarrow \mathbb R$ \\%& $\omega_p(v_1,v_2, v_3)=f(p) \cdot dVol( v_1,v_2,v_3)$
%\end{longtable}
%\end{table}

\subsection{Integration of Forms}
\label{sec:integralOfForms}
We started this chapter saying that we wanted to design objects that are well suited for integration. We ended up with differential forms that in 3-dimensional spaces turn out to be either vector fields or functions. Let's now look at how differential forms are integrated. We omit various technicalities-for a clean introduction of the integral see e.g. \cite{globalAnalysis}; for a very understandable introduction see \cite{bachman2006geometric}.

A $k$-form can be integrated over $k$-dimensional regions. If $\phi : U \subset \mathbb R^k \to M$ is a map, $M$ a $k$-dimensional manifold and $\omega^k$ a differential $k$-form on $M$ and a region $\Omega = \phi(U)$ parametrized with $\phi$. The integral 
\[\int_{\phi(U)} \omega^k \]
is defined and calculated by pulling everything back to $\mathbb R^l$:
\[\int_{\phi(U)} \omega^k = \int_{U\subset\mathbb R^k} \omega_{\phi(x_1,...,x_k)}(\frac{\partial \phi}{\partial x_1},...,\frac{\partial \phi}{\partial x_k}) d x_1...d x_k\]
The integral on the right integrates a function depending on $k$ variables over a region $U$ in $\mathbb R^k$ as usual. But note: the integral on the left  lacks any '$d x_i$`s. It is independent of local coordinates, and can be computed with any set of coordinates, see Appendix \ref{app:integrals}. As intended $\omega$ automatically scales according to the volume spanned by the vectors $\frac{\partial \phi}{\partial x_i}$ for any coordinates $\phi$, thereby canceling out the choice of parametrization.

The way differential forms are designed, they can be integrated only over sets of a fixed dimensionality- a 1-form over one dimensional smooth sets, a $2$-form over 2-dimensional sets etc. A $k$-form defined on a $n$-dimensional manifold $M$ can only be integrated over any $k$-dimensional subset of $M$. This restriction is somewhat softened by the Hodge star operator $\star$, introduced in Section \ref{sec:hodgeStar}. The $\star$ will allow to make a $n-k$-form out of a $k$-form. Like that for example a $0$-form can indirectly be integrated over a $n$-dimensional set too.

\subsection{Pull-Backs}
\label{sec:pullbacks}

\begin{figure}%
\begin{center}
\includegraphics[height= 5cm]{imgs/5_pullback.eps}%	
\end{center}
\caption{A mapping $h$ can be used to pullback a differential form $\omega$ from one manifold to another, as $h$ provides a mapping between the manifolds and their tangential spaces. }%
\label{fig:5_pullback}%
\end{figure}

The 'pulling back` used to define an integral can be done more generally. Suppose we have a mapping between two manifolds $N$ and $M$ $h: N\to M$, which has $det(Dh) \neq 0$ and for simplicity  is smooth. As seen in Section \ref{sec:derivativeBetweenMfs}, $Dh$ is a mapping between the tangential spaces of $N$ and $M$. Therefore if we have a differential $k$-form $\omega$ on $M$ we can 'pull it back' to $N$ via
\[(h^*\omega)_p (v_1,...,v_k) := \omega_{h(p)}(Dh v_1,...,Dh v_k). \]
The pullback is also depicted in Figure \ref{fig:5_pullback}. The action of pulling back $\omega$ using $h$ is denoted by $h^* \omega$. This mapping preserves the integral (check it by using the definitions!)
\[\int_{h(U)\subset M} \omega = \int_{U \subset N} h^*\omega \]
This means that we can integrate either $\omega$ over a subset of $M$ or the pulled back mapping over a subset of $N$. Usually, as in the definition of the integral, you pull back forms to $\mathbb R^k$, using one of the local maps.
The pullback is very powerful as it conserves properties under the integral. 

\section{Discrete Differential Forms}

Differential forms are defined on Manifolds and as it is to be expected, discrete forms are defined on discrete manifolds. But there are no tangential spaces on discrete manifolds. While differential forms are spatially varying multilinear mappings, a discrete differential is something much simpler - it simply is a set of averaged values, as depicted in Figure \ref{fig:5_discreteForms}

\begin{definition}[Discrete Form]
A discrete $j$-form on a discrete $k$-manifold assigns a real number to every $j$-simplex contained in the discrete manifold. This vector of values is also sometimes called a $j$ co-chain. 
\end{definition}
The question to be answered is how this set of values relates to a non-discrete differential form.

\subsection{Sampling Forms}
\label{subsec:samplingForms}

\begin{figure}%
\begin{center}
\def\svgwidth{0.8\columnwidth}
\input{imgs/5_discreteForms.eps_tex}
\end{center}
\caption{A discrete differential $k$-form is a set of values associated to the $k$-simplices of a discrete manifold. A value represents the integral of the sampled differential $k$-form over the associated simplex.}%
\label{fig:5_discreteForms}%
\end{figure}

To relate discrete forms with differential forms, the discrete manifold $K$ needs to be related somehow to a non-discrete manifold $M$. We will just assume that the discrete manifold $K$ approximates the manifold $M$ and for any simplex $\sigma \in K$ there is a continuous analogue $\sigma \subset M$ on the manifold $M$, as in Figure \ref{fig:5_simplexVsMF}. We denote both the discrete simplex and the continuous counterpart by the same symbol. Which is meant should be clear from the context.

The relation then is simple: given a differential $k$-form $\omega^k$ on $M$ the value of the discrete $k$ form $\textbf{w}$ on a $k$-simplex $\sigma$ is value of $\omega^k$ accumulated over $\sigma$.

 \[\textbf{w}(\sigma) = \int_\sigma \omega^k\]
In the following we look at some examples in $\mathbb R^2$ and $\mathbb R^3$.

\subsubsection{Sampling 0-Forms}
As seen in Section \ref{subsec:diffformsare}, $0$-forms can simply be represented as functions $f: M\to \mathbb R$. A $0$-form has to be integrated over 0-dimensional sets, i.e. points. The discrete 0-form is a set of values associated to vertices. The value at a vertex position $v$ or 0-simplex $v$
 is
\[\textbf{w}^0(v) = f(v)\]
i.e. $f$ evaluated at $v$.

\subsubsection{Sampling 1-Forms}
A $1$-form $\omega^1$ on a manifold $M$ can be represented by a tangential vector field $\nu:M\to TM$ via 
\[\omega^1_p(v) = \langle\nu(p),v\rangle.\] 
A $1$-form can be integrated over $1$-dimensional curves. A discrete $1$-form is therefore a set of values associated to the $1$-simplices i.e. edges of the discrete manifold. The value on an edge $e$ is
\[\textbf{w}^1(e) = \int_{e} \omega^1 = \int_{0}^1 \langle\nu(e(t)),\frac{\partial}{\partial t}e(t)\rangle dt\]
where in the last integral $e(t)$ is a parametrization of the curve on the manifold $M$ associated to the edge $e$. A discrete $1$-form samples a vectorfield by projecting the field on the edge and accumulating these values along the edge. The resulting value can be thought of as measuring how much the vectorfield 'flows' along the edge. If the edge $e$ is a straight line and the vectorfield a constant vector  $\textbf{w}^1(e)$ is simply the projection of the vector to the edge
\[\textbf{w}^1(e) = \langle \nu, e \rangle.\]

On a $2$-manifold there is a second way of how to sample the vectorfield by measuring the flow \emph{through} the edge instead of \emph{along} the edge; we will come back to this in a while when talking about the Hodge star $\star$ operator in Section \ref{sec:hodgeStar}. 
\[\textbf{w}^1(e) = \int_{0}^1 \langle\nu(e(t)),(\frac{\partial}{\partial t}e(t))^\perp \rangle dt\]
Here $^\perp$ denotes the vector rotated by $90^\circ$ according the orientation of the surface. The two sampling schemes are depicted in Figure \ref{fig:5_samplingForms}.


\begin{figure}%
\begin{center}
\includegraphics[height= 3.5cm]{imgs/5_simplexVsMF.eps}%	
\end{center}
\vspace{-0.5cm}
\caption{We assume that for a simplicial complex that samples a manifold every simplex has related smooth region on the manifold.}%
\label{fig:5_simplexVsMF}%
\end{figure}

\subsubsection{Sampling 2-Forms}
A $2$-form can be integrated over $2$D patches and the discrete 2-form associates values to the 2-simplices i.e. the triangles of the discrete manifold.

On a 2-manifold: here a differential $2$-form $\omega^2$ is represented by a function $f$. The value $\textbf{w}^2(t)$ on a triangle $t$ is the integral of $f$ over $t$:
\[\textbf{w}^2(t)= \int_{t} f\, dVol\]

On a 3-manifold: here a differential $2$-form $\omega^2$ is represented by a vectorfield $\nu: M\to TM$, but evaluating it on two vectors amounts to
\[\omega_p(a,b) = \langle \nu(p) , a \times b \rangle\]
such that the value of the discrete $2$-form associated to a triangle $t$ is
\[\textbf{w}^2(t)= \int_{t} \langle \nu(p), n(p) \rangle \, dp\]
where $n(p)$ denotes the normal on the surface $t$ at the point $p$. This measures the flow of the vectorfield \emph{through} the surface $t$, see Figure \ref{fig:5_samplingForms}.


%\subsubsection{Some Observations}
%\note{is this needed?? Is this the right place? Should Duality be introduced here for real? Decide this later} This is the right place to make some observations. As we have seen on 3 manifolds vectorfields can either be interpreted as $1$ OR as $2$ forms, which decides on how they are integrated. The same is true for functions $f$ that can be interpreted as $0$ forms or $3$ forms, which again decides on how to integrate them. 
%
%On 2D surfaces $0$ and $2$ forms are represented the same way and for $1$-forms there were 2 ways to integrate them. There is a principle behind this: on an $n$-dimensional manifold there is a strong relation between differential $(n-k)$-forms and differential $k$-forms. We can make a $(n-k)$-form out of a $k$-form and vice versa. This is mirrored in the representation of differential forms above: a $1$-form can be interpreted as a $(3-1) =2$ form in an $(n =3)$D spaces and so on. The 'related' form will be called 'dual' form and you will get from the $k$ form $\omega^k$ to its dual $\nu^{n-k}$ using the so called Hodge operator $\star$:
%\[\nu^{k} = \star \omega^{n-k}\]
%In a 2D setting the dual of a $1$-forms is again a $1$-form, which explains (or at least motivates) why we gave two ways to sample $1$ forms.

%Functions $f$ are special anyway as, restricted to any $k$-manifold they can be interpreted as a $k$-form and integrated over it.

\subsection{Integration of Discrete Forms}

\begin{figure}%
\includegraphics[width=\columnwidth]{imgs/5_samplingForms.eps}%
\caption{Possibilities to sample 1-forms on 2-manifolds: you can either measure the flow through or along the edge. 2-forms on 3-manifolds are sampled by measuring the flow through a face.}%
\label{fig:5_samplingForms}%
\end{figure}
A discrete form can very easily be integrated over a set of simplices. Integrating a discrete $k$-Form $\textbf{w}^k$ over a set of $k$-simplices $\{\sigma_1,...,\sigma_l\}$ can be done simply by summing up the values on those simplices. If $\textbf{w}^k$ is the sampled version of $\omega^k$ this sum is exactly the integral of $\omega^k$  over the k dimensional set $\{\sigma_1,...,\sigma_l\}$
\[\int_{\{\sigma_1,...,\sigma_l\}} \omega^k = \sum_{i=1}^l \textbf{w}^k(\sigma_i)\]
as
\[\textbf{w}^k(\sigma_i) = \int_{\sigma_i} \omega^k\]
As we have seen we can describe a set of simplices in a discrete manifold as a vector $\sigma$ of dimension $\# k$-$simplices$ consisting of plus/minus ones and zeros. The discrete form $\textbf{w}^k$ is a vector of the same dimension and the discrete integral over $\sigma$ is the scalar product of those two vectors
$$\langle \sigma , \textbf{w}^k \rangle$$
The discrete analogon of the integral is in our setting the scalar product!

%\newpage		
\chapter{Exterior Calculus \& Discrete Exterior Calculus}

This chapter is the core of this thesis. The things that have been explained were explained solely because they are needed to understand this crucial chapter. The anticlimactic thing now is that we can sum up most of this chapter in the following three line argument:
\[\int_{\delta \Omega} \omega = \int_{\Omega} d \omega\]
therefore
\[d_{discrete} := \delta_{discrete}^T\]
But then again, these three lines are very compact and use quite a few concepts. We have here an $\Omega$ which is an oriented bordered manifold, $\delta$ which is the border operator that induces an orientation to the border, $\omega$ which is a differential form and $\delta_{discrete}$ which is the discrete border operator for (oriented) discrete manifolds, written as a matrix. And there are integrals $\int$ that integrate differential forms. So we had to deal with each of those things before we could get here.
The big unknown in the argument above (aside from the equation itself) is the 'differential operator' $d$ which we have not seen yet. The $d$ is interesting, because it generalizes various operators like the divergence and curl and the very nice properties that $d$ has will directly hold for them.

This chapter is structured in the following way: we begin this chapter by introducing $d$ and explaining Stokes theorem, which is the name of the first equation. Then we describe
how to discretize $d$. In the fourth section we will introduce the Hodge star operator $\star$ and its discrete version which will be the last missing element we need to formulate various operators known from classical calculus. \note{...}

\begin{longtable}{|p{4.5cm}|p{4.5cm}|p{4.5cm}|}
\hline
Smooth Theory& Discrete Theory& Implementation (Notes)\\
\hline
	External Calculus
	\begin{packed_enum}
		\item[-] Gradient, Curl and Divergence
		\item[-] d
		\item[-] Stokes Theorem
		\item[-] Star and DeRham Complex
	\end{packed_enum}
	&
	Discrete External Calculus
	\begin{packed_enum}
		\item[-] Discrete d
		\item[-] Dual Mesh
		\item[-] Show also intuitive match to curl etc
	\end{packed_enum}
	 & 
	 A look at the Laplacian from chapter 2
	 \begin{packed_enum}
		\item[-] The DEC matrices (and tests)
	\end{packed_enum}
	 \\		
\hline
\end{longtable}
\section{(?) Gradient, Curl, Divergence}
\note{Geometric Definition of gradient, curl and divergence, maybe plus reduction to 'standard' $\nabla$ operator when using appropriate coordinates.}

\note{The question is where and how to do this. In an own chapter, together with Differential Structure? How general? Probably not too general... and more on an intuitive level then anything else... }
\note{Write this / Decide about this in next iteration}

\section{The Exterior Derivative $d$}
\note{The operators above take a form of one type and return an other.
Introduce the $d$.}

The differential operator $d$ (called exterior derivative) is somewhat a generalisation of the usual derivative. But maybe it is better to look at $d$ as something completely new and unknown, because thinking of it as the 'derivative for differential forms' might lead to wrong associations and expectations. For example the idea to apply the differential operator multiple times to get an 'nth derivative' does not make sense, as you will see.

But a good (and safe) idea is to look at the gradient of a function to motivate the differential operator $d$. The gradient takes a function $f:\mathbb R^n\to \mathbb R$ and returns a vector field of vectors $grad_p(f)$, such that
\[f(p + x) \approx f(p) + \langle grad_p(f), x \rangle\]
Lets translate this in terms of forms. We interpret $f$ as a $0$-form and the vector field as a $1$-form. The gradient therefore takes a $0$-form and returns $1$-form. And this is generalized by the differential operator $d$: it takes $k$-forms and returns $k+1$-forms.

\subsection{Defining $d$} 
The differential operator $d$ is easy to define on $\mathbb R^n$:
\begin{definition}[Exterior Derivative on $\mathbb R^n$]
The exterior derivative $d$ maps differential $k$-forms to differential $k+1$-forms. If $\omega^k$ is given in standard coordinates $x_1,..,x_n$ as 
\[\omega^k_p= \sum_{i_1<...<i_k} \omega_{i_1,..,i_k}(x) dx_{i_1} \wedge ... \wedge dx_{i_k}\]
the exterior derivative is given by
given by 
\[d\omega^k = \sum_{i_1<...<i_k}\sum_{\alpha = 1}^{n}\frac{\partial \omega_{i_1,..,i_k}(x)}{\partial x_\alpha} dx_i \wedge dx_{i_1} \wedge ... \wedge dx_{i_k}\]
\end{definition}

To define $d$ on manifolds in general we use pullbacks \note{as introduced in section...} namely: if $M$ is a manifold, $h$ a local map and $\omega^k$ a a $k$ form then
\[d\omega^k := (h^*)^{-1}d(h*\omega^k).\]
The first pullback $h^*$ transforms $\omega^k$ to a $k$-form on $\mathbb R^n$, where the exterior derivative is already defined and can be used. Then the result is pulled back to the manifold. \note{Any note on welldefinedness needed? Image?}


But why do we care about this rather unintuitive operator?
The exterior derivative is very useful because it has loads of great properties (like the Stokes Theorem in the next Section the Point Carr� Lemma  in Section \note{...} or the Hodge decomposition Theorem \note{...}). And because many differential operators from classical calculus are special cases of it or can be expressed with it as with the examples given in the following section.

\subsection{What $d$ is in $\mathbb R^3$}
Let's get to know the exterior derivative on the 'manifold' $\mathbb R^n$ with the standard base. The exterior derivative $d$ will act exactly the same way if written in other coordinates \note{example in...}. And that is quite nice: we have formulated $d$ such that we do not need to care if we are on a manifold or what coordinates we are using and we can readily describe $d$ if we have chosen any coordinates; just as it was with differential forms. 

\subsubsection{$0$-Forms}
If we have a $0$-Form on $\mathbb R^n$  given by $f:\mathbb R^n \to \mathbb R$ then the exterior derivative is (by definition)
\[df = \sum_{\alpha = 1}^n \frac{\partial f}{\partial x_\alpha} dx_\alpha \]
and $df$ at any point applied to a vector $df_p(v)$ is
\[df_p(v) = \langle \nabla f, v \rangle\]
and $\nabla f = (\frac{\partial f}{\partial x_1},...,\frac{\partial f}{\partial x_n})$ is just the usual gradient in euclidean coordinates.

\note{Auskommentiert: kommentar ueber die notation $dx_i$ / $du_i$... Braeuchtes das...?}
%On a side note: this also motivates the notation $dx_\alpha$ for the special one forms we use as a basis of the space of $1$-forms; $x_\alpha$ is the $\alpha$th coordinate of a point $x$, i.e. short for the function $f(x) = f(x_1,...,x_n) = x_\alpha$. Therefore if we interpret $d$ as the exterior derivative 
%\[(dx_\alpha)_p(v) = \langle e_\alpha,v \rangle \]
%is exactly what we defined it \note{in sec...} to be. The same is true for an arbitrary map $\phi(u_1,...,u_k)$ that locally assigns the coordinates $u_1,..,u_k$ to a manifold $M$:
%\[du_i = (\phi^*)^{-1}(d (\phi^*)(u_i)\]
%\[=(\phi^*)^{-1}(d e_i)\]
%\[= \langle D\phi \cdot e_i, D \phi v \rangle\]
%\note{or similar}
\subsubsection{$1$ Forms}
A vector field 
\[\mathcal V : \mathbb R^n \to \mathbb R^n\]
\[\mathcal V(x) = (v_1(x),...,v_n(x))\]
 interpreted as a one form on $\mathbb R^n$ with standard coordinates is
\[\omega^1_p = \sum_{i = 1}^n v_i(p) d x_i \]
and to apply the exterior derivative  to it amounts to
\[d\omega^1 = \sum_{i=1}^n \sum_{j = 1} ^n \frac{\partial v_i(p)}{\partial x_j} dx_j \wedge d x_i\]
if we reorder these terms
\[= \sum_{1\leq i < j \leq n } (\frac{\partial v_j(p)}{\partial x_i} - \frac{\partial v_i(p)}{\partial x_j}) dx_i \wedge d x_j\]
and on $\mathbb R^3$ this is exactly the $rot$ operator: if we represent the arising 2 form as a vector we get
\[d \begin{pmatrix}
v_1(x) \\ v_2(x) \\ v_3(x)
\end{pmatrix} = \begin{pmatrix}
\frac{v_3(x)}{\partial x_2} -\frac{v_2(x)}{\partial x_3}\\
\frac{v_1(x)}{\partial x_3} -\frac{v_3(x)}{\partial x_1}\\
\frac{v_2(x)}{\partial x_1} -\frac{v_1(x)}{\partial x_2}\\
\end{pmatrix}\]
\subsubsection{$2$-Forms on $\mathbb R^3$}
Lastly we have a look at $d$ on $2$-Forms on $\mathbb R^3$. Again the differential form can be represented as a vector field $\mathcal V = (v_1,v_2,v_3)$ and
\[\omega^2 = v_1 dx_2 \wedge dx_3 + v_2 dx_3 \wedge dx_1 + v_3 dx_1 \wedge dx_2\]
The exterior derivative is then
\[d \omega^2 = (\frac{\partial v_1}{\partial x_1} + \frac{\partial v_2}{\partial x_2} + \frac{\partial v_3}{\partial x_3})dx_1\wedge dx_2 \wedge dx_3\]
which is exactly the divergence operator.

\subsubsection{Summary}
This is a good place to summarise the relations between differential forms and exterior calculus and standard calculus as done in Figures \ref{fig::6_1_SC2d} and \ref{fig::6_1_SC3d}.

\begin{figure}
\begin{center}
\includegraphics[height=3.5cm]{imgs/6_1_SCvsEC_2d.eps}
\caption{Top: The differential forms arising on 2 dimensional manifolds and the exterior derivative. Bottom: The corresponding objects and operators in standard calculus}
\label{fig::6_1_SC2d}
\end{center}

\end{figure}

\begin{figure}
\begin{center}
\includegraphics[height=3.5cm]{imgs/6_1_SCvsEC_3d.eps}
\end{center}
\caption{Top: The differential forms arising on 3 dimensional manifolds and the exterior derivative. Bottom: The corresponding objects and operators in standard calculus}
\label{fig::6_1_SC3d}
\end{figure}





\subsection{Properties of the exterior derivative}
The exterior derivative has the following properties that are more or less straight forward to check by plugging in the definitions; you can find detials e.g. in [globalAnalysis]

\begin{enumerate}
\item $d(\omega^k + \psi^k) = d\omega^k + d\psi^k$
\item $d(\omega^k \wedge \psi^l) =( d\omega^k) \wedge \psi^l + (-1)^k \omega^k \wedge(d \nu^l)$
\item $d(d\omega^k) = 0$
\item $f^*(d\omega^k) = d(f^* \omega^k)$
\end{enumerate}
 The third and fourth property are the most note worthy.  Applying $d$ two times in a row always leads to zero (as you can check by simply writing it down). And the exterior derivative commutes with pullbacks. This means that you can freely chose where and in what map you want to work and calculate derivatives; just pull everything to a space where you want to have it.


\subsection{$d$ in another base}
\note{Optional Section... For fun and because it is simple: gradient in different coordinates}

\section{Stokes Theorem}
Now we have finally arrived to the magical chapter where we explain
\[\int_{\delta\Omega} \omega = \int_{\Omega} d \omega.\]
Actually to show or at least sketch why this theorem holds is not that complicated. It is a generalization of and follows from the fundamental theorem of calculus: if $f:\mathbb R \to \mathbb R $ has an antiderivative $F$ i.e. $F' = f$, then
\[\int_a^b f(x) dx = F(b) - F(a).\]
We can rewrite this in the differential form notation: say $\Omega = [a,b]$ is an oriented line (1 Manifold) such that the border is $\delta \Omega = -\{a\} + \{b\}$, $\omega^0 = F$, then $d F = F' =f$ is a 1 Form; as it is defined on a one manifold it can be represented as a function.
\[\int_{[a,b]} d\omega^0 = \int_{\delta [a,b]} \omega^0 = \int_{-\{a\}}\omega^0 + \int_{\{b\}} \omega^0 = -\omega^0(a) + \omega^0(b)\]
So the fundamental theorem is the Stokes theorem applied to $0$-forms. Note how it is important that $\omega^0$ respects the orientation of the points it is applied to; this equals the antisymmetry of higher order forms.

We will only shortly sketch a proof for Stokes theorem for forms defined on $\mathbb R^n$. If Stokes theorem is proven on $\mathbb R^n$ it is not complicated to to see how we would get the result on Manifolds; we have seen that the integral does not change under pullbacks 
\[\int_{\phi(U)} \omega = \int_{U} \phi^*\omega\]
and we have seen as well (or at least mentioned) that the derivative $d$ commutes with the pullback such that we can directly use the theorem for $\mathbb R^n$:
\[\int_{\phi(U)}d \omega = \int_{U} \phi^*(d\omega) = \int_{U} d (\phi^*(\omega)) = \int_{\delta U} \phi^*(\omega) = \int_{\delta(\phi(U))} \omega\] 
There are a few technical difficulties, mainly that you do not have a global parametrisation $\phi$ in general. A clean proof can for example be found in [GlobalAnalysis].

\subsection{A Proof Sketch for Stokes Theorem}

This sketch is following strongly the reasoning made in [globalAnalysis],

It will be enough to show the Theorem for a very simple geometric object: a so called singular cube. A $k$-dimensional singular cube $c^k$ is basically a manifold $C^k$ together with a global parametrisation \note{image}
\[c^k: [0,1]^k \to C^k \subset \mathbb R^n\] 
So we show
\[\int_{\delta C^{k+1}} \omega^{k} = \int_{C^{k+1}}d\omega^k\] 

\subsubsection{Proof (sketch)}
Given a singular cube $C^{k+1}$ with parametrisation $c^{k+1}$ we can pull the whole problem back to $[0,1]^{k+1}$. So actually it is enough to show the theorem for the standard cube $[0,1]^{k+1}$.

We write an arbitrary $k$-Form $\omega^k$ on $[0,1]^{k+1} \subset \mathbb R^{k+1}$ as 
\[\omega^k = \sum_{i=1}^{k+1} f_i dx_1 \wedge...dx_{i-1} \wedge dx_{i+1} ...\wedge dx_{k+1}\]
where in each term the $i$th $dx_i$ is omitted. Then
\[d\omega^k = \sum_{i=1}^{k+1}(-1)^{i-1}\frac{\partial f}{\partial x_i}dx^1\wedge ... \wedge dx^k\]
and
\[\int_{I^{k+1}} d \omega^k = \sum_{i=1}^{k+1}(-1)^{i-1} \int_{[0,1]^{k+1}} \frac{\partial f}{\partial x_i} dx_1 \wedge...\wedge dx_{k+1}\]
Now we can simply use the known funamental theorem to integrate the single terms in the sum relative to $x_i$
\[\int_{0}^1 \frac{\partial f}{\partial x_i} (x_1,...,x_{i-1},t,x_{i+1},...) dt = f(x_1,...x_{i-1},1,x_{i+1},...) - f(x_1,...x_{i-1},0,x_{i+1},...)\]
getting 
\[\int_{[0,1]^{k+1}} d \omega^k = \sum_{i=1}^{k+1}(-1)^{i-1} \underbrace{\int_{0}^1...\int_{0}^1}_{\textit{k+1 integrals}} \frac{\partial f}{\partial x_i} dx_1...dx_{k+1}\]
\[= \sum_{i=1}^{k+1}(-1)^{i-1} \left(\underbrace{\int_{0}^1...\int_{0}^1}_{\textit{k integrals}} f(x_1,...,x_{i-1},1,x_{i+1}...,x_{k+1}) dx \right.\]
\[- \left. \underbrace{\int_{0}^1...\int_{0}^1}_{\textit{k integrals}} f(x_1,...,x_{i-1},0,x_{i+1}...,x_{k+1}) dx \right)\]
where the integrals used are the 'common' integrals, not the one for forms. Every term in the last sum integrate $f$ over one side of the cube $[0,1]^{k+1}$, because plugging in a $1$ or a $0$ for one parameter and integrating over the others has exactly that effect. The factors $(-1)^{i-1}$ together with the minus from the fundamental theorem, result exactly in assigning to every term a sign matching the orientation induced by the border operator applied to $[0,1]^{k+1}$\note{image}. 

And therefore
\[\int_{[0,1]^{k+1}} d \omega^k = \int_{\delta [0,1]^{k+1}}\omega^k\]

.......................................

And with the sketch in Figure \ref{fig::6_1_singularCubes} and the less than exact observation that we can build more or less everything out of deformed cubes, we see Stokes theorem as sufficiently proven for manifolds :).

\begin{figure}[h]
\begin{center}
\includegraphics[height=3cm]{imgs/6_1_singularCubeChain.eps}
\end{center}
\caption{Some random bordered manifold that can be made out of 3 singular cubes. Some integral $\int_O$ over this object amounts to the sum of integrals over the singular cubes. For every cube the Stokes theorem is true. So the integral $\int_O$ is given by the sum of border integrals of the cubes. But inner edges cancel out because of their opposite orientations. Therefore $\int_O d\omega= \int_{\delta O}\omega$} 
\label{fig::6_1_singularCubes}
\end{figure}


\subsection{Special Cases of Stokes Theorem}
\note{needed? ....}

\note{Known Examples and special cases. Cases where it does not hold.}

\subsection{Geometry of $d$}

Stokes' theorem is more then a  valuable tool for calculations and for reformulations. It shows you the geometry of the exterior derivative $d$. Stokes theorem binds the exterior derivative strongly to the border operator; both operations are somewhat equivalent, as you can chose to either apply the border operator to a region or to apply the exterior derivative to the differential form at hand.
\[\int_{\Omega} d \omega = \int_{\delta \Omega} \omega\] 
We can make this even clearer by using a bracket notation for the integral; the first argument is the manifold, the second argument the differential form:
\[[\Omega, \omega] := \int_{\Omega} \omega\]
Then Stokes theorem can be formulated as
\[[\Omega, d\omega] = [\delta \Omega, \omega]\]
and the border operator and the exterior operator play an equivalent role.

\section{Discrete Differential Operator}
Stokes Theorem captures the geometry of the exterior derivative. And we can now define the discrete exterior derivative to conserve this geometric property i.e. conserve Stokes theorem. We can directly translate Stokes Theorem to the discrete setting:
\[[\Omega,d\omega] = [\delta \Omega, \omega]\]
becomes
\[\langle \sigma, d_{discrete}^k\textbf{w}^k \rangle = \langle \delta_k \sigma, \textbf{w}^k \rangle\]
as integrals are simply scalar products in the discrete setting. But this relation \emph{defines} the yet unknown $d_{discrete}$; it has to be the transposed of the border-operator matrix:
\[d_{discrete} = \delta^T\]
By discretizing the exterior derivative we get at once consistent discretizations of all its special cases: gradient, divergence and curl.

\subsection{Examples}
For example we know that $d$ applied to $0$ forms is the gradient. Our discrete realisation of the exterior derivative for $0$-forms is  

\section{Duality: We want more}
Not all operators can be built yet. Introduce Star, duality. The star could also be motivated with the obvious relation between k and n-k forms.
\section{Dual Mesh and Star Operator}
Discrete Version of this.
\section{This is Discrete Exterior Calculus}

\section{The 0 Form Laplacian from the beginning}
Now easy to write that guy down.

\newpage		
\chapter{Application: Mesh Parametrization}
	\begin{longtable}{|p{4.5cm}|p{4.5cm}|p{4.5cm}|}
		\hline
		Smooth Theory& Discrete Theory& Implementation (Notes)\\
		\hline
			Conformal Maps
			\begin{packed_enum}
				\item[-] Conformal Maps with Exterior Calculus
			\end{packed_enum}
			&
			Existance of Embeddings, The thingy theorem
			\begin{packed_enum}
				\item[-] Dimension of result spaces depending on topology?
			\end{packed_enum}
			 & 
			 Implementing it with DEC
			 \begin{packed_enum}
				\item[-] The equation
				\item[-] Border Constraints
				\item[-] Results
			\end{packed_enum}
			 \\		
		\hline
	\end{longtable}
	In this chapter we are all about 0 Forms.
	\subsection{Embeddings}
	Taking coordinates as functions associated to the Surface and not the other way round. The thingy theorem for graphs
	\subsection{Conformal Maps}
	Conformal maps properties, EC formulation
	\subsection{Implementaion: Conformal Embedding}
		How its done. Pretty straight forward if you have the machinery. Different Border Constraints. Mention MeanValue Weights?
	\subsection{Dimension of Harmonic Space?}
		Looking at results and topology. Genus, Bettinumbers, DeRham Complex?	Would be nice..
		Mention cutting algorithms like the quad mesh paper..
\newpage	
\section{Application: Vectorfield Design and the general Laplacian}
	\begin{longtable}{|p{4.5cm}|p{4.5cm}|p{4.5cm}|}
		\hline
		Smooth Theory& Discrete Theory& Implementation (Notes)\\
		\hline
			Important External Calculus Results:
			\begin{packed_enum}
				\item[-] Point Carre Lemma
				\item[-] Laplace Beltrami Operator: $d$ and $\delta$ free.
				\item[-] Hodge Decomposition
				\item[-] Here on the dim of harmonic spaces?
			\end{packed_enum}
			&
			The same as the smooth ones.
			\begin{packed_enum}
				\item[-] Properties still hold exactly.
				\item[-] Laplacian in Least square sence
				\item[-] Border Constraints
				\item[-] One Form interpolation
			\end{packed_enum}
			 & 
			 Implementing it with DEC
			 \begin{packed_enum}
				\item[-] Vector Field Design
				\item[-] =Least square harmonic 1 Form solving
			\end{packed_enum}
			 \\		
		\hline
	\end{longtable}
	\subsection{Point Carre Lemma}
	When is a form d or delta of another form?
	\subsection{General Laplacian}
	The Laplacian is as well d as delta of another form...
	\subsection{Hodge Decomposition}
	Splitting Forms
	\subsection{Application: Vector Field Design}
	Using all this for Vector Field Design
	\subsection{Border Constraints}
	How borders can be treated but it affects everything.
\newpage
\section{A Fluid Simulation with DEC}
	\begin{longtable}{|p{4.5cm}|p{4.5cm}|p{4.5cm}|}
		\hline
		Smooth Theory& Discrete Theory& Implementation (Notes)\\
		\hline
			\begin{packed_enum}
				\item[-] Introduction to the fluid equations and reformulation in DEC.
			\end{packed_enum}
			&
			\begin{packed_enum}
				\item[-] Second approach to Borderconstraints.
				\item[-] Reformulation / solving for rot part etc.
			\end{packed_enum}
			 & 
			 Implementing Fluid Sim with DEC
			 \begin{packed_enum}
				\item[-] Solving for Exact harmonic 1 Form
				\item[-] continuous Vfield interpolation
				\item[-] pathtracing
				\item[-] Results
			\end{packed_enum}
			 \\		
		\hline
	\end{longtable}
	All theory but the Fluid Simulation dependent theory is introduced, so this is a demonstration of DEC in use.
	\subsection{The Euler Equations}
	Short introduction to the meaning of the equation
	\subsection{The Algorithm}
	Approach and general Algorithm
	\subsection{Interpolation and Pathtracing}
	As it says. Issues on curved meshes.
	\subsection{Border Constraints}
	Need for exact Harmonic solution $=>$ Equation.
	\subsection{Influence of Mesh choice and parameter choice}
	Results and problems.
\end{document}