\documentclass[draft]{scrbook}
\usepackage{amsmath}
\usepackage{amssymb}
\usepackage{graphicx}
\usepackage{color}
\usepackage{algorithmic}
\usepackage[english]{babel}

%\usepackage[latin1]{inputenc}
\usepackage{tikz}
\usepackage{longtable}
\usetikzlibrary{shapes,arrows}

\title{Discrete Differential Calculus for Academics with No Time}
\author{Peter Bertholet}

\newenvironment{definition}[1][]{\begin{trivlist}
\item[\hskip \labelsep {\bfseries Definition (#1)}]\begin{itshape}}{\end{itshape}\end{trivlist}}

\newenvironment{packed_enum}{
\begin{enumerate}
  \setlength{\itemsep}{1pt}
  \setlength{\parskip}{0pt}
  \setlength{\parsep}{0pt}
}{\end{enumerate}}
\newenvironment{packed_itemize}{
\begin{itemize}
  \setlength{\itemsep}{1pt}
  \setlength{\parskip}{0pt}
  \setlength{\parsep}{0pt}
}{\end{itemize}}


\newcommand{\note}[1]{\textcolor{red}{\textit{(#1)}}}
\newcommand{\abs}[1]{\left| #1 \right|}

\begin{document}

	\maketitle
	\tableofcontents
	
%	\begin{abstract}
%		Discrete Exterior Calculus...
%\end{abstract}
	
\chapter{Introduction}

The goal of this Thesis is to present an introduction to Discrete External Calculus (DEC) without assuming more than some basic knowledge of linear Algebra and standard calculus. You should be able to use this text like a tutorial to get enough theoretical background and enough practical knowledge to understand DEC and apply it to problems.

\section{A glimpse of Discrete Exterior Calculus}
Before getting started, you need to get a glimpse at what Discrete External Calculus is and what it is good for. DEC is, broadly speaking, a way to state and treat a large class of differential equations on meshes. 

You most probably have already worked with some differential Operators. Some of the most common of them being the usual partial derivative $\frac{\partial}{\partial x_i}$, the gradient $\nabla = (\frac{\partial}{\partial x_1},\frac{\partial}{\partial x_2},...,\frac{\partial}{\partial x_n})$ or the Laplace Operator $\Delta = \frac{\partial^2}{\partial x_1^2} + \frac{\partial^2}{\partial x_2^2} +... + \frac{\partial^2}{\partial x_n^2}$. External Calculus provides a framework to treat those operators in a very clean and unified way. And it allows you to define and use these Operators on curved surfaces without much work.

But why should you care about this? As you probably know these operators are extremely powerful tools to describe all kind of problems. Examples---
And (non-discrete) external Calculus gives you tools to state and treat these problems very neatly.

You now might say that okay, external calculus is probably a nice way to write down equations that else would seem more cluttered and okay, it might help you describe problems even on curved surfaces, but how does this help with actual practical applications and computations? And this is what Discrete External Calculus is about: it provides a consistent and straight forward way to adapt the operators for the use on meshes for computations. 

\begin{figure}[bht]
\begin{center}
\includegraphics[width=6cm]{Imgs/1_1_line.eps}
\includegraphics[width=6cm]{Imgs/1_1_spline.eps}
\end{center}
\caption{Left: the discrete mesh you have to work with. Right: Splines are used to associate a differentiable function to the discrete set of points; saying this is a good way to approximate the underlying function makes assumptions about the underlying sampled function}
\label{fig::1_1_linevsspline}
\end{figure}

Consider this very simple situation: you have a 1-D function on a discrete set of points and want to calculate its derivative (Fig. \ref{fig::1_1_linevsspline}). Here you already encounter a problem. Obviously mathematically it is less than clear what derivatives should be on the pointset. You have different options. One would be to assume the points represent a differentiable curve which has a closed formula (depending on the pointset) such that a derivative can be calculated at any point. Another would be to embrace the situation, take the non-differentiable piecewise linear line and accept that you know only average values of the derivative on each line segment. I.e. you would say the derivative is the set of differences $f_{i+1} - f_i$ associated to the line segments $i$.

Both possibilities have advantages and drawbacks. While the first option (which leads to Finite Element like methods (FEM)) provides you a value for the derivative everywhere, the assumption that the spline is a good approximation of the ''real'' curve might be wrong and introduce errors. The second option gives you only a discrete set of values but you do not make any assumptions about the underlying curve, as for any differentiable function $f$ taking the values $f_i$, the averaged value is
\[\int_{line\;segment\;i} f^\prime dx = f_{i+1} - f_i\] 
So the second option never gives a wrong result. 

Discrete External Calculus is based on the second option. It takes the position that differential operations like taking the derivative on a discrete surface gives a discrete set of values (which have a true meaning for any underlying continuous function) and that interpolation is a somewhat independent problem.

\begin{figure}
\begin{center}
\includegraphics[width=6cm]{Imgs/1_1_1dLap.eps}
\includegraphics[width=6cm]{Imgs/1_1_1dLapSmoothing.eps}
\end{center}
\caption{An example illustrating how the DEC Laplace operator conserves a geometric property of the continuous Laplacian: When the Laplace operator is applied to the coordinates of the surface it is defined on, it defines an area minimizing flow. The image depicts the initial line with the initial Laplacian vectors (left) and the deformation of the line according to the Laplacian flow, where the border is fixed.}
\label{fig::1_1_1dSmoothing} 
\end{figure}

But the great strength of DEC is yet another one. Differential operators mostly are of a very geometric nature and have geometric properties. The gradient of a function for example always points in the direction of the highest local ascent. External Calculus is very well suited to capture some key geometric properties of those operators. 

While the piecewise linear line considered above does not allow differential operators to give concise values at all (or any) points, the geometric properties of the operators often are still meaningful. 
You can define the Laplace Operator $\Delta$ on and for surfaces; for now just assume that this is done somehow. If you then have a function defined ON the curve, you can apply $\Delta$ to this function (if both the surface and the function are 'nice' enough).  Some special functions that are defined on the line are the functions that assign coordinates to every point on the line, i.e. for a line in 2d the functions $x(p), y(p)$. 
The Laplace Operate has many very important properties. One of its (maybe less essential but rather plastic) geometric properties is that, if applied to the coordinate functions, it defines an area minimizing flow, as shown in Fig. \ref{fig::1_1_1dSmoothing}. This means that the vectors returned by the Laplacian point in a direction such that, if you move all points according to them, the area of the surface gets minimized in an optimal way.

While we can not calculate the usual Laplacian on the piecewise linear curve of Fig. \ref{fig::1_1_1dSmoothing} (it would be undefined between every two line segments and 0 everywhere else) the geometric property of area minimization stays meaningful. Thus this property can be directly used to define a discrete Laplacian. This discrete Laplacian then not merely approximates the continuous Laplacian, it also preserves some of its geometric properties \emph{exactly}.

And this is the fundament of DEC: differential operators are not merely approximated; some geometric key features from the continuous counterparts are identified and DEC is then designed to respect them exactly on a piecewise linear mesh. And, as you will see, external calculus serves you some rather deep geometric relations of these operators on a golden tablet.

% Define block styles
%\tikzstyle{decision} = [diamond, draw, fill=blue!20, 
%   text width=4.5em, text badly centered, node distance=3cm, inner sep=0pt]
%\tikzstyle{block} = [rectangle, draw, fill=blue!10, text width=7em, text centered, rounded corners, minimum height=4em]
%\tikzstyle{line} = [draw, -latex']
%\tikzstyle{cloud} = [draw, ellipse,fill=red!20, node distance=3cm,
%    minimum height=2em]


%\begin{tikzpicture}[node distance = 4cm, auto]
% Place nodes
%\node [block] (step1) {Identify Geometric Key Property};
%\node [block, right of= step1] (step2) {Formulate Property for Meshes};
%\node [block, right of= step2] (step3) {Define Operator such that the property is preserved};
% Draw edges
%\path [line] (step1) -- (step2);
%\path [line] (step2) -- (step3);
%\end{tikzpicture}    



\section{A Short Tour of this Thesis}
Before you lie dozens of pages full of incredible fun and full of mysteries which will be unravelled before your eyes. But so you don't have to take my word for it: take the small tour of this Thesis. 

\textcolor{red}{(TODO/REDO this is conceptual. Idea: about one or two sentences per theme and a nice looking image where a keyfeature is depicted. Plus the Text should have like different modules so it is easy to skip parts if you are more or less familiar with them)}

To get started some basics are covered; if you are already familiar with them you can skip them without any loss.
To begin with the geometric objects we want to work with are presented: smooth surfaces and meshes. Or more generally manifolds and simplicial complexes. Manifolds are any curved hyper-surfaces that are inbedded in a higher dimensional space and simplicial complexes are the higher dimensional equivalents of triangle meshes, including for example tetrahedra meshes.

Then the basics of differential geometry are treated; we look at local properties of surfaces and various kinds of curvature. We then try to define the equivalent on triangle meshes, which will lead to defining a Laplacian on meshes. As an application meshes get smoothed.

In the next section  differential forms are introduced and explained. This is the heart of the matter and explains why the core concepts of discrete differential forms make sense. Differential forms being a concept you might not have met as such this chapter is a crucial one. It will lead to Stokes Theorem which encodes the geometry of the differential Operator and also forms the heart of DEC.
\[\int_{\partial\Omega} \phi = \int_{\Omega} d\phi\]

The rest are various aspects, applications and refinements, where we will paralelly deepen the theory about differential forms (i.e. External Calculus) to understand aspects of practical problems and derive further parts of the discrete exterior calculus to computationally solve these problems.


%Smooth manifolds (i.e. curved surfaces) are introduced so we can later define differential operators ON them. Important notions are tangential spaces; as all applications presented here will take place on meshes (i.e. 2d surfaces embedded in 3d space) some basic notions from differential geometry are repeated; if you are more or less familiar with these notions you can skip this part.


What needs to be introduced:
\begin{enumerate}
\item Some basics about Manifolds (Background of the continuous Objects the Theory is based on)
	\begin{enumerate}
		\item Manifolds and Tangential Spaces
		\item Tangential Spaces, 
		\item Orientations
		\item how manifolds get a differential structure i.e. how differential Operators like a derivative can be defined ON surfaces
	\end{enumerate}
\item Basics about Meshes (About the Objects we are making computations on) HANDS ON
	\begin{enumerate}
		\item Simplices
		\item Orientations, Border Operator and Wind Edge structure
		\item dual meshs (and their orientations) Either here or later.
	\end{enumerate}
\item Some basic Differential Geometry
	\begin{enumerate}
		\item Curves and Tangents
		\item curvatures, curvature tensor
		\item how manifolds get a differential structure i.e. how differential Operators like a derivative can be defined ON surfaces
	\end{enumerate}
\item Differential Geometry on the Mesh: HANDS ON
	\begin{enumerate}
		\item Discrete Mean curvature
		\item Other curvatures
	\end{enumerate}
\item Differential forms:
	\begin{enumerate}
		\item What they are and how they arise
		\item How they form a linear space
		\item Operations: Stokes Theorem and the differential Operator.
		\item Identifying the linear spaces with vector fields etc,
		\item Mapping the differential operator d to classical differential operators
		\item The Hodge Star and the codifferential Operator (maybe later)
		\item Thingy when something is a closed form. (these things will be introduced when needed)
		\item The Hodge Decomposition
		\item DeRham Komplex: topological constraints to harmonic fields.
	\end{enumerate}
\item Basic Properties of differential Operators: (good question where to put this in, its quite important)
	\begin{enumerate}
		\item Divergence
		\item Gradient
		\item Curl
		\item Laplace Operator
	\end{enumerate}
\item Discrete Differential Forms HANDS ON
	\begin{enumerate}
		\item What they represent (avg values)
		\item How they are introduced
		\item Derivation of discrete d 
		\item dual mesh mesh stuff
	\end{enumerate}
\item Applications
	\begin{enumerate}
		\item Differential Geometry: Curvatures to analyse the local structure of a mesh, Smoothing, remeshing
		\item Conformal Maps
		\item Vector Field Design
		\item Fluid Simulation
	\end{enumerate}
\end{enumerate}
Oh lord, what shall i do...? Write a short! summary for each topic (MAX one Paragraph each), place them in a good order,work them out in a second iteration. Explain math without proofs, write proofs where helpfull later.

The next part 
The first practical part describes one or two applications where notions from differential geometry are discretized
First of all the smooth mathematical counterparts are treated. How to describe them and properties. Then 

What is contained in this Thesis: (TODO/REDO after the rest is written)
\begin{enumerate}
\item An explanation of the geometric nature of differential operators
\item Some basic properties of Manifolds i.e. curved surfaces.
\item The basics of external Calculus and refresh your knowledge of differential Geometry 
\item See how Discrete External Calculus is developed and which key features it tries to conserve.
\item Application of DEC to a group of problems out of various fields.
\item Cookbook recipe ?
\end{enumerate}
What is not treated in this Thesis but should not be ignored:
\begin{enumerate}
\item Other standard methods like finite differences or finite element method, even thought DEC is closely related to those and one would benefit of a closer look at FEM and its error analysis.
\item An analysis how well DEC methods perform in contrast to these standard methods or any Error Analysis of FEM.
\end{enumerate}
\chapter{Manifolds and Meshes}

\begin{figure}[h]
\begin{center}
\includegraphics[height = 4cm]{imgs/2_1_ECVsDEC.eps}
\end{center}
\vspace{-0.5cm}
\caption{Exterior calculus is defined on manifolds, discrete exterior calculus on discrete manifolds. This chapter therefore covers manifolds and discrete manifolds.}
\end{figure}

Throughout this thesis we will always have to deal with two worlds. On one side there is the continuous world, where classical calculus and exterior calculus can be used to describe problems or physical relations. On the other side is the discrete world, where computational calculations can be done and the the smooth world is approximated.

This chapter focuses the geometric objects in these worlds. On the continuous side these are manifolds, which are the mathematical objects used to describe smooth curved spaces. Manifolds are treated in Section \ref{sec::2_Manifolds}. On the other side we have discrete manifolds, which are multiple dimensional meshes and are introduced in Section \ref{sec::2_discreteManifolds}. Having a correct  understanding of both discrete and continuous manifolds is vital to understand exterior and discrete exterior calculus.

The last section of this chapter is of a more practical nature. It can be used as a guide when implementing discrete manifolds for DEC or as a hands-on section to get the most out of reading this thesis. As application some geometric operations on meshes and discrete manifolds are described.

%So lets get started. In this chapter we will have a closer look at the basic objects we will deal with throughout this text.  We start with the description of so called ''manifolds'' which describe smooth surfaces and more generally smooth spaces embedded in higher dimensional spaces.
%In the second section we have a closer look at the meshes (and more generally Simplicial Complexes) that we use in practice in the place of manifolds. The third section explains how to implement meshes and simplicial complexes in a way that is convenient for DEC purposes. 


\begin{figure}[ht]
\begin{center}
	\begin{longtable}{|p{4cm}|p{4cm}|p{4cm}|}
		\hline
		Smooth Theory& Discrete Theory& Implementation (Notes)\\
		\hline
		Smooth Manifolds (Surfaces)
		\begin{packed_itemize}
		\item[-] Maps and Coordinates
		\item[-] Tangential Space
		\item[-] Orientations
		\item[-] Bordered Manifolds
		\end{packed_itemize}
		&
		Discrete Manifolds
		\begin{packed_itemize}
		\item[-] Simplices and Simplicial Complexes
		\item[-] Discrete Manifolds
		\item[-] Orientations
		\item[-] Border Operator, Border Matrix
		\end{packed_itemize}
		&
		Meshes
		\begin{packed_itemize}
		\item[-] Winged Edge Structure
		\item[-] General Complexes
		\item[-] Sparse Matrices
		\item[-] Simple Geometric Operations
			\begin{packed_itemize}
				\item[-] Manifold Check
				\item[-] Border Computation
			\end{packed_itemize}
		\end{packed_itemize} \\
%			Smooth Surfaces & Meshes & General Meshes\\
%			-Maps and Coordinates & & \\
%			-Tangential Space & -Simplices / Simplicial Complexes & -Winged edge / Incidence\\
%			-Orientations & -Orientations& -Simple geometric operations:\\
%			-Functions on Surfaces & -Border Operator & --Orientation\\
%			-Derivative on Surfaces & & --Iterating over Neighborhoods\\
%			& & --Wellformedness \\
%			& & --Finding Border Components\\
%			& & Notes on Sparse Matrices\\
		\hline
	\end{longtable}
	\caption{Overview of the topics of this chapter}
\end{center}
\end{figure}

\section{Manifolds}
\label{sec::2_Manifolds}
In this section we introduce smooth curved spaces, known as manifolds. Some simple two-dimensional manifolds (in short: 2-manifolds) are depicted in Figure \ref{fig::2_1_manifold}.

The following topics on manifolds need to be covered:
\begin{enumerate}
\item Describing bordered manifolds using local maps
\item Tangential spaces
\item Orientation of manifolds
\item The border operator and the orientation of borders
\item Differential structure on manifolds (derivatives of functions on manifolds)
\end{enumerate}

While no application in this thesis actually exceeds 3 dimensions, discrete exterior calculus (DEC) and exterior calculus provide tools for arbitrary dimensions, without any complications. So there is no reason to cover only two or three dimensional spaces. 
		
\subsection{Describing Manifolds}	
A two dimensional manifold is an object that locally looks like a plane and possibly has one or multiple (one dimensional) borders. Three two dimensional surfaces are given in Figure \ref{fig::2_1_manifold}.

We describe manifolds using local maps. A local map at some point on the manifold describes the manifold locally; in the case of 2-manifolds as a function of two parameters. There are two types of maps: maps that describe the manifold close to borders an maps that describe the manifold at inner points, as depicted in the Figure \ref{fig::2_1_mapping}. This is taken account of by allowing the local map to either describe the manifold using a plane or a halfplane. A 2-manifold then is an object that can be described at every point locally with a 2 dimensional map (with or without border). This makes sure that the manifold looks at every point like a plane or halfplane.


Generally a $k$-dimensional manifold $M$ lying in $\mathbb R^n$ is a geometric object that locally looks like $\mathbb R^k$ or like the halfspace $\mathbb H^k = \{x= (x_1,...,x_k) \in \mathbb R^k : x_k \geq 0\}$.  Formally this is done by guarantying that at every point on the manifold there is a 'map' linking the manifold to $\mathbb R^k$ or $\mathbb H^k$.
		
For the sake of simplicity we will assume through the whole text that all functions and mappings considered are infinitely differentiable.
		
\begin{figure}
	\begin{center}
		\includegraphics[height=4cm]{imgs/2_1_2dmanifolds.eps}
		\caption{Three simple 2-manifold; the surfaces of the sphere and the torus do not have any borders, the third surface has. The inner of the sphere and the torus would be 3-manifolds with border (their borders being their surfaces)}
		\label{fig::2_1_manifold}
	\end{center}
\end{figure}
		
\begin{definition}[Map] A $k$-dimensional map is a differentiable mapping 
\[\phi: U \subset \mathbb R^k \rightarrow \phi(U)\] 
\[\begin{pmatrix}
	u_1\\ \vdots \\ u_k
\end{pmatrix} \rightarrow \begin{pmatrix}x_1(u_1,...,u_k)\\x_2(u_1,...,u_k)\\ \vdots \\ \vdots \\ x_n(u_1,...,u_k)\end{pmatrix}\]
that is injective and whose Jakobimatrix has rank $k$ on all $U$, where $U$ is some open subset of $\mathbb R^k$ or $\mathbb H^k$ (Figure \ref{fig::2_1_mapping}).
		
\end{definition} 

A $k$-manifold then is an object where you can find local maps everywhere:

\begin{definition}[Manifold] A subset $S\subset \mathbb R^n$ is a (regular) $k$ manifold, if for each point $p \in S$ there exist an open set $V\subset \mathbb R^k$ such that there is a map $\phi: U \rightarrow  V\cap S$.
\end{definition} 

	
\begin{figure}
	\begin{center}
		\includegraphics[width=14cm]{imgs/2_1_borderedmanifold_maps.eps}
		\caption{An example of a bordered 2-manifold. Left: a local map $\phi$ at some inner point. Right: a local map at a border point. The map at the border has the additional restriction that the border of the halfspace $H^2$ is mapped to the border of the manifold.}
		\label{fig::2_1_mapping}
	\end{center}
\end{figure}

There are some important details in the definition of maps. The injectivity of the maps prevents self intersections, and the non zero Jacobi determinant makes sure that the image of a map $\phi$ does not degenerate. For example the image of a two dimensional map should not degenerate to a point or a line.

\subsubsection{Local Coordinates}

A local map $\phi: U \subset \mathbb R^k \rightarrow M$ assigns local coordinates to a manifold $M$. The values of the tuple $u=(u1_,...,u_k)\in U$ are said to be the local coordinates of the point $\phi(u)$ on the manifold, in the map $\phi$. A classic example of this is the sphere parametrized by two angles using a map like
\[\phi(a,b) = (sin(a)sin(b), sin(a)cos(b), cos(a))\]
This assigns coordinates $(a,b)$ just as longitude and latitude are used as coordinate for the world. (But note that you can not parametrize the whole sphere at once if the source domain of your map is an open set and the map has to be injective)

If you have a local map $\phi$ you can also express functions $f:M \rightarrow ?$ defined on the manifold in the local coordinates given by $\phi$. This means that you consider $f \circ \phi : U \subset \mathbb R^k \rightarrow ?$ instead of $f$, such that you can use the local coordinates $(u_1,...,u_k)$ as parameter of $f$ instead of the position on the manifold. %(Image : local coordinates..) 

%\subsubsection*{A Note on Coordinates}
%Note that there are two types of coordinates here. Lets look at a 2-Manifold. You can have the local coordinates given by and depending on the local parametrisations you chose (giving you (u,v) coordinates). Then you have the global coordinates of the surrounding space $\mathbb R^3$, assigning (x,y,z) coordinates to every point.
%
%But in the end all coordinates are just descriptive tools; the geometric object we describe is assumed to exist independently of all these parameters. The 'tricky' part then is to get independence of parametrizations when we want to describe geometric properties of our manifolds. Even to get a definition of 'differentiability' on the manifold that is independent of the selected parametrization one must be careful with the definition.
%
%In fact in later applications  we will play around with coordinates, (with surface smoothing in chapter ... the global coordinates, with conformal maps in chapter ... the local coordinates). Rather than taking coordinates as describing the manifolds we take them as something associated to the geometric object to work with them or solve for them.
%
%External calculus (introduced in chapter...) will provide tools and results that do not use parametrizations.
%

	
\subsection{Tangential Spaces}		
Manifolds have tangential spaces.
For curves calculating tangents is easy. If you have a parametrization $\alpha(t): I \subset \mathbb R \to \mathbb R^n$, then $\alpha'(t) = (x_1'(t),x_2'(t),...,x_n'(t))$ is the direction of a tangent vector to the curve at the position $\alpha(t)$. 

\begin{figure}[tb]
\begin{center}
\includegraphics[width=8cm]{imgs/2_1_tangent.eps}
\end{center}
\caption{A parametrized curve, the vector $\alpha'$ and the tangential space $T_{\alpha(t_0)}$ at $t_0$.}
\label{fig::2_1_paramCurve}
\end{figure}

		While the length of $\alpha'(t)$ at the point $\alpha(t)$ depends on the parametrization $\alpha$ (for example $\alpha(2t)$ is a different parametrization of the same curve where the length doubles) the \textbf{tangential space} $ T_{\alpha(t)}S = span(\alpha'(t)) = \{x \in \mathbb R^n: x = c \alpha'(t), c \in \mathbb R\}$ does only depend of the position $\alpha(t)$ on the curve, as depicted in Figure \ref{fig::2_1_paramCurve}. 
		
We can do the same for a $k$-dimensional manifold $M$. The tangential space $T_p M$ at a point $p$ is:

\begin{enumerate}
	\item The space that approximates the surface in the best way, locally at $p$.
	\item The space that contains the tangents of all curves on the surface that go through $p$.
	\item For a given parametrization $\phi: \mathbb R^k \to \mathbb R^n$, $\phi(u) = (\phi_1(u),...,\phi_n(u))$ the tangential plane $T_{\phi(u)}$ is given by
			\[span(\frac{\partial \phi} {\partial u_1},..., \frac{\partial \phi} {\partial u_k}) = span(\begin{pmatrix}
	\frac{\partial \phi_1} {\partial u_1} \\
	\frac{\partial \phi_2} {\partial u_1}\\
	\vdots\\
	\frac{\partial \phi_n} {\partial u_1}
\end{pmatrix},...,\begin{pmatrix}
	\frac{\partial \phi_1} {\partial u_k} \\
	\frac{\partial \phi_2} {\partial u_k}\\
	\vdots\\
	\frac{\partial \phi_n} {\partial u_k}
\end{pmatrix}),\]
as depicted in Figure \ref{fig::2_1_mapping_coords} for a 2-manifold. Here the restriction that maps have a non-zero Jacobi determinant plays a role, as it means that the partial derivatives are linearly independent.
\end{enumerate}

\begin{figure}[tb]
\begin{center}
\includegraphics[width=12cm]{imgs/2_1_mapping_coords.eps}
\end{center}
\caption{A map $\phi$ of a 2 manifold $M$ is used to determine the tangential space $T_{p}M$ at some point $p$}
\label{fig::2_1_mapping_coords}
\end{figure}

The tangential space $T_p M$ is the vector space containing all tangential vectors at a point $p$. Through the tangential space, every point on a manifold gets an associated vector space. While the vector space itself is depending solely on $p$, the choice of a basis for $T_pM$ is open. Usually  the basis vectors $\frac{\partial \phi}{\partial u_i}$  are chosen, according to some local map $\phi$. But, just as it is the case in Fig. \ref{fig::2_1_mapping_coords}, these vectors are in general not orthogonal or normalized. This is something to take care of.

One special example of a manifold is  the space $\mathbb R^k$ seen as a manifold parametrized by $\mathbb R^k$ with $\phi = id$. Then the tangential space $T_p \mathbb R^k$ at any point $p$ is again $\mathbb R^k$. But still: the vectors in one tangential space $T_p \mathbb R^k$ and the vectors of another tangential space $T_q \mathbb R^k$ can not be mixed; every point gets its own proper tangential space, not shared with any other point.
		
\subsection{Orientations}
We only want to consider a special kind of manifolds: orientable manifolds. Orienting a volume is to assign a sign to the volume you are treating. Either your volume is positive or negative. 

For a vector space you can encode orientation in the ordering of basis vectors. Two ordered bases $v_1,...v_k$ and $w_1,...,w_k$ describe the same orientation if the matrix that describes the change of bases has a positive determinant. The determinant measures the signed volume spanned by a set of vectors.

In the previous section we introduced tangential spaces and emphasized that every point gets its own proper tangential vector space. Tangential spaces of points that are very close together are very similar and it makes sense to ask them to have the same orientation.

%In the last section we introduced tangential spaces and 

We saw that parametrizations can provide bases for tangential spaces. One single parametrization induces consistent orientations to the tangential spaces of all points it hits. Therefore we say that a Manifold can be oriented if all tangential spaces can be oriented consistently.

\begin{definition}[Oriented Manifold] A manifold is orientable if there exists a set of maps $\mathcal A = \{\phi: U_\phi \to \phi(U_\phi) \subset M\}$ such that the maps describe the whole manifold and any two maps $\phi$, $\psi$ which describe a common patch $\psi(U_\psi) \cap \phi(U_\phi)$ result in the same orientations i.e. the base change matrix $C$ from the base $D\phi$ to $D\psi$ has a positive determinant
\[det(C) >0\]
A manifold is oriented if for all tangential spaces a consistent orientation has been chosen.

\end{definition}

For 2d manifolds in 3d space this is the same as asking that you can consistently chose a surface normal, as tried in Fig. \ref{fig::2_1_mobius}.

\begin{figure}[t]
\begin{center}
\includegraphics[width = 6cm]{imgs/2_1mobius.eps}
\caption{The Moebius strip, the pathological example of a non orientable manifold}
\label{fig::2_1_mobius}
\end{center}
\end{figure}

\subsection{The Border Operator}
The border operator describes a special geometric operation for manifolds. 
We denote the border of a manifold $M$ by $\delta M$ and call $\delta$ the border operator. From the definition of maps at border points follows that the border of a manifold is again a manifold, where the dimension decreases by one. And from the definitions also follows, that the border of a manifold always is a manifold without border, as it is the case with spheres or tori (again see Figure \ref{fig::2_1_manifold}); this means that for bordered manifolds
\[\delta\delta M = \emptyset.\]

A central point (considering DEC) is that an oriented manifold induces an orientation to its border. This is sketched in Figure \ref{fig::2_1_borderManifold}.  What follows is a short technical description of how the orientation on the border is defined formally. The border operator and the border orientations play a central role in external calculus and throughout this thesis. Therefore it deserves to be introduced properly.

 As the orientation of a manifold is defined by the orientation of its tangential spaces we need to take a closer look at the tangential spaces of bordered manifolds.
While nothing is special for tangential spaces at non-border points, at border points two tangential spaces are present. One is the tangential space of the manifold $T_pM$ and $k$ dimensional, the other one is the $k-1$ dimensional tangential space of the border manifold $T_p \delta M$ (see Fig. \ref{fig::2_1_borderManifold}, left) . Inducing an orientation to the border means inducing an orientation in $T_p\delta M$ using the orientation of $T_p M$. This happens by defining normals on the border.

\begin{figure}
\begin{center}
\includegraphics[width = 13cm]{imgs/2_1_borderedManifold_combined.eps}
\end{center}
\label{fig::2_1_borderManifold}
\caption{On bordered manifold two tangential spaces $T_pM$ and $T_p\delta M$  are present at border points; $N$ is the outward pointing border normal (left image). The right image depicts how the manifold oriented according to the base ($b_1$,$b_2$) of some tangential space induces an orientation to the border: $N$ and a vector following the border orientation have to build a basis oriented like $b_1,b_2$.}
\end{figure}

For any border point you can define a border normal $N$. The border normal $N$ is the vector in $T_p N$ with:
\begin{itemize}
\item $N$ is orthogonal to $T_p \delta N$
\item $N$ has length 1
\item $N$ points 'outside'
\end{itemize}
Pointing outside is defined formally using the map $h$ at the border; $Dh$ is a linear bijective map from $\mathbb R^k$ to $T_pM$, so $N$ can be pulled back to $\mathbb R^k$ and it points 'outside' if the $k$th component of $Dh^{-1} N$ is negative \note{(Image?)}.

We defined orientation by the enumeration of basis vectors. So if a basis $b_1,...,b_k$ gives the orientation of $T_pM$,
a basis $\widetilde{b_1},...,\widetilde{ b_{k-1}}$ of the tangential space of the border $T_p\delta M$ is oriented according to the manifold if prepending the normal $N$ to the basis $N,\widetilde{b_1},...,\widetilde{ b_{k-1}}$ has the same orientation as $b_1,...,b_k$. This is also shown in Fig. \ref{fig::2_1_borderManifold}.


\subsection{Functions and Derivatives on Manifolds}
\label{sec::2_derivativesOnMF}

Exterior calculus is about differentiation  and integration of functions and more general things on manifolds. In this section we explain how differentiation of mappings $f:M\to M'$ between two manifolds $M$ and $M'$ is done ON manifolds.  In our setting manifolds are more than just geometric objects; they become spaces where differentiation is possible, just as it is in $\mathbb R^n$. The manifolds get a \textbf{differential structure}.

\subsubsection{Derivatives}
Given a manifold $M$ and a function $f: M \rightarrow \mathbb R^n$, what is the derivative of $f$? We want the derivative to be something very similar to the derivative $Dh$ of a function $h: \mathbb R^k \rightarrow \mathbb R^n$. In this case $Dh$ is the linear mapping that locally approximates $h$ and can be used to give the directional derivative for a direction $v$.
\[h( p + tv) \approx h(p) + Dh \cdot tv\]
We want the same for functions $f$ on manifolds: $Df$ should be a linear mapping that maps a direction to a vector that describes the change of $f$ when going in that direction. A direction on a manifold at some position is a tangential vectors. This is important: the differential $Df$ is a mapping from the \emph{tangential spaces} to vectors, as depicted in Figure \ref{fig::3_1_manifoldDerivative}. 

We can express the idea that $Df\cdot v$ describes the change of $f$ in the direction $v$ readily by using a curve $\alpha (t)$ with a tangent $\frac{\partial \alpha(0)}{\partial t} = v$ in the wished direction $v$:
\begin{equation} Df \cdot v := \frac{\partial}{\partial t} f(\alpha(t)) \label{eq:2_1_derivativeDef}\end{equation}
As $f(\alpha(t))$ is simply a function $\mathbb R \rightarrow \mathbb R^n$ we know how to calculate the right hand side $\frac{\partial}{\partial t} f(\alpha(t))$. This is not very handy for any calculations; but we can express the derivative in the local coordinates given by a parametrization $\phi(u_1,...,u_k)$.

\begin{figure}
\begin{center}
\includegraphics[width= 12.5cm]{imgs/3_1_manifoldDerivative.eps}
\end{center}
\caption{Construction of a derivative of a real valued function $f$ defined on a manifold locally parametrized by $\phi$. $Df$ at a point $p$ is a linear mapping from the tangential space $T_p M$ to $\mathbb R$}
\label{fig::3_1_manifoldDerivative}
\end{figure}

As we have seen, a parametrization provides a base of the tangential space, namely 
\[\frac{\partial\phi}{\partial u_1},..., \frac{\partial\phi}{\partial u_k}\] 
Curves can be expressed in this map and tangential vectors can be described in this base: $\alpha(t) = \phi(u_1(t),...,u_k(t))$ and $\alpha'(t) = \frac{\partial\phi}{\partial u_1} u_1' + ... + \frac{\partial\phi}{\partial u_k} u_k'$. The function $f$ also has to be given in that map , i.e. 
\begin{eqnarray*} f(u_1,...,u_k) &=& f(\phi(u_1,...,u_k)) \\
 &=& f_1(\phi(u_1,...,u_k)),...,f_n(\phi(u_1,...,u_k)). \end{eqnarray*} 
Then 
\[Df \cdot \alpha'(t) = (\frac{\partial f}{\partial u_1},..., \frac{\partial f}{\partial u_k}) \cdot \begin{pmatrix}
	u_1' \\ \vdots \\ u_k'\end{pmatrix}\]
and $Df$ is described \emph{in the local coordinates given by $\phi$} by the $ n \times k$ matrix $(\frac{\partial f}{\partial u_1},..., \frac{\partial f}{\partial u_k})$. $(u_1',...,u_k')$ is the description of the tangential vector $v = \alpha'$ in the base $D\phi$ . 

The curve $\alpha$ was only used to construct the derivative; written without $\alpha$ the derivative $Df$ in local coordinates $\phi$ is:
\[Df \cdot v = (\frac{\partial f}{\partial u_1},..., \frac{\partial f}{\partial u_k}) \cdot \begin{pmatrix}
	v_1 \\ \vdots \\ v_k\end{pmatrix}\]
where the tangential vector $v$ is expressed in the base of the tangential space induced by $\phi$
\[v = v_1 \frac{\partial \phi}{u_1} +...+ v_k \frac{\partial \phi}{u_k}\] 
 %This situation is depicted in Figure \ref{fig::3_1_manifoldDerivative}.
\subsubsection*{Derivatives of Mappings between Manifolds}
\label{sec:derivativeBetweenMfs}
A slight generalisation is considering mappings 
\[f:M\to M'\]
going from one manifold $M$ to an other manifold $M'$, as shown in Figure \ref{fig::3_1_manifoldDerivative2}. If we look again at Equation \ref{eq:2_1_derivativeDef}, we see that, as $f(\alpha(t))$ is a curve on $M'$ and $\frac{\partial}{\partial t}f(\alpha(t))$ is a tangential vector to this curve, $Df\cdot v$ has to be a vector in the tangential space of $M'$. This means that the derivative $D_pf$ at some point $p$ is a linear mapping from the tangential space $T_pM$  to the tangential space $T_{f(p)} M'$, i.e. 
\[D_p f = T_p M \rightarrow T_{f(p)} M'.\] 

If $M$ is a $k$-manifold and $M'$ a $l$-manifold $Df$ can be expressed as a $k\times l$ matrix, described relatively to two sets of local coordinates $\phi \rightarrow M$ and $\psi \rightarrow M'$.

\begin{figure}
\begin{center}
\includegraphics[width= 13cm]{imgs/3_1_manifoldDerivative2.eps}
\end{center}
\caption{Two 2-manifolds $M$ and $M'$ with local parametrizations $\phi$ and $\psi$. $f$ is a function $f: M \rightarrow M'$. $Df$ at a point $p$ is a linear mapping from the tangential space $T_p M$ to $T_{f(p)} M'$, choosing $D\phi$ and $D \psi$ to parametrize the tangential spaces $Df$ can be represented as a $2\times 2$ matrix (relative to these bases)}
\label{fig::3_1_manifoldDerivative2}
\end{figure}



%\note{	Tangential Spaces and Differential structure. Maybe put it in next chapter; here its all about geometry, not about functions.}

\newpage
\section{Discrete Manifolds}
\label{sec::2_discreteManifolds}
In the last sections we had a look at the geometric objects exterior calculus will be defined on, i.e. smooth surfaces and manifolds. The next step is to introduce the discrete analogues we want to do computations with: triangle meshes, or more generally simplices and simplicial complexes. Simplices are for example points (0-dimensional), lines (1-dimensional) triangles (2-dimensional) and tetrahedra (3-dimensional). Simplicial complexes are 'meshes' made out of them. The definitions are taken from \cite{DMK08} and \cite{FRANKEL11}.

\subsection{Simplices and Simplicial Complexes}

\begin{figure}[t]
\begin{center}
\includegraphics[height= 2cm]{Imgs/2_3_simplices.eps}
\end{center}
\caption{A 0-simplex (point) 1-simplex (line) 2-simplex (triangle) and 3-simplex (tetrahedron) }
\label{fig::2_3_simplices}
\end{figure}

A $k$-simplex is the most basic geometric object with a $k$-dimensional volume: the convex hull of $k+1$ points, as depicted in Fig. \ref{fig::2_3_simplices}. No point should lie in the convex hull of the others; else no $k$-dimensional volume is spanned and the simplex is called degenerated.

\begin{definition}[Simplex] A non degenerated $k$-simplex is the convex hull of $k + 1$ points $p_1,...,p_{k+1}$, where the vectors $p_2 -p_1, p_3,-p_1, ..., p_{k+1} -p_1$ are linearly independent. It is represented as a tuple of its corner vertices $\{p_1,...,p_{k+1}\}$.
\end{definition}

Every simplex has faces of various dimensions: any combination of $l+1$ of its corner vertices forms an $l$-dimensional face. For example a tetrahedron has 4 2-dimensional faces (triangles), 6 1-dimensional faces (edges) and 4 0-dimensional faces (vertices),see Figure \ref{fig::2_3_simplices}. A 4-simplex would have 5 tetrahedral faces and so on.

Out of simplices one can build simplicial complexes, in the same way as meshes are built out of triangles. The restrictions are the usual: the interior of any two simplices should not overlap, and if the intersection of two simplices is not empty, the intersection has to be a face of both simplices. A simplicial complex then is a list of simplices, following these restrictions. 

If a simplicial complex contains a  $k$-simplex $\sigma$, we also demand that all faces of $\sigma$ are part of the simplicial complex. This is not just a tedious technical detail;  we explicitly want to associate different values to all faces of simplices. In a triangle mesh, for example, we will need to keep track not only of triangles and vertices but also of the edges.

\begin{definition}[Simplicial Complex]
A simplicial  complex is a collection $\kappa$ of simplices, such that if a simplex is contained in $\kappa$, all its faces are too. Furthermore the intersection of any two simplices in  $\kappa$ is either empty or a common face.
\end{definition}

Lastly we do not want our discrete manifolds  to have the analogue of dangling triangles (Figures \ref{fig::2_2_dangling} and \ref{fig::2_2_dangling2}). To ensure this formally one has to make a restriction that is similar to the definition of manifolds. Just as we ensured that a $k$-manifold locally looks like $\mathbb R^k$ or $\mathbb H^k$, we want to make sure our discrete manifold looks locally like either a $k$-dimensional ball or a $k$-dimensional half-ball. This gets rid of dangling things.

\begin{figure}
\begin{center}
\includegraphics[height=2.5cm]{imgs/2_2_dangling.eps}
\caption{These are not discrete 2-manifolds: the first mesh has a dangling triangle, the second mesh has a 'wheel' and is not locally equivalent to a plane, the same holds for the third mesh}
\label{fig::2_2_dangling}
\end{center}
\end{figure}

\begin{definition}[Discrete Manifold]
A $k$-dimensional discrete Manifold is a simplicial complex where for every vertex in $\kappa$ the union of all incident simplices is equivalent to a $k$-dimensional ball or a $k$-dimensional half ball.
\end{definition} 

On discrete manifolds we can define orientations and a border operator with the same geometric meaning as on smooth manifolds.

\subsection{Orientations}
\label{subsec:SC_orientations}
As on manifolds we can treat orientations on discrete manifolds; they are quite of some practical importance and a notorious source of switched sign errors when implementing things using discrete exterior calculus. 

We can assign one of two orientations to any simplex of any dimension, meaning that the volume represented by the simplex should be considered as positive or negative. While we coded orientation before via the enumeration of basis vectors, for simplices we encode orientation via the enumeration of their corner vertices.  For edges it is the most intuitive what this means: we assign a direction to the edge $\{p_1,p_2\}$ by saying the first vertex listed is the start vertex of the edge. Note that for an edge or any geometric object there is not a strict 'positive` or a 'negative` orientation; we can only say how something is oriented relative to something else. For example the edge $\{p_1,p_2\}$ is oriented negatively to the edge $\{p_2,p_1\}$; this is noted as
\[-\{p_1,p_2\} = \{p_2,p_1\}.\]
So the orientation of a $k$-simplex depends on the way its corner vertices are enumerated. Two enumerations of corner vertices result in the same orientation if they are related by an even permutation. A permutation is called even, if it can be reproduced by switching pairs of vertices an even number of times. E.g.
\[\{a,b,c,d\} = \{c,a,b,d\}\]
\[\{a,b,c,d\} \rightarrow \{c,b,a,d\}\rightarrow \{c,a,b,d\}\]
by using two swaps,swapping  $a$ and $c$ and then $a$ and $b$.
You can also use the determinant to determine the sign of the permutation; just calculate the determinant of the permutation matrix
\[\{a,b,c,d\} \rightarrow \{c,a,b,d\}\]
\[\begin{pmatrix}c\\a\\b\\d \end{pmatrix}=\begin{pmatrix} 0 & 0 & 1 &0 \\ 1 &0&0&0 \\ 0&1&0&0 \\ 0&0&0&1 \end{pmatrix}\begin{pmatrix} a\\b\\c\\d \end{pmatrix}\]
Or again you can use the simplex to induce a base to the affine space it is aligned to
\[p_1 -p_2,...,p_{k}-p_{k+1}\]
and two enumerations induce the same orientation if these bases have the same orientation. This also shows that defining the orientation of a simplex by looking at the ordering of its corner vertices amounts to the same as orienting volumes by choosing bases.

\begin{figure}
\begin{center}
\includegraphics[height=2.5cm]{imgs/2_3_danglingTetrahedra2.eps}
\caption{If tetrahedra are not connected by two dimensional faces, they are 'dangling�.}
\label{fig::2_2_dangling2}
\end{center}
\end{figure}


One exception are vertices or 0-simplices $\{v_0\}$, where orientation is not encodable in the enumeration of the vertex. We need to assign orientations to single points too and say that $-\{v_0\}$ is the negatively oriented version of $\{v_0\}$. Orientation is 'imprinted` on the point. The best way of thinking of orientation is that orientation adds a sign to volumes. A negatively oriented point is then a point whose $0$-dimensional volume is negative. The $0$-dimensional volume of any single point is defined to be either $1$ or $-1$ and the 0 dimensional volume of a point set is $\#positive\; points - \#negative\; points$.

As long as you stick with calculations in $\mathbb R^3$ it stays pretty simple to determine if two orientations a simplex are the same, if you stick with triangle meshes it is trivial. Just make sure you always remember to respect orientations. In Section \ref{sec::2_handsOnSimplicialComplexes} we will also come back to the question of how to compute relative orientations in practice.


\subsection{The Border Operator}
\label{sec::2_borderOrientation}

Just as with manifolds we have a border operator for discrete manifolds. And just like with manifolds an oriented discrete manifold induces an orientation to its border.

We first introduce some notation: we can respresent collections of simplices as \emph{formal sums}, as depicted in Figure \ref{fig::2_2_formalsum}. The simplices are represented as tuples, a negaitve sign means a change of orientation. Simplices that are different from each other are not summed up, only if two tuples describe the same simplex but for orientation, the sum is taken. Particularly simplices of opposed orientation cancel out.
\begin{figure}
\begin{center}
\includegraphics[width = 12cm]{imgs/2_2_formalsum.eps}
\end{center}
\caption{Two sets of edges expressed as formal sums that get summed up}
\label{fig::2_2_formalsum}
\end{figure}

The border of a single $k$-simplex is  the following formal sum
\[\delta\{v_0,v_1,...,v_k\} = \sum_{j=0}^k (-1)^j\{v_0,...,\widehat{v_j},...,v_k\}\]
where  the $\widehat{v_j}$ means omitting $v_j$. This expresses that the border of the simplex is a set of $k-1$ simplices. The orientations they get are the ones that the simplex induces. Note that prepending the omitted vertex $v_j$ to $(-1)^j\{v_0,...,\widehat{v_j},...,v_k\}$ leads to a simplex with the orientation of $\{v_0,v_1,...,v_k\}$. This is consistent with the way we defined that orientation should be induced to borders of smooth manifolds. 

For example the border of a triangle $\{a,b,c\}$ is \[\{b,c\} -\{a,c\} + \{a,b\} = \{a,b\} + \{b,c\} + \{c,a\},\] just as it should be.

But if we can take the border of single simplices, we can also take the border of a set of simplices or of discrete manifolds; it is simply the formal sum of the borders of the $k$-simplices the discrete $k$-manifold is made out of. As you see in the Figure \ref{fig::2_2_borderUnoriented} this can go 'wrong` when the discrete manifold is oriented inconsistently. 

\begin{figure}
\begin{center}
\includegraphics[width = 6cm]{imgs/2_2_borderUnoriented.eps}
\end{center}
\caption{The border operator that respects orientation only makes sense with oriented discrete manifolds. Orientation of faces is depicted by an arrow that says what orientation a simplex induces to its border}
\label{fig::2_2_borderUnoriented}
\end{figure}

\subsubsection{The Border Operator as a Matrix}
If all simplices occurring in a complex are enumerated and have a fixed orientation, the border operator can be expressed as a matrix, the incidence matrix. A set of $j$ simplices is represented as vector of integers of dimension $\#(j\;simplices)$. The $k$-th entry of this vector represents the number of times the $k$-th $j$-simplex occurs. 
For example in the following enumeration
\begin{center}
\def\svgwidth{13cm}
\input{imgs/2_2_complexEnumeration.tex}
\end{center}
the set of edges $e0 - e1 + e2 $ is represented by the vector $(1,-1,1,0,0)$.
In this example there are two border matrices, one to compute the border of edge sets 
\[\delta_1 = \begin{pmatrix}
-1&-1&0 &0 & 0\\
0&1&1 &1 & 0\\
1&0&-1 &0&1\\
0&0&0&-1&-1\\
\end{pmatrix} \]
and one to compute the border of face sets:
\[\delta_2 = \begin{pmatrix}
1 & -1 & 1 &0&0\\
0& 0& -1 & 1 & -1
\end{pmatrix}^T\]
For example the border of the line segment $e_0 -e_4 + e_3$ is then given by
\[\begin{pmatrix}
-1\\ 1\\ 0 \\0 
\end{pmatrix} = \delta_1\begin{pmatrix}
1\\0\\ 0\\1\\-1
\end{pmatrix}\]
which is $-v_0 + v_1$, saying that $v_0$ is the 'start` and $v_1$ the 'end` border of the line. 

For a $k$-complex there is a total of $k$ border matrices: one border operator for sets of 1-simplices (edges), one for 2-simplices (triangle faces), one for 3-simplices and so on. We will always make the difference between these border operators; we add a $j$ as subscript to the border operator of $j$-simplices : $\delta_j$. 

The entry $(i,j)$ in a border matrix is the relative orientation of the two simplices concerned. For example $\delta_1(0,1) = -1$ because the vertex $v_0$ is oriented negatively relative to the edge $e_1$, considering the border induced orientation.

\subsection{Oriented Discrete Manifold}
\label{sec::2_orientedDiscreteMF}
Lastly we can not only orient single simplices, but also a whole discrete manifold. This leads to oriented discrete manifolds, which are the discrete analogue to smooth oriented manifolds. 

The orientation of a volume is strongly linked to the orientation of borders. For convenience we will define well orientedness of a discrete manifold using the border orientations.
Two $k$-simplices that share a $k-1$ dimensional face are oriented consistently exactly if the induced orientation of this face is opposed for both $k$-simplices, as depicted in Figure \ref{fig::2_2_borderUnoriented}.
A $k$-manifold is oriented if all-$k$ simplices are oriented consistently. 

\subsection{Summary}

\begin{figure}%
\begin{center}
\includegraphics[height=3cm]{imgs/2_2_snoothVSdiscreteBorderOp.eps}	
\end{center}
\caption{Applying the smooth border operator to a $k$-manifold returns a $k-1$-manifold, the same holds in the discrete setting. Also for \emph{oriented} smooth and discrete manifolds $\delta\delta = 0$ holds.}%
\label{fig:2_2_snoothVSdiscreteBorderOp}%
\end{figure}
Discrete manifolds are geometric objects made out of simplices and allow the definition of orientation and border operators, just like smooth manifolds. Also, applying the discrete border operator twice to an oriented discrete manifold leads to an empty set:
\[\delta_{j-1}\delta_{j} = 0\]
This mirrors the property of the smooth border operator. The smooth and the discrete border operator are depicted schematically in Figure \ref{fig:2_2_snoothVSdiscreteBorderOp}.

Subsets of $j$-simplices of a discrete $k$-manifold can simply be represented by vectors and the border operators by matrices.  Up to now only the geometry of smooth manifolds has been discretized; to develope an analogue to differential calculus on manifolds we will first need to introduce differential forms.


\newpage
\section{Implementation: Mesh and basic operations}
This is the hands-on part of this chapter. The implementation chapters provide a guideline of what you need to implement to get DEC and the later applications up and running. The components needed are described and some of the more tricky details are mentioned.

\subsection{A word on Sparse Matrices}
The point of DEC is to reformulate differential equations with sparse matrices. Therefore the implementation is somewhat centered around sparse matrices.

If you plan to implement your DEC framework you should start by looking for a sparse matrix solver. For all results in this thesis the sparse Solver from the Pardiso-Project of the University of Basel has been used as a black box solver. Unfortunately it is not free-ware but any other sparse solver will do as well.

Sparse matrices are matrices where most entries are zero. Instead of storing all $n\times m$ values of a matrix, you choose to only store the non-zero values and their indices. There are different ways to do this; the pardiso solver uses the so called Yale format.

The Yale format uses 3 vectors to describe an arbitrary $n\times m$ matrix $A$. The first vector $a$ stores all non-zero values of $A$, enumerated by row. The second vector $ja$ stores the column indices of the non-zero values, again enumerated by row. The third vector $ia$ stores for every row the index $i$, such that $a(i)$ and $ja(i)$ describe the value and the row of the first element in the row. Additionally one appends the number of values in $a$ (or $ja$) to $ia$.

For example 
\[\begin{pmatrix}
1 & 0 & 0 &3 \\
 0 & 0 & 0 &2 \\
 0 & 4&2&0
\end{pmatrix} \Rightarrow \begin{cases} a &= [1,3,2,4,2] \\ ia &= [0,2,3,5]  \\ ja &= [0,3,3,1,2]\end{cases}\]
Iterating over the values and indices of the $k$th row then amounts to
\begin{algorithmic}
\FOR{i = ia(k):ia(k+1)}
	\STATE out $\gets (k,ja(i))$   //the index pair
	\STATE out $\gets a(i)$  //of this value
\ENDFOR
\end{algorithmic}

Whatever implementation of sparse matrices you choose to use or implement, the usual basic operations need to be implemented:
\begin{itemize}
\item multiplication of matrices
\item transposing matrices
%\item iterate over the column indices and values of any row
\item adding matrices
\item inverting the elements of a matrix (replace the non zero values $a$ by $1/a$)
\item multiplication of vectors
\end{itemize}

\subsection{Implementing a Mesh for DEC}
\label{sec::2_handsOnSimplicialComplexes}
Most application in this thesis focus 2-complexes i.e. classical triangle meshes, you might not need any more general implementation so we treat general $k$ complexes separately in the next section. 

For DEC we need the complete geometric information of meshes; we explicitly keep lists of vertices, edges and faces, the full information about their incidence and border relations, as well as their assigned orientations. Edges are stored once, with an arbitrary chosen orientation.

For 2d meshes a winged edge structure is a convenient choice of representation. You could also choose to represent the mesh just by keeping lists of vertices, edges and faces plus the incidence matrices, as mentioned in the next section. \note{There might be better choices... what? References?}

\begin{figure}[tb]
\begin{center}
\includegraphics[width=8cm]{imgs/2_1_wingedEdge.eps}%	
\end{center}
\caption{The information stored on a winged structure edge}%
\label{figs::2_1_wingedEdge}%
\end{figure}

In a winged edge structure you have the following three objects:
\begin{figure}[h]
\begin{center}
\includegraphics[width = 11cm]{imgs/2_2_wingedEdge2.eps}
\caption{Impementation of a winged edge structure}
\end{center}
\end{figure}

With this information present it is easy to do things like iterating over the incident edges or faces of a vertex.

%\subsection{Implementing the Border Operator}
%\note{mention this here or include it in the border section?}


%\newpage
\subsection{Implementing k-Simplicial Complexes for DEC}

\begin{figure}%
	\begin{center}
	\includegraphics[width=10cm]{imgs/2_1_Complex.eps}%	
	\end{center}
	\caption{Implementation of a $k$ Complex; use tuple of ordered indices to characterize a simplex}%
	\label{fig::2_1_Complex}%
\end{figure}

Chances are you do not need simplicial complexes of higher dimensions other than tetrahedral meshes embedded in $\mathbb R^3$. 
	
Never the less one straight forward and for DEC suitable way to implement arbitrary $k$ complexes is to store lists of simplices and represent the incidence information explicitly as sparse matrices. The incidence matrices (border operator matrices) play a central role in DEC and need to be set up anyway.

An implementation of a $k$-complex then might look like this: the vertices (0-Simplices) are stored in a list and contain their positions. A single $j$-Simplex is then represented by a $j$ tuple of vertex indices. A $k$ complex consists then of $k+1$ simplex lists; for every dimension one list, as sketched in Figure \ref{fig::2_1_Complex}.

Setting up the border operator matrices $\delta_j$ for complexes of arbitrary dimensions is not completely trivial, as to compute relative orientations of simplices you need to find the sign of some permutation. It gets much easier if the index tuples describing the simplices are sorted i.e.
\[(i_1,i_2,...,i_j): i_1 < i_2 <...<i_j.\] 
With sorted indices we can directly use the definition of the border operator from Section \ref{sec::2_borderOrientation} to compute the relative orientation of a $j-1$ simplex $(v_0,...,\widehat{v_l},..., v_{j-1})$ lying on the border of a $j$ simplex $(v_0,...,\widehat{v_l},..., v_{j})$:
\[Orientation = (-1)^l\]

But while it is easier to compute relative orientations if the indices of your simplices are sorted, you loose the ability to store arbitrary simplex orientations using the ordering of vertices. For all but the $k$ simplices this does not matter, even for oriented discrete $k$-manifolds, as there is nor 'right' or 'wrong' orientation and all that matters is that you consistently use the same orientation all the time. But for the $k$ simplices in an oriented $k$ manifold you really need to be able to chose the orientation, so you have to keep track of the orientation independently in an additional variable (as is done in Figure \ref{fig::2_1_Complex}).

So when you resort the tuple of a $k$-simplex in a $k$-manifold you need to determine if an index tuple describes the same orientation as the sorted index tuple. This can be done using a so called inversion table. Lets say the tuple $(1,2...,n)$ is scramble to the tuple $(i_1,...,i_n)$. An inversion is then an index pair $(i_l,i_k)$, where $l<k$ but $i_l >i_k$, i.e. the order of $i_l,i_k$ was inverted. The number of inversions of a single index is then the number of indices left to it that are greater than the index. The relative orientation of a simplex represented by a scrambled tuple to a simplex with the sorted tuple is $(-1)^{\#inversions}$ where $\#inversions$ is the total number of inversions.

Example: 
\begin{eqnarray*}
3,2,5,4,1 \\
0,1,0,1,4
\end{eqnarray*}
The first line represents the permuted indices the lower the number of inversions of every index. The total number of inversion is 6 and the relative orientation of
$\{3,2,5,4,1\}$ to $\{1,2,3,4,5\}$ is $(-1)^6 = 1$ which would meant that both tuples represent simplices with the same orientation.

Setting up a $k$ complex  and all the border matrices $\delta_j$ might then look like this: start with the $k$ simplices; resort their indices if needed and adapt the stored orientation.
Then enumerate all occuring $k-1$ simplices (sort their indices, you do not need to adapt the orientation) and set up the border matrix $\delta_k$. Proceed with enumerating all $k-2$ simplices occuring as borders of $k-1$ simplices and the set up of the $\delta_{k-1}$ matrix and so on.

		
\subsection{Implementing Basic Mesh Operations}
To get later applications working you will need most of the following tools. For the applications presented in this thesis they do not need to perform extremely fast, as these operations will occur only once when setting up an application.

\subsubsection{Setup a DEC mesh from a Wireframe Mesh Representation}
The usual representation of meshes as e.g. with .obj files is by just giving a list of vertex positions and a list of faces; you need to set up the winged edge structure or $k$ complex from these.
	
\subsubsection{Set Up Border Matrices}
If you chose a winged edge structure to implement 2-complexes, you additionally need to set up the border matrices $\delta_2$ and $\delta_1$. Store them directly with your mesh. These matrices play a central role in DEC; you could say that this whole thesis is about these matrices, so test them well. If you have an oriented discrete manifold, you can also use the relation $\delta_{k}\cdot \delta_{k+1} = 0$ (the border of the border of an oriented manifold is empty) to check their correctness. Implementing the other tools in this section will test these matrices further.

\subsubsection{Check if a DEC Mesh is a Discrete Manifold}
\begin{figure}[tb]
	\begin{center}
	\includegraphics[width=12cm]{imgs/2_3_danglingTeapot.eps}
	\end{center}
	\caption{A teapot mesh that on the first look seems to be a discrete borderless 2 manifold but turns out to be a mesh with border and dangling triangles, which makes it a non-manifold mesh and therefore not suited for some DEC applications}
\end{figure}

To avoid singular matrices and to eliminate the possibility that bugs occur due to the ill-formedness of a $k$-complex or mesh (as in Fig. \ref{fig::2_2_dangling} and \ref{fig::2_2_borderUnoriented}) it is handy to have such a test method. I.e. you should test for orientation errors and the connectedness to avoid dangling simplices.
The orientation needs to be checked for $k$ simplices only, which would be for example the faces for a 2D-mesh. 

That a $k$-complex is oriented can be checked by looking at $\delta_k$. Any column has to have either exactly one entry or two entries that sum up to zero. This checks exactly the condition we gave in \ref{sec::2_orientedDiscreteMF}: Any 1 simplex is either on the border (therefore being part of exactly one $k$ simplex) or between 2 $k$-simplices, having once positive and once negative orientation.

If you are sticking with the winged edge mesh, finding dangling faces is fairly simple; at every vertex iterate over all edges and make sure that exactly 2 or no edges have only one neighbour face.
\begin{figure}
	\begin{center}
	\includegraphics[width=7cm]{imgs/2_3_danglingTriangles.eps}
	\vspace{0.5cm}
	
	\includegraphics[width=10cm]{imgs/2_3_DanglingTetrahedra.eps}
	\end{center}
	\caption{Top: On the left all 2-simplices (triangles) can be reached indirectly by hopping from triangle to triangle where every hopping pair shares a 1-simpex (considering only triangles that neighbour the marked vertex). On the right the manifold property is violated. Bottom: two 3-simplices that share a common face (no dangling), that share an edge (dangling) and that share a vertex (dangling)}
	\label{fig::2_3_dangling}
\end{figure}
Detecting dangling $k$-simplices in a $k$-complex needs slightly more work. Formally dangling was prevented by asking that at any vertex, the incident $k$-simplices form either a ball or a half ball. This is equivalent to asking that any two $k$ simplices neighbouring some vertex $v_0$ are connected via $k$ simplices neighbouring $v_0$ that share $k-1$ faces, as depicted in Figure \ref{fig::2_3_dangling}.  So to check wellformedness at a vertex, take all neighbour $k$-simplices, choose one and put it on a stack. Pop the stack and push all $k$-simplices that share a $k-1$ face with the popped simplex. Keep on popping and pushing like this. If any $k$ simplex remains you have a dangling situation. \note{(Clear enough?)}
	
	
Note that these non-manifold detection algorithms will not detect non-manifoldness due to self intersections (like the most right situation in Fig. \ref{fig::2_2_dangling}), as such problems do not show in the incidence matrices. But for most applications it is enough that we have well-formed incidence matrices and do not care if such self intersections occur.
	
\subsubsection{Find Borders}
Given an oriented discrete $k$-manifold: find the $(k-1)$-complex that represents its border. Finding the border can be easily done by applying the border Operator $\delta_k$ to the $(1,1,1,1,1...)$ vector. The resulting vector then exactly represents the border manifold. This also tests the correctness of your border matrix and the correctness of the orientation of your discrete $k$-manifold.
	
\subsubsection{Get Connected Components of a set of K-Simplices}
It also comes in handy to be able to identify the different connected components of a mesh, as the different components a mesh often need to be treated independently from each other as separate objects.
	
Given a list of $k$-simplices which forms one or more discrete manifolds you should identify the different connected components. You can also apply this
to the border of a $k$ manifold with multiple borders to get the various border components as in Figure \ref{fig::2_3_bunnyBorder}.
	
\begin{figure}[t]
	\begin{center}
	\includegraphics[width=7cm]{imgs/2_3_bunny_borders.eps}
	\end{center}
	\caption{A mesh with multiple border components.}
	\label{fig::2_3_bunnyBorder}
\end{figure}

	

\input{Chapters/TwoDimensionalSurfaces.tex}

\section{Differential Structure and Differential Calculus on Manifolds}

What we have seen up to here actually amounts to defining manifolds and providing them a differential structure. This means that manifolds become more than geometric objects that are measured and described. They become spaces on which you can define functions for which derivatives and other operations can be calculated directly ON the manifold. While all operations are defined by mapping everything back to $\mathbb R^n$, this differential structure exists on its own right.

The differential structure is 'glued' on a manifold using maps, as seen in the subsection Derivatives on Surfaces. Differentiability is then defined in the following way:

\begin{definition}[Differentiability] If $M^k$ and $N^l$ are two manifolds and $f: M^k \rightarrow N^l \subset \mathbb R^n$ is a continuous mapping we call $f$ differentiable if for every map $h:\mathbb R^k \rightarrow M^k$ the map $f \circ h :  \mathbb R^k \rightarrow \mathbb R^n$ is differentiable.
\end{definition}

As seen the differential of a function at some point is a linear mapping between two tangential spaces.
\[D_pf: T_p M^k \rightarrow T_{f(p)} N^l\]
A slightly more abstract way of viewing this is looking at $T M^k$ : the space of all tangential spaces, called the tangential bundle. Given a map $h:\mathbb R^k \rightarrow M^k$ we can get a local map of the tangential bundle 
\[h_* : \mathbb R^k \times \mathbb R^k \rightarrow T M^k\]
\[h_*(x,v) = (h(x), Dh\, v)\]
suggesting that the tangential bundle actually is $2k$-manifold. The differential of a function $f: M^k \rightarrow N^l$ then is a mapping between the two tangential bundles $T M^k$ and $T N^l$ which are a $2k$ and a $2l$ manifold.

For our first steps of differential calculus on manifolds we also need vector fields (which we will later generalize to Forms).  A vectorfield $\mathcal V$ on a manifold $M^k$ is simply the assignment of a vector from the tangential space to every point of $M^k$.
\begin{definition}[Vector Field]
A vector field is an assignment $\mathcal V: M^k \rightarrow T M^k$ with $\mathcal V(x) \in T_x M^k$.
\end{definition}
Using that the tangential bundle actually is a manifold with a differential structure we can ask from a vector field that it is smooth or differentiable.

We can then define a vectorfield in a coordinate/ map dependent way. We do this with a simple example taken from [Thomas Friedrich, Global Analysis].

\subsubsection{An example Vector Field} Taking $\mathbb R^2$ as a manifold parametrized by the identity map $\phi(u,v) = ID$, i.e. para-metrized with euclidean coordinates. The tangential spaces get the bases $\frac{\partial \phi}{\partial u}$ , $\frac{\partial \phi}{\partial v}$ which form simply the standard basis $(1,0), (0,1)$ at any point. We can use the (for now a bit alianating) notation $  \frac{\partial}{\partial u}$, $\frac{\partial}{\partial v}$ for the two bases vectors, dropping the $\phi$. We can the define a vector field
\[\mathcal V(u,v) = u \frac{\partial }{\partial v} - v \frac{\partial }{\partial u}.\]
Note that $\frac{\partial }{\partial v}$ really only is a strange notation for $\frac{\partial \phi}{\partial v} = (0,1)$ and the same for $\frac{\partial }{\partial u}$. We can now try to express the vector field $\mathcal V$ in a different map; in polar coordinates:
\[h(r,\omega) = (r \cos (\omega), r \sin (\omega))\]
The base of an arbitrary tangential space induced by this map is then
\[\frac{\partial}{\partial r} = (\cos(\omega), \sin(\omega)) \]
\[\frac{\partial}{\partial \omega} = (-r \sin(\omega), r\cos(\omega))\]
again using the fancy notation $\frac{\partial}{\partial r}$ to denote a vector. Expressed in euclidean coordinates given by $\phi$ these two vectors are
\[\frac{\partial}{\partial r} = \frac{1}{\abs{(u,v)}} ( u \frac{\partial }{\partial u} + v \frac{\partial }{\partial v} ) \]
\[\frac{\partial}{\partial \omega} = u \frac{\partial }{\partial v} - v \frac{\partial }{\partial u}\]
such that $\mathcal V$ expressed in polar coordinates is simply
\[\mathcal V = \frac{\partial}{\partial \omega}\]

\note{Image}

\subsection{Derivatives, Vectorfields and Differential Operators}
On manifolds you can calculate derivatives relative to a vector field. This is simply the directional derivative relative to the vector field's direction. Given a function $f$ we denote the derivative with respect to the vector field $\mathcal V$ as $\mathcal V (f)$. Geometrically we already did that, if $\alpha$ is a curve with $\alpha(0) = p$ and $\alpha'(0)= v = \mathcal V (p)$, then
\[\mathcal V (f) (p) = \frac{\partial }{\partial t} f \circ \alpha(t)\]
Now if $\mathcal V(u)$ is written in some map $\phi$ as $\sum_i v_i(u) \frac{\partial}{\partial u_i}$, again using the $\frac{\partial}{\partial u_i}$ as fancy notations of the induced base vectors the derivative with respect to the vector field becomes
\[\mathcal V (f) = \sum_i v_i(u)\frac{\partial(f \circ \phi)}{\partial u_i}\]
which motivates the 'strange' vector notation. \note{further reasoning needed?}

\subsection{Riemannian Metric}
From standard calculus we are used to that the gradient of a function $f:\mathbb R^k \rightarrow \mathbb R$ is a vector field with vectors pointing in the direction where $f$ has the largest increase. But in fact the gradient of a function $f$ is in fact only a linear mapping that approximates $f$ via $f(x + h) \approx f(x) + grad_x(f) (h)$ and the vector is merely a representation of the gradient. What actually happens here is that we represent the gradient using a vector AND a scalar product:
\[grad_x(f) (h) = \langle grad, h\rangle\]

To do the same on manifolds $M^k$ we need a scalar product on all tangential spaces. As we consider manifolds as objected embedded in a higher dimensional space $M^k \subset \mathbb R^n$ all tangential spaces are subspaces of the embedding space such that they inherit a scalar product. So if $\phi: \mathbb R^k \rightarrow M^k \subset \mathbb R^n$ is a local map inducing the local bases $\frac{\partial \phi}{\partial u_i}$, $i= 1...k$ to the tangential spaces and we have two vectors $v, w$ in some tangential space $T_p M^k$ expressed in the local bases as
\[v= v_1 \frac{\partial \phi}{\partial u_1} +...+ v_k\frac{\partial \phi}{\partial u_k} \]
\[w = w_1 \frac{\partial \phi}{\partial u_1} +...+ w_k\frac{\partial \phi}{\partial u_k}\]
the scalar product induced by the embedding space is
\[\langle v,w \rangle = \sum_{i,j = 1}^k v_iw_j\langle \frac{\partial \phi}{\partial u_i},\frac{\partial \phi}{\partial u_j}\rangle.\]
So if $v = (v_1,...,v_k)$ and $w = (w_1,...,w_k)$ in the map induced base, the induced scalar product is represented by the matrix
\[G= \begin{pmatrix}\langle \frac{\partial \phi}{\partial u_1},\frac{\partial \phi}{\partial u_1}\rangle &\cdots& \langle \frac{\partial \phi}{\partial u_1},\frac{\partial \phi}{\partial u_k}\rangle \\
\vdots &&\vdots\\
\langle \frac{\partial \phi}{\partial u_k},\frac{\partial \phi}{\partial u_1}\rangle &\cdots& \langle \frac{\partial \phi}{\partial u_k},\frac{\partial \phi}{\partial u_k}\rangle \end{pmatrix} = (D\phi)^T D\phi\]
This set of scalar products that is consistently defined for all tangential spaces $T_p M^k$ is the so called Riemannian metric.

Equipped with the Riemannian metric we can define a gradient and a gradient vector field for functions as well as divergience and a laplacian. \note{We will generalize these later but it is worth seeing them once for themselves before the generalization. TODO}

Note the Riemannian metric can also be used to measure volumes and angles. Angles are quite obvious; if we have two curves $\alpha$ and $\beta$ intersecting at some point $p$ with tangents $v$ and $w$, the angle between the curves is $\langle v,w\rangle$. The volume comes from the fact that $det(A A^T)$ equals the volume spanned by $A$'s row vectors squared. The determinant $det(G)$ then is the volume spanned by the column vectors of $D\phi$ squared and the $k$-dimensional volume of some subset $\phi(U) \subset M^k$ covered by a map $\phi$ is
\[vol(\phi(U))= \int_U \sqrt{det(G)}\;du_1,...,du_k\]
This measures 'absolute' orientation independent volume.






%\newpage
\chapter{Differential Forms}
	In this Chapter we introduce forms and differential forms. Differential forms are mathematical objects that allow to treat many things in a unified way such as real valued functions, vector fields. Differential forms can be integrated (they are predestined for integration) and differentiated. The point why we are interested in differential forms is that the differential operator $d$ unifies many differential operators, like divergence, gradient and curl. It will unmask common relations of the operators and also their geometry, which is given by Stokes Theorem
	\[\int_{\delta \Omega} \omega = \int_{\Omega} d\omega\]
and directly connects the differential operator $d$ to the border operator $\delta$. In the end it will be this relation that is exploited to define various differential operators on simplicial complexes and meshes and by doing so preserve many important properties of these differential operators.

We will start this chapter with a motivation for differential forms, then we will capture them a bit more formally, introducing a few operations for them. In the third section we will relate them to standard calculus objects like vector fields and end this chapter by having a quick look at integration and discrete differential forms for simplicial complexes.
\begin{table}[h]
	\begin{longtable}{|p{4.5cm}|p{4.5cm}|p{4.5cm}|}
		\hline
		Smooth Theory& Discrete Theory& Implementation (Notes)\\
		\hline
			Differential Forms: \begin{itemize}
			  \setlength{\itemsep}{1pt}
			  \setlength{\parskip}{0pt}
				\setlength{\parsep}{0pt}
				\item[-]Diff Form motivation
				\item[-]Forms (multilinear mappings) and the dimension of k Form space 
				\item[-]Differential Forms 
				\item[-]Riemann Integral of Diff Forms 
				\item[-]Interpretation of Diff Forms in $\mathbb R^3$ 
			\end{itemize}
			&
			\begin{packed_enum}
				\item[-] Discrete Form
				\item[-] Sampling Forms
			\end{packed_enum}
			 & - none
			 \\		
		\hline
	\end{longtable}
	\end{table}
	
\section{Motivation: The perfect thing to integrate}

\note{Have a look at what you need when taking an integral: an either symmetric or antisymmetric form taking
vectors from the tangential space}

The goal of this section is to get an intuition of what a differential form is. Differential forms arise very naturally when considering integrals. 

Suppose we have a 2D surface $M$ and a function $f$ defined on the surface and want to integrate $f$ over $M$ i.e. calculate the Riemann integral
\[\int_{M} f dA\]
To calculate the integral we can do the usual: we choose a grid, sum up the areas of the rectangles weighted by the  function value $f(s)$ somewhere in the parallelogram $s$ and take the limit under grid refinement
\[\lim_{diam(s\in grid)\rightarrow 0} \sum_{s \in grid} f(s)\cdot area(s) \]

\begin{figure}
\begin{center}
\includegraphics[width = 14cm]{imgs/5_1_riemann.eps}
\end{center}
\vspace{-1cm}
\caption{To calculate the Riemann integral over a surface we select a grid and refine it \note{Too Blabla?}}
\end{figure}

So what we actually need is a way to assign values to areas $s$- any 'function' that does this can be integrated easily. Say $\omega$ does just this, then
\[\int_M \omega = \lim_{diam(s\in grid)\rightarrow 0} \sum_{s \in grid} \omega(s)\]
Obviously $\omega$ has to follow some rules to be useful for integration. For one it should scale with the area of $s$. Assume a grid segment $s$ is spanned by two vectors $a_s$,$b_s$, then we can write $\omega(s)$ as $\omega(a_s,b_s)$ \note{image}. For $\omega$ to scale properly with the area of $s$ we need it to be linear in both $a_s$ and $b_s$
\[\omega(\lambda a_s, b_s) =\lambda \omega(a_s,b_s)\]
\[\omega(a_s , \lambda b_s) =\lambda \omega(a_s,b_s)\]
Furthermore, $\omega$ should behave well when the parameters are swapped:
\[\omega(a_s,b_s) = ? \omega(b_s,a_s)\]
There are two possibilities- we could chose ? to be 1 which makes sense if we want $\omega$ to depend solely on the absolute area of $s$. Or we can choose ? to be -1 and $\omega$ respects the orientation of $s$. We will chose the second variant where $\omega$ then is a \emph{differential form}. The first variant leads to so called \emph{pseudo forms}. Respecting the orientation of underlying surfaces will be crucial for many results.

One last note on $a_s$ and $b_s$. If you look at the grid in Figure \note{...} you see that they nearly lie in the tangential spaces of the surface $M$, anyway they do in the limit. Therefore some $\omega$ is very well suited for integration over a surface if, at any point $p$ on the surface $M$ it provides a function $\omega_p$ that takes 2 vectors from the tangential space $T_pM$, is linear in both arguments and antisymmetric. I.e. for all $p$
\[\omega_p: T_p M \to \mathbb R\]
\[\omega_p \text{ bilinear and antisymmetric } \forall p \in M\]
Obviously $\omega$ should also  change smoothly between neighbouring $p$'s; we will ensure this later.


\section{Forms and Differential Forms}
\note{Concretize The space of multilinear antisymmetric functions: bases, dimension, wedge. Define Differentialforms}
More formally: a $k$-form on $\mathbb R^l$ is a multi linear asymmetric mapping $\mathbb R^l \times ... \times \mathbb R^l \to \mathbb R$ which depends on $k$ vectors from the $\mathbb R^l$. 
\begin{definition}[$k$ Form]
 A $k$-Form (not a differential $k$ differential form, mind you) on a $l$ dimensional space $\mathbb R^l$ is a mapping $\omega : \mathbb R^l \times \mathbb R^l\times \cdots \times  \mathbb R^l \rightarrow \mathbb R$ with following properties:
\begin{enumerate}
\item $\omega(x_1,...,x_k)$ is linear in all $k$ parameters, meaning that \[\omega(x_1,..,\lambda a + b,..., x_k) = \lambda \omega(x_1,..,a,..., x_k) + \omega(x_1,.., b,..., x_k)\].
\item $\omega(x_1,...,x_k)$ is skew symmetric, meaning that switching any two variables leads to a change of sign:
\[\omega(x_1,...,x_i,...,x_j,...,x_k) = - \omega(x_1,...,x_j,...,x_i,...,x_k)\]
\item These properties also mean that 
\[\omega(x_1,...,x_k) = 0 \;\text{ if }\;x_1,...,x_k \text{ are linearly dependent}\]
\end{enumerate}
\end{definition}

The third property follows from $\omega$'s antisymmetry $\omega(...,v,...,v,...) = -\omega(...,v,...,v,...)$, i.e. $\omega(...,v,...,v,...) = 0$ and its
 linearity.  
The space of all $k$-Forms on $\mathbb R^l$ is denoted by $\Lambda^k (\mathbb R^l)$ and build a vector space: if $\omega$ and $\nu$ are $k$ forms so are $\omega + \nu$ and any multiples $\lambda \omega$. A natural question is what dimension $\Lambda^k(\mathbb R^l)$ has and to find a suitable basis of this space.

\subsection{Basis and Dimension of $\Lambda^k(\mathbb R^l)$}
Given a $k$-form $\omega$ on $\mathbb R^l$ and a basis $e_1,...,e_l$ then $\omega(a_1,...,a_k)$ can be rewritten the following way. We can express the $a_j$ explicitly as sum of basis vectors
\[a_j = \sum_{i}a_j^ie_i\]
and inserting this and using linearity we get
\[\omega(a_1,...,a_l) = \omega(\sum_{i}a_1^ie_i,...,\sum_{i}a_l^ie_i)\]
\[= \sum_{i_1,...,i_k \in\{1,...,l\}}a_1^{i_1}\cdot ... \cdot a_k^{i_k} \omega(e_{i_1},...,e_{i_k})\]
We can reorder this sum such that all terms treating the same set of basis vectors are grouped together:
\[=\sum_{i_1<...<i_k}\left(\sum_{\sigma \in S^k} a_1^{i_{\sigma(1)}}\cdot ... \cdot a_k^{i_{\sigma(k)}} \omega(e_{i_{\sigma(1)}},...,e_{i_{\sigma(k)}})\right)\]
where $S^k$ is the permutation group and the inner sum goes over all orderings of basis vectors.
Because of the antisymmetry reordering $e_{i_{\sigma(1)}},...e_{i_{\sigma(k)}}$ to $e_{i_1},...,e_{i_k}$ such that $i_1<...<i_k$ affects only the sign of $\omega(e_{i_1},...,e_{i_k})$:
\[\omega(e_{i_1},...,e_{i_k}) = sgn(\sigma)\omega(e_{i_{\sigma(1)}},...e_{i_{\sigma(k)}})\]
and rewriting the sum above yields
\[\sum_{i_1<...<i_k}\left(\sum_{\sigma \in S^k} sgn(\sigma) a_1^{i_{\sigma(1)}}\cdot ... \cdot a_k^{i_{\sigma(k)}}\right) \omega(e_{i_1},...,e_{i_k})\]
\[ = \sum_{i_1<...<i_k} det_{i_1,...,i_k}(a_{i_1},...,a_{i_{k}}) \omega(e_{i_1},...,e_{i_k})\]
where $det_{i_1,...,i_k}(a_{i_1},...,a_{i_{k}})$ is a subdeterminant of the matrix formed by the vectors $a_1,...,a_k$ restricted to the lines $i_1,...,i_k$.
\[det_{i_1,...,i_k}(a_1,...,a_k) = det \begin{pmatrix}
a_1^{i_1} &a_2^{i_1} &...&a_k^{i_1} \\
\vdots & & & \vdots \\
a_1^{i_k} &a_2^{i_k} &...&a_k^{i_k} 
\end{pmatrix}\]

We can read a few things out of this. For one, the $k$-form $\omega$ is determined uniquely by the values it assumes on $k$-tuples of basis vectors $e_{i_1},...,e_{i_k}$ with $i_1 <...< i_k$. And the $k$ forms (you can easily check that they are $k$-forms)
\[det_{i_1,...,i_k}(a_{1},...,a_{k})\]
which calculate $k$-subdeterminants of the input vectors, form a basis of $\Lambda^k(\mathbb R^l)$. From this follows directly that that the dimension of the  space of $k$-forms on $\mathbb R^l$ equals the number of ordered tuples $i_1<...<i_k$ of integers $i_1,...,i_k \in \{1,...,l\} $ i.e.
\[\dim (\Lambda^k(\mathbb R^l)) = \begin{pmatrix}
l \\
k
\end{pmatrix}\] 
In particular, the space of $k$ forms on $\mathbb R^l$ with $k>l$ is 0-dimensional, which means that there are no $k>l$-forms.

\subsection{The Wedge Product}
The wedge product for forms is a way to create higher order forms out of lower order forms, for example out of a $j$ form $\omega^j$ and a $k$ form $\nu^k$ you can make a $j+k$ form $\omega^j\wedge \nu^k$. The important points to grasp in this section are that you can use the wedge product to create higher order forms, you can use the wedge product to simply describe bases of the spaces of $k$-Forms $\Lambda^k(\mathbb R^l)$ and that the wedge product is associative, distributive and has some symmetry. 

We start wedging $1$-forms. The space $\Lambda^1(\mathbb R^l)$ has dimension $l$ and is spanned by a special set of basis forms
\[de_i(a) := det_i(a) = a^i\]
i.e. the projection to the $i$th coordinate of $a$ with respect to the chosen base $e_1,...,e_l$. Note that often $x_1,...,x_l$ or $x,y,z$ or similar is chosen to denote the base of $\mathbb R^l$ or $\mathbb R^3$ and these basis functions become $dx_1,..., dx_l$ or $dx,dy,dz$. For two arbitrary 1-forms $\omega^1$, $\nu^1$ we define
\[\omega^1\wedge\nu^1(a,b):= det \begin{pmatrix}
\omega(a) & \omega(b) \\
\nu(a) & \nu(b)
\end{pmatrix}\]
and for $k$ one forms $\omega_1^1,...\omega_k^1$
\[\omega_1^1\wedge\omega_2^1 \wedge...\wedge\omega^1_k(a_1,...,a_k):= det \begin{pmatrix}
\omega_1(a_1) &  ... & \omega_1(a_k) \\
\vdots & & \vdots \\
\omega_k(a_1) &... & \omega_k(a_k)
\end{pmatrix}\]
If we apply the wedge product to the 'standard' basis 1 forms $de_1,..., de_l$ 
\begin{eqnarray*}de_{i1}\wedge de_{i2} \wedge ... \wedge de_{ik}(a_1,...,a_k) &= &det \begin{pmatrix}
de_{i1}(a_1) &  ... & de_{i1}(a_k) \\
\vdots & & \vdots \\
de_{ik}(a_1) &... & de_{ik}(a_k)
\end{pmatrix} \end{eqnarray*}
which is $det_{i_1,...,i_k}(a_1,...,a_k)$ when $i_1 <...<i_k$. These are exactly the 'standard' basis forms for the space of $k$-forms from the last section. Finally for a $k$ and an $l$ form $\omega^j$ and $\nu^k$ we define the wedge product by writing them in such a basis
\[\omega^j = \sum_{i_1<...<i_j} w_{i_1,..,i_j} de_{i_1}\wedge...\wedge de_{i_j} \]
\[\nu^k = \sum_{i_1<...<i_k} v_{i_1,..,i_k} de_{i_1}\wedge...\wedge de_{i_k} \]
and then imposing associativity and distributivity to the wedge product
\begin{eqnarray*}
\omega^j\wedge \nu^k &=& \sum_{i_1<...<i_j} w_{i_1,..,i_j} de_{i_1}\wedge...\wedge de_{i_j} \\
& & \wedge \sum_{i_1<...<i_k} v_{i_1,..,i_k} de_{i_1}\wedge...\wedge de_{i_k} 
\end{eqnarray*}

Note that introducing the wedge like this is not very clean. You could directly define the wedge product as
\[\omega^j\wedge \nu^k (v_1,...,v_{l+k} = \frac{1}{k!l!}\sum_{\sigma \in S^{k+l}} sgn(\sigma) \omega^j(v_{\sigma(1)},...,v_{sigma(j)})\nu^l(v_{\sigma(k+1)},...,v_{\sigma(k+l)}).\]
Anyway; the wedge product has the following properties as is easy to show (and is done e.g. in [Global Analysis])
\emph{
\begin{enumerate}
\item Linearity in both arguments, i.e. $(\lambda\omega_1^k + \omega_2^k)\wedge\nu^l = \lambda(\omega_1^k \wedge\nu^l) + \omega_2^k \wedge \nu^l$ and the same for $\nu^l$
\item Associativity, i.e. $ (\omega^j \wedge \nu^k) \wedge \mu^l = \omega^j (\wedge \nu^k \wedge \mu^l)$
\item Symmetry: $\omega^k\wedge \nu^l = (-1)^{kl} \nu^l \wedge \omega^k$
\end{enumerate}
}

\subsection{Differential Forms}
Now we can formulate what a differential form is, what these 'perfectly integrable' objects are. As motivated in the motivational section our objects need to assign to each point $p$ of a manifold $M$ a form $\omega_p$ defined on the tangential space $T_pM$. Remember that a local map $\phi: \mathbb U \subset R^l \to M$ induces a basis to all  points in $\phi(U)$, namely $\frac{\partial \phi}{\partial u_i}$. We can directly use this basis to define a basis for the space of one forms (and then higher order forms)  as done in the last section. Instead of  writing $d\frac{\partial \phi}{\partial u_i}$ we will write $d u_i$; the $u_i$ are rad as the coordinates given by the local map $\phi$.

\begin{definition}[Differential Form]
A $k$-differential form $\omega^k$ is a mapping that assigns a $k$-form $\omega_p \in \Lambda^k(T_pM)$ to every point $p\in M$.

Given a local map $\phi: U \rightarrow M$ all $k$-Forms $\omega_p$ with $p\in \phi(M)$ can be expressed in the by $\phi$ induced coordinates:
\[\omega_p = \sum_{i_1<...<i_k}\omega_{i_1,...,i_k}(p) du_{i_1}\wedge...\wedge du_{i_k}\]
for some realvalued functions $\omega_{i_1,...,i_k}(p)$. We then say that the differential form $\omega$ is $k$ times differentiable if ex pressed in local coordinates $\omega_{i_1,...,i_k}(p)$ are $k$ times differentiable. For simplicity sake we will always assume that $\omega$ is infinitely often differentiable, i.e. smooth.

\end{definition}

So this is a differential form. We will see examples an relate differential forms to more common things like vector fields in the next section. Note that any operation defined for forms can point-wisely be defined for differential forms. For example the wedge product $\wedge$ for two differential forms $\omega^k$ and $\nu^l$ would be
\[(\omega^k\wedge \nu^l)_p := \omega^k_p\wedge \nu^l_p\]


\subsection{What Differential Forms are in $\mathbb{R}^3$}
\note{Using a metric (or simply by saying we look only at embedded manifolds with induced metric) we can associate
Vector fields etc to Forms.}
To get a feeling for differential forms we look at them in $\mathbb R^3$ or 3 Manifolds. In $\mathbb R^3$ differential forms have very simple presentations.

\subsubsection{Differential 0 Forms}
\note{curvature img}
Yes, differential 0 forms make sense! By definition a differential 0 form assigns a 0-form, i.e. a constant, to every point $p$ on a Manifold. This means a 0 differential form is simply a function $\omega: M \to \mathbb R$ that is infinitely  often differentiable we only want to talk about smooth differential forms. Some special examples would be the spacial coordinates of points $p \in M$ e.g. the $x_i$-coordinate function $ p = (x_1,...,x_n) \mapsto x_i$ or curvature.


\subsubsection*{Differential 1 Forms}
\note{image VField}
Differential $1$-Forms are equivalent to (tangential) vector fields. A 1-Form is a linear mappings from $\omega: \mathbb R^l \rightarrow \mathbb R$. And these can be simply represented as the scalar product of some vector $\omega^{\#} \in \mathbb R ^l$ with the input vector.
\[\omega(v) = \langle \omega^{\#}, v \rangle\]
This works just as well in tangential spaces. BUT there is a but, we need a scalar product on the tangential spaces that does not depend on the choice of basis vectors of the tangential spaces, or rather 'cancels out' the choice. As we look only at manifolds embedded in a higher dimensional space $M \subset \mathbb R^n$ the most natural choice of a scalar product on the tangential spaces is the scalar product induced by the surrounding space. We have already looked at this and established that this induced scalar product for all tangential spaces is described by the Riemannian metric (see Section \note{...}). 

Note that in principle you could use a different metric i.e. different scalar products in which case $\omega$ would be described by a different vector $\omega^{\#}$. This is why the sharp operator is \emph{depending on a metric}. The operation of making a 1 Form out of a vector $w$ is usually denoted by the 'flat' operator $\flat$, i.e. $w^\flat$.

\subsubsection*{Differential 2 Forms}
If we look only at three dimensional manifolds then the space of two forms on tangential spaces $\Lambda^2(T_pM)$ has dimension \[\dim\Lambda^2(T_pM) =\begin{pmatrix}
3\\
2
\end{pmatrix} =3\] such that a differential $2$-form can again be represented as a vector field. In the $\mathbb R^3$ with the standard basis and euclidean scalar product a basis of $\Lambda^2(\mathbb R^3)$ is given by $dy \wedge dz$, $dz \wedge dx$, $dx \wedge dy$. As $dy \wedge dz (a,b) = a_yb_z -a_zb_y$ (and so on) a 2-Form $\omega^2(a,b)$ can be written as
\[\omega^2(a,b) = w_1 dy \wedge dz + w_2 dz \wedge dx + w_3 dx \wedge dy = \langle \widehat{w}, a \times b \rangle\] 
where $\widehat{w} = (w_1,w_2,w_3)$ is a vector that can be used to represent $\omega^2$. This means if we want a vector $\widehat{w} \in \mathbb R^3$ to act like a two form on two input vectors $a,b$ in $\mathbb R^3$, it amounts to taking the scalar product of $\omega^\sharp$ and a vector normal to $a,b$ scaled by the area spanned by $a,b$.

If we are in some tangential space $T_pM$ with some (usually the induced) scalar product we can do the same. The representing vector is consistent under change of basis \note{i think, as this amounts to $(\star \omega)^\sharp$ } but again depends on the chosen scalar product /Riemannian metric.


\subsubsection*{Differential 3 Forms: Volume Forms}
In general there is a special differential $l$-form on an l dimensional manifold $M^l$: the 'volume form'. The volume on some space $V$ is sometimes denoted as $dV$ or $dVol$. As the dimension of the space of $l$ forms on $T_pM^l$ is
\[\dim(\Lambda^l(\mathbb R^l)) = 1\]
and every $l$-form is simply a multiple of the volume form $dV$. The volume form measures the signed (orientation dependent) volume spanned by the input vectors. In $\mathbb R^l$ with the standard basis $e_1,...,e_l$ and the euclidean scalar product this is exactly the determinant:
\[d\mathbb R^l (v_1,..., v_l) = det(v_1,...,v_l) = de_1\wedge ...\wedge d e_l\]
In $T_p M$ with an arbitrary basis and the induced scalar product, the notion of volume has to be treated a bit more carefully as we want it to be consistent under basis changes. The bottom line is that you need to chose a basis $e_1,...,e_n$ that is orthonormal relative to the chosen scalar product such that the volume form is given by
\[dT_pM = de_1 \wedge ... \wedge de_n\]

If your l dimensional manifold $M\subset \mathbb R^n$ has the induced  euclidean metric / scalar product and $\phi$ is a map assigning local coordinates $u_1,...,u_l$ to some tangential spaces $T_pM$
\[c \cdot du_1\wedge...\wedge du_l = dT_pM\]
has to be a multiple of the volume form; The volume spanned by $\frac{\partial \phi}{\partial u_i}$ is (as already seen) $\sqrt{det((D\phi)^TD\phi)}$ such that
\[c = \frac{dT_pM(...\frac{\partial \phi}{\partial u_i}...)}{du_1\wedge...\wedge du_l (...\frac{\partial \phi}{\partial u_i}...)} = \frac{\sqrt{det((D\phi)^TD\phi)}}{det((D\phi)^TD\phi)}\]
and therefore the volume form in local coordinates is
\[dT_pM =\frac{1}{\sqrt{det((D\phi)^TD\phi)}}  du_1\wedge...\wedge du_l\]
\note{Is this correct?NO  TODO}

Anyway as any $l$ form on an $l$ dimensional space is the multiple of the volume form any $l$ differential form can be represented as a real valued function $f:M \rightarrow \mathbb R$ and the related $l$ form $\omega^l$ is then
\[\omega^l = f \cdot dVol\]
Note that again, as the notion of volume is related to the chosen scalar product/ metric, this representation is also depending on the metric. This is also why the set of scalar products is called a riemannian metric as it is directly related to how volumes are measured. Note that even thought  in these sections there is always an emphasis that the riemannian metric can be chosen, as we work with embedded manifolds $M \subset \mathbb R^n$ we will always chose the metric induced by the euclidean metric on the embedding space. \note{Is this needed?}

\subsubsection{Differential $n$-Forms on $\mathbb R^n$}
The one-dimensionality of the space of $n$-forms on $\mathbb R^n$ can also be used to get a very simple description of what happens to a $n$ differential form under a base change.
If $A$ is a linear map $\mathbb R^n \to \mathbb R^n$ 
\[det(Av_1,Av_2,...,Av_n) = det(A)\cdot det(v_1,...,v_n)\]
and therefore if $\omega^n$ is a $n$ differential form on $\mathbb R^n$
\[\omega^n(Av_1,...,Av_n) = det(A)\cdot \omega^n(v_1,...,v_n)\]
as any $n$ form is simply a multiple of the volume form $\omega^n = c \cdot det$. We can play around with this a bit more such that for fixed $v_1,...,v_n$ building a matrix $V$ we get
\[\omega^n(Av_1,...,Av_n) = det(A)\cdot \omega^n(v_1,...,v_n)= det(V) \cdot \omega^n(a_1,...,a_n)\]

\subsubsection*{Summary}
Images or tables that summarise what differential forms are on 2 dimensional and 3 dimensional manifolds. \note{Todo, as image, consistent with later images...}

%\begin{table}[h]
%\begin{longtable}{lccc}
%& & 2-Manifolds &\\
%Forms & $\Lambda^0(T_pM)$ & $\Lambda^1(T_pM)$ & $\Lambda^2(T_pM)$\\
%Dimension & 1 & 2 & 1 \\
%Differential Form  & $f:M\rightarrow \mathbb R$ & $\omega^{\sharp}: M \rightarrow TM$  & $f:M\rightarrow \mathbb R$ \\
% & $\omega_p =f(p)$ & $\omega_p(v) = \langle \omega^{\sharp} ,v\rangle$ & $\omega_p(v_1,v_2)=   f(p) \cdot dArea(v_1,v_2)$
%\end{longtable}
%\end{table}
%
%\begin{table}[h]
%\begin{longtable}{lcccccr}
% Forms & $\dim$ & Diff. Forms & Representation & & \\
% $\Lambda^0(T_pM)$ & 1 & 0-Forms & Function & $f:M\rightarrow \mathbb R$ \\%& $\omega_p =f(p)$ \\
% $\Lambda^1(T_pM)$ & 3 & 1-Forms & VField & $\omega^{\sharp}: M \rightarrow TM$  \\%& $\omega_p(v) = \langle \omega^{\sharp} ,v\rangle$\\
% $\Lambda^2(T_pM)$ & 3 & 2-Forms & VField & $\widehat{w}: M \rightarrow TM$ \\%& $\omega_p(v_1,v_2) = \langle \widehat{w} ,v_1\times v_2\rangle$ \\
% $\Lambda^3(T_pM)$ & 1 & 3-Forms & Function & $f:M\rightarrow \mathbb R$ \\%& $\omega_p(v_1,v_2, v_3)=f(p) \cdot dVol( v_1,v_2,v_3)$
%\end{longtable}
%\end{table}

\section{Integrating Forms}
\note{Now again lets have a look at integration of Diffforms.} We started this chapter saying that we wanted to design objects that are perfect to be integrated. We ended up with differential forms that in 3-dimensional spaces turn out to be either vector fields or functions. Let's now look in detail at how differential forms are integrated. 

A $k$ form can be integrated over $k$-dimensional regions. There are a couple of technicalities like that regions on a manifold can not generally be covered by one map but these are omitted here, for a clean introduction of the integral see e.g. \note{[global analysis]}; for an even less clean introduction (but quite understandable) see \note{[geometric diffforms]}. If $\phi : U \subset \mathbb R^k \to M$ is a map, $M$ a $k$-dimensional manifold and $\omega^k$ a differential $k$-form on $M$ and a region $\Omega = \phi(U)$ parametrized with $\phi$. The integral 
\[\int_{\phi(U)} \omega^k \]
is defined and calculated by pulling everything back to $\mathbb R^l$:
\[\int_{\phi(U)} \omega^k = \int_{U\subset\mathbb R^k} \omega_{\phi(x_1,...,x_k)}(\frac{\partial \phi}{\partial x_1},...,\frac{\partial \phi}{\partial x_k}) d x_1...d x_k\]
Here there are some things to note: the the thing on the integral on the right is simply a function depending on $k$ variables and is integrated as such things are always integrated. But more importantly: the expression on the left  lacks any '$d x_i$'s. The integral is independent of local coordinates; just as intended in the motivating example $\omega$ automatically scales according to the volume spanned by the vectors $\frac{\partial \phi}{\partial x_i}$. 

That this definition really is independent of the chosen map $\phi$ follows directly from the common transformation formula. Say we use a different map $\psi: V \rightarrow \Omega \subset M$, then $\psi = \phi \circ h$ for some mapping $h:U \subset \mathbb R^k \rightarrow V \subset \mathbb R^k$ \note {image}. 
\[\int_{\psi(V)} \omega^k = \int_V \omega_{\psi(x_1,...,x_k)}(\frac{\partial \psi}{\partial x_1},...,\frac{\partial \psi}{\partial x_k}) d x_1...d x_k\]
\[= \int_V \omega_{\psi(x_1,...,x_k)}(\frac{\partial \phi \circ h}{\partial x_1},...,\frac{\partial \phi \circ h}{\partial x_k}) d x_1...d x_k\]
\[= \int_V \omega_{\phi\circ h(x_1,...,x_k)}(D\phi \cdot Dh) d x_1...d x_k\]
\[= \int_{V} det(Dh) \omega_{\phi\circ h(x_1,...,x_k)}(\frac{\partial \phi}{\partial x_1},...,\frac{\partial \phi}{\partial x_k}) d x_1...d x_k\]
and using the transformation formula
\[= \int_U \omega_{\phi(x_1,...,x_n)} (\frac{\partial \phi}{\partial x_1},...,\frac{\partial \phi}{\partial x_k}) d x_1...d x_k \]

Note that when $\omega$ is written in some local map it becomes $\sum ... \wedge d u_i$ gets '$du_i$'s. They play in the integral an equivalent roles to the $d x_i$'s in the old well known integrals, but for an additional sign from orientation, i.e $\int du_1\wedge du_2 = - \int du_2 \wedge du_1$.
\note{will have to overwork this section, dunno if it helps much...}
\note{Question: Do I really need to say how integration works in Detail? it's not really needed anywhere.... Emphasis that integrqals make sense only over the right dimension and a $j$ form on a $k$ dimensional manifold can be integrated over any $j$ dimensional submanifold. Example a one form can be integrated over any curve}

\note{Functions are somewhat special... as they can be interpreted as a $k$ form on any (sub) manifold it can be integrated over anything}

\subsection{Pull-Back}
\note{Pulling back differential forms from one manifold to another}
This 'Pulling back' used to define an integral can be done more generally. If we have a mapping between two manifolds $N$ and $M$ $h: N\to M$ (for simplicity say smooth and with $det(Dh) \neq 0$), then, as seen before $Dh$ is a mapping between the tangential spaces of $N$ and $M$. Therefore if we have a differential $k$-Form $\omega$ on $M$ we can 'pull it back' to $N$ via
\[(h^*\omega)_p (v_1,...,v_k) := \omega_{h(p)}(Dh v_1,...,Dh v_k) \]
\note{image} where the action of pulling back $\omega$ using $h$ is denoted by $h^* \omega$. This mapping preserves the integral (check it by using the definitions!)
\[\int_{h(U)\subset M} \omega = \int_{U \subset N} h^*\omega \]
This means that we can integrate either $\omega$ over a subset of $M$ or the pulled back mapping over a subset of $N$. Usually, as in the definition of the integral, you pull back forms to $\mathbb R^k$.


\section{Discrete Forms}
\note{What discrete forms are and where they live}
As we did earlier, we will introduce the discrete analogue of the continuous object at once. Differential forms are defined on Manifolds (and their tangential spaces). Just as it is to be expected discrete forms are defined on discrete manifolds. But we do not have tangential spaces on discrete manifolds. Therefore instead of (multi-) linear mappings a discrete differential form will simply be a set of averaged values. Thought this seems like a huge downgrade, with time we will see that this simple representation is still very powerful.

A discrete form is simply the following:
\begin{definition}[Discrete Form]
A discrete $j$ form on a discrete $k$ manifold assigns a real number to every $j$ simplex contained in the discrete manifold. \note{image} This vector of values is also sometimes called a $j$ co-chain. \note{it is, isn't it?}
\end{definition}
The question to be answered now is how this set of values relates to a non-discrete differential form.

\subsection{Sampling Forms}
\note{How you sample vector fields etc / what the sampled values mean. Here the duality of forms already emerges.}

First of all to relate discrete forms with differential forms the discrete manifold $K$ needs to be related somehow to a non-discrete manifold $M$. We will just assume that the discrete manifold $K$ approximates the manifold $M$ and for any simplex $\sigma \in K$ there is a continuous analogue on $M$ as in Figure \note{img}. We denote both the discrete and the continuous counterpart by the same symbol.

The relation then is very simple: given a $k$-form $\omega^k$ on $M$ the value of the discrete $k$ form $\textbf{w}$ on a $k$-simplex $\sigma$ is

 \[\textbf{w}(\sigma) = \int_\sigma \omega^k\]

To make the meaning of discrete forms completely clear we look at some examples in $\mathbb R^2$ and $\mathbb R^3$. In practice we will never sample any differential forms, these examples are only here to help you understand what the discrete forms actually represent.

\subsubsection{Sampling 0 Forms}
\note{img from talk}

As seen, $0$-Forms can simply be represented as functions $f: M\to \mathbb R$. A $0$ has to be 'integrated' over 0 dimensional sets, i.e. points. The discrete 0-form is therefore a set of values associated to vertices. The value at a vertex position $v$ (or differently frased: at a 0 simplex $v$) is
\[\textbf{w}^0(v) = f(v)\]
i.e. $f$ evaluated at $v$.

\subsubsection{Sampling 1 Forms}
\note{img(s) from talk in 2d: once as flux and once as flow}
A $1$-form $\omega^1$ on a manifold $M$ can be represented by a tangential vector field $\nu:M\to TM$ via 
\[\omega^1_p(v) = \langle\nu(p),v\rangle.\] 
A $1$-form can be integrated over $1$-dimensional curves. A discrete $1$-form is therefore a set of values associated to the $1$-simplices i.e. edges of the discrete manifold. The value on an edge $e$ is
\[\textbf{w}^1(e) = \int_{e} \omega^1 = \int_{0}^1 \langle\nu(e(t)),\frac{\partial}{\partial t}e(t)\rangle dt\]
where in the last integral (in a very dirty notation) $e(t)$ is thought of as some parametrization of the curve on the manifold $M$ associated to the edge $e$. This means a discrete $1$ -form samples (and represents) a vectorfield by projecting the field on the edge and 'summing' these values up along the edge. This value can be thought of as measuring how much the vectorfield 'flows' along the edge. If the edge $e$ is a straight line and the vectorfield a constant vector  $\textbf{w}^1(e)$ is simply the projection of the vector to the edge
\[\textbf{w}^1(e) = \langle \nu, e \rangle.\]

Note that on a $2D$ manifold there is a second way of how to sample the vectorfield by measuring the flow THROUGH the edge instead of ALONG the edge (we will come back to this in a while when talking about the Hodge star $\star$ operator and 'duality' in section \note{[...]}). \note{Image}
\[\textbf{w}^1(e) = \int_{0}^1 \langle\nu(e(t)),\left(\frac{\partial}{\partial t}e(t)\right)^\perp \rangle dt\]
Here $^\perp$ denotes the vector rotated by $90^\circ$ according the orientation of the surface.

\subsubsection{Sampling 2 Forms}
A $2$-form can be integrated over $2$D patches and the discrete 2 form associates values to the 2-simplices i.e. the triangles of the discrete manifold.

On a 2D manifold: Here a differential $2$-form $\omega^2$ again is represented by a function $f$ and the value on a triangle $t$ and the value $\textbf{w}^2(t)$ is simply the integral of $f$ over $t$.
\[\textbf{w}^2(t)= \int_{t} f\, dVol\]
Note that a $3$-Form on a 3 manifold, or generally a $n$-Form on an $n$ manifold is represented as a function $f$ and is sampled in the same way.

On a 3D manifold: Here a differential $2$-Form $\omega^2$ is again represented as a vectorfield $\nu: M\to TM$, but evaluating it on two vectors amounts to
\[\omega_p(a,b) = \langle \nu(p) , a \times b \rangle\]
such that the value of the discrete $2$-Form associated to a triangle $t$ is
\[\textbf{w}^2(t)= \int_{t} \langle \nu(p), n(p) \rangle \, dp\]
where $n(p)$ denotes the normal on the surface $t$ at the point $p$ (according to its orientation). This measures the flow of the vectorfield THROUGH the surface $t$. \note{image}

\subsubsection{Some Observations}
\note{is this needed?? Is this the right place? Should Duality be introduced here for real? Decide this later} This is the right place to make some observations. As we have seen on 3 manifolds vectorfields can either be interpreted as $1$ OR as $2$ forms, which decides on how they are integrated. The same is true for functions $f$ that can be interpreted as $0$ forms or $3$ forms, which again decides on how to integrate them. 

On 2D surfaces $0$ and $2$ forms are represented the same way and for $1$-forms there were 2 ways to integrate them. There is a principle behind this: on an $n$-dimensional manifold there is a strong relation between differential $(n-k)$-forms and differential $k$-forms. We can make a $(n-k)$-form out of a $k$-form and vice versa. This is mirrored in the representation of differential forms above: a $1$-form can be interpreted as a $(3-1) =2$ form in an $(n =3)$D spaces and so on. The 'related' form will be called 'dual' form and you will get from the $k$ form $\omega^k$ to its dual $\nu^{n-k}$ using the so called Hodge operator $\star$:
\[\nu^{k} = \star \omega^{n-k}\]
In a 2D setting the dual of a $1$-forms is again a $1$-form, which explains (or at least motivates) why we gave two ways to sample $1$ forms.

%Functions $f$ are special anyway as, restricted to any $k$-manifold they can be interpreted as a $k$-form and integrated over it.

\subsection{Integrating Discrete Forms}
\note{Its just a scalar product! Will be a very short section}

A discrete form can very easily be integrated over a set of simplices. Integrating a discrete $k$-Form $\textbf{w}^k$ over a set of $k$-simplices $\{\sigma_1,...,\sigma_l\}$ can be done simply by summing up the values on those simplices. If $\textbf{w}^k$ is the sampled version of $\omega^k$ this sum is exactly the integral of $\omega^k$  over the k dimensional set $\{\sigma_1,...,\sigma_l\}$
\[\int_{\{\sigma_1,...,\sigma_l\}} \omega^k = \sum_{i=1}^l \textbf{w}^k(\sigma_i)\]
as
\[\textbf{w}^k(\sigma_i) = \int_{\sigma_i} \omega^k\]
As we have seen in section \note{...} we can describe a set of simplices in a discrete manifold as a vector $\sigma$ of dimension $\# k-simplices$ consisting of plus/minus ones and zeros when the simplices have a fixed enumeration. $\textbf{w}^k$ is a vector as well and the discrete integral is just the scalar product of those two vectors
$$\langle \sigma , \textbf{w}^k \rangle$$

\subsection{(?) Interpolating discrete differential Forms and the $^\#$ }
\note{Either in this chapter or later: how to interpolate discrete forms and how to gain vectors from 1 forms. write this/ decide this later}

%\newpage		
\chapter{Exterior Calculus \& Discrete Exterior Calculus}
\label{chap:EC}

This chapter constitutes the core of the theory of this thesis. We introduce the exterior derivative, describe its geometry and use its geometry to build discrete exterior calculus on discrete manifolds. Yet, the main argument of this chapter can be summed up in the following three lines:
\begin{equation} \int_{\delta \Omega} \omega = \int_{\Omega} d \omega \label{eq:stokes}\end{equation}
therefore
\[d_{discrete} := \delta_{discrete}^T.\]
These three lines are rather compact and use most of the concepts introduced up to now. While manifolds ($\Omega$), the border operators ($\delta, \delta_{discrete}$), forms ($\omega$) and their integrals have already been treated, the discrete exterior derivative $d$ has yet to be introduced. The three line argument above says, that according to Stokes theorem (Equation \ref{eq:stokes}), the discrete exterior derivative should be defined as the transposed of the discrete border operator. 
Discretizing the differential operator $d$ is interesting, because $d$ generalizes various operators like the divergence and curl. By discretizing $d$ we directly get discrete versions of these operators. 

This chapter begins with an introduction of the exterior derivative in Section \ref{sec:EC_exteriorDerivative} and an explanation of Stokes theorem and the geometry of $d$ in Section \ref{sec:EC_stokes}. Then follows a description of
how $d$ is discretized. In the fourth section, Section \ref{sec:hodgeStar}, we introduce the Hodge star operator $\star$ and its discrete version, the last missing elements needed to formulate various operators known from classical calculus.

%\begin{longtable}{|p{4.5cm}|p{4.5cm}|p{4.5cm}|}
%\hline
%Smooth Theory& Discrete Theory& Implementation (Notes)\\
%\hline
%	External Calculus
%	\begin{packed_enum}
%		\item[-] Gradient, Curl and Divergence
%		\item[-] d
%		\item[-] Stokes Theorem
%		\item[-] Star and DeRham Complex
%	\end{packed_enum}
%	&
%	Discrete External Calculus
%	\begin{packed_enum}
%		\item[-] Discrete d
%		\item[-] Dual Mesh
%		\item[-] Show also intuitive match to curl etc
%	\end{packed_enum}
%	 & 
%	 A look at the Laplacian from chapter 2
%	 \begin{packed_enum}
%		\item[-] The DEC matrices (and tests)
%	\end{packed_enum}
%	 \\		
%\hline
%\end{longtable}

\section{The Exterior Derivative $d$}
\label{sec:EC_exteriorDerivative}
The exterior derivative $d$ is a generalization of the usual derivative. But it is better to look at $d$ as something completely new and unknown, because thinking of it as the 'derivative for differential forms' can lead to wrong associations and expectations. For example applying the differential operator multiple times to get an $n$th derivative does not make sense. But the exterior derivative generalizes the usual derivative in the sense that the exterior derivative is the counterpart to the integral, by Stokes theorem.
%An example of the exterior derivative is the gradient. The gradient takes a function $f:\mathbb R^n\to \mathbb R$ and returns a vector field $grad(f)$. In terms of forms the gradient takes a $0$-form $f$ and returns a $1$-form $grad(f)$. This is the general behavior of the differential operator $d$: it takes $k$-forms and returns $k+1$-forms.

The exterior derivative allows a rich theory and is of great practical relevance because  many differential operators from classical calculus can be expressed using it. We give examples in Section \ref{subsec:EDeuclidean} and also coordinate free examples in Section \ref{subsec:geometryCDO}.

\subsection{Definition} 
The differential operator $d$ is easy to define on $\mathbb R^n$, its relevance is not obvious on first sight, but is explained in the following sections.
\begin{definition}[Exterior Derivative on $\mathbb R^n$]
The exterior derivative $d$ maps differential $k$-forms to differential $k+1$-forms. If $\omega^k$ is given in standard coordinates $x_1,..,x_n$ as 
\[\omega^k_p= \sum_{i_1<...<i_k} \omega_{i_1,..,i_k}(x) dx_{i_1} \wedge ... \wedge dx_{i_k}\]
the exterior derivative is given by
given by 
\[d\omega^k = \sum_{i_1<...<i_k}\sum_{\alpha = 1}^{n}\frac{\partial \omega_{i_1,..,i_k}(x)}{\partial x_\alpha} dx_{\alpha} \wedge dx_{i_1} \wedge ... \wedge dx_{i_k}\]
\end{definition}

To define $d$ on manifolds in general we use pullbacks, as introduced in Section \ref{sec:pullbacks}, namely: if $M$ is a manifold, $h$ a local map and $\omega^k$ a $k$-form, then
\[d\omega^k := (h^*)^{-1}d(h^*\omega^k).\]
The first pullback $h^*$ transforms $\omega^k$ to a $k$-form on $\mathbb R^n$, where the exterior derivative is already defined and can be used. Then the result is pulled back to the manifold.\footnote{There are some technicalities about $(h^*)^{-1}$ that are omitted, see e.g. \cite{globalAnalysis}. Also note that this definition does not depend on the map $h$.}

\subsection{Properties}
The exterior derivative has the following properties that are more or less straight forward to check by plugging in the definitions; you can find detials e.g. in \cite{globalAnalysis}.

\begin{enumerate}
\item $d(\omega^k + \psi^k) = d\omega^k + d\psi^k$
\item $d(\omega^k \wedge \psi^l) =( d\omega^k) \wedge \psi^l + (-1)^k \omega^k \wedge(d \nu^l)$
\item $d(d\omega^k) = 0$
\item $f^*(d\omega^k) = d(f^* \omega^k)$
\end{enumerate}
 The third and fourth property are the most noteworthy.  Applying $d$ two times in a row always leads to zero (as you can check by simply writing out $dd$). And the exterior derivative commutes with pullbacks. This means that you can freely chose where and in what map you want to work and calculate derivatives; just pull everything to a space where you want to have it.


\subsection{The Exterior Derivative in Euclidean Coordinates}
\label{subsec:EDeuclidean}
Let's look at the exterior derivative on $\mathbb R^n$ with the standard basis. The differential forms are interpreted as functions and vectorfields, as described in the last chapter in Section \ref{subsec:diffformsare}.

\subsubsection{0-Forms}
If we have a $0$-form on $\mathbb R^n$  given by $f:\mathbb R^n \to \mathbb R$, then the exterior derivative is by definition the 1-form
\[df = \sum_{\alpha = 1}^n \frac{\partial f}{\partial x_\alpha} dx_\alpha .\]
Therefore, applying $df$ at any point $p$ to a vector $v$ is
\[df_p(v) = \langle \nabla f, v \rangle,\]
where $\nabla f = (\frac{\partial f}{\partial x_1},...,\frac{\partial f}{\partial x_n})$ is the gradient in euclidean coordinates. This means that $d$ acts on $0$-forms like the gradient operator.


\begin{figure}
\begin{center}
\includegraphics[height=3cm]{imgs/6_1_SCvsEC_2d.eps}
\caption{The differential forms arising on 2-manifolds and the exterior derivative (EC) and the corresponding objects and operators in standard calculus (SC).}
\label{fig::6_1_SC2d}
\end{center}

\end{figure}

%\note{Auskommentiert: kommentar ueber die notation $dx_i$ / $du_i$... Braeuchtes das...?}
%On a side note: this also motivates the notation $dx_\alpha$ for the special one forms we use as a basis of the space of $1$-forms; $x_\alpha$ is the $\alpha$th coordinate of a point $x$, i.e. short for the function $f(x) = f(x_1,...,x_n) = x_\alpha$. Therefore if we interpret $d$ as the exterior derivative 
%\[(dx_\alpha)_p(v) = \langle e_\alpha,v \rangle \]
%is exactly what we defined it \note{in sec...} to be. The same is true for an arbitrary map $\phi(u_1,...,u_k)$ that locally assigns the coordinates $u_1,..,u_k$ to a manifold $M$:
%\[du_i = (\phi^*)^{-1}(d (\phi^*)(u_i)\]
%\[=(\phi^*)^{-1}(d e_i)\]
%\[= \langle D\phi \cdot e_i, D \phi v \rangle\]
%\note{or similar}
\subsubsection{1-Forms}
A vector field 
\[\mathcal V : \mathbb R^n \to \mathbb R^n\]
\[\mathcal V(x) = (v_1(x),...,v_n(x))\]
 interpreted as a one form is
\[\omega^1_p = \sum_{i = 1}^n v_i(p) d x_i .\]
To apply the exterior derivative to $\omega^1$ amounts to
\[d\omega^1 = \sum_{i=1}^n \sum_{j = 1} ^n \frac{\partial v_i(p)}{\partial x_j} dx_j \wedge d x_i.\]
If we reorder these terms we get
\[= \sum_{1\leq i < j \leq n } (\frac{\partial v_j(p)}{\partial x_i} - \frac{\partial v_i(p)}{\partial x_j}) dx_i \wedge d x_j\]
and on $\mathbb R^3$ this is exactly the $rot$ operator: if we represent the arising 2 form as a vector we get
\[d \begin{pmatrix}
v_1(x) \\ v_2(x) \\ v_3(x)
\end{pmatrix} = \begin{pmatrix}
\frac{v_3(x)}{\partial x_2} -\frac{v_2(x)}{\partial x_3}\\
\frac{v_1(x)}{\partial x_3} -\frac{v_3(x)}{\partial x_1}\\
\frac{v_2(x)}{\partial x_1} -\frac{v_1(x)}{\partial x_2}\\
\end{pmatrix}\]
This means that $d$ applied to 1-forms can be interpreted as the curl operator.

\subsubsection{$2$-Forms on $\mathbb R^3$}
A differential 2-form can be represented as a vector field $\mathcal V = (v_1,v_2,v_3)$ as
\[\omega^2 = v_1 dx_2 \wedge dx_3 + v_2 dx_3 \wedge dx_1 + v_3 dx_1 \wedge dx_2.\]
The exterior derivative is then
\[d \omega^2 = (\frac{\partial v_1}{\partial x_1} + \frac{\partial v_2}{\partial x_2} + \frac{\partial v_3}{\partial x_3})dx_1\wedge dx_2 \wedge dx_3\]
which means that $d$ is exactly the divergence operator.

\subsubsection{Summary}
The Figures \ref{fig::6_1_SC2d} and  \ref{fig::6_1_SC3d} summarize the relation between differential forms and standard calculus on 2-manifolds and  on 3-manifolds.


\begin{figure}
\begin{center}
\includegraphics[height=3cm]{imgs/6_1_SCvsEC_3d.eps}
\end{center}
\caption{The differential forms arising on 3-manifolds and the exterior derivative (EC) and the corresponding objects and operators in standard calculus (SC)}
\label{fig::6_1_SC3d}
\end{figure}


\section{Stokes Theorem}
\label{sec:EC_stokes}

Now we finally explain Stokes theorem
\[\int_{\delta\Omega} \omega = \int_{\Omega} d \omega.\]
Stokes theorem is a generalization of and follows from the fundamental theorem of calculus: if $f:\mathbb R \to \mathbb R $ has an antiderivative $F$ i.e. $F' = f$, then
\[\int_a^b f(x) dx = F(b) - F(a).\]
The fundamental theorem of calculus can be rewritten in differential form notation: say $\Omega = [a,b]$ is an oriented line with border $\delta \Omega = -\{a\} + \{b\}$. If we identify $F$ with a 0-form $\omega^0 = F$, then $d F = F' =f$ is a 1-form. As $dF$ is a 1-form defined on a one manifold it can also be represented as a function. The fundamental theorem then becomes:
\[\int_{[a,b]} d\omega^0 = \int_{\delta [a,b]} \omega^0 = \int_{-\{a\}}\omega^0 + \int_{\{b\}} \omega^0 = -\omega^0(a) + \omega^0(b)\]
So the fundamental theorem is Stokes theorem applied to $0$-forms. Note how it is important that $\omega^0$ respects the orientation of the points it is applied to.

We will only sketch a proof for Stokes theorem. 
There are a few technical difficulties that are omitted. A clean proof can for example be found in \cite{globalAnalysis} or \cite{FRANKEL11}.

\subsection{Proof Sketch}

\begin{figure}%
\begin{center}
\includegraphics[height = 4cm]{imgs/6_singularcube.eps}%	
\end{center}
\vspace{-0.5cm}
\caption{A singular cube: a $k$-dimensional patch that can be parametrized by the cube $[0,1]^k$ using a single map.}%
\label{fig:6_singularCube}%
\end{figure}
This sketch is following strongly the reasoning made in \cite{globalAnalysis}.
It is enough to show the theorem for so called singular cubes. A $k$-dimensional singular cube $c^k$ is a manifold $C^k$ together with a global parametrisation (see Figure \ref{fig:6_singularCube})
\[c^k: [0,1]^k \to C^k \subset \mathbb R^n.\] 
We show
\[\int_{\delta C^{k+1}} \omega^{k} = \int_{C^{k+1}}d\omega^k.\] 
To extend the theorem to arbitrary manifolds consider the Figure \ref{fig::6_1_singularCubes}; the basic idea is that the theorem holds if a manifold is the union of a set of disjoint cubes $\Omega = \bigcup c^k_i$, as
\[\int_{\Omega} d\omega = \sum_i \int_{c^k_i} d\omega = \sum_i \int_{\delta c^k_i} \omega = \int_{\delta \Omega} \omega ,\]
where the second step is the Stokes theorem for singular cubes and the last step holds because internal boundaries cancel out, as motivated in Figure \ref{fig::6_1_singularCubes}.

\begin{figure}[t]
\begin{center}
\includegraphics[height=3cm]{imgs/6_1_singularCubeChain.eps}
\end{center}
\caption{Some random bordered manifold that can be made out of 3 singular cubes. Some integral $\int_O$ over this object amounts to the sum of integrals over the singular cubes. For every cube the Stokes theorem is true. So the integral $\int_O$ is given by the sum of border integrals of the cubes. But inner edges cancel out because of their opposite orientations. Therefore $\int_O d\omega= \int_{\delta O}\omega$} 
\label{fig::6_1_singularCubes}
\end{figure}

\subsubsection{Proof (sketch)}

Given a singular cube $C^{k+1}$ with parametrisation $c^{k+1}$ we can pull the whole problem back to $[0,1]^{k+1}$. If the theorem is proven for $[0,1]^{k+1}$ it also holds for singular cubes $C^{k+1}$:
\[\int_{c([0,1]^{k+1})}d \omega = \int_{[0,1]^{k+1}} c^*(d\omega) = \int_{[0,1]^{k+1}} d(c^*(\omega)) = \int_{\delta [0,1]^{k+1}} c^*(\omega) = \int_{\delta(c([0,1]^{k+1}))} \omega\] 
We write an arbitrary $k$-Form $\omega^k$ on $[0,1]^{k+1} \subset \mathbb R^{k+1}$ as 
\[\omega^k = \sum_{i=1}^{k+1} f_i dx_1 \wedge...dx_{i-1} \wedge dx_{i+1} ...\wedge dx_{k+1}\]
where in each term the $i$th $dx_i$ is omitted. Then
\[d\omega^k = \sum_{i=1}^{k+1}(-1)^{i-1}\frac{\partial f}{\partial x_i}dx^1\wedge ... \wedge dx^k\]
and
\[\int_{I^{k+1}} d \omega^k = \sum_{i=1}^{k+1}(-1)^{i-1} \int_{[0,1]^{k+1}} \frac{\partial f}{\partial x_i} dx_1 \wedge...\wedge dx_{k+1}\]


Now we can simply use the known fundamental theorem to integrate the single terms in the sum relative to $x_i$
\[\int_{0}^1 \frac{\partial f}{\partial x_i} (x_1,...,x_{i-1},t,x_{i+1},...) dt = f(x_1,...x_{i-1},1,x_{i+1},...) - f(x_1,...x_{i-1},0,x_{i+1},...)\]
getting 
\[\int_{[0,1]^{k+1}} d \omega^k = \sum_{i=1}^{k+1}(-1)^{i-1} \underbrace{\int_{0}^1...\int_{0}^1}_{\textit{k+1 integrals}} \frac{\partial f}{\partial x_i} dx_1...dx_{k+1}\]
\[= \sum_{i=1}^{k+1}(-1)^{i-1} \left(\underbrace{\int_{0}^1...\int_{0}^1}_{\textit{k integrals}} f(x_1,...,x_{i-1},1,x_{i+1}...,x_{k+1}) dx \right.\]
\[- \left. \underbrace{\int_{0}^1...\int_{0}^1}_{\textit{k integrals}} f(x_1,...,x_{i-1},0,x_{i+1}...,x_{k+1}) dx \right)\]
where the integrals used are the 'common' integrals. Every term in the last sum integrates $f$ over one side of the cube $[0,1]^{k+1}$, because plugging in a $1$ or a $0$ for one parameter and integrating over the others has exactly that effect. The factors $(-1)^{i-1}$ together with the minus from the fundamental theorem, result exactly in assigning to every term a sign matching the orientation of the respective cube side induced by the border operator applied to $[0,1]^{k+1}$, see Figure \ref{fig:6_stokesproof}. Therefore
\[\int_{[0,1]^{k+1}} d \omega^k = \int_{\delta [0,1]^{k+1}}\omega^k\]
\qed

\begin{figure}[t]%
\includegraphics[width=\columnwidth]{imgs/6_stokesproof.eps}%
\caption{On the left the orientations of the borders when computing the integrals $\int_0^1\int_0^1$ are depicted, on the right are the orientations induced by the border operator. The additional signs $(-1)^{i-1}$ for the faces $x_i=1$ and $(-1)^{i}$ for the faces $x_i=0$ lead to the induced border orientation.}%
\label{fig:6_stokesproof}%
\end{figure}

\subsection{Geometry of $d$}

Stokes theorem is more then a  valuable tool for calculations and for reformulations. It unmasks the geometry of the exterior derivative $d$. Stokes theorem binds the exterior derivative to the border operator, which is a purely geometric operation. Both operations are equivalent, as you can choose either to apply the border operator to a region or to apply the exterior derivative to the differential form at hand.
This can be made even clearer by using a bracket notation for the integral; the first argument is the manifold, the second argument is the differential form:
\[[\Omega, \omega] := \int_{\Omega} \omega\]
Then Stokes theorem can be formulated as
\[[\Omega, d\omega] = [\delta \Omega, \omega].\]
The border operator and the exterior operator play an equivalent role.

\subsection{The Geometry of Classical Differential Operators}
\label{subsec:geometryCDO}
That Stokes' theorem captures the geometry of $d$ can be highlighted by looking at the standard calculus operators $div$ and $curl$ on arbitrary 3-manifolds.  The geometric, coordinate free definition of the curl $\nabla \times v$ of a vector field $v$ is 

\[(\nabla \times v) \cdot n = \lim_{diam(A)\rightarrow 0 } \frac{\int_{\partial A} \langle v , t \rangle dx}{Area(A)} \]
where $n$ is a normal vector, $A$ is an area patch orthogonal to $n$ and $t$ is a tangent to the border of $A$, see Figure \ref{fig:6_curlDiv}. So the curl operator takes a vector field and returns a linear mapping that can be used to measure how much the vector field curls around a point. This geometric property is exactly captured by Stokes theorem applied to 1-forms, when forms are interpreted as in \ref{subsec:diffformsare}:
\[\lim_{diam(A)\rightarrow 0 }\frac{\int_{\partial A} \omega^1}{Area(A)} = \lim_{diam(A)\rightarrow 0 }\frac{\int_{A} d \omega^1}{Area(A)} = d \omega^1 (a,b)\]
where $a,b$ are vectors with $a\times b = n$ and $n$ normal to $A$.
The same is true for the geometric definition of divergence, where the net flow in and out of a volume is measured
\[div(v) = \lim_{diam(V)\rightarrow 0 } \frac{\int_{\partial V} \langle v , n \rangle dx}{Vol(V)}\]
where $n$ is the normal field on the border of the volume $V$, see Figure \ref{fig:6_curlDiv}. And again, this is exactly captured by Stokes theorem and the exterior derivative for $2$-forms
\[\lim_{diam(V)\rightarrow 0 }\frac{\int_{\partial V} \omega^2}{Vol(V)} = \lim_{diam(V)\rightarrow 0 }\frac{\int_{V} d \omega^2}{Vol(V)} = d \omega^2\]

\begin{figure}%
\begin{center}
\def\svgwidth{8cm}
\input{imgs/6_curlDiv.eps_tex}%	
\end{center}
\caption{The geometric definition of curl and divergence. Curl measures the flow around a infinitesimal patch orthogonal to a normal $n$. Divergence, here displayed in a 2D setting, measures the local netflow. }%
\label{fig:6_curlDiv}%
\end{figure}
\newpage
\section{The Discrete Exterior Derivative}

Stokes theorem captures the geometry of the exterior derivative. Both smooth and discrete manifolds have common geometric operations, namely the border operator. The common geometry of smooth and discrete manifolds can now be used to define a discrete exterior derivative on discrete manifolds, which preserves the geometry of the exterior derivative. The discrete exterior derivative forms the pillar of DEC.

To conserve the geometry of $d$ means to conserve Stokes theorem.  We can directly translate Stokes theorem to the discrete setting- In the discrete setting the integral of a discrete form $\discrete{w}$ over a set of simplices $\sigma$ is the scalar product $\langle \sigma, \discrete{w} \rangle$ and Stokes theorem
\[[\Omega,d\omega] = [\delta \Omega, \omega]\]
becomes
\[\langle \sigma, d_{discrete}^k\textbf{w}^k \rangle = \langle \delta_{k+1} \sigma, \textbf{w}^k \rangle,\]
where $d_{discrete}$ is yet unknown. But this relation \emph{defines} the yet unknown $d_{discrete}$; it has to be the transposed of the border-operator matrix:
\[d_{discrete} = \delta^T\]
By discretizing the exterior derivative like this we get at once consistent discretizations of all its special cases like gradient, divergence and curl. 

\subsection{Examples}
For example we know that $d$ applied to $0$-forms is the gradient. Our discrete realization of the exterior derivative for $0$-forms is the matrix
\[d_{discrete}^0 = \delta_1^T\]
which has size $(\# edges \times \#vertices)$. Applying this matrix to a discrete $0$-form yields a vector of dimension $\# edges$, a discrete $1$-form. As $\delta_1^T$ is the incidence matrix of the edges, it assigns the value $\textbf{w}^0(v_1) - \textbf{w}^0(v_0)$ to an edge $(v_0,v_1)$. The gradient is simply realized as a difference.

Another example is the curl operator, depicted in Figure \ref{fig::6_1_curl}. The curl operator is a realization of $d$ applied to differential $1$-forms. In the discrete setting curl therefore is realized as
\[d_{discrete}^1 = \delta_2^T\]
\begin{figure}[t]
\begin{center}
\includegraphics[height = 4cm]{imgs/6_4_curl.eps}
\end{center}
\caption{Curl is realized as the incidence matrix of the faces. By applying this matrix the values on the border edges of a face are summed up according to the orientation of the face (thus $-g_1$)}
\label{fig::6_1_curl}
\end{figure}

As this is the incidence matrix of faces, applying $d_1$ to a 1-form sums up the values of the discrete $1$-form on edges along a face and assigns the sum to the face. 

\subsection{Correctness}
The discrete exterior derivative $d_{discrete}$ does not introduce any new errors. It is consistent with the way we interpret and sample discrete differential forms- you can take the exterior derivative before sampling or the discrete exterior derivative after sampling, it does not matter.

Suppose that $\textbf{w}^0$ samples $\omega^0$. Then $d_{discrete}^0 \textbf{w}^0$ exactly samples $d\omega^0$:
\[\int_{[v_0,v_1]} d\omega^0 = \omega^0(v_1) - \omega^0(v_0)= \textbf{w}^0(v_1) - \textbf{w}^0(v_0)\] 
But, by design, this is true for all $d_{discrete}^k$: suppose $\textbf{w}^k$ samples $\omega^k$, then
\[\int_{\sigma^{k+1}} d\omega^k  = \int_{\delta \sigma^{k+1}} \omega^k = \langle \delta_{k+1}(\sigma^{k+1}), \textbf{w}^k\rangle = (d_{discrete}^{k} \textbf{w}^k) (\sigma^{k+1}).\]


\section{Duality: The Hodge Star}
\label{sec:hodgeStar}
By having a discrete exterior derivative, we have a discretization of all the operators that arise as special cases of the exterior derivative (see Figures \ref{fig::6_1_SC2d} and \ref{fig::6_1_SC3d}). But there is still an important element missing. For example we would like to apply the divergence operator to the gradient of a function $div(grad(f))$ to get the Laplacian, as it is done in standard calculus:
\[\Delta = div\circ grad\]
Looking at Figure $\ref{fig::6_1_SC3d}$ tells us that on 3-dimensional manifolds $grad$ is $d_0$ and $div$ is $d_2$.   This is a problem: $d_0$ takes a $0$-form and maps it to a $1$-Form and we can not apply $d_2$ to $1$-forms!
The key to this 'problem' lies in the duality of forms. 

We have already seen that different forms have the same representation. In three dimensions differential $0$-forms and differential $3$-forms can be represented as real valued functions, $1$-forms and $2$-forms as vectorfields. So it should not come as a surprise that you can generically make  a $0$-form out of a $3$-form, a $2$-form out of a $1$-form, and vice versa. 

\subsection{Intuition}
\label{subsec:hodgeintuition}
The idea is to treat a differential $k$-form like a differential $(n-k)$-form. But while $k$-forms measure $k$-dimensional volumes, $(n-k)$-forms measure $n-k$ dimensional volumes.
We can use a trick: given a small $k$-dimensional cube $c$ we can associate with $c$ the $n-k$-dimensional cube $c^\perp$ that is perpendicular to $c$ and has the same volume as $c$ i.e.
\[c\perp c^\perp \]
\[vol_k(c) = vol_{n-k}(c^\perp)\] 
Then, if we want to calculate an approximate integral of a $k$-form over a set of $n-k$-dimensional cubes $\{c_1,...,c_l\}$, thereby treating the $k$-form like an $n-k$-form, the sum
\[\sum_{j} \text{''}\omega^k(c_j)\text{``}\]
is instead calculated as 
\[\sum_{j} \omega^k(c_j^{\perp}),\]
see the sketch in Figure \ref{fig::6_1_dualIntegral}. To keep the notion of $k$-forms clean we associate a \emph{dual} $n-k$-form $\star\omega^k$ to a $k$-form $\omega^k$, that behave like described, i.e. in a very dirty notation
\[\star\omega^k(c) \approx  \omega^k(c^\perp)\]
\begin{figure}[t]
\begin{center}
\includegraphics[height = 4cm]{Imgs/6_4_dualiIntegral.eps}
\end{center}
\caption{Intuition for duality in $\mathbb R^3$: We have a one form $\omega^1$ that can be evaluated on lines and would like a $2$-Form $\star \omega^1$ that can be evaluated on 2-dimensional regions. The dual form $\star \omega$ evaluated on a small square should be the value of $\omega^1$ evaluated on the orthogonal complement of the small square with equal volume, in this case an orthogonal line with length $area(square)$. }
\label{fig::6_1_dualIntegral}
\end{figure}

\subsection{Dual Forms}
We will now formally define the dual of a form and then examine how it fits the 'intuition' developed above. 

\subsubsection{Definition of $\star$}
The spaces of forms $\Lambda^k(\mathbb R^n)$ and $\Lambda^{n-k}(\mathbb R^n)$ have the same dimension:
\[\dim(\Lambda^k(\mathbb R^n)) = \begin{pmatrix}
n \\ k
\end{pmatrix}= \begin{pmatrix}
n \\ n-k
\end{pmatrix} = \dim(\Lambda^{n-k}(\mathbb R^n))\]
As they have the same dimension we can find a bijective linear mapping between these two spaces. Usually this mapping is defined by first defining  a scalarproduct on the space of forms, and then using the scalar product to define the mapping. 

In order to define a scalar product on the space $\Lambda^k$ we select an orthogonal basis $e_1,...,e_n$ of $\mathbb R^n$ with respect to the euclidean scalar product\footnote{Note that the Hodge duality is usually defined with respect to an arbitrary scalar product, but for us this will do} and define for two $k$-forms $\omega^k, \nu^k$
\[\langle \omega^k, \nu^k \rangle = \sum_{i_1 <...<i_k} \omega^k(e_{i_1},...,e_{i_k})\cdot \nu^k(e_{i_1},...,e_{i_k}).\]
This definition does \emph{not} depend on the choice of the basis. If both forms are written in the base given by the $de_i$'s, the scalar product is
\[\omega^k = \sum_{i_1<...<i_k}w_{i_1,..,i_k} de_{i_1}\wedge ...\wedge de_{i_k}\]
\[\nu^k = \sum_{i_1<...<i_k}v_{i_1,..,i_k} de_{i_1}\wedge ...\wedge de_{i_k}\]
\[\langle \omega^k, \nu^k\rangle = \sum_{i_1<...<i_k} v_{i_1,..,i_k} \cdot w_{i_1,..,i_k}\]
i.e. the euclidean scalar product of the vectors describing $v$ and $w$. This really is a scalar product: it is linear in both parameters and symmetric.

Duality is then defined using the volume form $dVol$. The dual $\star\omega^k$ of a $k$-form $\omega^k$ is defined to be the $n-k$-form that fulfills
\[\langle\star\omega^k, \nu^{n-k}\rangle dVol  = \omega^k \wedge \nu^{n-k} \text{ for all $(n-k)$-forms $\nu^{n-k}$}\] 
The operator $\star$ is called the Hodge star. The Hodge star describes a linear mapping from $\Lambda^k$ to $\Lambda^{n-k}$:
 \[\star(\omega_1^k + \lambda \omega_2^k) = \star\omega_1^k + \lambda \star\omega_2^k \]

\subsubsection{Understanding $\star$}
We can understand the Hodge operator by looking at how it acts on basis elements $de_{i_1}\wedge...\wedge de_{i_k}$ for a positively oriented orthonormal basis $e_1,...,e_n$. From the definition of $\star$ follows:
\[\star (de_{i_1}\wedge...\wedge de_{i_k}) = sign \cdot de_{j_1}\wedge...\wedge de_{j_{n-k}} \]
where $j_1,...,j_{n-k}$ is the complement of the indices $i_1,...,i_k$ in the set $\{1,...n\}$ and $sign$ comes from reordering the indices, i.e. is the sign in the equation
\[de_{i_1}\wedge...\wedge de_{i_k} \wedge de_{j_1}\wedge...\wedge de_{j_{n-k}} = sign \cdot de_1\wedge de_2 \wedge ... \wedge de_n .\]
For example in a three dimensional setting with an orthonormal basis $e_1, e_2, e_3$
\[\star de_1 =  de_2 \wedge de_3\]
\[\star de_3 =  de_1 \wedge de_2\]
and 
\[\star de_2 =  - de_1 \wedge de_3 =  de_3 \wedge de_1\]
where we have a minus because
\[de_2 \wedge de_1 \wedge de_3 = - de_1 \wedge de_2 \wedge de_3\]
This shows that the $\star$ is close to the intuitive description in Section \ref{subsec:hodgeintuition}:  suppose $v = a \times b$ for vectors $a$,$b$ and $v$ in $\mathbb R^3$. The vectors $a,b$ span a patch perpendicular to $v$ with an area equal to the length of $v$. Furthermore 
\[de_1(v) = v_1 = (a_2b_3 -a_3b_2)\] 
\[\star de_1 (a,b)= (de_2 \wedge de_3)(a,b) = a_2b_3 -a_3b_2 \]
therefore
\[\star de_1 (a,b) = de_1(v)\]
so $\star de_1$ing the patch spanned by $a,b$ is like $de_1$ing the 'line' $v$. The same is true for $de_2$ and $de_3$. And as the $\star$ operator is linear and any one form is a sum $\omega^1 = \lambda_1de_1 + \lambda_2de_2 + \lambda_3de_3$ we get
\[\star\omega^1(a,b) = \omega^1(v)\]

\vspace{0.5cm}
\begin{center}
\includegraphics[height = 3cm]{imgs/6_4_staromega1.eps}
\end{center}
\newpage
\subsection{Realization of $\star$ in Standard Calculus}
Differential forms relate to objects from standard calculus. We now determine how the $\star$-operator acts on these presentations.

\subsubsection{0-forms and volume forms}
The dual of a 0-form $f$ is simply $f dVol$ and the dual of an $n$-form $f dVol$ is simply $f$- so the $\star$ operator denotes only a change of interpretation of a function $f$ as a $0$-form or an $n$-form and is the identity in standard calculus.
\[\star f = f\; dVol\]
\[\star f\;dVol = f\]
\subsubsection{1-forms and 2-forms in $\mathbb R^3$}
In three dimensions applying the hodge operator to the one form associated to a vectorfield leads to the two form associated to the same vectorfield. The star operator again denotes only a change of interpretation.
\[\star(v_1de_1 + v_2 de_2 + v_3 de_3) = v_1(de_2\wedge de_3) + v_2 (de_3\wedge de_1) + v_3(de_1\wedge de_2)\]
\[\star(v_1(de_2\wedge de_3) + v_2 (de_3\wedge de_1) + v_3(de_1\wedge de_2)) = v_1de_1 + v_2 de_2 + v_3 de_3\]

\subsubsection{1-forms in $\mathbb R^2$}
For 1-forms on two dimensional manifolds something happens. The dual of a one form is again a one form, and
\[\star de_1 = de_2\]
\[\star de_2 = -de_1\]
\[\star(a\;de_1 + b\;de_2) = b\;de_1 -a\;de_2\]
The operation equivalent to the Hodge $\star$ for a two dimensional vector field is a rotation by $90^\circ$, in orthonormal coordinates simply
\[\begin{pmatrix}
0 & 1 \\
-1 & 0
\end{pmatrix}\]
It also follows that $\star \star \omega^1 = -\omega^1$ as rotating a vector twice by $90^\circ$ changes its orientation. Note that this is directly related to the two sampling schemes for $1$-forms described in Section \note{...}. \note{TODO}.
\subsubsection{The dual of the dual k-form}
Differential $1$-forms on $2$-dimensional manifolds are not special, considering their behavior under the Hodge star. One can  show with little effort that
\[\star\star \omega ^k = (-1)^{k(n-k)}\omega^k.\]
Applying $\star$ twice will switch the orientation of $\omega^k$ exactly when $k(n-k)$ is not even. But for $n= 2,3$, the pair $n=2$ and $k=1$ is the only one where $k(n-k)$ is not even. 


\section{Exterior Calculus}

\begin{figure}[t]
\begin{center}
\includegraphics[height = 5cm]{imgs/6_4_exteriorCalculus}

\includegraphics[height = 5cm]{imgs/6_4_standardCalculus}
\end{center}
\caption{Top: the de Rham complex for an $n$-dimensional manifold. Bottom: the realization of the deRham complex in standard calculus.  The dashed arrows represent Laplacians defined by concatenating operators.}
\label{fig::deRhamComplex}
\end{figure}

Translated in standard calculus terms, it the $\star$ operator turns out to be often nothing more than a change of interpretation. But for exterior calculus the Hodge $\star$ plays an important role, as without it you could not higher order operators: applying $d$ once has only simple partial derivatives; applying $d$ twice is zero $dd=0$. With the $\star$ operator you can circumvent this : $d\star d$ is a second order operator.

With $\star$, $d$, $\wedge$ and also $^\#$ and $^\flat$, who formalize the relation between 1-forms and vectorfields (see Section \ref{subsec:diffformsare}), we have a very elegant and powerful language at hand.
The $\star$ and $d$ operator build the so-called de Rham complex, depicted in Figure
\ref{fig::deRhamComplex}. The de Rham complex is a so called chain complex which has all kinds of interesting properties, but goes beyond the scopes of this thesis. %But it is interesting to know that the border operator together with manifolds, the exterior derivative together with differential forms, simplicial complexes with the discrete border operator and discrete forms with the discrete exterior derivative each form closely related chain complexes. That the discrete exterior calculus operators form a chain complex is partially a reason for the strength of DEC.
This complex summarizes the operators of exterior calculus and we will meet quite a few important relations on this conglomerate of differential-form spaces and the operators $d$ and $\star$.

The realization of the de Rham complex in 2 dimensions is depicted in Figure \ref{fig:deRham2d} where $\nabla \times$ in $\mathbb R^2$ is
\[\nabla \times := \frac{\partial}{\partial x_2} - \frac{\partial}{\partial x_1}.\] 
The realization of $\star^{-1} d \star$ for 1-forms is minus the divergence operator $\nabla\cdot$, as you can easily check.

\begin{figure}
\begin{center}
\includegraphics[height = 5cm]{imgs/6_4_standardCalculusn2}
\end{center}
\caption{The realization of the de Rham complex in two dimensions.}
\label{fig:deRham2d}
\end{figure}

\subsection{The Coderivative}
\label{subsec::coderivativ}
It is useful to define the so called coderivative $\partial_k$, which is given by
\[\partial_k = (-1)^{n-k}\star^{-1}_{k+1}d_{n-k}\star_k,\]
takes a differential $k$-form and returns a differential $k-1$-form. In two dimensions $\partial_1$ is the divergence operator. Besides being a handy abbreviation, it is the adjoint to the exterior derivative on borderless compact manifolds $M$ or if one of the differential forms is zero on the border $\delta M$\footnote{The adjointness follows directly from Stokes Theorem and the behavior of $d$ and $\wedge$: $0=\int_{\delta M} \omega^k\wedge \star \nu^{k+1} = \int_M (d\omega\wedge\star \nu + (-1)^k\omega\wedge d\star \nu)$ and then setting $\omega\wedge d\star\nu = (-1)^{n-k}\omega\wedge \star \star d\star\nu$ we end $0= \int_M \langle d\omega,\nu\rangle dVol  - \int_M \langle \omega,\delta\nu \rangle dVol$}
\begin{equation}\langle d\omega, \nu\rangle = \langle \omega , \partial \nu \rangle \end{equation}
where the scalar for forms is extended to differential forms via
\[\langle \omega, \nu \rangle = \int_{M} \langle\omega^k,\nu^k\rangle dVol = \int_{M} \omega \wedge \star \nu.\]

%Not all operators can be built yet. Introduce Star, duality. The star could also be motivated with the obvious relation between k and n-k forms.

\section{Discrete Exterior Calculus}
What is left is finding a discrete star operator and finding a way to represent discrete dual forms $\star \textbf{w}$. If $\textbf{w}^k$ is a discrete $k$-form associated to $k$-simplices i.e. discrete objects of $k$-dimensional volume, the dual should be a discrete $n-k$-form, therefore associated to $n-k$-dimensional objects. For this we use a \textbf{dual mesh}. We will use Voronoi duality, because it facilitates the definition of a discrete $\star$.

\subsection{Dual Mesh}
Suppose we have a discrete $n$-manifold. Then the Voronoi dual of a $n$-simplex is its circumcenter and the dual of a $k$-simplex is the $n-k$ dimensional cell spanned by the circumcenters of the incident $n$-simplices, as depicted for $n=2$ in the inlined image. From now on we will always make the difference the primal mesh consisting of primal vertices, edges, faces etc and the dual mesh consisting of dual vertices, edges and so on. If $\sigma$ is a simplex, we will denote its dual cell by $\star \sigma$.

\begin{center}
\includegraphics[height=4cm]{imgs/6_4_dualMesh}
\end{center}

\subsubsection{Border Operator and Orientation of the dual mesh}

The border-relation on a dual complex is simply the incidence relation of the primary mesh; a dual $j$ cell $\star \sigma^{n-j}$ is on the border of the dual $j+1$-cell $\star \sigma^{n-j-1}$ exactly if $\sigma^{n-j-1}$ is on the border of $\sigma^{n-j}$.

But the dual border-matrix is not directly given by the transposed primary incidence matrix $\delta^T$; we have to take care of orientations of the dual cells, as a dual cell $\star \sigma$ on the dual mesh gets an orientation induced by the primary simplex $\sigma$, as depicted in Figure \ref{fig:6_inducedorientations}.
\begin{figure}%
\includegraphics[height=3cm]{imgs/6_inducedorientations.eps}
\caption{The dual cells (red) with the orientations induced by the primary simplices. The first three sketches are in $\mathbb R^2$, the last two in $\mathbb R^3$}%
\label{fig:6_inducedorientations}%
\end{figure}

Bottom line is that
\[\delta_{k}^{dual} = (-1)^{n-k+1} (\delta_{n-k+1}^{primal})^T\]
We give a proof of this in the Appendix \ref{app:dualborder}. An example for $n=2$ is given in Figure \ref{fig:6_dualborder}.

\begin{figure}%
\begin{center}
\includegraphics[height= 4cm]{imgs/6_7_dualborder.eps}%
\end{center}
\caption{The central vertex is negatively oriented with respect to the incident edges (left). The dual edges and the dual face have an induced orientation, but the dual edges are oriented positively with respect to the dual face. Therefore $\delta_{2}^{dual} = - (\delta_1^{primal})^T$ in the $n=2$ dimensional setting.}%
\label{fig:6_dualborder}%
\end{figure}



\subsection{Discrete Dual Forms and Star Operator}
While in the continuous case the dual of a differential form is again a differential form, we make a strict distinction between discrete primary forms and discrete dual forms. While the discrete primary forms are defined on the simplices of the primary mesh, the discrete dual forms are defined on the dual mesh.
We 'sample' dual forms on the dual mesh. The value of the discrete dual form $\star \textbf{w}$ sampling $\star \omega$ on the dual simplex $\star \sigma$ can be interpreted as
\[\star \textbf{w}(\star \sigma) = \int_{\star \sigma^k}  \star \omega^k.\]
The discrete exterior derivative  on the dual mesh is given by
\[d^{dual}_{n-k} = (\delta^{dual}_{n-k+1})^T \]
just as we defined the discrete exterior derivative on the primary mesh, to preserve the geometry of $d$ revealed by Stokes Theorem. Expressed with primal matrices this is
\[d^{dual}_{n-k}= (-1)^k(\delta_k^{primal}) = (-1)^k(d^{primal}_{k-1})^T \]

What is left is the question how to get form a discrete form $\textbf{w}^k$ to its dual $\star \textbf{w}^k$, i.e. how the two integrals
\[\int_{\star \sigma^k} \star \omega^k ,\;\;\; \int_{\sigma^k} \omega^k\]
relate. As we use the Voronoi duality, $\star \sigma$ is orthogonal to $\sigma$. If $\omega^k$ is constant on $\sigma^k$ and $\star \sigma^k$, we have, because of the way the Hodge star is defined
\[\int_{\star\sigma} \star\omega^k =  \frac{Vol_{n-k}(\star \sigma^k)}{Vol_k(\sigma^k)}\int_{\sigma} \omega^k\]
This motivates the use of the diagonal matrix
\[\star^{discrete}_k = \begin{pmatrix}
\frac{Vol_{n-k}(\star \sigma^k_1)}{Vol_k(\sigma_1^k)} \\
&\frac{Vol_{n-k}(\star \sigma_2^k)}{Vol_k(\sigma_2^k)} \\
& & \ddots \\
& & & \frac{Vol_{n-k}(\star \sigma_m^k)}{Vol_k(\sigma_m^k)}
\end{pmatrix}\]
as a discrete version of the $\star$ operator to relate the discrete dual and primary forms 
\[\star\textbf{w}^k  = (\star\textbf{w}^k(\star\sigma^k_1),...,\star\textbf{w}^k(\star\sigma^k_m))\]
\[\textbf{w}^k = (\textbf{w}^k(\sigma^k_1),...,\textbf{w}^k(\sigma^k_m))\]
i.e.
\[\star \textbf{w}^k = \star^{discrete}_k \textbf{w}^k.\]
This discrete $\star$ operator is \emph{not} compatible with the sampling scheme; the dual of a discrete form $\textbf{w}^k$ sampling $\omega^k$ only approximates a correctly sampled $\star \omega^k$. But if the size of the simplices gets smaller, the error of $\star^{discrete}$ goes to zero, as $\omega^k$ will be close to constant in small regions.

This simple $\star^{discrete}$ operator will still prove to be quite good. From a numerical point of view it is great that it is a diagonal matrix. And by associating dual forms to dual Voronoi cells, the geometry of the Hodge star is captured quite well by this discrete Hodge star. 

\subsubsection{Drawbacks of Voronoi Duality}
One drawback when choosing the dual mesh and discrete star as Voronoi cells is that Voronoi cells 'degenerate' in the presence of obtuse simplices. The circumcenter of a simplex can lie arbitrarily far away from the simplex such that $\star_{discrete}\textbf{w}^k$ is not a good estimation for a sampled dual form. Adapting the dual mesh and star matrices as with the Laplacian in Section \note{...} or using a different dual mesh and star operator might be beneficial. But this was not investigated in this thesis. 
 
\subsection{This is Discrete Exterior Calculus}
With the discrete star, forms, dual forms and discrete exterior derivative for dual and primal forms we can build a 'discrete de Rham' complex. \note{mention of more general chain complexes} The discrete de Rham complex keeps many properties from the continuous one. \note{Maybe list them here... PoinCarr� and Hodge ... TODO}

\note{Dual forms implemented as vectors. While related to a different geometric object it is a vector of same dimension}

\note{\textbf{Scalar product for discrete $k$-Forms}\newline
Adjointness of the covariant derivative with the discrete scalar product. Define the discrete co-derivative such that adjointness holds. Or the scalar product such that adjointness holds.
Example scalar product of $0$-Forms needs to be divided by area to sum up correctly. $1$-Forms: diamond shaped area...
}

\subsubsection{Discrete Scalarproduct and Adjointness}
\label{subsec:5_discreteScalarprod}
We define the scalar product for discrete differential forms such that the discrete coderivative is adjoint to the discrete exterior derivative.
\[\langle d\discrete{v}, \discrete{w} \rangle_{discrete} = \langle \discrete{v}, \partial \discrete{w}\rangle_{discrete}\]
But a scalar product on a finite dimensional vector space can be  described by a symmetric matrix $S$:
\[\langle \discrete{v}, \discrete{w} \rangle_{discrete} = \discrete{v}^TS\discrete{w}\]
And the first equation is fulfilled when $S = \star_{discrete}$ is used as then
\[\langle d\discrete{v}, \discrete{w} \rangle_{discrete} = (d\discrete{v})^T\star \discrete{w} =  
(\discrete{v})^T d^T\star \discrete{w} =  (\discrete{v})^T \star (\star^{-1}d^T\star \discrete{w}) =  \langle \discrete{v}, \partial \discrete{w}\rangle_{discrete}\]
%\[\langle \omega^k , \nu^k \rangle = \int_{M}  \omega^k\wedge \star \nu^k\]

To use the $\star$ matrix as scalar product also makes intuitively sense: for 0-forms
\[\int_{M}\langle \omega^0, \nu^0 \rangle = \sum \int_{\star \sigma^0} \omega^0 \cdot \nu^0 dVol \approx \sum \discrete{v}(\sigma^0) \cdot \discrete{w}(\sigma^0) \cdot Vol(\star\sigma^0) \]
\[= \discrete{v}^T \star^0_{discrete} \discrete{w}\]
as the matrix $\star^0$ stores the volumes of the dual cells of vertices $\sigma^0$. The scalar product for discrete 0-forms gets weighted by dual areas. Generally the values of the discrete form are scaled such that they represent values scaled to the diamond shaped areas spanned by a simplex and its dual, see Figure \note{Todo} for discrete 1-forms on a discrete 2-manifold
\[\langle \omega^k, \nu^k\rangle = \int_{M}\langle \omega^k, \nu^k \rangle = \sum \int_{hull(\sigma ,\star \sigma)} \omega^k \wedge \star \nu^k \approx \discrete{v}^T \star^k_{discrete} \discrete{w}\]
\note{
\textbf{The 0 Form Laplacian from the beginning}\newline
Now easy to write that guy down.}

\newpage		
\chapter{Application: Mesh Parametrization}
	\begin{longtable}{|p{4.5cm}|p{4.5cm}|p{4.5cm}|}
		\hline
		Smooth Theory& Discrete Theory& Implementation (Notes)\\
		\hline
			Conformal Maps
			\begin{packed_enum}
				\item[-] Conformal Maps with Exterior Calculus
			\end{packed_enum}
			&
			Existance of Embeddings, The thingy theorem
			\begin{packed_enum}
				\item[-] Dimension of result spaces depending on topology?
			\end{packed_enum}
			 & 
			 Implementing it with DEC
			 \begin{packed_enum}
				\item[-] The equation
				\item[-] Border Constraints
				\item[-] Results
			\end{packed_enum}
			 \\		
		\hline
	\end{longtable}
	In this chapter we are all about 0 Forms.
	\subsection{Embeddings}
	Taking coordinates as functions associated to the Surface and not the other way round. The thingy theorem for graphs
	\subsection{Conformal Maps}
	Conformal maps properties, EC formulation
	\subsection{Implementaion: Conformal Embedding}
		How its done. Pretty straight forward if you have the machinery. Different Border Constraints. Mention MeanValue Weights?
	\subsection{Dimension of Harmonic Space?}
		Looking at results and topology. Genus, Bettinumbers, DeRham Complex?	Would be nice..
		Mention cutting algorithms like the quad mesh paper..
\newpage	
\section{Application: Vectorfield Design and the general Laplacian}
	\begin{longtable}{|p{4.5cm}|p{4.5cm}|p{4.5cm}|}
		\hline
		Smooth Theory& Discrete Theory& Implementation (Notes)\\
		\hline
			Important External Calculus Results:
			\begin{packed_enum}
				\item[-] Point Carre Lemma
				\item[-] Laplace Beltrami Operator: $d$ and $\delta$ free.
				\item[-] Hodge Decomposition
				\item[-] Here on the dim of harmonic spaces?
			\end{packed_enum}
			&
			The same as the smooth ones.
			\begin{packed_enum}
				\item[-] Properties still hold exactly.
				\item[-] Laplacian in Least square sence
				\item[-] Border Constraints
				\item[-] One Form interpolation
			\end{packed_enum}
			 & 
			 Implementing it with DEC
			 \begin{packed_enum}
				\item[-] Vector Field Design
				\item[-] =Least square harmonic 1 Form solving
			\end{packed_enum}
			 \\		
		\hline
	\end{longtable}
	\subsection{Point Carre Lemma}
	When is a form d or delta of another form?
	\subsection{General Laplacian}
	The Laplacian is as well d as delta of another form...
	\subsection{Hodge Decomposition}
	Splitting Forms
	\subsection{Application: Vector Field Design}
	Using all this for Vector Field Design
	\subsection{Border Constraints}
	How borders can be treated but it affects everything.
\newpage
\section{A Fluid Simulation with DEC}
	\begin{longtable}{|p{4.5cm}|p{4.5cm}|p{4.5cm}|}
		\hline
		Smooth Theory& Discrete Theory& Implementation (Notes)\\
		\hline
			\begin{packed_enum}
				\item[-] Introduction to the fluid equations and reformulation in DEC.
			\end{packed_enum}
			&
			\begin{packed_enum}
				\item[-] Second approach to Borderconstraints.
				\item[-] Reformulation / solving for rot part etc.
			\end{packed_enum}
			 & 
			 Implementing Fluid Sim with DEC
			 \begin{packed_enum}
				\item[-] Solving for Exact harmonic 1 Form
				\item[-] continuous Vfield interpolation
				\item[-] pathtracing
				\item[-] Results
			\end{packed_enum}
			 \\		
		\hline
	\end{longtable}
	All theory but the Fluid Simulation dependent theory is introduced, so this is a demonstration of DEC in use.
	\subsection{The Euler Equations}
	Short introduction to the meaning of the equation
	\subsection{The Algorithm}
	Approach and general Algorithm
	\subsection{Interpolation and Pathtracing}
	As it says. Issues on curved meshes.
	\subsection{Border Constraints}
	Need for exact Harmonic solution $=>$ Equation.
	\subsection{Influence of Mesh choice and parameter choice}
	Results and problems.
\end{document}