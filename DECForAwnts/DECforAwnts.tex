\documentclass[draft]{scrbook}
\usepackage{amsmath}
\usepackage{amssymb}
\usepackage{graphicx}
\usepackage{color}
\usepackage{algorithmic}
\usepackage[english]{babel}

%\usepackage[latin1]{inputenc}
\usepackage{tikz}
\usepackage{longtable}
\usetikzlibrary{shapes,arrows}

\title{Discrete Differential Calculus for Academics with No Time}
\author{Peter Bertholet}

\newenvironment{definition}[1][]{\begin{trivlist}
\item[\hskip \labelsep {\bfseries Definition (#1)}]\begin{itshape}}{\end{itshape}\end{trivlist}}

\newenvironment{packed_enum}{
\begin{enumerate}
  \setlength{\itemsep}{1pt}
  \setlength{\parskip}{0pt}
  \setlength{\parsep}{0pt}
}{\end{enumerate}}
\newenvironment{packed_itemize}{
\begin{itemize}
  \setlength{\itemsep}{1pt}
  \setlength{\parskip}{0pt}
  \setlength{\parsep}{0pt}
}{\end{itemize}}


\newcommand{\note}[1]{\textcolor{red}{\textit{(#1)}}}
\newcommand{\abs}[1]{\left| #1 \right|}

\begin{document}

	\maketitle
	\tableofcontents
	
	%\begin{abstract}
	%	Discrete Exterior Calculus...
%\end{abstract}
	
\chapter{Introduction}

The goal of this Thesis is to present an introduction to Discrete External Calculus (DEC) without assuming more than some basic knowledge of linear Algebra and standard calculus. You should be able to use this text like a tutorial to get enough theoretical background and enough practical knowledge to understand DEC and apply it to problems.

\section{A glimpse of Discrete Exterior Calculus}
Before getting started, you need to get a glimpse at what Discrete External Calculus is and what it is good for. DEC is, broadly speaking, a way to state and treat a large class of differential equations on meshes. 

You most probably have already worked with some differential Operators. Some of the most common of them being the usual partial derivative $\frac{\partial}{\partial x_i}$, the gradient $\nabla = (\frac{\partial}{\partial x_1},\frac{\partial}{\partial x_2},...,\frac{\partial}{\partial x_n})$ or the Laplace Operator $\Delta = \frac{\partial^2}{\partial x_1^2} + \frac{\partial^2}{\partial x_2^2} +... + \frac{\partial^2}{\partial x_n^2}$. External Calculus provides a framework to treat those operators in a very clean and unified way. And it allows you to define and use these Operators on curved surfaces without much work.

But why should you care about this? As you probably know these operators are extremely powerful tools to describe all kind of problems. Examples---
And (non-discrete) external Calculus gives you tools to state and treat these problems very neatly.

You now might say that okay, external calculus is probably a nice way to write down equations that else would seem more cluttered and okay, it might help you describe problems even on curved surfaces, but how does this help with actual practical applications and computations? And this is what Discrete External Calculus is about: it provides a consistent and straight forward way to adapt the operators for the use on meshes for computations. 

\begin{figure}[bht]
\begin{center}
\includegraphics[width=6cm]{Imgs/1_1_line.eps}
\includegraphics[width=6cm]{Imgs/1_1_spline.eps}
\end{center}
\caption{Left: the discrete mesh you have to work with. Right: Splines are used to associate a differentiable function to the discrete set of points; saying this is a good way to approximate the underlying function makes assumptions about the underlying sampled function}
\label{fig::1_1_linevsspline}
\end{figure}

Consider this very simple situation: you have a 1-D function on a discrete set of points and want to calculate its derivative (Fig. \ref{fig::1_1_linevsspline}). Here you already encounter a problem. Obviously mathematically it is less than clear what derivatives should be on the pointset. You have different options. One would be to assume the points represent a differentiable curve which has a closed formula (depending on the pointset) such that a derivative can be calculated at any point. Another would be to embrace the situation, take the non-differentiable piecewise linear line and accept that you know only average values of the derivative on each line segment. I.e. you would say the derivative is the set of differences $f_{i+1} - f_i$ associated to the line segments $i$.

Both possibilities have advantages and drawbacks. While the first option (which leads to Finite Element like methods (FEM)) provides you a value for the derivative everywhere, the assumption that the spline is a good approximation of the ''real'' curve might be wrong and introduce errors. The second option gives you only a discrete set of values but you do not make any assumptions about the underlying curve, as for any differentiable function $f$ taking the values $f_i$, the averaged value is
\[\int_{line\;segment\;i} f^\prime dx = f_{i+1} - f_i\] 
So the second option never gives a wrong result. 

Discrete External Calculus is based on the second option. It takes the position that differential operations like taking the derivative on a discrete surface gives a discrete set of values (which have a true meaning for any underlying continuous function) and that interpolation is a somewhat independent problem.

\begin{figure}
\begin{center}
\includegraphics[width=6cm]{Imgs/1_1_1dLap.eps}
\includegraphics[width=6cm]{Imgs/1_1_1dLapSmoothing.eps}
\end{center}
\caption{An example illustrating how the DEC Laplace operator conserves a geometric property of the continuous Laplacian: When the Laplace operator is applied to the coordinates of the surface it is defined on, it defines an area minimizing flow. The image depicts the initial line with the initial Laplacian vectors (left) and the deformation of the line according to the Laplacian flow, where the border is fixed.}
\label{fig::1_1_1dSmoothing} 
\end{figure}

But the great strength of DEC is yet another one. Differential operators mostly are of a very geometric nature and have geometric properties. The gradient of a function for example always points in the direction of the highest local ascent. External Calculus is very well suited to capture some key geometric properties of those operators. 

While the piecewise linear line considered above does not allow differential operators to give concise values at all (or any) points, the geometric properties of the operators often are still meaningful. 
You can define the Laplace Operator $\Delta$ on and for surfaces; for now just assume that this is done somehow. If you then have a function defined ON the curve, you can apply $\Delta$ to this function (if both the surface and the function are 'nice' enough).  Some special functions that are defined on the line are the functions that assign coordinates to every point on the line, i.e. for a line in 2d the functions $x(p), y(p)$. 
The Laplace Operate has many very important properties. One of its (maybe less essential but rather plastic) geometric properties is that, if applied to the coordinate functions, it defines an area minimizing flow, as shown in Fig. \ref{fig::1_1_1dSmoothing}. This means that the vectors returned by the Laplacian point in a direction such that, if you move all points according to them, the area of the surface gets minimized in an optimal way.

While we can not calculate the usual Laplacian on the piecewise linear curve of Fig. \ref{fig::1_1_1dSmoothing} (it would be undefined between every two line segments and 0 everywhere else) the geometric property of area minimization stays meaningful. Thus this property can be directly used to define a discrete Laplacian. This discrete Laplacian then not merely approximates the continuous Laplacian, it also preserves some of its geometric properties \emph{exactly}.

And this is the fundament of DEC: differential operators are not merely approximated; some geometric key features from the continuous counterparts are identified and DEC is then designed to respect them exactly on a piecewise linear mesh. And, as you will see, external calculus serves you some rather deep geometric relations of these operators on a golden tablet.

% Define block styles
%\tikzstyle{decision} = [diamond, draw, fill=blue!20, 
%   text width=4.5em, text badly centered, node distance=3cm, inner sep=0pt]
%\tikzstyle{block} = [rectangle, draw, fill=blue!10, text width=7em, text centered, rounded corners, minimum height=4em]
%\tikzstyle{line} = [draw, -latex']
%\tikzstyle{cloud} = [draw, ellipse,fill=red!20, node distance=3cm,
%    minimum height=2em]


%\begin{tikzpicture}[node distance = 4cm, auto]
% Place nodes
%\node [block] (step1) {Identify Geometric Key Property};
%\node [block, right of= step1] (step2) {Formulate Property for Meshes};
%\node [block, right of= step2] (step3) {Define Operator such that the property is preserved};
% Draw edges
%\path [line] (step1) -- (step2);
%\path [line] (step2) -- (step3);
%\end{tikzpicture}    



\section{A Short Tour of this Thesis}
Before you lie dozens of pages full of incredible fun and full of mysteries which will be unravelled before your eyes. But so you don't have to take my word for it: take the small tour of this Thesis. 

\textcolor{red}{(TODO/REDO this is conceptual. Idea: about one or two sentences per theme and a nice looking image where a keyfeature is depicted. Plus the Text should have like different modules so it is easy to skip parts if you are more or less familiar with them)}

To get started some basics are covered; if you are already familiar with them you can skip them without any loss.
To begin with the geometric objects we want to work with are presented: smooth surfaces and meshes. Or more generally manifolds and simplicial complexes. Manifolds are any curved hyper-surfaces that are inbedded in a higher dimensional space and simplicial complexes are the higher dimensional equivalents of triangle meshes, including for example tetrahedra meshes.

Then the basics of differential geometry are treated; we look at local properties of surfaces and various kinds of curvature. We then try to define the equivalent on triangle meshes, which will lead to defining a Laplacian on meshes. As an application meshes get smoothed.

In the next section  differential forms are introduced and explained. This is the heart of the matter and explains why the core concepts of discrete differential forms make sense. Differential forms being a concept you might not have met as such this chapter is a crucial one. It will lead to Stokes Theorem which encodes the geometry of the differential Operator and also forms the heart of DEC.
\[\int_{\partial\Omega} \phi = \int_{\Omega} d\phi\]

The rest are various aspects, applications and refinements, where we will paralelly deepen the theory about differential forms (i.e. External Calculus) to understand aspects of practical problems and derive further parts of the discrete exterior calculus to computationally solve these problems.


%Smooth manifolds (i.e. curved surfaces) are introduced so we can later define differential operators ON them. Important notions are tangential spaces; as all applications presented here will take place on meshes (i.e. 2d surfaces embedded in 3d space) some basic notions from differential geometry are repeated; if you are more or less familiar with these notions you can skip this part.


What needs to be introduced:
\begin{enumerate}
\item Some basics about Manifolds (Background of the continuous Objects the Theory is based on)
	\begin{enumerate}
		\item Manifolds and Tangential Spaces
		\item Tangential Spaces, 
		\item Orientations
		\item how manifolds get a differential structure i.e. how differential Operators like a derivative can be defined ON surfaces
	\end{enumerate}
\item Basics about Meshes (About the Objects we are making computations on) HANDS ON
	\begin{enumerate}
		\item Simplices
		\item Orientations, Border Operator and Wind Edge structure
		\item dual meshs (and their orientations) Either here or later.
	\end{enumerate}
\item Some basic Differential Geometry
	\begin{enumerate}
		\item Curves and Tangents
		\item curvatures, curvature tensor
		\item how manifolds get a differential structure i.e. how differential Operators like a derivative can be defined ON surfaces
	\end{enumerate}
\item Differential Geometry on the Mesh: HANDS ON
	\begin{enumerate}
		\item Discrete Mean curvature
		\item Other curvatures
	\end{enumerate}
\item Differential forms:
	\begin{enumerate}
		\item What they are and how they arise
		\item How they form a linear space
		\item Operations: Stokes Theorem and the differential Operator.
		\item Identifying the linear spaces with vector fields etc,
		\item Mapping the differential operator d to classical differential operators
		\item The Hodge Star and the codifferential Operator (maybe later)
		\item Thingy when something is a closed form. (these things will be introduced when needed)
		\item The Hodge Decomposition
		\item DeRham Komplex: topological constraints to harmonic fields.
	\end{enumerate}
\item Basic Properties of differential Operators: (good question where to put this in, its quite important)
	\begin{enumerate}
		\item Divergence
		\item Gradient
		\item Curl
		\item Laplace Operator
	\end{enumerate}
\item Discrete Differential Forms HANDS ON
	\begin{enumerate}
		\item What they represent (avg values)
		\item How they are introduced
		\item Derivation of discrete d 
		\item dual mesh mesh stuff
	\end{enumerate}
\item Applications
	\begin{enumerate}
		\item Differential Geometry: Curvatures to analyse the local structure of a mesh, Smoothing, remeshing
		\item Conformal Maps
		\item Vector Field Design
		\item Fluid Simulation
	\end{enumerate}
\end{enumerate}
Oh lord, what shall i do...? Write a short! summary for each topic (MAX one Paragraph each), place them in a good order,work them out in a second iteration. Explain math without proofs, write proofs where helpfull later.

The next part 
The first practical part describes one or two applications where notions from differential geometry are discretized
First of all the smooth mathematical counterparts are treated. How to describe them and properties. Then 

What is contained in this Thesis: (TODO/REDO after the rest is written)
\begin{enumerate}
\item An explanation of the geometric nature of differential operators
\item Some basic properties of Manifolds i.e. curved surfaces.
\item The basics of external Calculus and refresh your knowledge of differential Geometry 
\item See how Discrete External Calculus is developed and which key features it tries to conserve.
\item Application of DEC to a group of problems out of various fields.
\item Cookbook recipe ?
\end{enumerate}
What is not treated in this Thesis but should not be ignored:
\begin{enumerate}
\item Other standard methods like finite differences or finite element method, even thought DEC is closely related to those and one would benefit of a closer look at FEM and its error analysis.
\item An analysis how well DEC methods perform in contrast to these standard methods or any Error Analysis of FEM.
\end{enumerate}
\chapter{Manifolds and Meshes}

\begin{figure}[h]
\begin{center}
\includegraphics[height = 4cm]{imgs/2_1_ECVsDEC.eps}
\end{center}
\vspace{-0.5cm}
\caption{Exterior calculus is defined on manifolds, discrete exterior calculus on discrete manifolds. This chapter therefore covers manifolds and discrete manifolds.}
\end{figure}

Throughout this thesis we will always have to deal with two worlds. On one side there is the continuous world, where classical calculus and exterior calculus can be used to describe problems or physical relations. On the other side is the discrete world, where computational calculations can be done and the the smooth world is approximated.

This chapter focuses the geometric objects in these worlds. On the continuous side these are manifolds, which are the mathematical objects used to describe smooth curved spaces. Manifolds are treated in Section \ref{sec::2_Manifolds}. On the other side we have discrete manifolds, which are multiple dimensional meshes and are introduced in Section \ref{sec::2_discreteManifolds}. Having a correct  understanding of both discrete and continuous manifolds is vital to understand exterior and discrete exterior calculus.

The last section of this chapter is of a more practical nature. It can be used as a guide when implementing discrete manifolds for DEC or as a hands-on section to get the most out of reading this thesis. As application some geometric operations on meshes and discrete manifolds are described.

%So lets get started. In this chapter we will have a closer look at the basic objects we will deal with throughout this text.  We start with the description of so called ''manifolds'' which describe smooth surfaces and more generally smooth spaces embedded in higher dimensional spaces.
%In the second section we have a closer look at the meshes (and more generally Simplicial Complexes) that we use in practice in the place of manifolds. The third section explains how to implement meshes and simplicial complexes in a way that is convenient for DEC purposes. 


\begin{figure}[ht]
\begin{center}
	\begin{longtable}{|p{4cm}|p{4cm}|p{4cm}|}
		\hline
		Smooth Theory& Discrete Theory& Implementation (Notes)\\
		\hline
		Smooth Manifolds (Surfaces)
		\begin{packed_itemize}
		\item[-] Maps and Coordinates
		\item[-] Tangential Space
		\item[-] Orientations
		\item[-] Bordered Manifolds
		\end{packed_itemize}
		&
		Discrete Manifolds
		\begin{packed_itemize}
		\item[-] Simplices and Simplicial Complexes
		\item[-] Discrete Manifolds
		\item[-] Orientations
		\item[-] Border Operator, Border Matrix
		\end{packed_itemize}
		&
		Meshes
		\begin{packed_itemize}
		\item[-] Winged Edge Structure
		\item[-] General Complexes
		\item[-] Sparse Matrices
		\item[-] Simple Geometric Operations
			\begin{packed_itemize}
				\item[-] Manifold Check
				\item[-] Border Computation
			\end{packed_itemize}
		\end{packed_itemize} \\
%			Smooth Surfaces & Meshes & General Meshes\\
%			-Maps and Coordinates & & \\
%			-Tangential Space & -Simplices / Simplicial Complexes & -Winged edge / Incidence\\
%			-Orientations & -Orientations& -Simple geometric operations:\\
%			-Functions on Surfaces & -Border Operator & --Orientation\\
%			-Derivative on Surfaces & & --Iterating over Neighborhoods\\
%			& & --Wellformedness \\
%			& & --Finding Border Components\\
%			& & Notes on Sparse Matrices\\
		\hline
	\end{longtable}
	\caption{Overview of the topics of this chapter}
\end{center}
\end{figure}

\section{Manifolds}
\label{sec::2_Manifolds}
In this section we introduce smooth curved spaces, known as manifolds. Some simple two-dimensional manifolds (in short: 2-manifolds) are depicted in Figure \ref{fig::2_1_manifold}.

The following topics on manifolds need to be covered:
\begin{enumerate}
\item Describing bordered manifolds using local maps
\item Tangential spaces
\item Orientation of manifolds
\item The border operator and the orientation of borders
\item Differential structure on manifolds (derivatives of functions on manifolds)
\end{enumerate}

While no application in this thesis actually exceeds 3 dimensions, discrete exterior calculus (DEC) and exterior calculus provide tools for arbitrary dimensions, without any complications. So there is no reason to cover only two or three dimensional spaces. 
		
\subsection{Describing Manifolds}	
A two dimensional manifold is an object that locally looks like a plane and possibly has one or multiple (one dimensional) borders. Three two dimensional surfaces are given in Figure \ref{fig::2_1_manifold}.

We describe manifolds using local maps. A local map at some point on the manifold describes the manifold locally; in the case of 2-manifolds as a function of two parameters. There are two types of maps: maps that describe the manifold close to borders an maps that describe the manifold at inner points, as depicted in the Figure \ref{fig::2_1_mapping}. This is taken account of by allowing the local map to either describe the manifold using a plane or a halfplane. A 2-manifold then is an object that can be described at every point locally with a 2 dimensional map (with or without border). This makes sure that the manifold looks at every point like a plane or halfplane.


Generally a $k$-dimensional manifold $M$ lying in $\mathbb R^n$ is a geometric object that locally looks like $\mathbb R^k$ or like the halfspace $\mathbb H^k = \{x= (x_1,...,x_k) \in \mathbb R^k : x_k \geq 0\}$.  Formally this is done by guarantying that at every point on the manifold there is a 'map' linking the manifold to $\mathbb R^k$ or $\mathbb H^k$.
		
For the sake of simplicity we will assume through the whole text that all functions and mappings considered are infinitely differentiable.
		
\begin{figure}
	\begin{center}
		\includegraphics[height=4cm]{imgs/2_1_2dmanifolds.eps}
		\caption{Three simple 2-manifold; the surfaces of the sphere and the torus do not have any borders, the third surface has. The inner of the sphere and the torus would be 3-manifolds with border (their borders being their surfaces)}
		\label{fig::2_1_manifold}
	\end{center}
\end{figure}
		
\begin{definition}[Map] A $k$-dimensional map is a differentiable mapping 
\[\phi: U \subset \mathbb R^k \rightarrow \phi(U)\] 
\[\begin{pmatrix}
	u_1\\ \vdots \\ u_k
\end{pmatrix} \rightarrow \begin{pmatrix}x_1(u_1,...,u_k)\\x_2(u_1,...,u_k)\\ \vdots \\ \vdots \\ x_n(u_1,...,u_k)\end{pmatrix}\]
that is injective and whose Jakobimatrix has rank $k$ on all $U$, where $U$ is some open subset of $\mathbb R^k$ or $\mathbb H^k$ (Figure \ref{fig::2_1_mapping}).
		
\end{definition} 

A $k$-manifold then is an object where you can find local maps everywhere:

\begin{definition}[Manifold] A subset $S\subset \mathbb R^n$ is a (regular) $k$ manifold, if for each point $p \in S$ there exist an open set $V\subset \mathbb R^k$ such that there is a map $\phi: U \rightarrow  V\cap S$.
\end{definition} 

	
\begin{figure}
	\begin{center}
		\includegraphics[width=14cm]{imgs/2_1_borderedmanifold_maps.eps}
		\caption{An example of a bordered 2-manifold. Left: a local map $\phi$ at some inner point. Right: a local map at a border point. The map at the border has the additional restriction that the border of the halfspace $H^2$ is mapped to the border of the manifold.}
		\label{fig::2_1_mapping}
	\end{center}
\end{figure}

There are some important details in the definition of maps. The injectivity of the maps prevents self intersections, and the non zero Jacobi determinant makes sure that the image of a map $\phi$ does not degenerate. For example the image of a two dimensional map should not degenerate to a point or a line.

\subsubsection{Local Coordinates}

A local map $\phi: U \subset \mathbb R^k \rightarrow M$ assigns local coordinates to a manifold $M$. The values of the tuple $u=(u1_,...,u_k)\in U$ are said to be the local coordinates of the point $\phi(u)$ on the manifold, in the map $\phi$. A classic example of this is the sphere parametrized by two angles using a map like
\[\phi(a,b) = (sin(a)sin(b), sin(a)cos(b), cos(a))\]
This assigns coordinates $(a,b)$ just as longitude and latitude are used as coordinate for the world. (But note that you can not parametrize the whole sphere at once if the source domain of your map is an open set and the map has to be injective)

If you have a local map $\phi$ you can also express functions $f:M \rightarrow ?$ defined on the manifold in the local coordinates given by $\phi$. This means that you consider $f \circ \phi : U \subset \mathbb R^k \rightarrow ?$ instead of $f$, such that you can use the local coordinates $(u_1,...,u_k)$ as parameter of $f$ instead of the position on the manifold. %(Image : local coordinates..) 

%\subsubsection*{A Note on Coordinates}
%Note that there are two types of coordinates here. Lets look at a 2-Manifold. You can have the local coordinates given by and depending on the local parametrisations you chose (giving you (u,v) coordinates). Then you have the global coordinates of the surrounding space $\mathbb R^3$, assigning (x,y,z) coordinates to every point.
%
%But in the end all coordinates are just descriptive tools; the geometric object we describe is assumed to exist independently of all these parameters. The 'tricky' part then is to get independence of parametrizations when we want to describe geometric properties of our manifolds. Even to get a definition of 'differentiability' on the manifold that is independent of the selected parametrization one must be careful with the definition.
%
%In fact in later applications  we will play around with coordinates, (with surface smoothing in chapter ... the global coordinates, with conformal maps in chapter ... the local coordinates). Rather than taking coordinates as describing the manifolds we take them as something associated to the geometric object to work with them or solve for them.
%
%External calculus (introduced in chapter...) will provide tools and results that do not use parametrizations.
%

	
\subsection{Tangential Spaces}		
Manifolds have tangential spaces.
For curves calculating tangents is easy. If you have a parametrization $\alpha(t): I \subset \mathbb R \to \mathbb R^n$, then $\alpha'(t) = (x_1'(t),x_2'(t),...,x_n'(t))$ is the direction of a tangent vector to the curve at the position $\alpha(t)$. 

\begin{figure}[tb]
\begin{center}
\includegraphics[width=8cm]{imgs/2_1_tangent.eps}
\end{center}
\caption{A parametrized curve, the vector $\alpha'$ and the tangential space $T_{\alpha(t_0)}$ at $t_0$.}
\label{fig::2_1_paramCurve}
\end{figure}

		While the length of $\alpha'(t)$ at the point $\alpha(t)$ depends on the parametrization $\alpha$ (for example $\alpha(2t)$ is a different parametrization of the same curve where the length doubles) the \textbf{tangential space} $ T_{\alpha(t)}S = span(\alpha'(t)) = \{x \in \mathbb R^n: x = c \alpha'(t), c \in \mathbb R\}$ does only depend of the position $\alpha(t)$ on the curve, as depicted in Figure \ref{fig::2_1_paramCurve}. 
		
We can do the same for a $k$-dimensional manifold $M$. The tangential space $T_p M$ at a point $p$ is:

\begin{enumerate}
	\item The space that approximates the surface in the best way, locally at $p$.
	\item The space that contains the tangents of all curves on the surface that go through $p$.
	\item For a given parametrization $\phi: \mathbb R^k \to \mathbb R^n$, $\phi(u) = (\phi_1(u),...,\phi_n(u))$ the tangential plane $T_{\phi(u)}$ is given by
			\[span(\frac{\partial \phi} {\partial u_1},..., \frac{\partial \phi} {\partial u_k}) = span(\begin{pmatrix}
	\frac{\partial \phi_1} {\partial u_1} \\
	\frac{\partial \phi_2} {\partial u_1}\\
	\vdots\\
	\frac{\partial \phi_n} {\partial u_1}
\end{pmatrix},...,\begin{pmatrix}
	\frac{\partial \phi_1} {\partial u_k} \\
	\frac{\partial \phi_2} {\partial u_k}\\
	\vdots\\
	\frac{\partial \phi_n} {\partial u_k}
\end{pmatrix}),\]
as depicted in Figure \ref{fig::2_1_mapping_coords} for a 2-manifold. Here the restriction that maps have a non-zero Jacobi determinant plays a role, as it means that the partial derivatives are linearly independent.
\end{enumerate}

\begin{figure}[tb]
\begin{center}
\includegraphics[width=12cm]{imgs/2_1_mapping_coords.eps}
\end{center}
\caption{A map $\phi$ of a 2 manifold $M$ is used to determine the tangential space $T_{p}M$ at some point $p$}
\label{fig::2_1_mapping_coords}
\end{figure}

The tangential space $T_p M$ is the vector space containing all tangential vectors at a point $p$. Through the tangential space, every point on a manifold gets an associated vector space. While the vector space itself is depending solely on $p$, the choice of a basis for $T_pM$ is open. Usually  the basis vectors $\frac{\partial \phi}{\partial u_i}$  are chosen, according to some local map $\phi$. But, just as it is the case in Fig. \ref{fig::2_1_mapping_coords}, these vectors are in general not orthogonal or normalized. This is something to take care of.

One special example of a manifold is  the space $\mathbb R^k$ seen as a manifold parametrized by $\mathbb R^k$ with $\phi = id$. Then the tangential space $T_p \mathbb R^k$ at any point $p$ is again $\mathbb R^k$. But still: the vectors in one tangential space $T_p \mathbb R^k$ and the vectors of another tangential space $T_q \mathbb R^k$ can not be mixed; every point gets its own proper tangential space, not shared with any other point.
		
\subsection{Orientations}
We only want to consider a special kind of manifolds: orientable manifolds. Orienting a volume is to assign a sign to the volume you are treating. Either your volume is positive or negative. 

For a vector space you can encode orientation in the ordering of basis vectors. Two ordered bases $v_1,...v_k$ and $w_1,...,w_k$ describe the same orientation if the matrix that describes the change of bases has a positive determinant. The determinant measures the signed volume spanned by a set of vectors.

In the previous section we introduced tangential spaces and emphasized that every point gets its own proper tangential vector space. Tangential spaces of points that are very close together are very similar and it makes sense to ask them to have the same orientation.

%In the last section we introduced tangential spaces and 

We saw that parametrizations can provide bases for tangential spaces. One single parametrization induces consistent orientations to the tangential spaces of all points it hits. Therefore we say that a Manifold can be oriented if all tangential spaces can be oriented consistently.

\begin{definition}[Oriented Manifold] A manifold is orientable if there exists a set of maps $\mathcal A = \{\phi: U_\phi \to \phi(U_\phi) \subset M\}$ such that the maps describe the whole manifold and any two maps $\phi$, $\psi$ which describe a common patch $\psi(U_\psi) \cap \phi(U_\phi)$ result in the same orientations i.e. the base change matrix $C$ from the base $D\phi$ to $D\psi$ has a positive determinant
\[det(C) >0\]
A manifold is oriented if for all tangential spaces a consistent orientation has been chosen.

\end{definition}

For 2d manifolds in 3d space this is the same as asking that you can consistently chose a surface normal, as tried in Fig. \ref{fig::2_1_mobius}.

\begin{figure}[t]
\begin{center}
\includegraphics[width = 6cm]{imgs/2_1mobius.eps}
\caption{The Moebius strip, the pathological example of a non orientable manifold}
\label{fig::2_1_mobius}
\end{center}
\end{figure}

\subsection{The Border Operator}
The border operator describes a special geometric operation for manifolds. 
We denote the border of a manifold $M$ by $\delta M$ and call $\delta$ the border operator. From the definition of maps at border points follows that the border of a manifold is again a manifold, where the dimension decreases by one. And from the definitions also follows, that the border of a manifold always is a manifold without border, as it is the case with spheres or tori (again see Figure \ref{fig::2_1_manifold}); this means that for bordered manifolds
\[\delta\delta M = \emptyset.\]

A central point (considering DEC) is that an oriented manifold induces an orientation to its border. This is sketched in Figure \ref{fig::2_1_borderManifold}.  What follows is a short technical description of how the orientation on the border is defined formally. The border operator and the border orientations play a central role in external calculus and throughout this thesis. Therefore it deserves to be introduced properly.

 As the orientation of a manifold is defined by the orientation of its tangential spaces we need to take a closer look at the tangential spaces of bordered manifolds.
While nothing is special for tangential spaces at non-border points, at border points two tangential spaces are present. One is the tangential space of the manifold $T_pM$ and $k$ dimensional, the other one is the $k-1$ dimensional tangential space of the border manifold $T_p \delta M$ (see Fig. \ref{fig::2_1_borderManifold}, left) . Inducing an orientation to the border means inducing an orientation in $T_p\delta M$ using the orientation of $T_p M$. This happens by defining normals on the border.

\begin{figure}
\begin{center}
\includegraphics[width = 13cm]{imgs/2_1_borderedManifold_combined.eps}
\end{center}
\label{fig::2_1_borderManifold}
\caption{On bordered manifold two tangential spaces $T_pM$ and $T_p\delta M$  are present at border points; $N$ is the outward pointing border normal (left image). The right image depicts how the manifold oriented according to the base ($b_1$,$b_2$) of some tangential space induces an orientation to the border: $N$ and a vector following the border orientation have to build a basis oriented like $b_1,b_2$.}
\end{figure}

For any border point you can define a border normal $N$. The border normal $N$ is the vector in $T_p N$ with:
\begin{itemize}
\item $N$ is orthogonal to $T_p \delta N$
\item $N$ has length 1
\item $N$ points 'outside'
\end{itemize}
Pointing outside is defined formally using the map $h$ at the border; $Dh$ is a linear bijective map from $\mathbb R^k$ to $T_pM$, so $N$ can be pulled back to $\mathbb R^k$ and it points 'outside' if the $k$th component of $Dh^{-1} N$ is negative \note{(Image?)}.

We defined orientation by the enumeration of basis vectors. So if a basis $b_1,...,b_k$ gives the orientation of $T_pM$,
a basis $\widetilde{b_1},...,\widetilde{ b_{k-1}}$ of the tangential space of the border $T_p\delta M$ is oriented according to the manifold if prepending the normal $N$ to the basis $N,\widetilde{b_1},...,\widetilde{ b_{k-1}}$ has the same orientation as $b_1,...,b_k$. This is also shown in Fig. \ref{fig::2_1_borderManifold}.


\subsection{Functions and Derivatives on Manifolds}
\label{sec::2_derivativesOnMF}

Exterior calculus is about differentiation  and integration of functions and more general things on manifolds. In this section we explain how differentiation of mappings $f:M\to M'$ between two manifolds $M$ and $M'$ is done ON manifolds.  In our setting manifolds are more than just geometric objects; they become spaces where differentiation is possible, just as it is in $\mathbb R^n$. The manifolds get a \textbf{differential structure}.

\subsubsection{Derivatives}
Given a manifold $M$ and a function $f: M \rightarrow \mathbb R^n$, what is the derivative of $f$? We want the derivative to be something very similar to the derivative $Dh$ of a function $h: \mathbb R^k \rightarrow \mathbb R^n$. In this case $Dh$ is the linear mapping that locally approximates $h$ and can be used to give the directional derivative for a direction $v$.
\[h( p + tv) \approx h(p) + Dh \cdot tv\]
We want the same for functions $f$ on manifolds: $Df$ should be a linear mapping that maps a direction to a vector that describes the change of $f$ when going in that direction. A direction on a manifold at some position is a tangential vectors. This is important: the differential $Df$ is a mapping from the \emph{tangential spaces} to vectors, as depicted in Figure \ref{fig::3_1_manifoldDerivative}. 

We can express the idea that $Df\cdot v$ describes the change of $f$ in the direction $v$ readily by using a curve $\alpha (t)$ with a tangent $\frac{\partial \alpha(0)}{\partial t} = v$ in the wished direction $v$:
\begin{equation} Df \cdot v := \frac{\partial}{\partial t} f(\alpha(t)) \label{eq:2_1_derivativeDef}\end{equation}
As $f(\alpha(t))$ is simply a function $\mathbb R \rightarrow \mathbb R^n$ we know how to calculate the right hand side $\frac{\partial}{\partial t} f(\alpha(t))$. This is not very handy for any calculations; but we can express the derivative in the local coordinates given by a parametrization $\phi(u_1,...,u_k)$.

\begin{figure}
\begin{center}
\includegraphics[width= 12.5cm]{imgs/3_1_manifoldDerivative.eps}
\end{center}
\caption{Construction of a derivative of a real valued function $f$ defined on a manifold locally parametrized by $\phi$. $Df$ at a point $p$ is a linear mapping from the tangential space $T_p M$ to $\mathbb R$}
\label{fig::3_1_manifoldDerivative}
\end{figure}

As we have seen, a parametrization provides a base of the tangential space, namely 
\[\frac{\partial\phi}{\partial u_1},..., \frac{\partial\phi}{\partial u_k}\] 
Curves can be expressed in this map and tangential vectors can be described in this base: $\alpha(t) = \phi(u_1(t),...,u_k(t))$ and $\alpha'(t) = \frac{\partial\phi}{\partial u_1} u_1' + ... + \frac{\partial\phi}{\partial u_k} u_k'$. The function $f$ also has to be given in that map , i.e. 
\begin{eqnarray*} f(u_1,...,u_k) &=& f(\phi(u_1,...,u_k)) \\
 &=& f_1(\phi(u_1,...,u_k)),...,f_n(\phi(u_1,...,u_k)). \end{eqnarray*} 
Then 
\[Df \cdot \alpha'(t) = (\frac{\partial f}{\partial u_1},..., \frac{\partial f}{\partial u_k}) \cdot \begin{pmatrix}
	u_1' \\ \vdots \\ u_k'\end{pmatrix}\]
and $Df$ is described \emph{in the local coordinates given by $\phi$} by the $ n \times k$ matrix $(\frac{\partial f}{\partial u_1},..., \frac{\partial f}{\partial u_k})$. $(u_1',...,u_k')$ is the description of the tangential vector $v = \alpha'$ in the base $D\phi$ . 

The curve $\alpha$ was only used to construct the derivative; written without $\alpha$ the derivative $Df$ in local coordinates $\phi$ is:
\[Df \cdot v = (\frac{\partial f}{\partial u_1},..., \frac{\partial f}{\partial u_k}) \cdot \begin{pmatrix}
	v_1 \\ \vdots \\ v_k\end{pmatrix}\]
where the tangential vector $v$ is expressed in the base of the tangential space induced by $\phi$
\[v = v_1 \frac{\partial \phi}{u_1} +...+ v_k \frac{\partial \phi}{u_k}\] 
 %This situation is depicted in Figure \ref{fig::3_1_manifoldDerivative}.
\subsubsection*{Derivatives of Mappings between Manifolds}
\label{sec:derivativeBetweenMfs}
A slight generalisation is considering mappings 
\[f:M\to M'\]
going from one manifold $M$ to an other manifold $M'$, as shown in Figure \ref{fig::3_1_manifoldDerivative2}. If we look again at Equation \ref{eq:2_1_derivativeDef}, we see that, as $f(\alpha(t))$ is a curve on $M'$ and $\frac{\partial}{\partial t}f(\alpha(t))$ is a tangential vector to this curve, $Df\cdot v$ has to be a vector in the tangential space of $M'$. This means that the derivative $D_pf$ at some point $p$ is a linear mapping from the tangential space $T_pM$  to the tangential space $T_{f(p)} M'$, i.e. 
\[D_p f = T_p M \rightarrow T_{f(p)} M'.\] 

If $M$ is a $k$-manifold and $M'$ a $l$-manifold $Df$ can be expressed as a $k\times l$ matrix, described relatively to two sets of local coordinates $\phi \rightarrow M$ and $\psi \rightarrow M'$.

\begin{figure}
\begin{center}
\includegraphics[width= 13cm]{imgs/3_1_manifoldDerivative2.eps}
\end{center}
\caption{Two 2-manifolds $M$ and $M'$ with local parametrizations $\phi$ and $\psi$. $f$ is a function $f: M \rightarrow M'$. $Df$ at a point $p$ is a linear mapping from the tangential space $T_p M$ to $T_{f(p)} M'$, choosing $D\phi$ and $D \psi$ to parametrize the tangential spaces $Df$ can be represented as a $2\times 2$ matrix (relative to these bases)}
\label{fig::3_1_manifoldDerivative2}
\end{figure}



%\note{	Tangential Spaces and Differential structure. Maybe put it in next chapter; here its all about geometry, not about functions.}

\newpage
\section{Discrete Manifolds}
\label{sec::2_discreteManifolds}
In the last sections we had a look at the geometric objects exterior calculus will be defined on, i.e. smooth surfaces and manifolds. The next step is to introduce the discrete analogues we want to do computations with: triangle meshes, or more generally simplices and simplicial complexes. Simplices are for example points (0-dimensional), lines (1-dimensional) triangles (2-dimensional) and tetrahedra (3-dimensional). Simplicial complexes are 'meshes' made out of them. The definitions are taken from \cite{DMK08} and \cite{FRANKEL11}.

\subsection{Simplices and Simplicial Complexes}

\begin{figure}[t]
\begin{center}
\includegraphics[height= 2cm]{Imgs/2_3_simplices.eps}
\end{center}
\caption{A 0-simplex (point) 1-simplex (line) 2-simplex (triangle) and 3-simplex (tetrahedron) }
\label{fig::2_3_simplices}
\end{figure}

A $k$-simplex is the most basic geometric object with a $k$-dimensional volume: the convex hull of $k+1$ points, as depicted in Fig. \ref{fig::2_3_simplices}. No point should lie in the convex hull of the others; else no $k$-dimensional volume is spanned and the simplex is called degenerated.

\begin{definition}[Simplex] A non degenerated $k$-simplex is the convex hull of $k + 1$ points $p_1,...,p_{k+1}$, where the vectors $p_2 -p_1, p_3,-p_1, ..., p_{k+1} -p_1$ are linearly independent. It is represented as a tuple of its corner vertices $\{p_1,...,p_{k+1}\}$.
\end{definition}

Every simplex has faces of various dimensions: any combination of $l+1$ of its corner vertices forms an $l$-dimensional face. For example a tetrahedron has 4 2-dimensional faces (triangles), 6 1-dimensional faces (edges) and 4 0-dimensional faces (vertices),see Figure \ref{fig::2_3_simplices}. A 4-simplex would have 5 tetrahedral faces and so on.

Out of simplices one can build simplicial complexes, in the same way as meshes are built out of triangles. The restrictions are the usual: the interior of any two simplices should not overlap, and if the intersection of two simplices is not empty, the intersection has to be a face of both simplices. A simplicial complex then is a list of simplices, following these restrictions. 

If a simplicial complex contains a  $k$-simplex $\sigma$, we also demand that all faces of $\sigma$ are part of the simplicial complex. This is not just a tedious technical detail;  we explicitly want to associate different values to all faces of simplices. In a triangle mesh, for example, we will need to keep track not only of triangles and vertices but also of the edges.

\begin{definition}[Simplicial Complex]
A simplicial  complex is a collection $\kappa$ of simplices, such that if a simplex is contained in $\kappa$, all its faces are too. Furthermore the intersection of any two simplices in  $\kappa$ is either empty or a common face.
\end{definition}

Lastly we do not want our discrete manifolds  to have the analogue of dangling triangles (Figures \ref{fig::2_2_dangling} and \ref{fig::2_2_dangling2}). To ensure this formally one has to make a restriction that is similar to the definition of manifolds. Just as we ensured that a $k$-manifold locally looks like $\mathbb R^k$ or $\mathbb H^k$, we want to make sure our discrete manifold looks locally like either a $k$-dimensional ball or a $k$-dimensional half-ball. This gets rid of dangling things.

\begin{figure}
\begin{center}
\includegraphics[height=2.5cm]{imgs/2_2_dangling.eps}
\caption{These are not discrete 2-manifolds: the first mesh has a dangling triangle, the second mesh has a 'wheel' and is not locally equivalent to a plane, the same holds for the third mesh}
\label{fig::2_2_dangling}
\end{center}
\end{figure}

\begin{definition}[Discrete Manifold]
A $k$-dimensional discrete Manifold is a simplicial complex where for every vertex in $\kappa$ the union of all incident simplices is equivalent to a $k$-dimensional ball or a $k$-dimensional half ball.
\end{definition} 

On discrete manifolds we can define orientations and a border operator with the same geometric meaning as on smooth manifolds.

\subsection{Orientations}
\label{subsec:SC_orientations}
As on manifolds we can treat orientations on discrete manifolds; they are quite of some practical importance and a notorious source of switched sign errors when implementing things using discrete exterior calculus. 

We can assign one of two orientations to any simplex of any dimension, meaning that the volume represented by the simplex should be considered as positive or negative. While we coded orientation before via the enumeration of basis vectors, for simplices we encode orientation via the enumeration of their corner vertices.  For edges it is the most intuitive what this means: we assign a direction to the edge $\{p_1,p_2\}$ by saying the first vertex listed is the start vertex of the edge. Note that for an edge or any geometric object there is not a strict 'positive` or a 'negative` orientation; we can only say how something is oriented relative to something else. For example the edge $\{p_1,p_2\}$ is oriented negatively to the edge $\{p_2,p_1\}$; this is noted as
\[-\{p_1,p_2\} = \{p_2,p_1\}.\]
So the orientation of a $k$-simplex depends on the way its corner vertices are enumerated. Two enumerations of corner vertices result in the same orientation if they are related by an even permutation. A permutation is called even, if it can be reproduced by switching pairs of vertices an even number of times. E.g.
\[\{a,b,c,d\} = \{c,a,b,d\}\]
\[\{a,b,c,d\} \rightarrow \{c,b,a,d\}\rightarrow \{c,a,b,d\}\]
by using two swaps,swapping  $a$ and $c$ and then $a$ and $b$.
You can also use the determinant to determine the sign of the permutation; just calculate the determinant of the permutation matrix
\[\{a,b,c,d\} \rightarrow \{c,a,b,d\}\]
\[\begin{pmatrix}c\\a\\b\\d \end{pmatrix}=\begin{pmatrix} 0 & 0 & 1 &0 \\ 1 &0&0&0 \\ 0&1&0&0 \\ 0&0&0&1 \end{pmatrix}\begin{pmatrix} a\\b\\c\\d \end{pmatrix}\]
Or again you can use the simplex to induce a base to the affine space it is aligned to
\[p_1 -p_2,...,p_{k}-p_{k+1}\]
and two enumerations induce the same orientation if these bases have the same orientation. This also shows that defining the orientation of a simplex by looking at the ordering of its corner vertices amounts to the same as orienting volumes by choosing bases.

\begin{figure}
\begin{center}
\includegraphics[height=2.5cm]{imgs/2_3_danglingTetrahedra2.eps}
\caption{If tetrahedra are not connected by two dimensional faces, they are 'dangling�.}
\label{fig::2_2_dangling2}
\end{center}
\end{figure}


One exception are vertices or 0-simplices $\{v_0\}$, where orientation is not encodable in the enumeration of the vertex. We need to assign orientations to single points too and say that $-\{v_0\}$ is the negatively oriented version of $\{v_0\}$. Orientation is 'imprinted` on the point. The best way of thinking of orientation is that orientation adds a sign to volumes. A negatively oriented point is then a point whose $0$-dimensional volume is negative. The $0$-dimensional volume of any single point is defined to be either $1$ or $-1$ and the 0 dimensional volume of a point set is $\#positive\; points - \#negative\; points$.

As long as you stick with calculations in $\mathbb R^3$ it stays pretty simple to determine if two orientations a simplex are the same, if you stick with triangle meshes it is trivial. Just make sure you always remember to respect orientations. In Section \ref{sec::2_handsOnSimplicialComplexes} we will also come back to the question of how to compute relative orientations in practice.


\subsection{The Border Operator}
\label{sec::2_borderOrientation}

Just as with manifolds we have a border operator for discrete manifolds. And just like with manifolds an oriented discrete manifold induces an orientation to its border.

We first introduce some notation: we can respresent collections of simplices as \emph{formal sums}, as depicted in Figure \ref{fig::2_2_formalsum}. The simplices are represented as tuples, a negaitve sign means a change of orientation. Simplices that are different from each other are not summed up, only if two tuples describe the same simplex but for orientation, the sum is taken. Particularly simplices of opposed orientation cancel out.
\begin{figure}
\begin{center}
\includegraphics[width = 12cm]{imgs/2_2_formalsum.eps}
\end{center}
\caption{Two sets of edges expressed as formal sums that get summed up}
\label{fig::2_2_formalsum}
\end{figure}

The border of a single $k$-simplex is  the following formal sum
\[\delta\{v_0,v_1,...,v_k\} = \sum_{j=0}^k (-1)^j\{v_0,...,\widehat{v_j},...,v_k\}\]
where  the $\widehat{v_j}$ means omitting $v_j$. This expresses that the border of the simplex is a set of $k-1$ simplices. The orientations they get are the ones that the simplex induces. Note that prepending the omitted vertex $v_j$ to $(-1)^j\{v_0,...,\widehat{v_j},...,v_k\}$ leads to a simplex with the orientation of $\{v_0,v_1,...,v_k\}$. This is consistent with the way we defined that orientation should be induced to borders of smooth manifolds. 

For example the border of a triangle $\{a,b,c\}$ is \[\{b,c\} -\{a,c\} + \{a,b\} = \{a,b\} + \{b,c\} + \{c,a\},\] just as it should be.

But if we can take the border of single simplices, we can also take the border of a set of simplices or of discrete manifolds; it is simply the formal sum of the borders of the $k$-simplices the discrete $k$-manifold is made out of. As you see in the Figure \ref{fig::2_2_borderUnoriented} this can go 'wrong` when the discrete manifold is oriented inconsistently. 

\begin{figure}
\begin{center}
\includegraphics[width = 6cm]{imgs/2_2_borderUnoriented.eps}
\end{center}
\caption{The border operator that respects orientation only makes sense with oriented discrete manifolds. Orientation of faces is depicted by an arrow that says what orientation a simplex induces to its border}
\label{fig::2_2_borderUnoriented}
\end{figure}

\subsubsection{The Border Operator as a Matrix}
If all simplices occurring in a complex are enumerated and have a fixed orientation, the border operator can be expressed as a matrix, the incidence matrix. A set of $j$ simplices is represented as vector of integers of dimension $\#(j\;simplices)$. The $k$-th entry of this vector represents the number of times the $k$-th $j$-simplex occurs. 
For example in the following enumeration
\begin{center}
\def\svgwidth{13cm}
\input{imgs/2_2_complexEnumeration.tex}
\end{center}
the set of edges $e0 - e1 + e2 $ is represented by the vector $(1,-1,1,0,0)$.
In this example there are two border matrices, one to compute the border of edge sets 
\[\delta_1 = \begin{pmatrix}
-1&-1&0 &0 & 0\\
0&1&1 &1 & 0\\
1&0&-1 &0&1\\
0&0&0&-1&-1\\
\end{pmatrix} \]
and one to compute the border of face sets:
\[\delta_2 = \begin{pmatrix}
1 & -1 & 1 &0&0\\
0& 0& -1 & 1 & -1
\end{pmatrix}^T\]
For example the border of the line segment $e_0 -e_4 + e_3$ is then given by
\[\begin{pmatrix}
-1\\ 1\\ 0 \\0 
\end{pmatrix} = \delta_1\begin{pmatrix}
1\\0\\ 0\\1\\-1
\end{pmatrix}\]
which is $-v_0 + v_1$, saying that $v_0$ is the 'start` and $v_1$ the 'end` border of the line. 

For a $k$-complex there is a total of $k$ border matrices: one border operator for sets of 1-simplices (edges), one for 2-simplices (triangle faces), one for 3-simplices and so on. We will always make the difference between these border operators; we add a $j$ as subscript to the border operator of $j$-simplices : $\delta_j$. 

The entry $(i,j)$ in a border matrix is the relative orientation of the two simplices concerned. For example $\delta_1(0,1) = -1$ because the vertex $v_0$ is oriented negatively relative to the edge $e_1$, considering the border induced orientation.

\subsection{Oriented Discrete Manifold}
\label{sec::2_orientedDiscreteMF}
Lastly we can not only orient single simplices, but also a whole discrete manifold. This leads to oriented discrete manifolds, which are the discrete analogue to smooth oriented manifolds. 

The orientation of a volume is strongly linked to the orientation of borders. For convenience we will define well orientedness of a discrete manifold using the border orientations.
Two $k$-simplices that share a $k-1$ dimensional face are oriented consistently exactly if the induced orientation of this face is opposed for both $k$-simplices, as depicted in Figure \ref{fig::2_2_borderUnoriented}.
A $k$-manifold is oriented if all-$k$ simplices are oriented consistently. 

\subsection{Summary}

\begin{figure}%
\begin{center}
\includegraphics[height=3cm]{imgs/2_2_snoothVSdiscreteBorderOp.eps}	
\end{center}
\caption{Applying the smooth border operator to a $k$-manifold returns a $k-1$-manifold, the same holds in the discrete setting. Also for \emph{oriented} smooth and discrete manifolds $\delta\delta = 0$ holds.}%
\label{fig:2_2_snoothVSdiscreteBorderOp}%
\end{figure}
Discrete manifolds are geometric objects made out of simplices and allow the definition of orientation and border operators, just like smooth manifolds. Also, applying the discrete border operator twice to an oriented discrete manifold leads to an empty set:
\[\delta_{j-1}\delta_{j} = 0\]
This mirrors the property of the smooth border operator. The smooth and the discrete border operator are depicted schematically in Figure \ref{fig:2_2_snoothVSdiscreteBorderOp}.

Subsets of $j$-simplices of a discrete $k$-manifold can simply be represented by vectors and the border operators by matrices.  Up to now only the geometry of smooth manifolds has been discretized; to develope an analogue to differential calculus on manifolds we will first need to introduce differential forms.


\newpage
\section{Implementation: Mesh and basic operations}
This is the hands-on part of this chapter. The implementation chapters provide a guideline of what you need to implement to get DEC and the later applications up and running. The components needed are described and some of the more tricky details are mentioned.

\subsection{A word on Sparse Matrices}
The point of DEC is to reformulate differential equations with sparse matrices. Therefore the implementation is somewhat centered around sparse matrices.

If you plan to implement your DEC framework you should start by looking for a sparse matrix solver. For all results in this thesis the sparse Solver from the Pardiso-Project of the University of Basel has been used as a black box solver. Unfortunately it is not free-ware but any other sparse solver will do as well.

Sparse matrices are matrices where most entries are zero. Instead of storing all $n\times m$ values of a matrix, you choose to only store the non-zero values and their indices. There are different ways to do this; the pardiso solver uses the so called Yale format.

The Yale format uses 3 vectors to describe an arbitrary $n\times m$ matrix $A$. The first vector $a$ stores all non-zero values of $A$, enumerated by row. The second vector $ja$ stores the column indices of the non-zero values, again enumerated by row. The third vector $ia$ stores for every row the index $i$, such that $a(i)$ and $ja(i)$ describe the value and the row of the first element in the row. Additionally one appends the number of values in $a$ (or $ja$) to $ia$.

For example 
\[\begin{pmatrix}
1 & 0 & 0 &3 \\
 0 & 0 & 0 &2 \\
 0 & 4&2&0
\end{pmatrix} \Rightarrow \begin{cases} a &= [1,3,2,4,2] \\ ia &= [0,2,3,5]  \\ ja &= [0,3,3,1,2]\end{cases}\]
Iterating over the values and indices of the $k$th row then amounts to
\begin{algorithmic}
\FOR{i = ia(k):ia(k+1)}
	\STATE out $\gets (k,ja(i))$   //the index pair
	\STATE out $\gets a(i)$  //of this value
\ENDFOR
\end{algorithmic}

Whatever implementation of sparse matrices you choose to use or implement, the usual basic operations need to be implemented:
\begin{itemize}
\item multiplication of matrices
\item transposing matrices
%\item iterate over the column indices and values of any row
\item adding matrices
\item inverting the elements of a matrix (replace the non zero values $a$ by $1/a$)
\item multiplication of vectors
\end{itemize}

\subsection{Implementing a Mesh for DEC}
\label{sec::2_handsOnSimplicialComplexes}
Most application in this thesis focus 2-complexes i.e. classical triangle meshes, you might not need any more general implementation so we treat general $k$ complexes separately in the next section. 

For DEC we need the complete geometric information of meshes; we explicitly keep lists of vertices, edges and faces, the full information about their incidence and border relations, as well as their assigned orientations. Edges are stored once, with an arbitrary chosen orientation.

For 2d meshes a winged edge structure is a convenient choice of representation. You could also choose to represent the mesh just by keeping lists of vertices, edges and faces plus the incidence matrices, as mentioned in the next section. \note{There might be better choices... what? References?}

\begin{figure}[tb]
\begin{center}
\includegraphics[width=8cm]{imgs/2_1_wingedEdge.eps}%	
\end{center}
\caption{The information stored on a winged structure edge}%
\label{figs::2_1_wingedEdge}%
\end{figure}

In a winged edge structure you have the following three objects:
\begin{figure}[h]
\begin{center}
\includegraphics[width = 11cm]{imgs/2_2_wingedEdge2.eps}
\caption{Impementation of a winged edge structure}
\end{center}
\end{figure}

With this information present it is easy to do things like iterating over the incident edges or faces of a vertex.

%\subsection{Implementing the Border Operator}
%\note{mention this here or include it in the border section?}


%\newpage
\subsection{Implementing k-Simplicial Complexes for DEC}

\begin{figure}%
	\begin{center}
	\includegraphics[width=10cm]{imgs/2_1_Complex.eps}%	
	\end{center}
	\caption{Implementation of a $k$ Complex; use tuple of ordered indices to characterize a simplex}%
	\label{fig::2_1_Complex}%
\end{figure}

Chances are you do not need simplicial complexes of higher dimensions other than tetrahedral meshes embedded in $\mathbb R^3$. 
	
Never the less one straight forward and for DEC suitable way to implement arbitrary $k$ complexes is to store lists of simplices and represent the incidence information explicitly as sparse matrices. The incidence matrices (border operator matrices) play a central role in DEC and need to be set up anyway.

An implementation of a $k$-complex then might look like this: the vertices (0-Simplices) are stored in a list and contain their positions. A single $j$-Simplex is then represented by a $j$ tuple of vertex indices. A $k$ complex consists then of $k+1$ simplex lists; for every dimension one list, as sketched in Figure \ref{fig::2_1_Complex}.

Setting up the border operator matrices $\delta_j$ for complexes of arbitrary dimensions is not completely trivial, as to compute relative orientations of simplices you need to find the sign of some permutation. It gets much easier if the index tuples describing the simplices are sorted i.e.
\[(i_1,i_2,...,i_j): i_1 < i_2 <...<i_j.\] 
With sorted indices we can directly use the definition of the border operator from Section \ref{sec::2_borderOrientation} to compute the relative orientation of a $j-1$ simplex $(v_0,...,\widehat{v_l},..., v_{j-1})$ lying on the border of a $j$ simplex $(v_0,...,\widehat{v_l},..., v_{j})$:
\[Orientation = (-1)^l\]

But while it is easier to compute relative orientations if the indices of your simplices are sorted, you loose the ability to store arbitrary simplex orientations using the ordering of vertices. For all but the $k$ simplices this does not matter, even for oriented discrete $k$-manifolds, as there is nor 'right' or 'wrong' orientation and all that matters is that you consistently use the same orientation all the time. But for the $k$ simplices in an oriented $k$ manifold you really need to be able to chose the orientation, so you have to keep track of the orientation independently in an additional variable (as is done in Figure \ref{fig::2_1_Complex}).

So when you resort the tuple of a $k$-simplex in a $k$-manifold you need to determine if an index tuple describes the same orientation as the sorted index tuple. This can be done using a so called inversion table. Lets say the tuple $(1,2...,n)$ is scramble to the tuple $(i_1,...,i_n)$. An inversion is then an index pair $(i_l,i_k)$, where $l<k$ but $i_l >i_k$, i.e. the order of $i_l,i_k$ was inverted. The number of inversions of a single index is then the number of indices left to it that are greater than the index. The relative orientation of a simplex represented by a scrambled tuple to a simplex with the sorted tuple is $(-1)^{\#inversions}$ where $\#inversions$ is the total number of inversions.

Example: 
\begin{eqnarray*}
3,2,5,4,1 \\
0,1,0,1,4
\end{eqnarray*}
The first line represents the permuted indices the lower the number of inversions of every index. The total number of inversion is 6 and the relative orientation of
$\{3,2,5,4,1\}$ to $\{1,2,3,4,5\}$ is $(-1)^6 = 1$ which would meant that both tuples represent simplices with the same orientation.

Setting up a $k$ complex  and all the border matrices $\delta_j$ might then look like this: start with the $k$ simplices; resort their indices if needed and adapt the stored orientation.
Then enumerate all occuring $k-1$ simplices (sort their indices, you do not need to adapt the orientation) and set up the border matrix $\delta_k$. Proceed with enumerating all $k-2$ simplices occuring as borders of $k-1$ simplices and the set up of the $\delta_{k-1}$ matrix and so on.

		
\subsection{Implementing Basic Mesh Operations}
To get later applications working you will need most of the following tools. For the applications presented in this thesis they do not need to perform extremely fast, as these operations will occur only once when setting up an application.

\subsubsection{Setup a DEC mesh from a Wireframe Mesh Representation}
The usual representation of meshes as e.g. with .obj files is by just giving a list of vertex positions and a list of faces; you need to set up the winged edge structure or $k$ complex from these.
	
\subsubsection{Set Up Border Matrices}
If you chose a winged edge structure to implement 2-complexes, you additionally need to set up the border matrices $\delta_2$ and $\delta_1$. Store them directly with your mesh. These matrices play a central role in DEC; you could say that this whole thesis is about these matrices, so test them well. If you have an oriented discrete manifold, you can also use the relation $\delta_{k}\cdot \delta_{k+1} = 0$ (the border of the border of an oriented manifold is empty) to check their correctness. Implementing the other tools in this section will test these matrices further.

\subsubsection{Check if a DEC Mesh is a Discrete Manifold}
\begin{figure}[tb]
	\begin{center}
	\includegraphics[width=12cm]{imgs/2_3_danglingTeapot.eps}
	\end{center}
	\caption{A teapot mesh that on the first look seems to be a discrete borderless 2 manifold but turns out to be a mesh with border and dangling triangles, which makes it a non-manifold mesh and therefore not suited for some DEC applications}
\end{figure}

To avoid singular matrices and to eliminate the possibility that bugs occur due to the ill-formedness of a $k$-complex or mesh (as in Fig. \ref{fig::2_2_dangling} and \ref{fig::2_2_borderUnoriented}) it is handy to have such a test method. I.e. you should test for orientation errors and the connectedness to avoid dangling simplices.
The orientation needs to be checked for $k$ simplices only, which would be for example the faces for a 2D-mesh. 

That a $k$-complex is oriented can be checked by looking at $\delta_k$. Any column has to have either exactly one entry or two entries that sum up to zero. This checks exactly the condition we gave in \ref{sec::2_orientedDiscreteMF}: Any 1 simplex is either on the border (therefore being part of exactly one $k$ simplex) or between 2 $k$-simplices, having once positive and once negative orientation.

If you are sticking with the winged edge mesh, finding dangling faces is fairly simple; at every vertex iterate over all edges and make sure that exactly 2 or no edges have only one neighbour face.
\begin{figure}
	\begin{center}
	\includegraphics[width=7cm]{imgs/2_3_danglingTriangles.eps}
	\vspace{0.5cm}
	
	\includegraphics[width=10cm]{imgs/2_3_DanglingTetrahedra.eps}
	\end{center}
	\caption{Top: On the left all 2-simplices (triangles) can be reached indirectly by hopping from triangle to triangle where every hopping pair shares a 1-simpex (considering only triangles that neighbour the marked vertex). On the right the manifold property is violated. Bottom: two 3-simplices that share a common face (no dangling), that share an edge (dangling) and that share a vertex (dangling)}
	\label{fig::2_3_dangling}
\end{figure}
Detecting dangling $k$-simplices in a $k$-complex needs slightly more work. Formally dangling was prevented by asking that at any vertex, the incident $k$-simplices form either a ball or a half ball. This is equivalent to asking that any two $k$ simplices neighbouring some vertex $v_0$ are connected via $k$ simplices neighbouring $v_0$ that share $k-1$ faces, as depicted in Figure \ref{fig::2_3_dangling}.  So to check wellformedness at a vertex, take all neighbour $k$-simplices, choose one and put it on a stack. Pop the stack and push all $k$-simplices that share a $k-1$ face with the popped simplex. Keep on popping and pushing like this. If any $k$ simplex remains you have a dangling situation. \note{(Clear enough?)}
	
	
Note that these non-manifold detection algorithms will not detect non-manifoldness due to self intersections (like the most right situation in Fig. \ref{fig::2_2_dangling}), as such problems do not show in the incidence matrices. But for most applications it is enough that we have well-formed incidence matrices and do not care if such self intersections occur.
	
\subsubsection{Find Borders}
Given an oriented discrete $k$-manifold: find the $(k-1)$-complex that represents its border. Finding the border can be easily done by applying the border Operator $\delta_k$ to the $(1,1,1,1,1...)$ vector. The resulting vector then exactly represents the border manifold. This also tests the correctness of your border matrix and the correctness of the orientation of your discrete $k$-manifold.
	
\subsubsection{Get Connected Components of a set of K-Simplices}
It also comes in handy to be able to identify the different connected components of a mesh, as the different components a mesh often need to be treated independently from each other as separate objects.
	
Given a list of $k$-simplices which forms one or more discrete manifolds you should identify the different connected components. You can also apply this
to the border of a $k$ manifold with multiple borders to get the various border components as in Figure \ref{fig::2_3_bunnyBorder}.
	
\begin{figure}[t]
	\begin{center}
	\includegraphics[width=7cm]{imgs/2_3_bunny_borders.eps}
	\end{center}
	\caption{A mesh with multiple border components.}
	\label{fig::2_3_bunnyBorder}
\end{figure}

	

\input{Chapters/TwoDimensionalSurfaces.tex}
\section{Differential Structure and Differential Calculus on Manifolds}

What we have seen up to here actually amounts to defining manifolds and providing them a differential structure. This means that manifolds become more than geometric objects that are measured and described. They become spaces on which you can define functions for which derivatives and other operations can be calculated directly ON the manifold. While all operations are defined by mapping everything back to $\mathbb R^n$, this differential structure exists on its own right.

The differential structure is 'glued' on a manifold using maps, as seen in the subsection Derivatives on Surfaces. Differentiability is then defined in the following way:

\begin{definition}[Differentiability] If $M^k$ and $N^l$ are two manifolds and $f: M^k \rightarrow N^l \subset \mathbb R^n$ is a continuous mapping we call $f$ differentiable if for every map $h:\mathbb R^k \rightarrow M^k$ the map $f \circ h :  \mathbb R^k \rightarrow \mathbb R^n$ is differentiable.
\end{definition}

As seen the differential of a function at some point is a linear mapping between two tangential spaces.
\[D_pf: T_p M^k \rightarrow T_{f(p)} N^l\]
A slightly more abstract way of viewing this is looking at $T M^k$ : the space of all tangential spaces, called the tangential bundle. Given a map $h:\mathbb R^k \rightarrow M^k$ we can get a local map of the tangential bundle 
\[h_* : \mathbb R^k \times \mathbb R^k \rightarrow T M^k\]
\[h_*(x,v) = (h(x), Dh\, v)\]
suggesting that the tangential bundle actually is $2k$-manifold. The differential of a function $f: M^k \rightarrow N^l$ then is a mapping between the two tangential bundles $T M^k$ and $T N^l$ which are a $2k$ and a $2l$ manifold.

For our first steps of differential calculus on manifolds we also need vector fields (which we will later generalize to Forms).  A vectorfield $\mathcal V$ on a manifold $M^k$ is simply the assignment of a vector from the tangential space to every point of $M^k$.
\begin{definition}[Vector Field]
A vector field is an assignment $\mathcal V: M^k \rightarrow T M^k$ with $\mathcal V(x) \in T_x M^k$.
\end{definition}
Using that the tangential bundle actually is a manifold with a differential structure we can ask from a vector field that it is smooth or differentiable.

We can then define a vectorfield in a coordinate/ map dependent way. We do this with a simple example taken from [Thomas Friedrich, Global Analysis].

\subsubsection{An example Vector Field} Taking $\mathbb R^2$ as a manifold parametrized by the identity map $\phi(u,v) = ID$, i.e. para-metrized with euclidean coordinates. The tangential spaces get the bases $\frac{\partial \phi}{\partial u}$ , $\frac{\partial \phi}{\partial v}$ which form simply the standard basis $(1,0), (0,1)$ at any point. We can use the (for now a bit alianating) notation $  \frac{\partial}{\partial u}$, $\frac{\partial}{\partial v}$ for the two bases vectors, dropping the $\phi$. We can the define a vector field
\[\mathcal V(u,v) = u \frac{\partial }{\partial v} - v \frac{\partial }{\partial u}.\]
Note that $\frac{\partial }{\partial v}$ really only is a strange notation for $\frac{\partial \phi}{\partial v} = (0,1)$ and the same for $\frac{\partial }{\partial u}$. We can now try to express the vector field $\mathcal V$ in a different map; in polar coordinates:
\[h(r,\omega) = (r \cos (\omega), r \sin (\omega))\]
The base of an arbitrary tangential space induced by this map is then
\[\frac{\partial}{\partial r} = (\cos(\omega), \sin(\omega)) \]
\[\frac{\partial}{\partial \omega} = (-r \sin(\omega), r\cos(\omega))\]
again using the fancy notation $\frac{\partial}{\partial r}$ to denote a vector. Expressed in euclidean coordinates given by $\phi$ these two vectors are
\[\frac{\partial}{\partial r} = \frac{1}{\abs{(u,v)}} ( u \frac{\partial }{\partial u} + v \frac{\partial }{\partial v} ) \]
\[\frac{\partial}{\partial \omega} = u \frac{\partial }{\partial v} - v \frac{\partial }{\partial u}\]
such that $\mathcal V$ expressed in polar coordinates is simply
\[\mathcal V = \frac{\partial}{\partial \omega}\]

\note{Image}

\subsection{Derivatives, Vectorfields and Differential Operators}
On manifolds you can calculate derivatives relative to a vector field. This is simply the directional derivative relative to the vector field's direction. Given a function $f$ we denote the derivative with respect to the vector field $\mathcal V$ as $\mathcal V (f)$. Geometrically we already did that, if $\alpha$ is a curve with $\alpha(0) = p$ and $\alpha'(0)= v = \mathcal V (p)$, then
\[\mathcal V (f) (p) = \frac{\partial }{\partial t} f \circ \alpha(t)\]
Now if $\mathcal V(u)$ is written in some map $\phi$ as $\sum_i v_i(u) \frac{\partial}{\partial u_i}$, again using the $\frac{\partial}{\partial u_i}$ as fancy notations of the induced base vectors the derivative with respect to the vector field becomes
\[\mathcal V (f) = \sum_i v_i(u)\frac{\partial(f \circ \phi)}{\partial u_i}\]
which motivates the 'strange' vector notation. \note{further reasoning needed?}

\subsection{Riemannian Metric}
From standard calculus we are used to that the gradient of a function $f:\mathbb R^k \rightarrow \mathbb R$ is a vector field with vectors pointing in the direction where $f$ has the largest increase. But in fact the gradient of a function $f$ is in fact only a linear mapping that approximates $f$ via $f(x + h) \approx f(x) + grad_x(f) (h)$ and the vector is merely a representation of the gradient. What actually happens here is that we represent the gradient using a vector AND a scalar product:
\[grad_x(f) (h) = \langle grad, h\rangle\]

To do the same on manifolds $M^k$ we need a scalar product on all tangential spaces. As we consider manifolds as objected embedded in a higher dimensional space $M^k \subset \mathbb R^n$ all tangential spaces are subspaces of the embedding space such that they inherit a scalar product. So if $\phi: \mathbb R^k \rightarrow M^k \subset \mathbb R^n$ is a local map inducing the local bases $\frac{\partial \phi}{\partial u_i}$, $i= 1...k$ to the tangential spaces and we have two vectors $v, w$ in some tangential space $T_p M^k$ expressed in the local bases as
\[v= v_1 \frac{\partial \phi}{\partial u_1} +...+ v_k\frac{\partial \phi}{\partial u_k} \]
\[w = w_1 \frac{\partial \phi}{\partial u_1} +...+ w_k\frac{\partial \phi}{\partial u_k}\]
the scalar product induced by the embedding space is
\[\langle v,w \rangle = \sum_{i,j = 1}^k v_iw_j\langle \frac{\partial \phi}{\partial u_i},\frac{\partial \phi}{\partial u_j}\rangle.\]
So if $v = (v_1,...,v_k)$ and $w = (w_1,...,w_k)$ in the map induced base, the induced scalar product is represented by the matrix
\[G= \begin{pmatrix}\langle \frac{\partial \phi}{\partial u_1},\frac{\partial \phi}{\partial u_1}\rangle &\cdots& \langle \frac{\partial \phi}{\partial u_1},\frac{\partial \phi}{\partial u_k}\rangle \\
\vdots &&\vdots\\
\langle \frac{\partial \phi}{\partial u_k},\frac{\partial \phi}{\partial u_1}\rangle &\cdots& \langle \frac{\partial \phi}{\partial u_k},\frac{\partial \phi}{\partial u_k}\rangle \end{pmatrix} = (D\phi)^T D\phi\]
This set of scalar products that is consistently defined for all tangential spaces $T_p M^k$ is the so called Riemannian metric.

Equipped with the Riemannian metric we can define a gradient and a gradient vector field for functions as well as divergience and a laplacian. \note{We will generalize these later but it is worth seeing them once for themselves before the generalization. TODO}

Note the Riemannian metric can also be used to measure volumes and angles. Angles are quite obvious; if we have two curves $\alpha$ and $\beta$ intersecting at some point $p$ with tangents $v$ and $w$, the angle between the curves is $\langle v,w\rangle$. The volume comes from the fact that $det(A A^T)$ equals the volume spanned by $A$'s row vectors squared. The determinant $det(G)$ then is the volume spanned by the column vectors of $D\phi$ squared and the $k$-dimensional volume of some subset $\phi(U) \subset M^k$ covered by a map $\phi$ is
\[vol(\phi(U))= \int_U \sqrt{det(G)}\;du_1,...,du_k\]
This measures 'absolute' orientation independent volume.






\newpage
\chapter{Differential Forms}
	\begin{longtable}{|p{4.5cm}|p{4.5cm}|p{4.5cm}|}
		\hline
		Smooth Theory& Discrete Theory& Implementation (Notes)\\
		\hline
			Differential Forms: \begin{itemize}
			  \setlength{\itemsep}{1pt}
			  \setlength{\parskip}{0pt}
				\setlength{\parsep}{0pt}
				\item[-]Diff Form motivation
				\item[-]Forms (multilinear mappings) and the dimension of k Form space 
				\item[-]Differential Forms 
				\item[-]Riemann Integral of Diff Forms 
				\item[-]Interpretation of Diff Forms in $\mathbb R^3$ 
			\end{itemize}
			&
			\begin{packed_enum}
				\item[-] Discrete Form
				\item[-] Sampling Forms
			\end{packed_enum}
			 & - none
			 \\		
		\hline
	\end{longtable}
	\subsection{Motivation: The perfect thing to integrate}
		Have a look at what you need when taking an integral: an either symmetric or antisymmetric form taking
		vectors from the tangential space
	\subsection{Forms and Differential Forms}
		Concretize The space of multilinear antisymmetric functions: bases, dimension, wedge. Define Differentialforms
	\subsection{What Forms are in $\mathbb{R}^3$}
		Using a metric (or simply by saying we look only at embedded manifolds with induced metric) we can associate
		Vector fields etc to Forms.
	\subsection{Integrating Forms}
		Now again lets have a look at integration of Diffforms.
	\subsection{Discrete Forms}
		What discrete forms are and where they live
	\subsubsection{Sampling Forms}
		How you sample vectorfields etc / what the sampled values mean. Here the duality of forms already emerges.
	
\newpage		
\section{External Calculus \& Discrete External Calculus}
	\begin{longtable}{|p{4.5cm}|p{4.5cm}|p{4.5cm}|}
		\hline
		Smooth Theory& Discrete Theory& Implementation (Notes)\\
		\hline
			External Calculus
			\begin{packed_enum}
				\item[-] Gradient, Curl and Divergence
				\item[-] d
				\item[-] Stokes Theorem
				\item[-] Star and DeRham Complex
			\end{packed_enum}
			&
			Discrete External Calculus
			\begin{packed_enum}
				\item[-] Discrete d
				\item[-] Dual Mesh
				\item[-] Show also intuitive match to curl etc
			\end{packed_enum}
			 & 
			 A look at the Laplacian from chapter 2
			 \begin{packed_enum}
				\item[-] The DEC matrices (and tests)
			\end{packed_enum}
			 \\		
		\hline
	\end{longtable}
	\subsection{Gradient, Curl, Divergence}
		Geometric Definition of gradient, curl and divergence, maybe plus reduction to 'standard' $\nabla$ operator when using apporpriate coordinates.
	\subsection{Differential Operator}
		The operators above take a form of one type and return an other.
		Introduce the $d$.
	\subsection{Stokes Theorem}
		To what depth this is proven is open, as this needs bordered manifolds and i don't know if I want to look at them
		Maybe proof SKETCH, because this result is astonishing.
		Known examples and special cases. Cases where it does not hold.
	\subsection{Discrete Differential Operator}
		Using stokes Theorem we introduce the discrete $d$ as the conjugate of the border operator
	\subsection{Duality: We want more}
		Not all operators can be built yet. Introduce Star, duality
	\subsection{Dual Mesh and Star Operator}
		Discrete Version of this.
	\subsection{The 0 Form Laplacian from the beginning}
		Now easy to write that guy down.
\newpage		
\section{Application: Mesh Parametrization}
	\begin{longtable}{|p{4.5cm}|p{4.5cm}|p{4.5cm}|}
		\hline
		Smooth Theory& Discrete Theory& Implementation (Notes)\\
		\hline
			Conformal Maps
			\begin{packed_enum}
				\item[-] Conformal Maps with Exterior Calculus
			\end{packed_enum}
			&
			Existance of Embeddings, The thingy theorem
			\begin{packed_enum}
				\item[-] Dimension of result spaces depending on topology?
			\end{packed_enum}
			 & 
			 Implementing it with DEC
			 \begin{packed_enum}
				\item[-] The equation
				\item[-] Border Constraints
				\item[-] Results
			\end{packed_enum}
			 \\		
		\hline
	\end{longtable}
	In this chapter we are all about 0 Forms.
	\subsection{Embeddings}
	Taking coordinates as functions associated to the Surface and not the other way round. The thingy theorem for graphs
	\subsection{Conformal Maps}
	Conformal maps properties, EC formulation
	\subsection{Implementaion: Conformal Embedding}
		How its done. Pretty straight forward if you have the machinery. Different Border Constraints. Mention MeanValue Weights?
	\subsection{Dimension of Harmonic Space?}
		Looking at results and topology. Genus, Bettinumbers, DeRham Complex?	Would be nice..
		Mention cutting algorithms like the quad mesh paper..
\newpage	
\section{Application: Vectorfield Design and the general Laplacian}
	\begin{longtable}{|p{4.5cm}|p{4.5cm}|p{4.5cm}|}
		\hline
		Smooth Theory& Discrete Theory& Implementation (Notes)\\
		\hline
			Important External Calculus Results:
			\begin{packed_enum}
				\item[-] Point Carre Lemma
				\item[-] Laplace Beltrami Operator: $d$ and $\delta$ free.
				\item[-] Hodge Decomposition
				\item[-] Here on the dim of harmonic spaces?
			\end{packed_enum}
			&
			The same as the smooth ones.
			\begin{packed_enum}
				\item[-] Properties still hold exactly.
				\item[-] Laplacian in Least square sence
				\item[-] Border Constraints
				\item[-] One Form interpolation
			\end{packed_enum}
			 & 
			 Implementing it with DEC
			 \begin{packed_enum}
				\item[-] Vector Field Design
				\item[-] =Least square harmonic 1 Form solving
			\end{packed_enum}
			 \\		
		\hline
	\end{longtable}
	\subsection{Point Carre Lemma}
	When is a form d or delta of another form?
	\subsection{General Laplacian}
	The Laplacian is as well d as delta of another form...
	\subsection{Hodge Decomposition}
	Splitting Forms
	\subsection{Application: Vector Field Design}
	Using all this for Vector Field Design
	\subsection{Border Constraints}
	How borders can be treated but it affects everything.
\newpage
\section{A Fluid Simulation with DEC}
	\begin{longtable}{|p{4.5cm}|p{4.5cm}|p{4.5cm}|}
		\hline
		Smooth Theory& Discrete Theory& Implementation (Notes)\\
		\hline
			\begin{packed_enum}
				\item[-] Introduction to the fluid equations and reformulation in DEC.
			\end{packed_enum}
			&
			\begin{packed_enum}
				\item[-] Second approach to Borderconstraints.
				\item[-] Reformulation / solving for rot part etc.
			\end{packed_enum}
			 & 
			 Implementing Fluid Sim with DEC
			 \begin{packed_enum}
				\item[-] Solving for Exact harmonic 1 Form
				\item[-] continuous Vfield interpolation
				\item[-] pathtracing
				\item[-] Results
			\end{packed_enum}
			 \\		
		\hline
	\end{longtable}
	All theory but the Fluid Simulation dependent theory is introduced, so this is a demonstration of DEC in use.
	\subsection{The Euler Equations}
	Short introduction to the meaning of the equation
	\subsection{The Algorithm}
	Approach and general Algorithm
	\subsection{Interpolation and Pathtracing}
	As it says. Issues on curved meshes.
	\subsection{Border Constraints}
	Need for exact Harmonic solution $=>$ Equation.
	\subsection{Influence of Mesh choice and parameter choice}
	Results and problems.
\end{document}