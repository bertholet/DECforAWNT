\chapter{A Fluid Simulation with DEC}

\begin{figure}[h]%
\begin{center}
\includegraphics[height = 5cm]{imgs/8_fluidsphereI.eps}%	
\end{center}
\caption{The simulation of a viscous fluid on a sphere with 1920 triangles. Motion was induced by stirring.}%
\label{fig:8_fluidsphere}%
\end{figure}

%\begin{longtable}{|p{4.5cm}|p{4.5cm}|p{4.5cm}|}
%	\hline
%	Smooth Theory& Discrete Theory& Implementation (Notes)\\
%	\hline
%		\begin{packed_enum}
%			\item[-] Introduction to the fluid equations and reformulation in DEC.
%		\end{packed_enum}
%		&
%		\begin{packed_enum}
%			\item[-] Second approach to Borderconstraints.
%			\item[-] Reformulation / solving for rot part etc.
%		\end{packed_enum}
%		 & 
%		 Implementing Fluid Sim with DEC
%		 \begin{packed_enum}
%			\item[-] Solving for Exact harmonic 1 Form
%			\item[-] continuous Vfield interpolation
%			\item[-] pathtracing
%			\item[-] Results
%		\end{packed_enum}
%		 \\		
%	\hline
%\end{longtable}
In this chapter a fluid simultation using DEC is presented, following the approach from \note{names [reference]}. 
All theory but fluid dynamics has been introduced, so this chapter is a demonstration of DEC in use. 

One of the challenges when simulating fluids is to keep geometric features like incompressibility or vortices. Because of the geometric nature of DEC and because DEC conserves the geometry of the curl and divergence operator, these difficulties drop away. On the other hand simplifying assumptions are taken, and together with the interpolation scheme used, it is not clear how accurate the simulation is. The authors of \note{reference} have taken the pragmatic point of view that their simulation is designed to preserve visually important features of fluids and left thoughts about accuracy for future work. 

%Problem of standard fluid simulations.

%\newpage
\section{Continuous Problem Statement}

%\subsection{Properties of Fluids}
%Formally, fluids are described by the Navier Stokes equation in the next section. Viscosity, incompressibility, vorticity.  \note{Todo: The Karman Vortex street. Viscous fluids have an interesting behaviour close to boundaries of non moving objects. As they have internal friction the velocity on the boundaries of such objects is zero. This leads to some vorticity which then are carried away}

\subsection{Navier Stokes}

\begin{figure}%
\begin{center}
\includegraphics[height=3cm]{imgs/7_vfvarying2.eps}%	
\end{center}
\caption{A comparision of the derivative and the derivative along the motion. The velocity field is constant over time and its derivative $\frac{\partial u}{\partial t}$ is zero. But the velocity of a tracked particle changes; the material derivative describes this change.}%
\label{fig:fsmaterialderivative}%
\end{figure}

The first step is to formulate the behaviour of fluids appropriately with vector calculus and exterior calculus. 
Physics provide the following description of incompressible viscous fluids, that is fluids with interior friction; the Navier-Stokes equation:
\[\frac{\partial u}{\partial t} + Du \cdot u = -\nabla p + \nu \Delta u\]
\begin{equation}div(u) = 0\label{eq:NS}\end{equation} 
The single terms have the following meanings:

\begin{itemize}
	\item $u(x,t)$ is a time and position dependent vector field of velocities. The velocities describe the speed and the direction of 'particles` at fixed positions
	\item The condition $div(u) = 0$ describes that the fluid is incompressible; at every point all incoming flow has to go out. This ensures that the total volume of the fluid is preserved.
	\item $\frac{\partial u}{\partial t} + D u \cdot u$ is the material derivative or derivative along the motion (see Figure \ref{fig:fsmaterialderivative}). The material derivative is based on the idea that you track particles or parcels and compute the change of their velocity. The trajectory $\alpha$ of a particle is defined by:
\[\frac{\partial \alpha(t)}{\partial t} = u(\alpha(t), t)\]
The change of velocity of such a particle can then simply be computed by using the chain rule  
\begin{align*}\frac{\partial}{\partial t} u(\alpha(t),t) &= Du \cdot \frac{\partial}{\partial t}\begin{pmatrix}
\alpha(t) \\
t
\end{pmatrix}\\
&= (\frac{\partial u}{\partial x}, \frac{\partial u}{\partial y},\frac{\partial u}{\partial z}, \frac{\partial }{\partial t}) \cdot \begin{pmatrix}
u(\alpha(t),t) \\
1
\end{pmatrix} \\
&= \frac{\partial u}{\partial t} + Du \cdot u
\end{align*}
\item $p(x,t)$ is a scalar field that describes the pressure at the position $x$ at time $t$. The vectors $-\nabla p$ therefore point in the direction of the largest pressure decrease.
\item The factor $\nu \Delta u$ is the diffusion factor and $\nu$ is the viscosity, a real valued material dependent constant. The viscosity $\nu$ describes how 'thick` the fluid is, how much internal friction it has. For example honey is thicker than water and its viscosity $\nu$ is higher; water having a viscosity of $0.001$. The term $\nu \Delta u$ describes that the higher the friction is, the more the velocities get diffused.
\end{itemize}

The Navier-Stokes equation therefore states that the movement of small parcels is influenced by pressure differences and a diffusion term. Every particle is pulled both in the direction of the largest pressure decrease and in a direction such that curl and divergence are decreased (diffusion), where the influence of the diffusion depends on the fluid's viscosity.

\subsubsection{Navier Stokes for Vorticities}
To put an emphasis on the behavior of vortices in the fluid simulation, the authors of \note{[Fluid sim]} use the Navier Stokes equation reformulated in terms of vorticities. You get such a description by applying $\nabla \times$ to both sides of Equation \ref{eq:NS}
\[\nabla \times \frac{\partial u}{ \partial{t}} + \nabla \times (u \cdot \nabla u) = - \underbrace{\nabla\times \nabla}_{=0} p + \nu \nabla \times \Delta u,\]
leading to\footnote{You need to use $\nabla u = 0$ to show the identity $\nabla \times (Du \cdot u) = D w \cdot u - Du \cdot w$ where $w = \nabla \times u$. Also $\nabla \times \Delta u = \Delta (\nabla \times u)$ is used. Both identities are easy to check by writing them out. }
\[\frac{\partial w}{\partial t} + Dw\cdot u - Du \cdot w =  \nu \Delta w,\]
\begin{equation}w = \nabla \times u,\hspace{0.5cm} div(u) = 0.\label{eq:NS2}\end{equation}
Simulating the curl $\nabla \times u$ of the velocity field instead of the velocities $u$ has benefits. The pressure term disappears. And simulating the curl leads to a better simulation of the vortices. 

The left hand side of Equation \ref{eq:NS2} describes the derivative of curl along the motion, similarly to the derivative of $u$ along the motion that appeared in Equation \ref{eq:NS}. Equation \ref{eq:NS2} therefore states that  vorticity is carried along the flow while being diffused with time. The viscosity $\nu$ controls how fast vorticities get diffused. For an inviscid fluid there is no internal friction,  the viscosity $\nu = 0$ and the derivative of the curl along the motion is 0. Therefore the curl along the motion is constant and vorticities are carried with the flow without being diminished.

\subsubsection{Stirring}
To take account of additional forces $F$, for example induced by stirring, the Equations \ref{eq:NS} and \ref{eq:NS2} need to be slightly adapted to
\[\frac{\partial u}{\partial t} + Du \cdot u = -\nabla p + \nu \Delta u + F\]
and
\begin{equation}\frac{\partial w}{\partial t} + Dw\cdot u - Du \cdot w =  \nu \Delta w + \nabla \times F \label{eq:NS3},\end{equation}
where $F$ is a vector valued function.

\subsection{A numerical integration scheme}

\begin{figure}[t]
\begin{center}\fbox{\parbox{12cm}{
\emph{
\begin{enumerate}
\item Velocities $u_{t_n}$ are known at time $t_n$
\item Compute vorticity $w_{t_{n+1}}$ at time $t_{n+1}$ at some positions $p'$:
\begin{enumerate}
	\item Backtrack $p'$ according to the velocity field $u_{t_n}$ to the position $p$ it was at time $t_n$
	\item Use $u_{t_n}$ to compute $w_{t_n}(p,t_n)$.
	\item Solve $w_{t_{n+1}}(p',t_{n+1}) = w_{t_n}(p,t_n) + (t_{n+1} - t_n)\Delta w_{t_{n+1}}(p',t_{n+1})$ for $w_{t_{n+1}}(p',t_{n+1})$ to add diffusion.
\end{enumerate}
\item Add forces $\omega_{t_{n+1}} += (t_{n+1}-t_n)\cdot\nabla \times F$
\item Use the vorticities $w_{t_{n+1}}$ to compute a velocity field $u_{t_{n+1}}$
\end{enumerate}}
}}\end{center}
\caption{The numerical integration scheme}
\label{fig:fd_numericalIntegration}
\end{figure}


The authors of \note{...} propose to solve the equation \ref{eq:NS3} for the simulation of the fluid, using what they call a geometric integration scheme. The scheme is a combination of implicit numerical integration and the tracking of particles and mimics the geometric behavior of the flow. Suppose a fluid parcel with space time position $(p_{t_n},t_n)$ is carried to the position $(p_{t_{n+1}},t_{n+1})$. Then Equation \ref{eq:NS3} means that
\begin{align}w(p_{t_{n+1}},t_{n+1}) \approx \;& w(p_{t_n},t_n) + (t_{n+1}-t_n) \cdot \nu \Delta w(p_{t_{n+1}},t_{n+1}) \nonumber \\ &{}+ (t_{n+1}-t_n) \cdot\nabla \times F_{t_{n+1}}\label{eq:vortDiffusion}\end{align}


%They compute the vorticity at some time $t_{n+1}$ by using that vorticity is simply carried along the flow plus some diffusion. Assuming that all velocities are known at the time $t$ and the fluid parcel at position $p$ is carried to some position $p'$ at some time $t'$. Then implicit numerical integration according to the Equation \ref{eq:NS3} amounts to
%\begin{equation}w(p',t') \approx w(p,t) + (t'-t) \cdot \nu \Delta w(p',t')\label{eq:vortDiffusion}\end{equation}
This leads to the integration procedure listed in Figure \ref{fig:fd_numericalIntegration}. Its greatest issue is that in step 4 a velocity field needs to be computed from vorticities alone. Before we discretize the equation we will consider this step.

\subsection{From Vorticities to Velocities}
\label{subsec:fd_vort2vel}
%\note{Constraint on border to get the adjointness of the operators, hodge decomp?}
This step is the most problematic of the algorithm. From the Pointcarr\'e Lemma we know that   in general  the problem $\partial u = w$ can not uniquely be solved for $u$. But by using the incompressibility of fluids and making additional assumptions about $u$, $u$ can be computed. The setting is the following: given a $2$-form $w$ we want to solve for $u$ such that
\begin{equation*}\left.\begin{aligned}\partial u &= 0 \;\; (incompressibility) \\ 
du &= w \end{aligned}\right\}\end{equation*}

The first step is to apply the Hodge decomposition theorem to $u$:
\[u = \partial a + db + c \]
where $c$ is harmonic. But from the incommpressibility condition $\partial u = 0$ we know that the non divergence free part $db$ above needs to be zero.\nobreak
%
\footnote{ We can get this by the usual argument using the no divergence constraint: $0 = \partial u = \partial \partial a + \partial db + \partial c = \partial d b$ and then the adjointness of $d$ and $\partial$: $\partial d b = 0 \Rightarrow \langle \partial d b , b \rangle = 0 \Rightarrow \langle db, db \rangle = 0 \Rightarrow db = 0 $}
%
Therefore: 
\[u = \partial a + c\]
And as the harmonic part $c$ fulfils $dc = 0$,
\begin{equation} w = du = d\partial a \label{eq:fd_vort2vel0}\end{equation}
There is no way to get the harmonic component $c$ out of the vorticities $w$. This is not astonishing, as by the Hodge decomposition it is orthogonal to vorticities. The harmonic part $c$ of $u$ therefore needs to be computed separately. From $w = d \partial a$ not even $a$ is well defined. But we are solely interested in $\partial a$ and using the adjointness of $d$ and $\partial$ you can show 
%
\footnote{If $ d \partial a = d \partial b$, then $d \partial (a- b) = 0$ and therefore, using the adjointness of $d$ and $\partial$ $\partial (a-b) = 0.$}
%
$$d \partial a = d \partial b \;\;\;\Rightarrow\;\;\; \partial a = \partial b$$
%
So it does not matter what $a$ with $w = d \partial a$ we choose. From the Hodge decomposition theorem we also know that we can constrain $da$ independently from $\partial a$, so we can add the constraint $da = 0$ to Equation \ref{eq:fd_vort2vel0}. Then the term $\partial d a$ vanishes, and Equation \ref{eq:fd_vort2vel0} can be transformed to
\begin{equation} w = d\partial a + \partial d a = \Delta a .\label{eq:fd_vort2vel}\end{equation}
This is the Poisson problem, which we are able to solve for $a$.

So from the vorticities $w$ we can get the vorticity part $da$ of the velocity field $u$, but not the harmonic part $c$. The authors of \note{[...]} simply assume that the harmonic part $c$ is not time dependent and can be computed once and for all from application dependent constraints. Lets reformulate this section in a Theorem:

\begin{thm} \label{thm:vort2vel}(Vorticity to Velocity) If the harmonic component $c$ of a $1$-form $u$ is given, the space of forms allows a Hodge decomposition and the operators $d$ and $\partial$ are adjoint, then the system
\begin{equation}\left.\begin{aligned}\partial u &= 0 \;\; (incompressibility) \\ 
du &= w \end{aligned}\right\}\end{equation}
can be solved by solving for $a$:
\[\Delta a = w\]
\[u = \partial a + c\]
\end{thm}

From Section \ref{subsec:vf_dimHarm} we know that depending on the topology of the manifold harmonic forms $c$ do not need to exist. Namely on the surface of a sphere there are no harmonic 1-forms. A fluid simulation on the surface of a sphere therefore is less complicated than one on a plane with borders. 

From Sections \note{[coderiv]} and \note{[hodge dcomp]} we also know that on manifolds with borders, the operators $d$ and $\partial$ need not to be adjoint and the Hodge decomposition needs not to hold either. In \note{[Physics Book]14.34 p377} more general results are mentioned. The Hodge decomposition does for example hold for 'tangential' forms, that is forms $\omega$ for which $\star\omega$ restricted to the border $\delta M$ is 0. One example of a tangential form would be vector fields whose vectors are tangential to borders. Tangential forms arises naturally in this setting, as the boundary of the mesh often represents a impenetrable wall through which a fluid can not flow.

\section{Implementation}
We restrict ourselves to simulations of fluids on two dimensional surfaces. There is no difference in the theory or the algorithm when working on three dimensions, but the display of flow that is restricted to two dimensions is simpler. And that DEC works just as well on curved manifolds can only be demonstrated in two dimensions. 

%Vectorfields and vorticities can directly be mapped to discrete forms and the steps $2 c)$ and $3)$ of \note{...} can be directly translated to DEC. The backtracing of vorticities is described in Section \note{...}
\subsection{Who is who}
To simplify the specification of border constraints the role of the dual mesh and the primary mesh are switched such that what we interpreted so far as dual forms are stored on the primary mesh and vice versa.  Then primary edges store the flow through them instead of the flow along them, see also the different sampling schemes from Section \ref{subsec:samplingForms}. As a result of changing the interpretation of the discrete forms, the operators $d$ and $\partial$ also have  different interpretation in vector calculus, see Figure \ref{fig:fd_whoiswho}. 

The changed roles make it is easy to constrain the flow through edges and boundaries. For example the constraint on obstacles and walls is that there is no flow across the boundary. This is enforced by $\discrete{v}^1(e_{boundary}) = 0$.

\begin{figure}%
\includegraphics[width=\columnwidth]{imgs/8_fd_dualpprimalproblem.eps}%
\caption{By switching the roles of the dual and primal mesh, vector fields are sampled differently. This results in the operators $d$ and $\partial$ also changing their meaning. }%
\label{fig:fd_whoiswho}%
\end{figure}

Changing the roles of the dual and primal mesh is consistent with the remark, that tangential forms have the property $\star w |_{\delta M} = 0 $. This can not be inforced on the dual mesh, as the border of the dual mesh is not the geometric border of the mesh. Therefore the change of roles is necessary.

\subsection{Discrete Equations}
We now translate the vector calculus equation \ref{eq:NS3} and the integration scheme into DEC terms. 
The velocity field $u$ is described as a discrete  $1$-form $\discrete{v}^1$, as is the user provided forces $\discrete{f}^1$ and the precomputed harmonic part of the flow $\discrete{c}^1$. %But instead of associating $\discrete{v}^1$ to the velocity $u$ as representing flows along edges according to the first sampling scheme described in \note{...} $\discrete{v}^1$ is associated to $u$ using the second sampling scheme \note{...} where $\discrete{v}^1$ describes the flux through the edges. In comparison to the vector field design implementation the roles of the dual and primary mesh gets switched, what were the discrete dual forms are now stored on the primary mesh. Accordingly the role of the operators $d$ and $\partial$ switch, as depicted in the schematic \note{img: Todo standard operators and relation to ec now}.
%The reason to chose to represent the velocity field by fluxes on the primary mesh is that it greatly facilitates the enforcement of border constrains. Boundaries of obstacles then simply consist of edges, where no flux passes, i.e. $\discrete{v}^1$ will be constrained to $\discrete{v}^1(e_{obstacle}) = 0$. \note{img}.

The vorticity of $u$  is a discrete $0$-form $ \discrete{w}^0 = \delta_{discrete}^1 \discrete{v}^{1}$, consisting of values defined on vertices. Steps $(2a)$ and $(2b)$ of the integration scheme, i.e. the backtracking of parcels and computation of vorticities are described in the next section. All other parts of the numerical integration are straight forward to translate. The step $(2 c)$, the diffusion of vorticity as given by the Equation \ref{eq:vortDiffusion},
\[\discrete{w}^0_{t_{n+1}} = \discrete{w}^0 + \nu t \Delta \discrete{w}^0_{t_{n+1}}\]
i.e.
\begin{equation}(Id - \nu t  \star_0 d^{dual}_0\star_1d_0)\discrete{w}^0_{t_{n+1}} = \discrete{w}^0 \label{eq:fd_discreteDiffusion}\end{equation}
The step $(3)$, the addition of forces is given by
\begin{equation}\discrete{w}^0_{t_{n+1}} += \partial^1_{discrete} \discrete{f}^1\end{equation}
The step $(4)$, the reconstruction of the updated $1$-form $\discrete{v}^1$ works as described in Theorem \ref{thm:vort2vel} in Section \ref{subsec:fd_vort2vel}:
\begin{equation}\Delta \discrete{a}^1 = \discrete{w}^0_{t_{n+1}}\label{eq:fd_discreteVort2Vel}\end{equation}
\begin{equation}\discrete{v}^1_{updated} =  d_0 \discrete{a}^1 + \discrete{c}^1\end{equation}
where $c$ is the precomputed harmonic solution, $\Delta$ is the discrete Laplacian. Equations \ref{eq:fd_discreteDiffusion} and \ref{eq:fd_discreteVort2Vel} are to be solved with a sparse solver.

\subsection{Interpolation and Pathtracing}
Vorticities are calculated by summing up the flow around dual faces \note{specify...}. To calculate the backtraced vorticities we therefore backtrace the dual faces, or more specifically the dual vertices. To trace the path of a particle, we need a continuous vector field defined on the mesh \note{img}. Note that therefore the interpolation scheme using Whitney forms described in Section \note{vfDesing} is not suited, as it does not produce a continuous vector field. 

\subsubsection{A continuous vector field on the mesh}
The authors of \note{...} propose the following interpolation scheme to get a  continuous vector field. First, velocity vectors are computed at all the positions of dual vertices; this can be done using the sharp operator from \note{vf}. A velocity vector at an arbitrary position $p$ on the mesh then is calculated by first identifying the dual face the position $p$ lies in, i.e. identifying the closest vertex,  and then smoothly interpolating the velocity vectors given on the corners of the dual face.
The smooth interpolation of vectors of the velocities given on the dual face are done using weights given in \note{Baricentric coords}. For a flat convex polygon with vertices $p_1,...,p_k$ the normalized  weight of a value given on the vertex $p_j$, where the incident edges have normals $n_{j1}$ and $n_{j2}$, is given by $w_j$
\[c_j(x) = \frac{\abs{n_{j1} \times n_{j2}}}{\langle n_{j1}, p_j -x\rangle\cdot\langle n_{j2}, p_j -x\rangle}\]
\[w_j(x) = \frac{c_j}{\sum_{i=1}^k c_i}\]
On the edges of the polygon, these weights interpolate the values on the vertices linearly and thus the weights can be used to get a continuous vector field on the mesh. While these weights are designed for flat polygons they work well for the potentially curved dual faces \note{img}.

On flat meshes these weights can be used directly to interpolated the velocities computed on dual vertices via
\[u(x) = \sum_{j} w_j(x) u_j\] 
On a curved mesh such an interpolation will not lead to a tangential vector. To reproject the interpolated vector on the appropriate triangle leads to a uncontinous vector field which might lead to problems in the pathtracing step. In general we propose to  flatten  the dual face the velocities by projecting them along the curvature normal from Section \note{...} associated to the dual face, do the interpolation with the projected vector field and then project the interpolated vector back to the appropriate triangle. \note{img problem and solved problem: peak where vf does not vanish}
\[u_{flat}(x) = \sum (u_j - N_{curv} \langle N_{curv}, u_j \rangle) w_j\]
\[= (\sum u_j w_j )- N_{curv} (\sum \langle N_{curv}, w_j u_j \rangle)\]
The reprojection onto a triangle $t$ with normal $N_t$ along $N_{curv}$ is given by
\[u(x) = u_{flat}(x) - N_{curv} \frac{\langle u_{flat}(x), N_{t}\rangle}{ \langle N_{curv}, N_t \rangle}\]


\note{These weights are defined similarely for higher dimensional convex polygons too, see \note{Bari coordinates}.}
 \note{image of the4 weights}

\subsubsection{Pathtracing on a mesh}
Given a continuous vector field on the mesh, the pathtracing step is straight forward. Let $u(x)$ be the interpolated velocity field, then this field can be pathtraced by doing small steps of size $h$ according to $u$ \note{img}. A simple pathtracing algorithm is also given in Listing \ref{fig:fd_pathtracing}. %The only issue when pathtracing a point is that you need to closely keep track what triangle you actually are on and when you change triangle, as you need this information to correctly compute $u$, and $u$ has to always be a tangential vector.

\begin{listing}
%\caption{backtrace a position}
\begin{algorithmic} %\REQUIRE T , stepSize  
\STATE \hspace{-0.3cm}\textbf{backtrace}:
\STATE t = T;
\WHILE {$( t > 0)$}
\STATE backtrace(t, stepSize, pos, triangle);
\ENDWHILE
\STATE
\STATE \hspace{-0.3cm}\textbf{backtrace}(t,stepSize, pos, triangle):
\STATE $u = u(pos, triangle)$;
\STATE $maxT = maxt(pos, triangle, u)$;
\STATE $step = min(t, stepSize, maxT)$
\IF{$step < maxT$}
\STATE $pos += step \cdot u;$
\STATE $t-= step;$
\ENDIF
\IF{triangle boundary reached}
\STATE $triangle = neighborTriangle$; %//or stop tracing if border of the mesh is hit.
\ENDIF
\end{algorithmic}
\caption{A straight forward algorithm to trace  the trajectory of a particle. $u(...)$ is the interpolated vector field and $maxt$ is a helper function that computes the maximal possible time step before hitting the border of the current triangle.}
\label{fig:fd_pathtracing}
\end{listing}

\subsubsection{Backtracing Vorticities}



\section{The harmonic component} 

\subsection{Fluid simulation on a sphere}
On a sphere there are no borders and there is no harmonic $1$-form, therefore no harmonic part $c$ needs to be considered in the reconstruction step and the computation of the harmonic field falls away.

\subsection{Flat Bordered Meshes}
On a flat mesh with boundaries the harmonic part $c$ of the flow can be determined through boundary constraints. Rather than using the boundary adapted matrices introduced for vector field design in Section \note{...} to compute a harmonic $1$-Form, we propose here to bundle up all border vertices and define that their dual face is one outer face. 

In the vector field design context the matrix was adapted to correctly enforce divergence constraint on boundary vertices, or rather their dual faces. Here the roles of the primary and dual mesh being switched, this corresponds to a correct zero curl enforcement on the boundary. The adaptation of the matrices introduces an error by assuming that half of the flow over a boundary edge is a correct guess for the flow through the part of the edge belonging to the dual face, as depicted in \note{img...}. Adding an outer face results in more accurate harmonic fields, as depicted in Figure \note{...}.


We interprete the border vertices as one vertex with one dual face with a closed boundary, as depicted in Figure \note{todo}. The boundary of this dual face is closed. Then the dual border operator matrix $\delta^{dual}_2$ needs to encode the border relation of the additional faces. The border of the dual outer face is the union of the borders of the dual faces of the border vertices. Therefore for every border component $border$ a column $\delta_{border}$ has to be added to $\delta_0^{dual}$
\[\delta_{border} = \delta_0^{dual} \cdot 1_{border}\]
where $1_{border}$ describes the set of vertices of $border$, i.e. is one for every vertex in $border$ and 0 else.
The dual derivative $d_1^{dual}$ is therefore adapted by adding one line per 
\[\widetilde{d_1^{dual}} = \begin{pmatrix} d_1^{dual} \\
(1_{border_1})^T \cdot d_1^{dual} \\
\vdots \\
(1_{border_k})^T \cdot d_1^{dual} \\
\end{pmatrix}\]
As the old dual faces of boundary vertices are ignored, the corresponding lines need to be dropped out of the matrices as well. Or equivalently the corresponding values of $\star_0$ can be set to zero:
\[(\star_0)_{borderVertex} = 0\]
additionally some positive dual primal weight $W$ for the newly added composed outer faces has to be added, leading to
\[\widetilde{\star_0} = \begin{pmatrix}((\star_0)_{=0\;for\; border\; vertices} \\
W\\
\vdots \\
W \end{pmatrix}\]
The weight $W$ could be taken to be the combined area of the dual faces of border vertices, which would be a consistent thing to to, but any weight will work.
Note that it is enough to do so for each 'hole' and not for the 'outer' border, as a no curl constraint is then automatically enforced on the outer border too.

The values of the harmonic 1-Form on the boundaries need to be carefully constrained such that the no divergence and no curl constraints for harmonicity can be met. At the very least the border constraints have to fulfill $\sum_{border} a_j = 0$, else it is impossible to have a no divergence field. The simplest constrains are to set for all edges of a border component
\begin{equation}\discrete{w}(edge) = 0 = \discrete{a}_{edge} \label{eq:fsobstacle}\end{equation}
to describe obstacles with no flow through its border, or to set for all edges
\begin{equation}\discrete{w}(edge) = \langle edge^\perp, v\rangle= \discrete{a}_{edge} \label{eq:fspermeable}\end{equation}
with a fixed direction $v$. This second constraint can be used for example to describe a flow that flows in on one side of a rectangular region and out on the other \note{img}. 

With these adapted matrices we use the same formulation as in the vector field chapter to get an equation for the harmonic field:
\begin{equation}\begin{pmatrix} d^T & \widetilde{\partial}^T & Id_{border}^T \end{pmatrix} \begin{pmatrix}
\star_2 & & \\
 & \widetilde{\star_0} & \\
 & & \lambda
\end{pmatrix} \begin{pmatrix} d \\
\widetilde{\partial}\\
Id_{border}
\end{pmatrix} \discrete{w}^1= \lambda\discrete{a}.\label{eq:fsharmonic}\end{equation}
Here $\lambda$ is some positive weight for the constraints from Equations \ref{eq:fsobstacle} and \ref{eq:fspermeable} and $\discrete{a}$ has according values for boundary edges and is zero for all internal edges. 
Note that the adapted matrix $\widetilde{d_1^{dual}}^T \star_0 \widetilde{d_1^{dual}}$ is considerably less sparse than the matrix used in the vector field design chapter, but in practice this was not an issue.

\note{Rather than to enforce a no curl constraint on the bordering dual faces with non closed borders, these are discarded from the equations and replaced by one constraint per border component, i.e. per outer face. \note{img}. 
To get a perfect discrete harmonic field to some border constraints, the border constraints have to lie in the kernel of the Laplacian, they have to fulfil }

\section{?The complete algorithm?}
\note{Direct force input is missing!}
\section{Influence of Mesh choice and parameter choice}
Results and problems.

\chapter{Further Literature}
This section gives a short overview where to go from here, after reading this thesis. The thesis should have helped a reader to develop a feeling for DEC while covering the most important theoretical foundations. There is a plenty of interesting topics related to DEC either not mentioned or only superficially treated here. 

Various texts introduce DEC rigorously as an independent theory, targeting an audience well acquainted with differential forms and possible applications of differential forms. After reading this thesis you should have enough background to understand the relevance and context of the different DEC operators for these texts being of interest. \note{....}

\subsection{?? Alternative $\star_0$ matrices ??}
\note{Where to put this section? And i don't know how to put mean value coordinates in the perspective}
As mentioned when introducing it, the discrete $\star$ is less than perfect. That it is a simple diagonal matrix is a great strength, but as seen there are some issues. For one, other than the discrete $d$ operator, it is not wholly consistent with the theoretical sampling scheme. But more importantly the weights degenerate if the mesh degenerates because of choosing the Voronoi mesh as dual mesh: the Voronoi cells for obtuse simplices are not bounded and therefore the approximation of the $\star$ operator which relies on a guess of the dual form integrated over the dual cell, gets arbitrary bad.

We have considered exactly this problem in Section \note{...} and came up with the mixed area weights \note{...}. In DEC the interpretation is ...

But even thought the mixed area weights are pretty good we want to mention some further alternatives:

\subsubsection{Mean value weights}

\subsubsection{Uniform weights}

\subsubsection{Alternative $\star_k$ matrices}

On the advanced pure math side a more rigorous study of chain complexes and co chain complexes will give a more abstract view on DEC emphasising the important common properties that EC and DEC share, as they are always what you call 'exact sequences`. 
Hodge theory and the de-Rham complex will give deeper insights but the mathematics are very involved. \note{...}
Interesting in this context also is the relation to Differential topology where DEC can be used  to get nearly equivalent discrete results on simplicial complexes.\note{...}

No less intersting is the relation between DEC and standard finite element methods \note{....}