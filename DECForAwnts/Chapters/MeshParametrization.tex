\chapter{Application: Mesh Parametrization}
	\begin{longtable}{|p{4.5cm}|p{4.5cm}|p{4.5cm}|}
		\hline
		Smooth Theory& Discrete Theory& Implementation (Notes)\\
		\hline
			Conformal Maps
			\begin{packed_enum}
				\item[-] Conformal Maps with Exterior Calculus
			\end{packed_enum}
			&
			Existance of Embeddings, The thingy theorem
			\begin{packed_enum}
				\item[-] Dimension of result spaces depending on topology?
			\end{packed_enum}
			 & 
			 Implementing it with DEC
			 \begin{packed_enum}
				\item[-] The equation
				\item[-] Border Constraints
				\item[-] Results
			\end{packed_enum}
			 \\		
		\hline
	\end{longtable}
	
With the machinery of discrete and non discrete exterior calculus at hand we can solve various problems with relative elegance . The way to use this machinery is always similar. The first step is to formulate a problem using exterior calculus, then to translate it to the discrete exterior calculus setting and use the sparse matrix formulation to solve the given problem computationally.

The problem under consideration in this chapter is two dimensional surface parametrization as it happens for example with texture mapping. The goal is to find a parametrization of a mesh, i.e. texture coordinates for all its vertices, such that the parametrization has some nice properties, here conformality. \note{[...] approaches} \note{reference to paper}. 

This chapter highlights the connection of discrete exterior calculus and conformal maps. The deduced algorithms are not state of the art. \note{[.....]}


	%\subsection{Embeddings}
	%Taking coordinates as functions associated to the Surface and not the other way round. The thingy theorem for graphs
	\section{Problem Statement: Conformal Maps}
	Given a smooth two dimensional surface $M$, we try to find a map $\phi$ that parametrizes the whole surface, $\phi: U \subset [0,1]\times [0,1] \rightarrow M$. Such a global map exists if the surface $M$ has the shape of a deformed disk with holes. Therefore, we assume that the mesh $M$ has the topology of a disk with holes. In Section \note{...} we describe how the topology of a manifold affects the solution space for parametrization and similar problems.
	
In the setting of texture mapping we would like the map $\phi$ to distort the texture $[0,1]\times [0,1]$ as less as possible. In general there is no perfect map. As soon as $M$ has points where both principle curvatures are non zero the surface is non-developable; there exists no map that preserves angles and areas at once\note{[...]}. As distortion freeness is not an option, the goal is to minimize distortion. There are various distortion measures that can be minimized. One possibility is to ask the map $\phi$ to be conformal: angles should be preserved under the map \note{img}. 
%Conformality is a local property; angle preservation can be enforced separately at every point. Minimizing the distortion of distances on the other hand affects a map globally, as it can not be enforced by only looking at a map locally. Because of its locality, angle preservation is relatively easy to impose.
Conformality can be expressed very well in terms of exterior calculus. 

\subsubsection{Conformal parametrization}
A map is conformal exactly if its inverse is conformal. Rather than to express conformality for a map $\phi:[0,1]^2\rightarrow M$, we express conformality for its inverse, the coordinate functions
\[(x,y): M \rightarrow [0,1]^2\]
\[x:M \rightarrow [0,1]\]
\[y:M \rightarrow [0,1]\]

\begin{figure}%
\begin{center}
\includegraphics[height = 3.5cm]{imgs/6_conformality.eps}%	
\end{center}
\caption{Because $dx$ and $dy$ (depicted by their related vectors) are orthogonal and of the same length, the linear mapping $D\phi$ has to be angle preserving, which means $\phi$ is angle preserving.}%
\label{fig:6_conformality}%
\end{figure}
%Note that angle preservation works in both ways: if $\phi$ preserves angles, so does $\phi^{-1}$.

The two coordinate functions $x,y$ are 0-forms. In terms of exterior calculus conformality is ensured by
\begin{equation}
 \star dx = dy.
 \label{eqn::conformality}
\end{equation}
As the exterior derivative $d$ is the gradient operator and the star operator represents a rotation by $90^\circ$ (compare with Figure \ref{fig:deRham2d}), Equation \ref{eqn::conformality} expresses that the gradients of the coordinate functions are orthogonal and of the same length. This ensures angle preservation compare with Figure \ref{fig:6_conformality}.\footnote{Interpreting $\tilde{\phi}(x,y,z) = \phi(x,y) + z\cdot N(x,y)$ where $N$ is the surface normal, $D\tilde{\phi}^{-1} = (dx ,dy,N)^T$ is a linear mapping represented by an orthogonal matrix, where $dx,dy$ are identified with the gradient vectors. Its inverse is $D\tilde{\phi} = (1/\lambda dx, 1/\lambda dy, DN)$.  This leads to $D\phi^T D\phi$ being the multiple of the identity matrix, therefore $angle (D\phi v , D\phi w )= angle(v,w)$}

From Equation \ref{eqn::conformality} one can directly deduce that $x$ and $y$ are harmonic 0-forms, i.e. 
\[\Delta x = - \star^{-1} d \star d x =0\]
\[\Delta y = - \star^{-1} d \star d y = 0.\] 
You get the harmonicity of $x$ and $y$ directly by using $\star d x = dy$ or $dy = -\star dx$  and $dd = 0$: 
\[\star d \star d x = \star ddy=  \star 0 = 0\]
\[\star d \star d y = -\star d d x = 0\]


Furthermore, a harmonic function on $M$ is uniquely determined by the values it assumes on the border. If all participating functions are sufficiently smooth, the following system has a unique solution \note{reference...}:
\[\left.\begin{matrix}
 \Delta f(x) = 0 & \text{ for $x \in M$ } \\
  f(x) = g & \text{ for $x \in \delta M$ } \\
\end{matrix}\right\rbrace\]
Here $M$ is a smooth bordered orientable 2-manifold with disk topology, and $g$ describes fixed constraints on the border. Therefore, in order to find a conformal parametrization, it is enough to find two harmonic 0-forms $x$ and $y$ and choose the border constraint such that the conformality holds. I.e. 
\begin{equation}\left.\begin{matrix}
 \Delta x = 0 & \text{ inside $ M$ } \\
 \Delta y = 0 & \text{ inside $ M$ } \\
 \text{ Border Constraint for $x,y$} & \text{ on $ \delta M$} \\
\end{matrix}\right\rbrace \label{eq:conformalProblemStatement}\end{equation}
where the border constraint uniquely determines the values of $x$ and $y$ on the border and should be chosen such that $\star dx = dy$ inside the mesh. The choice of good border constraints is problematic, but at least in principle a conformal parametrization can be found only by finding an optimal mapping for the border.


\section{Conformal Embedding with DEC}

We can directly express the problem statement \ref{eq:conformalProblemStatement} in discrete exterior calculus. To begin with we assume that the mesh we want to parametrize is a discrete $2$-manifold with the topology of a disk and \emph{no} holes. We solve for two discrete $0$-forms $x$ and $y$, i.e. two vectors of per vertex texture coordinates. Using the discrete star and $d$ operator the harmonicity constraint is given by the two linear equations
\[\star_0^{-1} d_1^{dual} \star_1 d_0^{primal} x = 0\]
\begin{equation}\star_0^{-1} d_1^{dual} \star_1 d_0^{primal} y = 0 \label{eq:conformal}\end{equation}
The border constraints are handled separately: we compute $x$ and $y$ coordinates for border vertices, leading to some coordinate vectors $b_x, b_y$ and, for border vertices, replace the equation above by
\[ x = b_x\]
\[y = b_y.\]
Obviously you can not choose arbitrary functions $x,y$ if you want the map $p\in M \to (x(p),y(p))$ to be a bijective mapping; you have to ensure that $(x(p),y(p)) \neq (x(q),y(q))$ for $p\neq q$ \note{img overlapping}. The Tutte theorem and its generalization by Gortler et Al. presented in the following Section explain appropriate conditions.

\subsection{Graph Theory}

It is not clear when solving the linear system \ref{eq:conformal} with some border constraints leads to a valid parametrization. A parametrization $p\in M \mapsto (x,y)$ has to be bijective, without any overlaps . It should \emph{embed} the mesh $M$ in $\mathbb R^2$.

Tutte's theorem gives simple constraints that guaranty that a mesh with a single border component is embedded correctly in $\mathbb R^2$. It applies to '3-connected planar graphs`, which for example are any discrete 2-manifolds with disk topology. Tutte stated that if the border of any such graph is mapped to an arbitrary convex polygon and every vertex is mapped to convex combination of its neighbors, then this mapping is an embedding. Tutte's theorem, stated as in \cite{Gortler}:

\begin{thm}(Tutte)\cite{Gortler} Let $G$ be a $3$-connected graph and $\delta G$ be its border component. If $\delta G$ is embedded in the plane as a convex polygon and every vertex is positioned as a strictly convex combination of its neighbors, then the drawing of $G$ with these vertex positions is an embedding. 
\end{thm}

This result means that choosing any convex polygon $b_x, b_y$ as image for the border vertices of a mesh and solving a linear system
\begin{align*}Ax &= 0 & x_{border} &= b_x\\
Ay &= 0 &y_{border} &= b_y\end{align*}
leads to an embedding without self intersections, if $A$ describes convex equations for all vertices, i.e. every line $i$ of the matrix fulfils
\begin{equation}
a_{ij} = 0 \text{ if $v_i$ and $v_j$ are not neighbors} \label{eqn:tutte1}
\end{equation}
\begin{equation}\sum_{j} a_{ij} = 0\label{eqn:tutte2}
\end{equation}
\begin{equation}
a_{ii} <0, a_{ij} \geq 0 \text{ for $i\neq j$}\label{eqn:tutte3}
\end{equation}
as such a line expresses that the vertex $i$ is a convex combination $\sum_{j\neq i} a_{ij} v_j = -a_{ii}v_i$ or stated equivalently, with the positive weights $w_j$ fulfilling $\sum_{j} w_j = 1$:
\[w_j := -\frac{a_{ij}}{a_{ii}}\]
\[v_i = \sum_{j \neq i} w_j v_j\]
Tutte's theorem does not completely apply to our setting. Having only convex polygons as borders is to restrictive and the matrix
\[\star_0^{-1} d_1^{dual} \star_1 d_0\]
does not fulfil Equation \ref{eqn:tutte3} when there are obtuse angles in the mesh; the cotan weights in the $\star_1$ matrix can be negative. But the Equations \ref{eqn:tutte1} and \ref{eqn:tutte2} are always met. 
To also meet Equation \ref{eqn:tutte3}, we can either forbid obtuse triangles or use the improved mixed area weights from Section \note{...} or so called mean value weights \note{...} which both are guaranteed to be non-negative.

In \cite{Gortler} Gortler et al. proved an extension of Tutte's theorem, which relaxes the conditions needed to get embeddings and also is applicable when multiple borders are present and they are mapped to non-convex polygons:

\begin{thm}\label{thm:gortler}(Gortler et al.) Under the following conditions a parametrization describes an embedding without self intersection:
\begin{packed_enum}
\item $G$ is an oriented 2-manifold with disc topology and multiple exterior faces (one exterior boundary and additional holes)
\item The exterior boundary is mapped to a polygon with winding number $2\pi$ and no self intersections.
\item The interior boundaries are mapped to non intersecting polygons with winding numbers $-2\pi$
\item Every mapped non boundary vertex lies in the convex hull of its neighbors
\item The reflex (i.e. non-convex) vertices on boundaries of the mapped graph lie in the convex hull of their neighbors
\end{packed_enum}
\end{thm}
For proofs of both theorems we refer to \cite{Gortler}. We do not reproduce any proof, as most concepts needed for the proof would not be needed anywhere else in this thesis.


\begin{figure}%
\begin{center}
\includegraphics[height=3cm]{imgs/7_innerborderorientation.eps}%	
\vspace{0.5cm}
\includegraphics[height=3cm]{imgs/6_Gortler.eps}%	
\end{center}

\caption{Top: Inner borders of an embedded surface have an orientation opposite of the outer border. Bottom: The conditions (4) and (5) of Gortler's theorem. The boundary vertex is reflex and therefore has to lie in the marked region on the left of the line. For triangle meshes the conditions (4) and (5) are necessary for a map of the mesh being an embedding}%
\label{fig:7_gortler}%
\end{figure}

Consider the five conditions of Theorem \ref{thm:gortler}. The first condition ensures that the problem is solvable from a topological point of view and that there exist global parametrizations of the mesh. Conditions 2 and 3 are needed because the mapped oriented manifold is still an oriented manifold, now embedded in $[0,1]^2$ and therefore the interior borders  have to be oriented negatively relative to the exterior border, see Figure \ref{fig:7_gortler}. 

The fourth condition is the same as in the Tutte theorem. The last property is the additional property to guaranty a map to be an embedding when you map multiple borders to possibly non-convex polygons. Border vertices that are mapped to reflex vertices have to be treated separately and have to fulfill the same constraint like inner vertices: to lie in the convex hull of its neighbors. Note also that inner borders always have reflex vertices and can consist only of reflex vertices.

While Tutte's theorem can be used directly to construct an embedding by choosing any convex weights and any convex polygon as mapped border, the extension of Gortler et Al. is not as straight forward to use. The fifth property is not easy to guaranty. If a parametrization is given, it is easy to check, but before the mapping is giving, you do not know which border vertices are mapped to reflex border vertices and on which you therefore have to enforce (5). 
%Simply fixing all border vertices as is proposed in the last section, does not enforce (5). But replacing the fixed constraint for a reflex vertex by a constraint that leads to a fulfillment of (5) might lead to its neighbors newly becoming reflex vertices \note{IMG}. Still this is the simple approach we are going to take.
%Still, the theorem is very useful: having no overlaps in a parametrization looks like a global property, which might need to be enforce by looking at all triangle pairs. But the theorem states that ensuring the local property (4) and taking care that the borders respect (5) is enough.

\subsection{Fixed Border Constraints}
The quality of a solution depends strongly on the quality of the border constraint. In the following we give some simple options. Note that we still assume that the mesh only has one border component.

\subsubsection*{Convex Polygon}
A very simple, robust approach is to force the $x$ and $y$ coordinates of the border component to form some convex polygon, for example a circle. If the border component has $n$ vertices, then for the $k$th border vertex $b_x$ and $b_y$ are simply set to
\[(b_x,b_y) = (sin((2k/n)\pi),cos((2k/n)\pi)).\] 
You can get a slight improvement by spacing the values on the circle as they are spaced on the border of the mesh, i.e. by using the factor 
\[(borderDist(v_k,v_0)/length(border)\] 
instead of the factor $k/n$. By Tutte's theorem solving Equation \ref{eq:conformal} with these border constraints always leads to an embedding. On the other hand this border constraint does not enforce conformality of the parametrization.

\subsubsection*{Conformal Border Constraint}

\begin{figure}%
\begin{center}
\includegraphics[height=4cm]{imgs/7_conformalborder.eps}%	
\end{center}

\caption{The linear equation to preserve the border angle.}%
\label{fig:6_conformalBorder}%
\end{figure}
Conformality means preserving angles. We now formulate a simple linear border constraint that tries to preserve the angles of border triangles. 

Let $v$ be a border vertex and $v^+$ and $v^-$ its predecessor and successor on the border (Figure \ref{fig:6_conformalBorder}). Then the angle $\alpha_v$ between the edges $(v^-,v)$ and $(v,v^+)$ measured on the mesh
should be preserved. When mapped to the plane, the total angle of the image of the border has to be $(n-2)\pi$, so we choose
\begin{equation}\alpha_v^{mapped} = \alpha_v \frac{(n-2) \pi}{\sum_{v \in border} \alpha_v}\label{eq:conformalAngles}\end{equation}
as target angle. To get a simple linear system for the mapped border vertices $v_{mapped}^*$ we constrain
\begin{equation} (v^+_{mapped}-v_{mapped} ) + \frac{\abs{v^+-v}}{\abs{v^- -v}}rot_{\alpha_v^{mapped}} \cdot (v_{mapped} - v_{mapped}^- )= 0,\label{eq:conformalBorder}\end{equation}
which takes into account the relative distances between neighboring border vertices in addition to the angle $\alpha_v$, see Figure \ref{fig:6_conformalBorder}. This linear equation for the mapped vertex positions is described by the following matrix
\[\begin{pmatrix}
-\frac{l^+}{l^-}cos(\alpha_{m})& \frac{l^+}{l^-}cos(\alpha_{m}) -1 & 1 & \frac{l^+}{l^-}sin(\alpha_{m}) & -\frac{l^+}{l^-}sin(\alpha_{m}) & 0\\
-\frac{l^+}{l^-}sin(\alpha_{m}) & \frac{l^+}{l^-}sin(\alpha_{m})  & 0  & - \frac{l^+}{l^-}cos(\alpha_{m}) & \frac{l^+}{l^-}cos(\alpha_{m}) -1 & 1\\
 
\end{pmatrix} \begin{pmatrix}
v_{m}^-.x \\
v_{m}.x \\
v_{m}^+.x \\
v_{m}^-.y \\
v_{m}.y \\
v_{m}^+.y \\
\end{pmatrix} = \begin{pmatrix}
0 \\
0
\end{pmatrix}\]

The set of equations for the border positions has two degrees of freedom: the scale and the orientation  of the mapped border. To get a solution we can fix two arbitrary boundary vertices to be mapped to  (1,0) and (0,0). The resulting values for border positions are rescaled and translated to $[0,1]^2$. While this constraint will ensure better conformality if the overall Gaussian curvature is low and the border is not too exotic, i.e. the outer angles can be preserved very well, there is in general no guaranty that the conditions of Gortler's theorem are met and this border constraint might not produce embeddings.

\subsubsection{Neumann Border Constraints}
An alternative to fixed border constraints that fits very well in this setting is a boundary condition derived by Desbrun et Al. in \note{[Intrinsic Parametrization of Surface Meshes]}. Again it is used that the Laplacian describes the gradient of the area, as described in Section [...]. 

For a single triangle $(u_i,u_j,u_k)$ in $\mathbb R^2$, fixing two of its vertices $u_j, u_k$ and taking the area as a function of $u_i$, the gradient of $area_{u_i,u_j,u_k}(u_i)$ simply is
\[ \nabla area_{u_i,u_j,u_k}(u_i) = rot^{90}(u_k-u_j)/2\]
because the area of the triangle is $\abs{u_j-u_k}\abs{u_i-p}/2$ (see sketch \note{todo...}) and
\[ \nabla area_{u_i,u_j,u_k}(u_i) = \abs{u_j-u_k}/2  \nabla \abs{u_i-p}= \abs{u_j-u_k}/2 \frac{p-u_i}{\abs{p-u_i}}\]
and $\abs{u_j-u_k}/2 \frac{p-u_i}{\abs{p-u_i}}$ is exactly $rot^{90}(u_k-u_j)/2$. Desbrun et Al. then propose to enforce
\[\sum_{(u_i,u_j,u_k)}cot(\alpha) (u_i-u_j) + cot(\beta) (u_i - u_k) = \frac{1}{2}\sum_{(u_i,u_j,u_k)}rot^{90}(u_k-u_j).\]
on boundary vertices $u_i$, i.e.
\[\Delta_{mesh} (x,y) = \Delta_{mappedmesh}(x,y)\]
Using this linear constraint there is no need to precompute any border positions, they are solved for together with the positions of the inner vertices. Only two arbitrary vertices have to be fixed as the linear system using these constraints has two degrees of freedom, one for scale and one for orientation. This constraint can be used as it is if multiple borders are present, without further adaptations. But again there is no guaranty that Gortler's conditions are met, in practice the border often has a winding number greater than $2\pi$.


\subsection{Multiple Borders}
With the Gortler's generalized Tutte theorem in our back we can consider the case of having multiple borders when not using the Neumann border constraint. We will treat the exterior and the interior borders differently, as well as reflex and convex border vertices. 
\subsubsection{Exterior Border}
First we compute border positions $b_x$ and $b_y$ for the exterior border in any described way. But we enforce $x = b_x$ and $y= b_y$ only for border vertices that are not reflex according to the estimated target angles $\alpha^{mapped}$. For those that are reflex we replace this constraint by some convexity constraint, we propose to simply treat them like all interior vertices and take the convexity constraint given by Equation \ref{eq:conformal}:
\[\star_0^{-1} d_1^{dual} \star_1 d_0=0\]
If the conformal border constraint leads to a good border, meaning that it allows a close to conformal map at the border, and replacing $x=b_x, y=b_y$ by a convex constraint leads to no new reflex vertices, Gortler's theorem ensures that the calculated solution is an embedding.

\subsubsection{Interior Borders}
For interior borders we procede similarly; we choose to enforce a conformal constraint on the inner borders. But we do not precompute any border positions.

We start by computing target angle sizes at all vertex positions, as with the exterior border, using Equation \ref{eq:conformalBorder}. Note that this gives us outer angles \note{img...} and when a target angle is less than $\pi$ the vertex is a reflex vertex. Again we treat all vertices that would be reflex vertices after this guess like inner vertices, i.e. choosing the convex constraint given by Equation \ref{eq:conformal} for these vertices. For the vertices estimated to be non reflex we enforce the linear equation derived for the conformal border constraint, i.e. Equation \ref{eq:conformalBorder}.

\subsubsection{Overall linear equation}
The linear equation to get a conformal parametrization for a mesh with disk topology and multiple holes then is the following.
There is one equation for each vertex x and y coordinate. Let $b_x$ and $b_y$ be some precomputed exterior border positions and $\alpha_v^{mapped}$ be the target angle for a border vertex $v$.

The linear equations then are the following: we fix convex vertices ($\alpha < \pi$) of the outer border,
\begin{eqnarray*}
 &x = b_x & \\
 &y = b_y &\\
\end{eqnarray*}
choose the conformal border constraint \ref{eq:conformalBorder} for convex vertices on inner borders 
\[(v^+_{mapped}-v_{mapped} ) + \frac{\abs{v^+-v}}{\abs{v^- -v}}rot_{\alpha_v^{mapped}} \cdot (v_{mapped} - v_{mapped}^- )= 0 \]
and for all reflex border vertices and inner vertices we choose the constraint deduced using DEC:
\[\star_0^{-1} d_1^{dual} \star_1 d_0^{primal} x = 0\]
\[\star_0^{-1} d_1^{dual} \star_1 d_0^{primal} y = 0\]


\subsection{Discussion}

\begin{figure}%
\includegraphics[width=\columnwidth]{imgs/6_trivialmesh.eps}%
\caption{}%
\label{}%
\end{figure}
As is to be expected the conformal border constraint has the potential to give the best results, but for some borders this constraint fails to meet the requirements of the extended Tutte theorem and does not produce an embedding. It may even produce boundary polygons with self intersections. But as long as the sum of the inner angles are close to the sum of angles of a planar polygon the approach works well. 

Choosing a convex polygon as a border on the other hand always leads to embeddings but angles are not preserved very well. 
\note{More images}

The Neumann border constraint fails often to produce embeddings, but the conformality is well ensured.

Bottom line: if you need a conformal parametrization of patches with borders that are reasonably shaped, the simple approach described in this chapter is reasonable and easy to implement. Else you need to resort to more refined border conditions e.g. in \note{Desbrun et Al. evolving borders} it is proposed to start with a convex border polygon and let the border evolve in a way to minimize an angle-distortion energy. There are also various different, non linear approaches for conformal or area preserving parametrizations \note{[...]}.  

This thesis being an introduction to DEC, this application is presented to illustrate how DEC concepts can directly be used in relation with conformal mappings - and how DEC explains the use of the $cotan$ weights in connection with conformal parametrizations. This chapter should also give an understanding of the theoretical constraints and issues when searching parametrizations.

%\subsection{Topology?}
%In this Chapter we constantly made the restriction that the mesh should have the topology of a disk (with or without holes). Why? The answer is simple: in most other cases there would not exist a global parametrization. To understand the why we have to fall back to (differential) topology. Differential topology treats very general properties of manifolds with a differential structure. Topology treats properties of objects that stay true if the objects are deformed in a continuous way. An important topological invariant is the Euler characteristic or the genus, which basically depends only on the number of holes a geometric object has. For example a doughnut (one hole or genus 1) has an other genus then a pretzel (3 holes, genus 3) or a sphere (no holes, genus 0).

%In the following we mention some results that clarify the topological constraints on parametrization and similar problems.



%We did use DEC and more graph theoretic results to treat the mesh parametrization problem. 
%		Looking at results and topology. Genus, Bettinumbers, DeRham Complex?	Would be nice..
%		Mention cutting algorithms like the quad mesh paper..
		
%\subsection{Generalization and further notes}
%Cutting algorithms, better border constraints, iterative process. Multiple holes: hole filling and recutting.

