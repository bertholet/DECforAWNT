\chapter{Vectorfield Design and the general Laplacian}
\label{chap:vfs}

\begin{figure}[h]%
\begin{center}\hspace{2cm}
\includegraphics[height = 5cm]{imgs/7_vfcube.eps}%	
\end{center}
\caption{A generated vector field, generated with a user given source sink and directional constraint.}%
\label{fig:vfCube}%
\end{figure}

%	\begin{longtable}{|p{4.5cm}|p{4.5cm}|p{4.5cm}|}
%		\hline
%		Smooth Theory& Discrete Theory& Implementation (Notes)\\
%		\hline
%			Important External Calculus Results:
%			\begin{packed_enum}
%				\item[-] Point Carre Lemma
%				\item[-] Laplace Beltrami Operator: $d$ and $\delta$ free.
%				\item[-] Hodge Decomposition
%				\item[-] Here on the dim of harmonic spaces?
%			\end{packed_enum}
%			&
%			The same as the smooth ones.
%			\begin{packed_enum}
%				\item[-] Properties still hold exactly.
%				\item[-] Laplacian in Least square sence
%				\item[-] Border Constraints
%				\item[-] One Form interpolation
%			\end{packed_enum}
%			 & 
%			 Implementing it with DEC
%			 \begin{packed_enum}
%				\item[-] Vector Field Design
%				\item[-] =Least square harmonic 1 Form solving
%			\end{packed_enum}
%			 \\		
%		\hline
%	\end{longtable}
	
The goal of this chapter is to generate natural looking tangential vector fields on two dimensional meshes, using a few intuitive, user-provided constraints, following the approach from Matthew Fisher et al. \cite{vField}. Possible constraints are the positions of sources and sinks or painted general flow directions. Tangential vector fields are naturally described by 1-forms and therefore the use of exterior and discrete exterior calculus suggests itself. 

This chapter is divided in four sections. In the first (short) section the problem statement is formulated with exterior calculus. The problem statement naturally leads to the formulation of harmonicity for 1-forms.  The second  and third section give the theoretical background for a better understanding of the harmonicity of forms and of the topological constraints connected to vector field design. In this context we describe the following exterior calculus and differential topology results:
\begin{enumerate}
\item The Hopf Index Theorem, which describes the possible vector fields on a surface, based on its topology.
\item The Poincar\'e Lemma, which answers the question when a differential form $\omega$ can be 'integrated', i.e. when there exist differential forms $\nu$ with $d\nu = \omega$
\item The Hodge Decomposition, which allows to split any differential form in three 'orthogonal' parts that can be treated somewhat independently, one of them harmonic.
\item A result from Hodge theory, which describes the dimension of the spaces of harmonic forms, solely depending on the topology of the surface.
\end{enumerate}
In the third section we use the continuous theory to  exactly formulate the properties of designed vectorfields and translate this directly to DEC. And we consider vector field design on meshes with borders or holes. 

\section{Problem Statement}
\label{sec:vfproblemstatement}
To produce a nice looking vector field, its visually prominent features need to be controlled. These are general flow directions and singularities.
Singularities are points where the vector field vanishes. These can be sources, sinks, vortices or of a more exotic nature, see Figure \ref{fig:vfsingularities}. The Hopf index theorem described in the next section sates that depending on the topology of the mesh such singularities are unavoidable. Therefore we want to control their type and positions.

We can do so by controlling the curl and the divergence of a vector field. We want to allow divergence and curl only in small, chosen environments. In terms of exterior calculus, this means that we want a 1-form for which $d\omega^1 = 0$ and $\partial \omega^1 = 0$, but for selected points, or rather for very small areas. These are exactly a zero curl and a zero divergence condition, see the schematic in Figure \ref{fig:deRham2d}.  Additionally we want to constrain the vector field to follow some painted strokes but still look natural. This is done by constraining the vector field to fixed vectors on the stroke. But then the zero curl and divergence condition needs to be relaxed. Instead of requiring $d\omega = 0$ and $\partial \omega = 0$ we require the $1$-form only to have a minimal exterior (co)-derivative:
\begin{flalign*}
%\begin{matrix}
\begin{aligned} &\omega^1 = v_{fixed} \\
& d\omega^1 = +-c_{curl} \\
& \partial \omega^1 = +-c_{div} \\
&\langle d\omega^1, d\omega^1\rangle + \langle\partial \omega^1, \partial \omega^1 \rangle \; minimal \\
\end{aligned} & \hspace{0.5cm}
\begin{aligned}
& \text{on strokes} & \\ 
 & \text{for vortices} &\\
 & \text{for source / sink vertices} &\\
& \text{else} &\\
\end{aligned}
\end{flalign*}	
Note that the manifold we are treating is two dimensional; $d\omega^1$ and $\partial \omega^1$ are therefore scalar functions. So $c_{div}$ and $c_{curl}$ are real valued constants. $v_{fixed}$ is a 1-form or in standard calculus terms, just some fixed vectors.
%\subsection{Point Carre Lemma}
%When is a form d or delta of another form?

\section{The Hopf Index Theorem}


%\subsection{Hopf Index Theorem}

\begin{figure}%
\begin{center}
\includegraphics[height=5cm]{imgs/6_vfsingularities.eps}%	
\end{center}
\caption{The computation of the index of some singularities, which is the winding number of $f = v / \abs{v}$. $f$ is depicted by the dashed arrows. Sources (a) and sinks (b) have index 1, saddles (c) have index -1 and (d) is a singularity with index 2. }%
\label{fig:vfsingularities}%
\end{figure}

The Hopf index theorem describes the composition of the singularities of a vector field (proof and an introduction to differential topology in \cite{guillemin1974differential}). Singularities are isolated points where the vector field vanishes. It states that, depending only on the topology, singularities can not be avoided and gives an invariant that is always fulfilled by the singularities. 
The formulation of the Hopf index theorem is the following:
\begin{thm}(Poincar\'e-Hopf Index Theorem) Let $M$ be a compact orientable smooth manifold. Let $v$ be a smooth tangential vector field (i.e. a 1-form) on $M$ with isolated zeros. If $M$ has a boundary $v$ has to point in the outward normal direction on the border. Then
\[\sum_{x_i: v(x_i) = 0} index_v(x_i) = \chi(M),\]
where $\chi(M)$ is the Euler characteristic of $M$.
\end{thm}
Let's clarify the different ingredients of this theorem. The Euler characteristic on a $k$-dimensional manifold is similar as for 2 dimensions. Instead of taking a triangulation you take a simplicial $k$-cell complex and the Euler characteristic is:
\[\chi(M) = \# \textit{0-cells} - \# \textit{1-cells} + \# \textit{2-cells} - \# \textit{3-cells} ...\]
The index of a zero $x_i$ is the degree of the map $\frac{v}{\abs{v}}$ at $x_i$, also called turning number. If $M$ is two dimensional, the index basically counts the number of times 
\[f: x \mapsto \frac{v(x)}{\abs{v(x)}}\] 
rotates when you follow a small circle around $x_i$.
%\footnote{ In higher dimensions, you choose a small $k$-dimensional ball around $x_i$; then the mapping $f$ maps this ball to the $k$ dimensional unit ball and you calculate the \emph{degree} of $f$. Basically you chose some point $z$ and take the number of source points in $f^{-1}(z)$ where $f$ is orientation preserving minus the number of source points where $f$ is not. See e.g. \cite{guillemin1974differential}} 
In Figure \ref{fig:vfsingularities} you find some images to develop an intuition for the index. For example sources and sinks have index $+1$.
 %You look at the source image of some point $z$ of the unit ball, which has a finite set of source points $f^{-1}(z)$. For every of these source points you decide if $x \mapsto \frac{v(x)}{\abs{v(x)}}$ is orientation preserving or not and assign a $+1$ to the point if it is and a $-1$ if it is not. Summing up all these $+-1$ then gives you the index. Provably this sum does not depend on the chosen $z$.


So what is the meaning of the Poincar\'e-Hopf index theorem? 
The theorem ties the existence of vanishing points of a vector field to the topology of the surface. By tying the sum of indices to the Euler characteristic, it gives a constraint on the composition of the singularities that depends only on the topology of the surface.

For example, by the Poincar\'e-Hopf theorem, all vector fields on borderless compact manifolds with non zero Euler characteristic have singularities. A sphere has Euler characteristic $2$. This means that any smooth vector field has to have one vanishing point with index two or two vanishing points with index one (e.g sources and sinks). If there are two sources and one sink then there has to be at least one additional vanishing point with characteristic $-1$ e.g. a saddle. 

A torus has characteristic $0$- it is possible to have vector fields without sources and sinks. It is not possible to have a field with exactly one source and one sink (giving a total index 2), but you could have one source and one saddle as in Figure \ref{fig:torusvf}. 

\begin{figure}%
\includegraphics[height = 5cm]{imgs/6_torusvf.eps}%
\caption{An example for the Hopf index theorem: a torus with a vector field that has one source (yellow) and one saddle (blue). The torus has Euler characteristic 0, a source index +1 and a saddle index -1.}%
\label{fig:torusvf}%
\end{figure}

%Now how does this relate to the search for parametrization of surfaces? The coordinate $0$-Forms $x$ and $y$ that we seek should have non vanishing gradients $dx$ and $dy$ which are 1-Forms or vectorfields. And here the Hopf theorem says that on any compact manifold without border with nonzero Euler characteristic this is not possible, e.g. for spheres or torus. As soon as there are holes present you can not say anything in general, as you can 'cheat' by moving all singularities in the hole. Still, in the setting of parametrizations we do not allow arbitrary behavior on the exterior border. As it is mapped to a simple polygon, the border has to act like a singularity of type (d) when pulled to a point. Therefore we can not parametrize a torus like this. 

\section{Harmonic, Closed and Coclosed Forms}
The formulation of vector field design in Section \ref{sec:vfproblemstatement} actually aims to find  nearly \emph{harmonic} 1-forms with the additional constraint on $d$ and $\partial$ of the form. Here we give some background for the operators $d$, $\partial$ and $\Delta$. The space of $k$-forms on closed  manifolds can be split into three orthogonal spaces: the image of $d$, the image of $\partial$ and the kernel of $\Delta$ which is the space of harmonic forms. The kernel of the Laplacian $\Delta$ also turns out to be the intersection of the kernels of $d$ and $\partial$.

\subsection{General Laplacian}
For $k$-forms the Laplacian is defined as 
\[\Delta_k = (-1)^{n-k+1}\star_{k} d_{n-k+1}^{dual} \star_{k+1}d_k + (-1)^{n-k}d_{k-1}\star_{k-1}d_{n-k}^{dual} \star_{k}\]
or ignoring the implicitely clear sub-indices and using the covariant derivative $\partial_k = (-1)^{n-k}\star^{-1} d \star$ introduced in Section \ref{subsec::coderivativ} 
\[\Delta = \partial d + d \partial .\]
which generalizes the usual Laplacian. We are interested in harmonic $k$-forms, i.e.
\[\Delta_k \omega^k = 0\]
and want to understand what it means for a form to be harmonic. 

\subsection{Curl- and Divergence-Freeness}
First of all remember that the coderivative is adjoint to the derivative with respect to the scalar product for $k$-forms, as seen in Section \ref{subsec::coderivativ}
Being the adjoint means that for borderless Manifolds or when the involved forms are zero on the borders of the Manifold
\[\langle d\omega^k, \nu^{k+1} \rangle = \langle \omega^k,\partial\nu^{k+1}\rangle.\]
Therefore, for harmonic $k$-forms
\begin{eqnarray*}0=\langle\Delta\omega^k,\omega^k\rangle &= &\langle \partial d\omega^k + d \partial \omega^k, \omega^k\rangle \\
 &=& \langle \partial d\omega^k, \omega^k\rangle + \langle d \partial \omega^k, \omega^k\rangle \\
 &=& \langle d\omega^k,d\omega^k\rangle + \langle \partial\omega^k,\partial\omega^k\rangle
\end{eqnarray*}
But because the scalar product is non-degenerated, meaning that $\langle \nu,\nu\rangle \geq 0$  with $=0$ iff $\nu=0$, it follows that
\[d\omega^k = 0\]
\[\partial\omega^k = 0\]
and we get on borderless manifolds or if $\omega^k = 0$ on the border
\[\Delta \omega^k = 0 \Leftrightarrow \: \left\{\begin{matrix}
d\omega^k = 0\\
\partial \omega^k = 0
\end{matrix}\right.\]
For 1-forms, $d$ is the curl operator and $\partial$ is the divergence operator. Therefore, on compact borderless manifolds, harmonic vector fields are curl and divergence free. Harmonic forms are $d$ and $\delta$ free. 
In the vector field design set up the aim is to minimize
$$\langle d\omega^1,d\omega^1\rangle + \langle \partial\omega^1,\partial\omega^1\rangle$$
So we solve for vector fields that are as harmonic as possible. Minimality under $\langle d\omega^1,d\omega^1\rangle + \langle \partial\omega^1,\partial\omega^1\rangle$ is also called weak formulation of harmonicity.
%A slight generalisation is for example a harmonic vector field on $\mathbb R^n$ for which $\lim_{r\to \infty} \int_{\delta B_r} \langle \omega , \omega \rangle  = 0$, as you can show adjointness of the two operators $d$, $\delta$ for such forms as well. That means that harmonic vector fields on $\mathbb R^n$ that decrease fastly enough are also curl and divergence free. \note{think about images}

\subsection{Poincar\'e-Lemma}
The Poincar\'e-Lemma describes when a differential form $\omega$ is the derivative of some other form $\nu$, i.e. when there exists a $\nu$ such that
\begin{equation} d\nu = \omega .\label{eq:exact}\end{equation}
The Poincar\'e-Lemma states that if a manifold $M$ can be parametrized by one open contractible subset\footnote{For example a ball-shaped region} of $\mathbb R^n$ this is exactly the case if
\begin{equation} d\omega = 0 \label{eq:closed}\end{equation}
A differential form $\omega$ for which there exists a $\nu$ such that Equation \ref{eq:exact} holds is called \emph{exact}, while an $\omega$ fulfilling Equation \ref{eq:closed} is called \emph{closed}. The Poincar\'e Lemma is also true for the codifferential operator $\partial$. Under the same conditions as above a form is co-exact exactly if it is co-closed
\[\exists \nu : \partial \nu = \omega\] \[\Leftrightarrow\]\[ \partial \omega = 0.\]
This means that, for general manifolds at least locally, a form is the exterior derivative of another form, exactly if it is $d$-free. The same is true for the coderivative $\partial$. 

%The Lemma is usually proven by construction; you can always construct a solution $\nu$ with $d\nu = \omega$ if $d\omega = 0$. 

The lemma also answers the question how unique a $\nu^{k}$ is, when you require $d\nu^{k} = \omega^{k+1}$. If $\nu_1$ and $\nu_2$ are solutions to Equation \ref{eq:exact}, then $d(\nu_1-\nu_2) = 0$. By the Pointcar\'e-Lemma the difference $ \nu_1 - \nu_2$ is again the exterior derivative of a $(k-1)$-form $\xi^{k-1}$, therefore 
\[\nu_2 = \nu_1 +  d\xi^{k-1}.\]
This means that given one special solution $\nu$ you get all solutions by adding the exterior derivative of arbitrary $k-1$ forms.

%\note{Image of curl free vector field that is only locally a gradient}
%\note{sketch what we know in a diagram}
\subsubsection{Example}
On any disc shaped manifold a $1$-form $\omega^1$ is the $d$ of a $0$-form exactly if $d\omega^1 = 0$. In vector calculus terms this means that a vectorfield is the gradient of a function exactly if the vector field is curl free. Or in 3 dimensions a vector field is the curl of another vector field exactly if it is divergence free.


\subsection{Hodge Decomposition}
\label{sec:VF_hodgeDecomp}

We have seen three kinds of differential forms: exact forms, co-exact forms and harmonic forms. 
The Hodge decomposition theorem states that any $k$-form can be split in three orthogonal complements: a harmonic part $\gamma: \Delta \gamma = 0$, an exact part $d\alpha^{k-1}$ and a co-exact part $\partial \beta^{k+1}$:
\begin{thm}(Hodge Decomposition)(p372 in \cite{FRANKEL11})
On \emph{borderless, compact manifolds} $M$ any differential $k$-form can be expressed as
\[\omega^k = d\alpha^{k-1} + \partial \beta^{k+1} + \gamma\]
Exact forms, co-exact forms and harmonic forms are mutually orthogonal to each other:
\[\langle d \alpha^{k-1}, \partial \beta^{k+1}\rangle = \langle d \alpha^{k-1}, \gamma\rangle = \langle \gamma, \partial \beta^{k+1}\rangle =0\]
\end{thm}

The idea  behind this result is that the spaces of exact and co-exact differential forms are orthogonal to each other and their orthogonal complement is the space of harmonic forms (Figure \ref{fig:hodgeDecomp}). The actual proof is complicated and technical.
We show that the Hodge decomposition works in Discrete Exterior Calculus. And in DEC the proof is very simple, because all involved spaces have finite dimensions. 

Realising that these three spaces are orthogonal to each other simplifies the vectorfield design problem: vorticity constraints can be met independently of divergence constraints.
Also, if there exist harmonic $1$-forms on a manifold, then specifying only divergence and curl constraints does not uniquely determine a vectorfield, as there is no constraint on the harmonic component $\gamma$.

The Hodge decomposition also generalized the Pointcar\'e lemma to arbitrary closed manifolds. The closedness of a form $d\omega = 0$ is not enough for the form to be exact, i.e. to be the exterior derivative of another form $d\nu = \omega$. But additionally the harmonic part of $\omega$ needs to be zero too. Or put differently:
\[d \omega = 0\]
\[\Leftrightarrow\]
\[\omega = d \nu + h\]
where $h$ is a harmonic form, see also Figure \ref{fig:hodgeDecomp}

\begin{figure}%
\begin{center}
\includegraphics[height=5cm]{imgs/7_hodgethmSchema.eps}%	
\end{center}
\vspace{-0.5cm}
\caption{By the Hodge decomposition theorem the space of $k$-forms on a closed manifold can be decomposed in the three spaces  of exact forms $d\Omega^{k-1}$, coexact forms $\partial\Omega^{k+1}$ and harmonic forms $Ker(\Delta_k)$. In general, closedness is not enough for a form to be exact, its harmonic component has to vanish too.}%
\label{fig:hodgeDecomp}%
\end{figure}

\subsection{Discrete Hodge decomposition}
In DEC the discrete version of the Hodge Decomposition holds. This  means that in practice, when using DEC, constraints on $d\omega$ and $\partial \omega$ can be enforced independently from each other. As we have seen with conformal maps, when border constraints are enforced, some DEC matrices have to be adapted. To be able to use the Hodge decomposition also when using adapted matrices, we formulate the discrete Hodge theorem generally. We state fundamental properties for the matrices $d$, $\star$ and $\partial$ that need to hold for the Hodge decomposition to hold.

\begin{thm}(Discrete Hodge Decomposition) If the matrices $d_k$,$\partial_k$, $\star_k$ and the scalar product $\langle , \rangle_k$ on the space of discrete $k$-forms fulfill
\begin{align}
d_{k-1} d_{k} &= 0 \\
\langle  d_k\discrete{v}^k,\discrete{w}^{k+1} \rangle_{k+1} &= \langle  \discrete{v}^k,\delta_{k+1}\discrete{w}^{k+1} \rangle_k \label{eq:adjoint}
\end{align}
then the following three subspaces are orthogonal to each other, with respect to the scalar product $\langle.,.\rangle_k$, and span the space of all discrete $k$-forms:
\[Img(d_{k-1}) \perp Img (\partial_{k+1}) \perp \left(Ker(d_k) \cap Ker(\delta_k)\right)\]
This means that any discrete $k$-form can be written uniquely as a sum
\[\discrete{w}^k = d_{k-1} \discrete{a}^{k-1} + \partial_{k+1} \discrete{b}^{k+1} + \discrete{c}^k\]
where $\discrete{c}^k$ is a discrete harmonic $k$-form.
\end{thm}

To see this we just need to use the adjointness \ref{eq:adjoint}: $Img(d_{k-1}) \perp Img (\partial_{k+1})$ is true as
\[\langle d \discrete{w} , \partial \discrete{v}\rangle = \langle \underbrace{dd}_{0} \discrete{w} , \discrete{v} \rangle = 0 \;\;\; \forall\; \discrete{v,w}\]
Furthermore $(Img(d_{k-1}) \perp Img (\partial_{k+1}))^\perp = \left(Ker(d_k) \cap Ker(\delta_k)\right)$ because
\begin{align*}
(Img(d_{k-1}) \perp Img (\partial_{k+1}))^\perp &= \discrete{w}^k: \begin{cases} \langle \partial_{k+1}\discrete{v}^{k+1},\discrete{w}^k \rangle = 0 & \forall \discrete{v}^{k+1}\\
\langle d_{k-1}\discrete{v}^{k-1},\discrete{w}^k \rangle = 0 & \forall \discrete{v}^{k-1}\end{cases} \\
&= \discrete{w}^k: \begin{cases} \langle \discrete{v}^{k+1},d_{k}\discrete{w}^k \rangle = 0 & \forall \discrete{v}^{k+1}\\
\langle \discrete{v}^{k-1},\partial_{k}\discrete{w}^k \rangle = 0 & \forall \discrete{v}^{k-1}\end{cases} \\
&= \discrete{w}^k: \begin{cases} d_{k}\discrete{w}^k = 0 \\
\partial_{k}\discrete{w}^k  = 0 \end{cases}\\
&= Ker(d_k) \cap Ker(\delta_k)
\end{align*}


%\subsection{Dimension of Harmonic spaces}
%\note{Section as written above, yet to be re placed, reorganized.}
%\note{Note: adjointness  is dirichelet enrgey to laplacian relation for 1d as $\int \abs{\nabla}^2 = \langle d,d \rangle ...$}



\subsection{Dimension of Harmonic spaces }
\label{subsec:vf_dimHarm}
\begin{figure}%
\begin{center}
\includegraphics[height = 3.5cm]{imgs/7_torusharmonic.eps}%	
\end{center}
\caption{The first Bettinumber for the torus is 2, as it consists of one loop. Therefore the space of harmonic vectorfields on the torus has dimension 2, every harmonic vector field is a combination of the two fields depicted here.}%
\label{fig:vftorusharmonic}%
\end{figure}

To round off the description of harmonic forms we want to mention the following reformulation of a theorem by de Rham.
The result we are interested in states (p373 in \cite{FRANKEL11})
\vspace{0.5cm}

\emph{The dimension of the kernel of the Laplacian on the space of $k$-forms on a compact borderless manifold is equal to the dimension of the de Rham cohomology group of degree $k$, the dimension of which is the $k$-th Bettinumber.} 

\vspace{0.5cm}
We mention cohomology groups here only because they have to be mentioned in this context at least once; we are interested in the statement that the $\dim ker(\Delta) = Betti_p$.

 The Bettinumbers are finite natural numbers that depend only on the topology of the manifold. 
The zeroth and the first Bettinumber capture the following properties on two dimensional compact Riemannian manifolds: the 0-Bettinumber is the number of connected components of the manifold. The first Bettinumber is two times the number of ''handles`` or ''loops`` the manifold has. Here ''handle`` is meant in its plainest sense. A coffee cup usually has exactly one loop-like handle, therefore the first Betti number of its surface is 2. A donut consist of exactly one loop-like handle, so its first Bettinumber is 2 as well. It is also merely a deformation of a classical coffee cup. A pretzel consist of three handles so its surface has Bettinumber 6. And a sphere has no handles at all, therefore it has Betti number 0.

The result therefore states not only that the space of harmonic forms always has a finite dimension, but the dimension also only depends on the topology of the manifold, and this, again, is very nice. Figure \ref{fig:vftorusharmonic} shows the result on the torus for vector fields. The overview over the different $k$-form subspaces on closed manifold is depicted in Figure \ref{fig:hodgeDecomp}.


The theory about harmonic forms, closed forms and co-closed forms on closed (i.e. compact and borderless) manifolds is rich and beautiful but also deep. To see more of it I recommend the book 'The Geometry of Physics` \cite{FRANKEL11}, Chapters 13 and 14 which describes the theory and its backgrounds in a concise, yet intuitive way, mentioning also the technical problems that would arise in a thorough treatment.

%The result we are interested in states \note{[p373 physics book]}: 
%\[\]
%\emph{The dimension of the kernel of the Laplacian on the space of $k$-forms is equal to the dimension of the de Rham cohomology group of degree $k$; the dimension of this kernel is the $k$-th Betti number.} 
%\[\]
%This needs some explanation; we go through this statement term by term. The kernel of the Laplacian on the space of $k$-Forms is easy to understand: it is the vector space of $k$-Forms for which $\Delta \omega^k = 0$ i.e. the space of harmonic $k$-forms. The harmonic $k$-forms form a vector space over $\mathbb R$ as any multiple of a harmonic $k$-form is a harmonic $k$-form and the sum of two harmonic $k$-forms is again harmonic.
%
%The $k$-th de Rham cohomology group is quite abstract. It is the space $Ker(d_k)/Img(d_{k-1})$ where $/$ is the 'modulo' operator. \note{Just now it is of no immediate importance; but we will meet this again when treating the Hodge decomposition and here is the right context to mention it. .... do i need to mention it?}
%
%A Betti number is a natural number and a topological invariant. Topological invariants stay the same when the underlying space is deformed. For example if you deform a manifold to an other manifold without tearing new holes in it or glueing multiple points to each other, topological invariants stay the same. You can really think of the manifold being made out of plasticine and any deformation which does not tear new holes in it or where you merge outer faces is allowed. 
%
%Even without understanding Betti numbers any further you can see that the result is quite strong: the solution space of harmonic $k$-forms on compact Riemannian manifolds is of some fixed finite dimension that depends only on the topology of the manifold i.e. its very general form. 
%
%The zeroth, and the first Bettinumber capture the following properties on two dimensional compact Riemannian manifolds: the 0-Betti number is simply the number of connected components of the manifold. The first Betti number is two times the number of ''handles`` or ''loops`` the manifold has. Here ''handle`` is meant in its plainest sense. A coffee cup usually has exactly one loop like handle, therefore the first Betti number of its surface is 2. A donut consist of exactly one loop like handle so its first Betti number is 2, it is also merely a deformation of a classical coffee cup. A pretzel consist of three handles so its surface has Betti number 6. And a sphere has no handles at all, therefore it has Betti number 0. \note{image}
%
%
%Just to mention it: the $k$-th Betti number for smooth Riemannian manifolds is more or less directly defined as the dimension of a cohomology group $Ker(d_k)/Img(d_{k-1})$. But we don't care about this. The important point is that they are topological invariants and therefore the dimension of the space of harmonic $k$-forms also only depends on the topology of the manifold.
%
%
%\note{Note that here there is a lot that is left unsaid. Bettinumbers, chain complexes, cochain complexes - you can get abstracted and abstracter - and see the symmetry between border operator, simplicial complexes etc clearer and clearer.}


\section{Application: Vector Field Design}
By the Hodge decomposition theorem, constraints on the divergence of a vector field are independent of constraints on the curl of a vector field. We solve for a $1$-form $\discrete{w}^1$ with prescribed divergence and curl and additional directional constraints.
First of all we note that in DEC the divergence of a discrete vector field $\partial \discrete{w}^1$ is a discrete $0$-form, therefore divergence constraints are given per vertex on a mesh. Rotational constraints are given by triangle, as the curl $d\discrete{w}^1$ is a discrete $2$-form. Finally directional constraints constrain specific values of the discrete $1$-form $\discrete{w}^1$ and therefore are given on selected edges.

All the constraints can be compiled easily in one linear system:
\begin{equation}\begin{pmatrix} d \\
\partial \\
Z
\end{pmatrix} \discrete{w}^1
= \begin{pmatrix}
r_t \\
s_v \\
c_z
\end{pmatrix} \label{eq:vfdesign1}
\end{equation}
where $Z$ is the identity matrix, constrained to the edges  affected by directional constraints, $r_t$ is the vector of target rotation values of defined by triangle, $s_v$ is the  target divergence values, defined by vertex and $c_z$ is the vector of constraints per edge, given by the directional constraints.

The constraints $r_t$, $s_v$ and $c_z$ are provided by a user and are very likely to be contradictory. For example constraints on singularities as vortices given by the Hopf Index theorem are easily violated, if an impossible combination of singularities is prescribed. Or directional constraints might induce divergence or curl. Therefore the linear system is solved in a least square sense.

But the correct norm to for the least squares problem is not the Euclidean norm. For example we want to minimize $d\discrete{w}^1-r_t$ as a discrete $2$-form and as seen in Section \ref{subsec:5_discreteScalarprod} the correct norm for $0$-forms is given by $\star_0$, meaning that
\[\langle d\discrete{w}^1, d\discrete{w}^1\rangle \rangle_2 = (d\discrete{w}^1)^T \star_2 (d\discrete{w}^1)
\]
should be minimal and analogously
\[\langle \partial \discrete{w}^1, \partial\discrete{w}^1\rangle \rangle_0 = (\partial\discrete{w}^1)^T \star_0 (\partial\discrete{w}^1)\]
therefore the linear equation \ref{eq:vfdesign1}
should be minimized relative to the scalar product or norm given by the diagonal matrix
\[L = \begin{pmatrix}
\star_2 & & \\
 & \star_0 & \\
 & & W
\end{pmatrix}\]
\[\langle \cdot , \cdot\rangle_{vfdesign} = (\cdot)^T L (\cdot)\]
\[\abs{\abs{ \cdot}}_{vfdesign} = \langle \cdot,\cdot \rangle_{vfdesign}\]
where $W$ is a diagonal weight matrix introduced to additionally control the influence of directional constraints. The least square solution $\discrete{w}^1$ 
\[\arg\min_{\discrete{w}^1}\abs{\abs{\begin{pmatrix} d \\
\partial \\
Z
\end{pmatrix} 
\discrete{w}^1 - \begin{pmatrix}
r_t \\
s_v \\
c_z
\end{pmatrix}}}_{vfdesign}\]
is then the solution of the normal equation 
\begin{equation}\begin{pmatrix} d^T & \partial^T & Z^T \end{pmatrix} L \begin{pmatrix} d \\
\partial \\
Z
\end{pmatrix} \discrete{w}^1= \begin{pmatrix} d^T & \partial^T & Z^T \end{pmatrix} L \begin{pmatrix}
r_t \\
s_v \\
c_z
\end{pmatrix}. \label{eq:vfdesignLQ}\end{equation}
This linear system can be implemented directly and solved with any sparse solver.

Again note the elegance of DEC: the vector field design problem can be formulated directly in DEC and leads to a simple sparse linear system. But there are two points not yet answered: how to treat meshes with borders and how to gain a vectorfield on the mesh surface from a discrete $1$-form, which is only a set of values stored on edges.

\subsection{1-Form to Vector Field}
\label{sec:vf_1form2vf}
Up to now we designed a discrete $1$-form $\discrete{w}^1$, without clarifying how the discrete 1-form can be used to generate a tangential vector field on the mesh. So given a discrete  1-form, i.e. a set of values defined per edge that represent values sampled as described in Section  \ref{subsec:samplingForms} how to generate a tangential vector field? In other words we want a discrete analogue to the continuous $\sharp$ operator from Section \ref{subsec:diffformsare}.

The definition of sharp operators for discrete exterior calculus is not obvious and closely related to interpolation schemes. One approach is to interpolate the values of the discrete $1$-forms to define a continuous differential $1$-form on every triangle. The interpolated $1$-forms integrated along an edge $e$ has to lead to the value $\discrete{w}^1(e)$. Then the non-discrete sharp operator can be applied to the interpolated 1-form to get a vector at any position on the mesh.

In \cite{hirani03} it is argued that there are at least four different $\sharp$ operators that can be gained from any single interpolation scheme. Also discrete dual forms need to be interpolated differently than discrete primary forms, as they are defined on different simplices or cells. 

The authors of the vector field design paper \cite{vField} propose to use so called Whitney forms or Whitney elements for the interpolation step. This leads to differentiable $1$-forms on every triangle, but with discontinuities on vertices and edges. We skip all details here and simply give the resulting $\sharp$ operator. For any triangle $t_{ijk}$ with vertices $i,j,k$ and edges $e_{ij}, e_{jk},e_{ki}$ the proposed vector at a point with barycentric coordinates $\alpha_i, \alpha_j, \alpha_k$ is
\begin{align} u(\alpha_i, \alpha_j, \alpha_k) = \frac{1}{2area(t_{ijk})} ( &(\discrete{w}_{ki} \alpha_k - \discrete{w}_{ij} \alpha_j)e_{jk}^\perp ) + {}\nonumber\\
&(\discrete{w}_{ij} \alpha_i - \discrete{w}_{jk} \alpha_k)e_{ki}^\perp + {}\nonumber\\ 
&(\discrete{w}_{jk} \alpha_j - \discrete{w}_{ki} \alpha_i)e_{ij}^\perp
\label{eq:vfSharp}
\end{align}
where $e^\perp$ denotes the edge $e$ rotated by $90^\circ$ in the plane of the triangle $t_{ijk}$, according to the orientation of $t_{ijk}$. We then have as a discrete sharp operator
$$\discrete{w}^{1\sharp}_{ijk}(\alpha_i, \alpha_j, \alpha_k) = u(\alpha_i, \alpha_j, \alpha_k).$$

Note that in the context of fluid simulation in the next Chapter we will need to get a continuous vector field from discrete $1$-form, such that this $\sharp$ will not be suited. But the $1$-form will be strictly divergence free and we will give a $\sharp$ operator for this special case.


\subsection{Directional Constraints}

\begin{figure}%
\begin{center}
\includegraphics[height= 2cm]{imgs/7_mousestroke.eps}%	
\end{center}
\caption{Getting a directional constraint (blue) from the mouse stroke}%
\label{fig:7_mouse}%
\end{figure}

The directional constraints $Z \discrete{w}^1 = c_z$ are provided by the user using strokes (Figure \ref{fig:7_mouse}). The simplest way to translate a stroke into $1$-form constraints is to capture the path of the mouse stroke and translate it to vectors that describe the direction of the mouse motion per triangle.
A single vector $v$ on a triangle $t$ can then be interpreted as describing a constant vector field on the triangle. According to the sampling scheme described in Section \ref{subsec:samplingForms}, the target value for an edge $e$ of the triangle $t$ simply is the projection of $v$ on $e_t$:
$$c_z = \langle v, (e.end - e.start) \rangle$$
If an edge $e$ is part of two triangles with different vector valued constraints $v_1,v_2$, simply taking the value
\[c_z= \frac{1}{2}\left(\langle v_1, (e.end - e.start) \rangle + \langle v_2, (e.end - e.start) \rangle\right)\]
works well.

For the magnitude of the directional constraints $v$ we propose two schemes. The first is to norm all directional constraints to the same user-specified length $\lambda$, i.e.
\begin{equation} \abs{v} = \lambda \label{eq:vfconstraint1}\end{equation}
The second one is to solve Equation \ref{eq:vfdesignLQ} once without any directional constraints to generate a first vector field $\nu_{pre}$ and then to scale the directional constraints to have the same length as the precomputed vector field:
\begin{equation} \abs{v} = \abs{\nu_{pre}} \label{eq:vfconstraint2}\end{equation}

\subsection{Results}

Both directional constraint types have different properties. Specifying a constraint of the type in Equation \ref{eq:vfconstraint1} can introduce new divergence or influence the vector field globally in an unintuitive way. Especially if the constraint is drawn in an environment where the vector field nearly vanishes, the enforcement of the no divergence constraint might lead to a global change of the vectorfield flow, see Figure \ref{fig:vf_dirconstr}. On the other hand you can use the introduction of divergence to your advantage: you can directly draw sources or sinks or more complex singularities which then are enforced \ref{fig:7_paintedSing}.

\begin{figure}%
\begin{center}
\includegraphics[width=0.75\columnwidth]{imgs/7_cube_directionals.eps}%	
\end{center}
\vspace{-15pt}
\caption{(a) On the front of a cube a source and a sink have been set. On the back of the cube a directional constraint is given by a stroke but is not yet enforced. (b) The directional constraint is enforced following Equation \ref{eq:vfconstraint1}. Because the length of the constraint is not well chosen the field is affected globally.
(c) The directional constraint is enforced using precalculated vector lengths following Equation \ref{eq:vfconstraint2}. This introduces no new divergence and the front of the cube stays unaffected. }%
\label{fig:vf_dirconstr}%
\end{figure}


Constraints described by Equation \ref{eq:vfconstraint2} are less aggressive; the impact on divergence is less pronounced. The constraint makes the designed vector field  follow the given stroke without affecting the global look of the vector field. But you can not use such a constraint to specify new singularities.

\begin{figure}%
\includegraphics[width=\columnwidth]{imgs/7_paintedsingularities.eps}%
\caption{The constraint from Equation \ref{fig:7_paintedSing} can be used to induce singularities, e.g. vortices (a), saddles (b) or spirals (c)}%
\label{fig:7_paintedSing}%
\end{figure}

\subsection{Border Constraints}


Meshes with boundaries are not treated correctly or intuitively without further adaptations, see Figure \ref{fig:7_adaptedvsUnadapted}. The problem is that the discrete zero divergence condition
\[\partial \discrete{w}^1 = 0\]
is not enforced correctly on boundaries. The problem is that border vertices do not have a full $1$-neighborhood of faces. 

Let's look at the discrete co-derivative $\partial_1 = \star^{-1}_0 d^{dual} \star_1$. The $\star_1$ applied to the discrete $1$-form $\discrete{w}^1$ leads to a dual $1$-form; the values associated to the dual edges represent the flow of the vector field going through the dual edge. The dual discrete derivative $d^{dual}$ sums up the values on the dual edges along the border of dual faces. At interior vertices the value $d^{dual} \star_1 \discrete{w}^1$ describes the overall flow in or out of the associated dual faces and is a correct approximation of divergence. But dual faces of border vertices are not closed and no flow through the boundary is considered (see Figure \ref{fig:vf_borderconstr}). Therefore, $\partial^{discrete} \discrete{w}^1 = 0$ implicitly enforces that there is no flow through the boundary of meshes, and that the designed vector field is parallel to borders.

In order to compute the total divergence on a dual face of a border vertex we need to know the flow through primary border edges. But this is a problem. The value stored on primary edges is the flow along the edge. Put differently we would need to know the dual form on primary vertices. %\note{It might be an interesting approach to store the dual and primal values on all simplices. this would fascilitate border constraint treatment and might allow non voronoi duality: from primary and dual value it might be possible to make good guesses of primary and dual value on non orthogonal simplices.}

The authors of \cite{vField} propose to use the zero curl condition $d\discrete{w}^1 = 0$ together with the Whitney based interpolation scheme to compute the flow through a primary border edge. The curl freeness on a triangle $t_{ijk}$ means $\discrete{w}^1_{ij} = - \discrete{w}^1_{jk} - \discrete{w}^1_{ki}$ and using the interpolation scheme from Equation \ref{eq:vfSharp} they compute the flow $f_{ij}$ through the edge $e_{ij}$
\begin{equation}f_{ij} = \int_{e_{ij}}\langle u(t, 1-t, 0), \frac{1}{\abs{e_{ij}}}e_{ij}^{T}\rangle = \discrete{w}^1_{jk} \cot(\theta^i) - \discrete{w}^1_{ki} \cot(\theta^j)\label{eq:vfFluxBorder}\end{equation}



\begin{figure}%
\includegraphics[width=\columnwidth]{imgs/7_borderunadapted.eps}%
\caption{A square with border. Without further adaptations of the DEC matrices the vectorfield is implicitely enforced to be tangential to the border (a). The adapted matrices allow the vector field to behave freely on the boundary (b). }%
\label{fig:7_adaptedvsUnadapted}%
\end{figure}


\begin{figure}
\begin{center}
\includegraphics[height=2.5cm]{imgs/7_vfborderconstraint.eps}%	
\end{center}
\caption{A discrete dual 1-form $\star_1\discrete{w}$ stores the flux through the edges (blue arrows) on the dual edges. Inner vertices have a full 1-neighborhood (a) and summing up the dual edges around a vertex is a correct estimation of divergence. Border vertices do not have a full neighborhood (b), summing up the dual edges around a vertex neglect any flow through the mesh boundary. Therefore $\partial_1$ needs to be adapted to take the estimated flux over the boundary into account as well (c).}%
\label{fig:vf_borderconstr}%
\end{figure}


The estimated flow through the border edges can be used to control the behaviour of the vector field on the boundaries. When computing $\delta_1$, we can adapt the matrix $d_1^{dual} \star_1$ to use the estimated flux through the boundary edges as well (see Figure \ref{fig:vf_borderconstr}).



\begin{figure}
\begin{center}
\includegraphics[height = 4cm]{imgs/8_5_borderAdapt.eps}
\end{center}
\caption{The flow along the edges $e_{ki}= (v_k,v_i)$ and $e_{jk} = (v_j, v_k)$ is known. The flow through the edge $e_{ij}$ can then be estimated as the flow through $x$ plus the flow through $y$. Some basic trigonometry shows that the length of $x$ is $\abs{e_{ki}}\cot(\theta_j)$ and the length of $y$ is $\abs{e_{jk}}\cot(\theta_i)$. As $x$ and $y$ are orthogonal to $e_{ki}$ and $e_{jk}$ respectively, the flow through $x$ and $y$ can be estimated as $-\discrete{w}^1_{ki} \cot(\theta^j)$ and $\discrete{w}^1_{jk} \cot(\theta^i)$ respectively, leading to Equation \ref{eq:vfFluxBorder}}
\label{fig:8_3_borderWeights}
\end{figure}


\subsubsection{Implementation}
All that needs to be changed is that the matrix $d_1^{dual} \star_1$ needs to be adapted on all rows corresponding to boundary vertices to take the estimated boundary flux into account. Supposing the edges $e_{ij}$, $e_{jk}$ are two boundary edges at the vertex $v_j$, then
\[(d_1^{dual} \star_1)_{adapted} (v_j) := d_1^{dual} \star_1(v_j) + 0.5 f_{ij} + 0.5 f_{jk}\]
The factor $+ 0.5 f_{ij} + 0.5 f_{jk}$ for boundary vertices can be easily described as a matrix
\begin{align*}M_{borderdiff}& = \bordermatrix{ 
& & \downarrow v \in e_1 \in border & &\downarrow v\in e_{2} \in border& & \cr
v \notin border \rightarrow & 0 & 0& ...& 0& 0 \cr
v\in border \rightarrow & 0 & 0.5 \cdot sgn(e_{1}, brdr) & ... & 0.5 \cdot sgn(e_{2}, brdr)&0\cr
& 0 & & ...& & 0 \cr
& 0 & & ...& & 0
} \cdot \nonumber \\
& \bordermatrix{ 
& & & & &\cr
e \notin \; border & 0 & 0 & ...&0 & 0 \cr
e_{ij} \in\;border& & \cot(\theta^i)sgn(e_{jk}, t_{ijk}) & ... & \cot(\theta^j)sgn(e_{ki}, t_{ijk}) & \cr
& 0 & & ...& & 0
}\end{align*}
The second matrix is a square matrix of size $(\# edges \times \# edges)$ that computes the flux for border edges following Equation \ref{eq:vfFluxBorder} and is zero for all non border edges. The first matrix computes the border flux relevant for border vertices and is of size $(\#vertices \times \#edges)$. The $sgn$ denote the relative orientations of the edges to the border or the face they lie in.

%\note{could simply use the equality $e_{1}^\perp = cot(\theta_2)e_2 + cot \theta_3 e_3$}

\subsubsection{ Pros and Cons of this approach} In the vector field design setting using the adapted matrix works very well. But to generally use the adapted $(d_1^{dual} \star_1)_{adapted}$ on meshes with borders might not be a good idea. First of all the constraint works with the assumption that there is no curl on boundary triangles. Secondly  with the adapted matrix, the property $\partial^{dual}_1\partial^{dual}_2 = 0$ does not hold any more and the adjointness of $d_0$ and $\delta_1$ also fails. And with it the discrete Hodge theorem. 
From a theoretical standpoint where we want the DEC operators to have properties similar to the properties of the continuous operators the loss of adjointness and $\partial^{dual}_1\partial^{dual}_2 = 0$ is bad. 

On the other hand it is natural for the Hodge decomposition theorem to fail on meshes with boundaries- only if the forms fulfill additional boundary constraints the Hodge decomposition holds. That without further adaptation  of the DEC matrices forms tangential to the boundary appear is interesting, as it mirrors the smooth setting, where the Hodge decomposition can be shown to hold on arbitrary manifolds for forms tangential to the border \cite{FRANKEL11}. 

Finding good adaptations for DEC matrices to enforce boundary conditions seems to be highly application dependent.
In the next Chapter where a fluid simulation using DEC is presented there will also be the need to solve for a harmonic $1$-form with given boundary constraints. But instead of using the approach described here, the tangentiality of the $1$-form searched is used and matrices are derived that are consistent with the discrete Hodge decomposition theorem, in order to achieve higher precision.

Nevertheless, in this vector design setting, or in any setting where the emphasis is on visually pleasant results rather than on optimal precision, the adapted $d^{dual}$ operator is very suitable.