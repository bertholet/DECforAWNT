\newpage
\chapter{Two Dimensional Surfaces}

\begin{figure}[h]
\begin{center}
\includegraphics[width = 7cm]{imgs/3_1_curvaturezebra.EPS}
\end{center}
\caption{A zebra mesh that is coloured according to its local mean curvature}
\end{figure}

In this chapter we focus completely on two dimensional surfaces. The theory here is not vital to understand exterior calculus or discrete exterior calculus and, if you are in a hurry, you may skip this chapter. Nevertheless as surfaces and triangle meshes play an important role in many applications it is worth to get acquainted with ways to describe and compute local mesh properties, as developed here. And you can see some of the concepts introduced for manifolds are applied to 2-dimensional surfaces.

In this chapter we start with a short differential geometry primer where we introduce basics like curvatures and a 'metric', a scalar product consistently defined on all tangential spaces. In the second section we will then thrive to get a notion of curvatures on meshes and thereby have a first encounter with a Laplace operator defined on a surface. The third section again is the practical section where we give some implementation details for curvatures and present surface smoothing as an application.

\begin{figure}[ht]
\begin{longtable}{|p{4.5cm}|p{4.5cm}|p{4.5cm}|}
\hline
Smooth Theory& Discrete Theory& Implementation (Notes)\\
\hline
Differential Geometry & Discrete Diff. Geo. & Implementation\\
-Curves & - A geometric go at the DEC $\Delta$ & -Implement mean curv by splitting matrices\\
-Curvatures & - Mean curv, principle curv etc.& -Application: Surface Smoothing\\
-Euler Characteristic? & & -Application: Remeshing? \\			
\hline
\end{longtable}
%\caption{The topics treated in this chapter}
\end{figure}

\section{Differential Geometry}
Differential geometry studies local and global properties of 2 dimensional surfaces embedded in $\mathbb R^3$. This section represents a dirty introduction to the basic notions of differential geometry, such as curvature. Many (not unimportant) details are omitted and technical problems ignored, be aware of this when reading. There are many good books introducing differential geometry, for example .... if you are interested in a clean introduction.

Differential geometry deals with surfaces, but in order to understand surfaces we need to understand curves. So we start with:

\subsection{Curves in $\mathbb R^3$}
Curves are 1-manifolds but somewhat special in that curves can be parametrized by one global map, while with higher dimensional manifolds you have to fall back to local maps. 

The manifold definition for curves translates to: 

\begin{definition}[Curve] A curve is a set of points $C \subset \mathbb R^3$ for which there exists a (differentiable, injective) parametrization $\alpha : (a,b) \to C$ with $\left|\alpha'(t)\right| >0$ for all $t\in (a,b)$. We make a strict difference between the parametrization of a curve and the curve itself. The curve is the point set described by the parametrization.
\end{definition}

%\note{Note on length? Nah trivial enough}

We have already seen that the first derivative of a parametrization describes tangents. To interpret higher derivatives we have to look at a special parametrization of curves: the parametrization by arc length.

\subsubsection{Parametrization by Arc Length}

A curve is parametrized by arclength if $|\alpha'(t)|=1$ for all $t$. This means that the curve is run through with constant speed (Figure \ref{fig::3_1_paramByArclength}). The name ``parametrized by arclength'' comes from the simple fact that the length of a curve segment 
\[l_a^b = \int_a^b |\alpha'| dt = b-a\]
is directly given by the parameter. It is not completely obvious, that such a parametrization by arclength always exists, but it does. Actually there exist always two parametrizations by arclength with different orientations. Note also, that even if you only have an arbitrary parametrization you can still easily calculate what the derivatives by arclength are at some given point; see the Appendix for this and for a sketch of the proof of existence of the parametrization. A curve parametrized by arc length is shown in Figure \ref{fig::3_1_paramByArclength}.

\begin{figure}[h]
\begin{center}
\includegraphics[width= 10cm]{imgs/3_1_paramByArclength.eps}
\end{center}
\caption{This curve is parametrized by arclength: lengths are preserved under $\alpha$ and the length of the curvesegment $\alpha([a,b])$ is simply $b-a$. The vectors $\alpha'$ (blue) have constant length 1.}
\label{fig::3_1_paramByArclength}
\end{figure}

\subsubsection{Curvature}
The second derivative of a parametrization by arclength has geometric meaning: it describes the curvature of a curve. Considering only parametrizations by arc length is not a restriction, as such parametrizations always exist and all derivatives of a parametrization by arclength can be gained from any parametrization.
	
Let $\alpha$ be a parametrization by arc length. The first derivative $\alpha'$ gives the (normalized) tangent direction; the second derivative $\alpha''$ then describes how fast and in what direction the tangent of the curve changes. More formally:
\begin{enumerate}
\item The vector $\frac{\partial^2 }{\partial t^2} \alpha$ is always orthogonal to the tangent $\frac{\partial}{\partial t} \alpha$. 

Because $|\alpha'|= 1$ is constant 
\[\langle \frac{\partial}{\partial t} \alpha,\frac{\partial}{\partial t} \alpha\rangle = const\]
\[0=\frac{\partial}{\partial t} \langle \frac{\partial}{\partial t} \alpha,\frac{\partial}{\partial t} \alpha\rangle = 2 \langle \frac{\partial^2}{\partial t^2} \alpha,\frac{\partial}{\partial t} \alpha\rangle\]
and this means orthogonality. You can check the last equation by plugging in the definition of the dot product and using the product rule of the derivation.
\item The absolute value of the second derivative $|\frac{\partial^2}{\partial t^2} \alpha|$ is the \textbf{curvature} (denoted by $\kappa$) of the curve. It describes how much the curve curves at a given point. 

A geometric interpretation of curvature is given by the curvature radius $r= \frac{1}{\kappa}$. In the plane given by the tangent and the second derivative, this radius is the radius of the circle that approximates the curve in the best way. More precisely: the curvature circle approximates the curve up to the second derivative. In comparison the tangent approximates the curve up to the first derivative (see Fig. \ref{fig::3_1_curvatureRadius}). 
\begin{figure}[h]
\begin{center}
\includegraphics[width=5cm]{imgs/3_1_curavtureRadius.eps}
\end{center}
\caption{The normed tangent $\alpha'$ (blue), the curvature $\alpha'' = \kappa \cdot n$ (green) and circles with curvature radius $\frac{1}{\kappa}$ (red)}
\label{fig::3_1_curvatureRadius}
\end{figure}
\end{enumerate}

\subsubsection{Torsion}
\note{Again this is a nice to have section, treating torsion and the third derivative. No relevance for the current thesis.}

\subsection{Curvatures on Surfaces}
Now that we have the notion of curvature for curves we will develop curvature for surfaces. The curvature of curves measures how fast the tangential spaces change. We want to measure the same thing on our 2D surfaces and do this by looking at the normal field.

\subsubsection{Normals and Orientation}
For oriented 2D surfaces embedded in $\mathbb R^3$ we can do something we can not do for general manifolds: we can find a well defined normal on every tangential space. Given a map $\phi$ matching the orientation, the normal at some position $\phi(p)$ is given by
\[N(p) = \frac{\frac{\partial \phi(p)} {\partial u} \times \frac{\partial \phi(p)} {\partial v}}{\abs{\frac{\partial \phi(p)} {\partial u} \times \frac{\partial \phi(p)} {\partial v}}}\]
The normal solely depends on the orientation. And, as mentioned in the section about oriented manifolds, finding a consistent normal field on a surface is equivalent to the finding an orientation of the surface.

%\subsubsection{Derivatives on surfaces}
%\note{Is this the right way to introduce this stuff? Should this section be more general for any manifolds and in an other chapter? Answer these questions in a later iteration.}
%Given a normal field $N$ on a surface we would like to find the derivative of $N$ to describe curvature. As $N$ is a function defined ON the surface we need to have a look at derivatives of functions on manifolds.

%Given a surface $S$ and a function $f: S \rightarrow \mathbb R^n$ what does it even mean for $f$ to be differentiable? We want the derivative to be something very similar to the derivative $Dh$ of a function $h: \mathbb R^2 \rightarrow \mathbb R^n$. In this case $Dh$ is the linear mapping that locally approximates $h$ and can be used to give the directional derivative for a direction $v$.
%\[h( p + tv) \approx h(p) + Dh \cdot tv\]
%We want the same for functions $f$ on surfaces: $Df$ should be a linear mapping, that maps a direction i.e. an element from the tangential plane to a vector that describes the change of $f$ when going in that direction. This is important: the differential $Dh$ is a mapping from the \emph{tangential spaces} to vectors. 
%
%We can express this readily by using a curve $\alpha (t)$ with a tangent $\frac{\partial \alpha(0)}{\partial t} = v$ in a wished direction.
%\[Df \cdot v := \frac{\partial}{\partial t} f(\alpha(t))\]
%As $f(\alpha(t))$ is simply a function $\mathbb R \rightarrow \mathbb R^n$ we know how to calculate the right hand side $\frac{\partial}{\partial t} f(\alpha(t))$. This is not very handy for any calculations; but we can express the derivative in the local coordinates given by a parametrization $\phi(u,v)$.
%
%\begin{figure}
%\begin{center}
%\includegraphics[width= 12.5cm]{imgs/3_1_manifoldDerivative.eps}
%\end{center}
%\caption{Construction of a derivative of a real valued function $f$ defined on a manifold locally parametrized by $\phi$. $Df$ at a point $p$ is a linear mapping from the tangential space $T_p M$ to $\mathbb R$}
%\label{fig::3_1_manifoldDerivative}
%\end{figure}
%
%As we have seen, a parametrization provides a base of the tangential space, namely 
%\[\frac{\partial\phi}{\partial u}, \frac{\partial\phi}{\partial v}\] 
%Curves can be expressed in this map and tangential vectors can be described in this base: $\alpha(t) = \phi(u(t),v(t))$ and $\alpha'(t) = \frac{\partial\phi}{\partial u} u' + \frac{\partial\phi}{\partial v} v'$. The function $f$ also has to be given in that map , i.e. $f(u,v) = f(\phi(u,v)) = (f_1(\phi(u,v)),...,f_n(\phi(u,v)))$. Then 
%\[Df \cdot \alpha'(t) = (\frac{\partial f}{\partial u}, \frac{\partial f}{\partial v}) \cdot \begin{pmatrix}
%	u' \\ v'\end{pmatrix}\]
%and $Df$ is described \emph{in the local coordinates given by $\phi$} by the $ n \times 2$ matrix $(\frac{\partial f}{\partial u}, \frac{\partial f}{\partial v})$. This situation is depicted in Figure \ref{fig::3_1_manifoldDerivative}.
%
%A slight generalisation is considering functions $f$ going from one manifold $M$ to an other manifold $M'$, as shown in Figure \ref{fig::3_1_manifoldDerivative2}. The derivative $Df$ at some point $p$ then is a linear mapping from the tangential space $T_pM$  to the tangential space $T_{f(p)} M'$, i.e. $D_p f = T_p M \rightarrow T_f(p) M'$. If $M$ is a $k$-manifold and $M'$ a $l$-manifold $Df$ can be expressed as a $k\times l$ matrix, described relatively to two sets of local coordinates $\phi \rightarrow M$ and $\psi \rightarrow M'$.
%
%\begin{figure}
%\begin{center}
%\includegraphics[width= 13cm]{imgs/3_1_manifoldDerivative2.eps}
%\end{center}
%\caption{Two 2-manifolds $M$ and $M'$ with local parametrizations $\phi$ and $\psi$. $f$ is a function $f: M \rightarrow M'$. $Df$ at a point $p$ is a linear mapping from the tangential space $T_p M$ to $T_{f(p)} M'$, choosing $D\phi$ and $D \psi$ to parametrize the tangential spaces $Df$ can be represented as a $2\times 2$ matrix (relative to these bases)}
%\label{fig::3_1_manifoldDerivative2}
%\end{figure}
%

\subsubsection{The derivative of the Normal-Field}
We got curvature of curves by looking at the change of unit tangents, i.e. how the tangential space changes. By looking at the derivative of the normal function we do the very same for the two dimensional tangential spaces of surfaces. As $N$ is a function defined ON the surface, we can apply what we developed in the section about derivatives of functions on manifolds (Section \ref{sec::2_derivativesOnMF}).

As mentioned above, $N$ maps a point on a surface $S$ to its unit normal, so $N$ is a mapping from the surface to the unit sphere i.e. between two manifolds (see Figure \ref{fig::3_1_normalDerivative}).

We are interested to express $N$ in local coordinates for a given map $\phi$. This is easy, as we know that the normal at a given point $\phi(u,v)$ is perpendicular to the tangential plane spanned by $\phi_u, \phi_v$.
\[N_{loc}(u,v) = \frac{\frac{\partial \phi}{\partial u} \times \frac{\partial \phi}{\partial v}}{|\frac{\partial \phi}{\partial u} \times \frac{\partial \phi}{\partial v}|} (u,v) \]

\begin{figure}[ht]
\begin{center}
\includegraphics[width=13cm]{imgs/3_1_normalDerivative.eps}
\end{center}
\caption{a) $N$ is a mapping from the surface $S$ to the sphere $S^2$. b) $DN$ at some point $p$ therefore is a linear mapping from the tangential space $T_pS$ to $T_{N(p)}S^2$. c) As $T_pS$ and $T_{N(p)}S^2$ are both orthogonal to $N(p)$ they can be identified and $DN$ is a mapping from $T_p S$ onto itself, why it can be expressed totally using only $\phi$}
\label{fig::3_1_normalDerivative}
\end{figure}

The derivative of the function $N$ maps vectors from the tangential space $T_pS$ to vectors in the tangential space $T_{N(p)}S^2$. Note that both $T_{N(p)}S^2$ and $T_pS$ are orthogonal to $N(p)$. This means both $T_pS$ and $T_{N(P)}S^2$ describe the same subspace of $\mathbb R^3$ and $dN$ can be thought of as a map $dN: T_pS \rightarrow T_pS$ from $T_pS$ onto itself. Therefore we can use the local map $\phi$ to induce a basis in the source AND the image space and $dN$ can be completely expressed in the local coordinates given by $\phi$. This is also described in Figure \ref{fig::3_1_normalDerivative}.


It turns out that $dN$ in local coordinates can be expressed solely using the coordinate function $\phi$ and its derivatives. For a deduction in detail see e.g. \note{p114 of do carno}. The formula is the following:

\[dN_{local} = \begin{pmatrix} a_{11} & a_{12} \\ a_{21} & a_{22} \end{pmatrix} = - \begin{pmatrix} e & f \\ f & g \end{pmatrix} \begin{pmatrix} E & F \\ F & G \end{pmatrix}^{-1}\]

Here $(E,F,F,G)$ is the so called first fundamental form and $(e,f,f,g)$ the second fundamental form, and they are given by
\[\begin{pmatrix} E & F \\ F & G \end{pmatrix} = \begin{pmatrix} \langle \phi_u, \phi_u \rangle & \langle \phi_u, \phi_v \rangle \\ 
							\langle \phi_u, \phi_v \rangle & \langle \phi_v, \phi_v \rangle \end{pmatrix}\]
\[\begin{pmatrix} e & f \\ f & g \end{pmatrix} = \begin{pmatrix} \langle N, \phi_{uu} \rangle & \langle N, \phi_{uv} \rangle \\ 
							\langle N, \phi_{uv} \rangle & \langle N, \phi_{vv} \rangle \end{pmatrix}.\]


\subsubsection{Curvature on Surfaces}
The derivative of the normal field $DN$ encodes all the curvature information of a surface. To understand it precisely we fall back to the curvature of curves.

A curve lying on a surface has two kinds of curvatures: a curvature induced by the surface, as well as a curvature that only depends on the curve. A curve lying in a plane gets no curvature induced by the plane, a curve on a sphere will always have some curvature. As seen in the section about curves, curvature of a curve is associated with a vector $\kappa n$, describing in what direction the curve curves. If a curve lies on a surface we can split $\kappa n$ in a component lying in the tangential plane and a component aligned with the surface normal.

\begin{definition}[Normal Curvature] The normal curvature $\kappa_n$ of a curve lying on a surface $S$ at a point $p$ is $\langle \kappa n, N(p)\rangle$ \note{Image}
\end{definition}
The normal curvature of a curve at some point only depends on the direction of the curve, i.e. its tangent, which lies in the tangential space of the surface at that point:
let $\phi: U \subset \mathbb R^2 \rightarrow S$ be a local parametrization and $\alpha$ be a parametrization by arc length of a curve. We can express $\alpha$ in local coordinates $\alpha(s) = \phi(u(s), v(s))$.  The functions $u(s)$ and $v(s)$ describe the curve in the local coordinates given by the map $\phi$. The curvature $\kappa n$ of the curve is 
\[\kappa n(s) = \frac{\partial^2 \alpha(s)}{\partial s^2}\]
and denoting the surface normals along the curve $\alpha(s)$ by $N(s)$, the normal curvature at the point $p = \alpha(s)$ is 
\[\kappa_n = \langle\frac{\partial^2 \alpha(s)}{\partial s^2}, N(s)\rangle\]
Using that the tangent $\frac{\partial \alpha}{\partial s}$ is orthogonal to the surface normal $N$ we get
\[\langle\frac{\partial \alpha(s)}{\partial s}, N(s)\rangle = 0 \]
\[ \frac{\partial}{\partial s} \langle\frac{\partial \alpha(s)}{\partial s}, N(s)\rangle = 0\]
\[\langle\frac{\partial^2 \alpha(s)}{\partial s^2}, N(s)\rangle + \langle\frac{\partial \alpha(s)}{\partial s}, \frac{\partial N(s)}{\partial s}\rangle  = 0\]
and therefore 
\[\kappa_n = -\langle\frac{\partial \alpha}{\partial s}, \frac{\partial N(s)}{\partial s}\rangle\]
Stated slightly differently the normal curvature at a point $p$ of a curve in some direction $v \in T_pS$ is given by
\[\kappa_n(v) =  -\langle v, dN \cdot v\rangle \]
and does only depends on the direction $v$. So $dN$ at some point $p$ encodes the normal curvatures in all directions at the point $p$.


The mapping $dN$ is a linear mapping from the linear space $T_pS$ onto itself. It has a very important further property, namely that it is self adjugated which formally means that
\[\langle v, dNw\rangle = \langle dN v, w \rangle\]
and is equivalent to saying that it can be represented by a symmetric matrix if an orthonormal base is selected for $T_pS$. The consequence is that $dN$ has a full set of Eigenvalues and that it has an orthonormal base of Eigenvectors. Say $\kappa_1 > \kappa_2$ are the Eigenvalues and $v_1$, $v_2$ the Eigenvectors.
\begin{enumerate}
\item $\kappa_1$ and $\kappa_2$ are called the principal curvatures and $v_1, v_2$ principle curvature directions. $\kappa_1$ is the largest normal curvature and $\kappa_2$ the smallest and the directions are orthogonal to each other. \note{(Image)}
\item The mean curvature , henceforth denoted by $H$, is given by $H = (\kappa_1 + \kappa_2)/2 = tr(dN)/2$ and equals the average normal curvature at the point.
\item The so called gauss curvature, denoted by $\kappa_G$, is given by $\kappa_G = \kappa_1 \cdot \kappa_2 = det(dN)$ and beyond other important properties captures the local structure of a surface:

\note{(IMAGE)}
 
\begin{enumerate}
\item $p$ is an elliptic point if $\kappa_G >0$
\item $p$ is a hyperbolic point if $\kappa_G <0$
\item $p$ is a parabolic point if $\kappa_G =0$ and $dN \neq 0$
\item $p$ is a planar point if $dN =0$
\end{enumerate}
\end{enumerate}
 Note that while $dN$ is represented as a matrix in some base, the Eigenvalues of this matrix do NOT depend on the base, just as the determinant and the trace don't either. This means all curvatures introduced can be directly calculated from any local representation. And we can associate a determinant $det (dN)$  directly to the map $dN$, as well as a trace $tr(dN)$.
 
\subsubsection{A metric on a surface}
\note{The first fundamental form as riemannian metric. This is not needed in this chapter only in connection with the hodge star. I will write it when i write the hodge star and decide then in what chapter to put it.}

\note{ introduce  maybe the euler characteristic?}

\section{Discrete Curvature and a Discrete Laplacian}
In this part we will translate the notion of curvature to 2D meshes. We will follow the approach given in \note{...} There are quite a few different approaches to define curvature on meshes, the problem is not trivial. \note{Could reference a few papers and approaches here.}


The main idea in the approach described in \note{chagah} is to use a relation between mean curvature with the Laplacian operator on surfaces. This relation exposes a geometric property of the mean curvature that can be  used directly to get a meaningful mean curvature definition on the mesh, alongside with a motivation for a Laplacian operator.

\subsection{Minimal Surfaces and the Mean Curvature Flow}
The geometric key property of mean curvature and the Laplacian that is exploited is the so called area minimizing flow. Minimal surfaces are a quite complex field; a quite readable introduction is given in [Osserman survey of minimal surfaces 1986], a rather advanced introduction is given in [T. H. Colding and W. Mincozzi]. Further computational aspects of the mean curvature flow in a similar setting as this thesis are described in [Konrad Polthier Computational Aspects of Discrete Minimal Surfaces]. To explain what the mean curvature flow is and how it relates to the Laplacian we sketch two 'facts' and their implications.

The first fact considers the relation between the manifold-Laplacian, the spacial coordinates of the manifold and the mean curvature normal. As we have seen, manifolds can get a differential structure. This can be used to define a divergence, a gradient and a Laplace operator ON manifolds. We will come back to such operators on manifolds in Section \note{...} when talking about exterior calculus; for now you will have to accept that they do exist, and that these differential operators can be applied to (smooth) function on the surface. 

 One special kind of smooth functions are the coordinate functions: every point on a manifold $M^k \subset \mathbb R^n$ has a spacial position $(x_1,...,x_n)$ and the function that assigns every point $p\in M^k$ its $j$-th coordinate is a smooth function; the $j$th coordinate function. \note{img}

The manifold Laplacian applied by element to the coordinates $x$ of 2-dimensional surface then turns out to be a multiple of the curvature normal:
\begin{equation} \Delta_{M^2}x= -2H\cdot N \label{eqn::laplaceMeancurve}\end{equation}
The second fact considers surface area as a function of the manifold's coordinates.
\[area(M^2)\]
Area is a functional and from a functional we can take a so called variational derivative. What follows is merely a sketch, for a clean reproduction of these arguments see e.g. [osserman or t.h.colding and W.minicozzi]. A variation is gotten by perturbing the function / object the functional is measuring. Here we take an arbitrary vector field of NON tangential vectors $V$ on the manifold and consider the area of the manifold after perturbing the coordinates by $tV$, where $t$ is a scalar. Denoting '$M^2 + tV$' by $M^2_t$ the area of the perturbed manifold is:

\[area(M^2_t)\]

This is a function in $t$ so we can calculate a derivative
\[\frac{\partial}{\partial t}area(M^2_t).\]
Setting $t=0$ then turns out to produce
\[\frac{\partial}{\partial t}area(M^2_t)|_{t=0} = 2\int_{M^2}H\langle V, N \rangle\]
where $N$ is the surface normal field of $M^2$ and $H$ is the mean curvature. This has various implications: if the mean curvature vanishes everywhere on a surface, the surface is a so called minimal surface: perturbing it locally always increases the surface area. (Well, this is not completely true: you would have to look at the second variational derivative to make this sure, but this is of no importance for us)

The second implication is that selecting $V = -H\cdot N$ leads to

\[\frac{\partial}{\partial t}area(M^2_t)|_{t=0} = -2\int_{M^2}H^2\langle N, N \rangle\]
which is always negative, meaning that perturbing any manifold in the direction $-H\cdot N$ will minimize its area. The deformation of a surface according to the mean curvature normal is therefore also called 'area minimizing flow'.
\note{Image}

An other implication is that taking $V= N$ and and looking only at a small neighborhood $U \subset M$ of a point $p$ we get
\[\frac{\partial}{\partial t}area(U_t)|_{t=0} = -2\int_{U}H\]
and therefore mean curvature at a point $p$ can also be obtained by
\[ 2H=\lim_{diam(U) \rightarrow 0}\frac{\frac{\partial}{\partial t}area(U_t)|_{t=0}}{area(U)}\]
and more generally
\[2H\cdot N =\lim_{diam(U) \rightarrow 0}\frac{grad(area(U))}{area(U)}\]

\subsection{Discrete Mean Curvature}

Now we can define a discrete mean curvature on meshes at vertex positions. In the last section we identified a geometric property of mean curvature that retains a concise meaning for meshes: area minimization. In comparison: in the Differential Geometry section mean curvature arises as the average of the two principal curvatures, i.e. the maximal and minimal normal curvature that occurs at some point. On a mesh this definition is meaningless, as curves on the mesh going through a vertex will not be differentiable and have no well defined curvature at vertex positions.

At any vertex $v$ consider the area of the one neighbourhood of $v$.
\[area_{v}= \sum_{T:neighbour\;triangle (v)} area(T)\]
this can be expressed as a function of the vertex positions. Fixing all vertex positions but $v$'s position $x_v$, this area is a function of $x_v$ and we can calculate the gradient
\[grad(area(x_v)) = \nabla area(x_v)\]
which turns out to be
\begin{equation}\frac{1}{2}\sum_{j\in N_1(v)} (\cot \alpha_{ij} + \cot \beta_{ij})(x_v-x_j) \label{eqn::cotan1}\end{equation}
\note{$\alpha \beta$ image}
as is done in detail in the Appendix of [Desbrun et Al.]. The question then is how, given a  mesh that samples a continuous 2 manifold, how this quantity relates to the mean curvature of the actual 2 manifold. In [Desbrun et Al.] it is shown that this quantity describes $\int_{\mathcal A} Hn$
\[\int_{\mathcal A} H\cdot N = \frac{1}{2}\sum_{j\in N_1(v)} (\cot \alpha_{ij} + \cot \beta_{ij})(x_v-x_j)\]
where $\mathcal A$ is an area around $v$ corresponding to an area on the mesh $\mathcal A_v$ around the vertex $v$ with the following two properties:
\begin{itemize}
\item $\mathcal A_v$ is completely contained in the triangles of the one ring of $v$.
\item The border of $\mathcal A_v$ cuts all incident edges at their mid points.
\end{itemize}
\note{image} 
and the right way to estimate $Hn$ at $v$ is
\[H\cdot N (v) = \frac{1}{2\mathcal A_v}\sum_{j\in N_1(v)} (\cot \alpha_{ij} + \cot \beta_{ij})(x_v-x_j)\] 
The question that remains open is what area $\mathcal A_v$ is optimal. [Desbrun et Al.] do an error estimation between $\int_{\mathcal A} Hn\; dA$ (the value you get by the cotangent formula \ref{eqn::cotan1}) and $\mathcal A_v \cdot Hn(v)$ (the correct value). The total error
\begin{eqnarray*}E &=& \sum_{i}||\int_{\mathcal A_{v_i}} Hn(x)\; dA - \mathcal A_{v_i} \cdot Hn(v_i)||^2 \\
&=&\sum_{i}||\int_{\mathcal A_{v_i}} Hn(x) - Hn(v_i)\; dA||^2 \\
&\leq & C \sum_{i}\int_{\mathcal A_{v_i}} ||x - v_i||^2\; dA
\end{eqnarray*}
where $C$ is some constant depending only on the mean curvature normal field $Hn$ and not the choice of areas $\mathcal A_v$. This motivates that choosing areas  minimizing $\int_{\mathcal A_{v_i}} ||x - v_i||^2\; dA$ gives optimal error bounds. This is fulfilled by choosing Voronoi areas $\mathcal  A_v = \mathcal A_{vornoi}$. 

\note{In Text image}

As long as triangles are non-obtuse this works well. For obtuse triangles this fails because, \note{...} This is why an alternative to Voronoi areas is introduced: mixed Areas.

\note{etc, TODO}

The relation between the surface Laplacian and the mean curvature in Equation \ref{eqn::laplaceMeancurve} then motivates to define a discrete Laplacian acting on a discrete function $f$ defined per vertex as
\[\Delta_{discrete} f = -\frac{1}{\mathcal A_v}\sum_{j\in N_1(v)} (\cot \alpha_{ij} + \cot \beta_{ij})(f_v-f_j)\] 
\note{A cleaner but less geometric approach to the discrete Lalacian in a Finite Element setting is also given in section.... and the much more general approach to differential operators using discrete differential forms in Chapter ... will also lead to this formula.}

\subsection{Discrete Gauss Curvature, Principle Curvatures}
\note{Is this Section needed?Claude's Opinion? Nothing new would be gained an they are described very understandably in the desbrun paper}
For the sake of completeness we present the discrete Gaussian and principle curvatures, as introduced in [desbrun et Al.], such that all the curvature operators introduced in the differential geometry section get some discrete analogues. While the discrete mean curvature performs very well and because of its strong relation to the Laplacian fits perfectly in the discrete differential forms setting introduced in the next chapters, the Gaussian as well as the principal curvatures use an independent approach. In the experiments done by [desbrun et Al.] they did also perform less well than or at best comparatively well as other approaches. 


\subsection{Note: Discrete Laplacian in a Finite Element Setup}
\note{Another not absolutely crucial section, decide about it / write it later}
So what actually happened here is that we used an energy (the area) minimization to define a Laplacian (or mean curvature) on a mesh, which is a piecewisely linear object. This can be looked at more generally; a variational definition of differential operators allows the differential operators to stay meaningful when working with objects and functions that have a lower regularity (not or not often differentiable functions/surfaces). In such a setting you can get a Laplacian on the space of piecewisely linear functions, which turns out to be exactly the Laplacian of the above section.

The Key is that the Laplacian arises as the first variation of the Dirichelet Energy: 
Dirichelet energy etc as in [computational analysis of Discrete Minimal Surfaces].

\section{Implementation}
This is the hands-on section. In the first section some details on an implementation of the discrete mean curvature normal are given, mainly because in the perspective of later chapters it makes sense to split up the calculation of the mean curvature normal in multiple matrices. In the second section surface smoothing after [desrbun paper2] is presented as a nice application of the things implemented up to here.

\subsection{Discrete $\Delta$ and Mean Curvature}
The discrete mean curvature is given by
\[Hn (v) = \frac{1}{2\mathcal A_{voronoi}}\sum_{j\in N_1(v)} (\cot \alpha_{vj} + \cot \beta_{vj})(x_v-x_j)\] 
which, as noted before, is half of the discrete Laplacian applied to the vertex coordinates. This is a linear function in the vertex coordinates and the calculation of all mean curvature normals at once can be expressed using matrices. Instead of storing the Laplacian in one matrix we will split it in 4 simpler matrices; these matrices can be reused in later chapters.

We first note that the term $\cot \alpha_{vj} + \cot \beta_{vj}$ arises once for every edge; we store these weights in a (sparse) diagonal matrix of dimension $\# edges$ and call it $\star_1$, for fun.
\[\star_1 = \begin{pmatrix}
\cot \alpha_{edge_1} + \cot \beta_{edge_1} & & \\
& \ddots & \\
& & \cot \alpha_{edge_{\#e}} + \cot \beta_{edge_{\#e}}
\end{pmatrix}\]
The values $x_j$ are defined per vertex (the $x_j$ consist of three coordinates; every one has to be treated separately), $(x_v-x_j)$ are defined per edge. This is exactly what we get when we apply the incidence Matrix $\delta_1^T$ (the transposed border matrix from the edges) to the coordinates of the vertices.
\[\delta_1^T \cdot \begin{pmatrix}
x_1 \\
\vdots\\
x_{\#v}
\end{pmatrix}\]
such that the terms $(\cot \alpha_{vj} + \cot \beta_{vj})(x_v-x_j)$ are represented by 
\[\star_1\cdot \delta_1^T \cdot x\]
Summing up over all incident edges  of a vertex can be achieved simply by applying the border matrix $\delta_1$ to the values defined by edge \note{note?}, and the quantity $\sum_{j\in N_1(v)} (\cot \alpha_{vj} + \cot \beta_{vj})(x_v-x_j)$ is given by vertex via
\[\delta_1\cdot\star_1\cdot \delta_1^T \cdot x\]
Lastly we have to take care of the normalisation factors $\frac{1}{2 \mathcal A_{voronoi}}$. We put the Voronoi areas $\mathcal A_{voronoi}$, which represent per vertex weights, again in a diagonal matrix of dimension $\# vertices$ and call it $\star_0$.
\[\star_0 =\begin{pmatrix}
\mathcal A_{voronoi}(vertex_1) & & \\
& \ddots & \\
& & \mathcal A_{voronoi}(vertex_{vertex_{\#v}})
\end{pmatrix}\] 
the matrix to compute all curvature normals $Hn$ is then given by
\[H\cdot N = \frac{1}{2} \star_0^{-1}\cdot\delta_1\cdot\star_1\cdot \delta_1^T \cdot x \]
and the discrete Laplacian is
\[\Delta_{discrete} = \star_0^{-1}\cdot\delta_1\cdot\star_1\cdot \delta_1^T.\]
The inversion of $\star_0$ is obviously trivial. In the hands- on section of chapter \note{ } $\delta_1$ was already implemented, such that the only missing components are the matrices $\star_0$ and $\star_1$.

\note{Image curvatured mesh}

\note{generalization higher dims ?}

\subsubsection*{Computing Cotan and Voronoi areas}
\note{Just a very few words.. needed?}

\subsection{Implementation: Surface Smoothing}
The implemented mean curvature/Laplacian operator can be directly used for surface smoothing, as described in [desbrun2]. The idea is that a surface is smoother if its surface area is smaller. But this is equivalent to deforming a surface according to the mean curvature normal $HN$. As the mean curvature is directly related to the Laplacian we can reformulate this idea as 
\[\frac{\partial}{\partial t} X = \lambda \Delta (X)\]
where $X$ represents the spacial positions of the (discrete) manifold to smooth and $\lambda$ is just a factor which controls how fast the smoothing is done. This equation simply expresses that the positions $X$ should be changed (over 'time' $t$) according to the Laplacian or mean curvature normal, which amounts to area minimization. 
An other interpretation is that this equation actually describes a diffusion process by which the noise of a surface gets diffused.

We have developed a discrete version of the mean curvature operand and can therefore directly reformulate the equation above using discrete time steps $\Delta t$,
\[X_{j+1} = X_{j} + \Delta t \cdot \lambda\cdot\Delta_{discrete} X_j\]

By doing this we do numerical integration by the forward Euler method \note{image 1D}. 

In this case the process is much stabler if the backward Euler method is used i.e.
\[X_{j+1} = X_{j} + \Delta t \cdot \lambda\cdot\Delta_{discrete} X_{j+1}\]
which amounts to solving
\[(Id - \Delta t \cdot \lambda\cdot\Delta_{discrete}) X_{j+1} = X_{j}\]
for every time step. \note{image}

\note{Images of results, explicit vs implicit}

\subsection{(?) Implementation: remeshing (?)}
\note{You could use the laplacian also for remeshing to get very nicely meshed meshes where circumcenters lie inside the triangles (the mesh gets slightly smoothed in the remeshing step... a fluid sim on a bunny would be funny; maybe a bit like hair in the wind...). The implementation is simple. These meshes would probably be nice enough for fluid simulations as described... This section is a 'nice to have' section i maybe do if everything else is done.}
