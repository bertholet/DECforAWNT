\newpage
\chapter{Two Dimensional Surfaces}

\begin{figure}[h]
\begin{center}
\includegraphics[width = 7cm]{imgs/3_1_curvaturezebra.EPS}
\end{center}
\caption{A zebra mesh that is coloured according to its local mean curvature}
\end{figure}

In this chapter we focus completely on two dimensional surfaces. The theory here is not vital to understand DEC and if you are in a hurry you may skip this chapter. Nevertheless as surfaces and triangle meshes play a vital role in many applications it is worth to get acquainted with ways to describe and compute local mesh properties, as developed here.

In this chapter we start with a short differential geometry primer where we introduce basics like curvatures and a 'metric', a scalar product consistently defined on all tangential spaces. In the second section we will then thrive to get a notion of curvatures on meshes and thereby have a first encounter with a Laplace operator defined on a surface. The third section again is the practical section where we give some implementation details for curvatures and present surface smoothing as an application.

\begin{figure}[b]
\begin{longtable}{|p{4.5cm}|p{4.5cm}|p{4.5cm}|}
\hline
Smooth Theory& Discrete Theory& Implementation (Notes)\\
\hline
Differential Geometry & Discrete Diff. Geo. & Implementation\\
-Curves & - A geometric go at the DEC $\Delta$ & -Implement mean curv by splitting matrices\\
-Curvatures & - Mean curv, principle curv etc.& -Application: Surface Smoothing\\
-Euler Characteristic? & & -Application: Remeshing? \\			
\hline
\end{longtable}
%\caption{The topics treated in this chapter}
\end{figure}

\section{Differential Geometry}
Differential geometry studies local and global properties of 2 dimensional surfaces embedded in $\mathbb R^3$. This section represents a dirty introduction to the basic notions of differential geometry, such as curvature. Many (not unimportant) details are omitted and technical problems ignored, be aware of this when reading. There are many good books introducing differential geometry, for example .... if you are interested in a clean introduction.

Differential geometry deals with surfaces, but in order to understand surfaces we need to understand curves. So we start with:

\subsection{Curves in $\mathbb R^3$}
Curves are 1-manifolds but somewhat special in that curves can be parametrized by one global map, while with higher dimensional manifolds you have to fall back to local maps. 

The manifold definition for curves translates to: 

\begin{definition}[Curve] A curve is a set of points $C \subset \mathbb R^3$ for which there exists a (differentiable, injective) parametrization $\alpha : (a,b) \to C$ with $\left|\alpha'(t)\right| >0$ for all $t\in (a,b)$. We make a strict difference between the parametrization of a curve and the curve itself. The curve is the point set described by the parametrization.
\end{definition}

%\note{Note on length? Nah trivial enough}

We have already seen that the first derivative of a parametrization describes tangents. To interpret higher derivatives we have to look at a special parametrization of curves the parametrization by arc length.

\subsubsection{Parametrization by Arc Length}

A curve is parametrized by arclength if $|\alpha'(t)|=1$ for all $t$. This means that the curve is run through with constant speed (img). The name ``parametrized by arclength'' comes from the simple fact that the length of a curve segment 
\[l_a^b = \int_a^b |\alpha'| dt = b-a\]
is directly given by the parameter. It is not completely obvious, that such a parametrization by arclength always exists, but it does. Actually there exist always two parametrizations by arclength with different orientations. Note also, that even if you only have an arbitrary parametrization one can still easily calculate what the derivatives by arclength are at some given point; see the Appendix for this and for a sketch of the proof of existence of the parametrization. A curve parametrized by arc length is shown in Figure \ref{fig::3_1_paramByArclength}.

\begin{figure}[h]
\begin{center}
\includegraphics[width= 10cm]{imgs/3_1_paramByArclength.eps}
\end{center}
\caption{This curve is parametrized by arclength: lengths are preserved under $\alpha$ and the length of the curvesegment $\alpha([a,b])$ is simply $b-a$. The vectors $\alpha'$ (blue) have constant length 1.}
\label{fig::3_1_paramByArclength}
\end{figure}

\subsubsection{Curvature}
The second derivative of a parametrization by arclength has geometric meaning: it describes the curvature of a curve. Considering only parametrizations by arc length is not a restriction, as such parametrizations always exist and all derivatives of a parametrization by arclength can be gained from any parametrization.
	
Let $\alpha$ be a parametrization by arc length. The first derivative $\alpha'$ gives the (normalized) tangent direction; the second derivative $\alpha''$ then describes how fast and in what direction the tangent of the curve changes. More formally:
\begin{enumerate}
\item The vector $\frac{\partial^2 }{\partial t^2} \alpha$ is always orthogonal to the tangent $\frac{\partial}{\partial t} \alpha$. 

Because $|\alpha'|= 1$ is constant 
\[\langle \frac{\partial}{\partial t} \alpha,\frac{\partial}{\partial t} \alpha\rangle = const\]
\[0=\frac{\partial}{\partial t} \langle \frac{\partial}{\partial t} \alpha,\frac{\partial}{\partial t} \alpha\rangle = 2 \langle \frac{\partial^2}{\partial t^2} \alpha,\frac{\partial}{\partial t} \alpha\rangle\]
and this means orthogonality. You can check the last equation by plugging in the definition of the dot product and using the product rule of the derivation.
\item The absolute value of the second derivative $|\frac{\partial^2}{\partial t^2} \alpha|$ is the \textbf{curvature} (denoted by $\kappa$) of the curve. It describes how much the curve curves at a given point. 

A geometric interpretation of curvature is given by the curvature radius $r= \frac{1}{\kappa}$. In the plane given by the tangent and the second derivative, this radius is the radius of the circle that approximates the curve in the best way. More precisely: the curvature circle approximates the curve up to the second derivative. In comparison the tangent approximates the curve up to the first derivative (see Fig. \ref{fig::3_1_curvatureRadius}). 
\begin{figure}[h]
\begin{center}
\includegraphics[width=5cm]{imgs/3_1_curavtureRadius.eps}
\end{center}
\caption{The normed tangent $\alpha'$ (blue), the curvature $\alpha'' = \kappa \cdot n$ (green) and circles with curvature radius $\frac{1}{\kappa}$ (red)}
\label{fig::3_1_curvatureRadius}
\end{figure}
\end{enumerate}

\subsubsection{Torsion}
\note{Again this is a nice to have section, treating torsion and the third derivative. No relevance for the current thesis.}

\subsection{Curvatures on Surfaces}
Now that we have the notion of curvature for curves we will develop curvature for surfaces. The curvature of curves measures how fast the tangential spaces change. We want to measure the same thing on our 2D surfaces and do this by looking at the normal field.

\subsubsection{Normals and Orientation}
For oriented 2D surfaces embedded in $\mathbb R^3$ we can do something we can not do for general manifolds: we can find a well defined normal on every tangential space. Given a map $\phi$ matching the orientation, the normal at some position $\phi(p)$ is given by
\[N(p) = \frac{\frac{\partial \phi(p)} {\partial u} \times \frac{\partial \phi(p)} {\partial v}}{\abs{\frac{\partial \phi(p)} {\partial u} \times \frac{\partial \phi(p)} {\partial v}}}\]
The normal solely depends on the orientation. And, as mentioned in the section about oriented manifolds, finding a consistent normal field on a surface is equivalent to the finding an orientation of the surface.

\subsubsection{Derivatives on surfaces}
\note{Is this the right way to introduce this stuff? Should this section be more general for any manifolds and in an other chapter? Answer these questions in a later iteration.}
Given a normal field $N$ on a surface we would like to find the derivative of $N$ to describe curvature. As $N$ is a function defined ON the surface we need to have a look at derivatives of functions on manifolds.

Given a surface $S$ and a function $f: S \rightarrow \mathbb R^n$ what does it even mean for $f$ to be differentiable? We want the derivative to be something very similar to the derivative $Dh$ of a function $h: \mathbb R^2 \rightarrow \mathbb R^n$. In this case $Dh$ is the linear mapping that locally approximates $h$ and can be used to give the directional derivative for a direction $v$.
\[h( p + tv) \approx h(p) + Dh \cdot tv\]
We want the same for functions $f$ on surfaces: $Df$ should be a linear mapping, that maps a direction i.e. an element from the tangential plane to a vector that describes the change of $f$ when going in that direction. This is important: the differential $Dh$ is a mapping from the \emph{tangential spaces} to vectors. 

We can express this readily by using a curve $\alpha (t)$ with a tangent $\frac{\partial \alpha(0)}{\partial t} = v$ in a wished direction.
\[Df \cdot v := \frac{\partial}{\partial t} f(\alpha(t))\]
As $f(\alpha(t))$ is simply a function $\mathbb R \rightarrow \mathbb R^n$ we know how to calculate the right hand side $\frac{\partial}{\partial t} f(\alpha(t))$. This is not very handy for any calculations; but we can express the derivative in the local coordinates given by a parametrization $\phi(u,v)$.

\begin{figure}
\begin{center}
\includegraphics[width= 12.5cm]{imgs/3_1_manifoldDerivative.eps}
\end{center}
\caption{Construction of a derivative of a real valued function $f$ defined on a manifold locally parametrized by $\phi$. $Df$ at a point $p$ is a linear mapping from the tangential space $T_p M$ to $\mathbb R$}
\label{fig::3_1_manifoldDerivative}
\end{figure}

As we have seen, a parametrization provides a base of the tangential space, namely 
\[\frac{\partial\phi}{\partial u}, \frac{\partial\phi}{\partial v}\] 
Curves can be expressed in this map and tangential vectors can be described in this base: $\alpha(t) = \phi(u(t),v(t))$ and $\alpha'(t) = \frac{\partial\phi}{\partial u} u' + \frac{\partial\phi}{\partial v} v'$. The function $f$ also has to be given in that map , i.e. $f(u,v) = f(\phi(u,v)) = (f_1(\phi(u,v)),...,f_n(\phi(u,v)))$. Then 
\[Df \cdot \alpha'(t) = (\frac{\partial f}{\partial u}, \frac{\partial f}{\partial v}) \cdot \begin{pmatrix}
	u' \\ v'\end{pmatrix}\]
and $Df$ is described \emph{in the local coordinates given by $\phi$} by the $ n \times 2$ matrix $(\frac{\partial f}{\partial u}, \frac{\partial f}{\partial v})$. This situation is depicted in Figure \ref{fig::3_1_manifoldDerivative}.

A slight generalisation is considering functions $f$ going from one manifold $M$ to an other manifold $M'$, as shown in Figure \ref{fig::3_1_manifoldDerivative2}. The derivative $Df$ at some point $p$ then is a linear mapping from the tangential space $T_pM$  to the tangential space $T_{f(p)} M'$, i.e. $D_p f = T_p M \rightarrow T_f(p) M'$. If $M$ is a $k$-manifold and $M'$ a $l$-manifold $Df$ can be expressed as a $k\times l$ matrix, described relatively to two sets of local coordinates $\phi \rightarrow M$ and $\psi \rightarrow M'$.

\begin{figure}
\begin{center}
\includegraphics[width= 13cm]{imgs/3_1_manifoldDerivative2.eps}
\end{center}
\caption{Two 2-manifolds $M$ and $M'$ with local parametrizations $\phi$ and $\psi$. $f$ is a function $f: M \rightarrow M'$. $Df$ at a point $p$ is a linear mapping from the tangential space $T_p M$ to $T_{f(p)} M'$, choosing $D\phi$ and $D \psi$ to parametrize the tangential spaces $Df$ can be represented as a $2\times 2$ matrix (relative to these bases)}
\label{fig::3_1_manifoldDerivative2}
\end{figure}


\subsubsection{The derivative of the Normal-Field}
Now we have the tools to analyse the normal function. We got curvature of curves by looking at the change of unit tangents, i.e. how the tangential space changes. By looking at the derivative of the normal function we do the very same for the two dimensional tangential spaces of surfaces.


As mentioned above, $N$ is a function that maps a point on a surface $S$ to its unit normal, so $N$ is a mapping from the surface to the unit sphere.

We are interested to express $N$ in local coordinates for a given map $\phi$. This is easy, as we know that the normal at a given point $\phi(u,v)$ is perpendicular to the tangential plane spanned by $\phi_u, \phi_v$.
\[N_{loc}(u,v) = \frac{\frac{\partial \phi}{\partial u} \times \frac{\partial \phi}{\partial v}}{|\frac{\partial \phi}{\partial u} \times \frac{\partial \phi}{\partial v}|} (u,v) \]

\begin{figure}[ht]
\begin{center}
\includegraphics[width=13cm]{imgs/3_1_normalDerivative.eps}
\end{center}
\caption{a) $N$ is a mapping from the surface $S$ to the sphere $S^2$. b) $DN$ at some point $p$ therefore is a linear mapping from the tangential space $T_pS$ to $T_{N(p)}S^2$. c) As $T_pS$ and $T_{N(p)}S^2$ are both orthogonal to $N(p)$ they can be identified and $DN$ is a mapping from $T_p S$ onto itself, why it can be expressed totally using only $\phi$}
\label{fig::3_1_normalDerivative}
\end{figure}

The derivative of the function $N$ maps vectors from the tangential space $T_pS$ to vectors in the tangential space $T_{N(p)}S^2$. Note that both $T_{N(p)}S^2$ and $T_pS$ are orthogonal to $N(p)$. This means both $T_pS$ and $T_{N(P)}S^2$ describe the same subspace of $\mathbb R^3$ and $dN$ can be thought of as a map $dN: T_pS \rightarrow T_pS$ from $T_pS$ onto itself. Therefore we can use the local map $\phi$ to induce a basis in the source AND the image space and $dN$ can be completely expressed in the local coordinates given by $\phi$. This is also described in Figure \ref{fig::3_1_normalDerivative}.


It turns out that $dN$ in local coordinates can be expressed solely using the coordinate function $\phi$ and its derivatives. For a deduction in detail see e.g. \note{p114 of do carno or appendix, to decide yet}. The formula is the following:

\[dN_{local} = \begin{pmatrix} a_{11} & a_{12} \\ a_{21} & a_{22} \end{pmatrix} = - \begin{pmatrix} e & f \\ f & g \end{pmatrix} \begin{pmatrix} E & F \\ F & G \end{pmatrix}^{-1}\]

Here $(E,F,F,G)$ is the so called first fundamental form and $(e,f,f,g)$ the second fundamental form, and they are given by
\[\begin{pmatrix} E & F \\ F & G \end{pmatrix} = \begin{pmatrix} \langle \phi_u, \phi_u \rangle & \langle \phi_u, \phi_v \rangle \\ 
							\langle \phi_u, \phi_v \rangle & \langle \phi_v, \phi_v \rangle \end{pmatrix}\]
\[\begin{pmatrix} e & f \\ f & g \end{pmatrix} = \begin{pmatrix} \langle N, \phi_{uu} \rangle & \langle N, \phi_{uv} \rangle \\ 
							\langle N, \phi_{uv} \rangle & \langle N, \phi_{vv} \rangle \end{pmatrix}.\]


\subsubsection{Curvature on Surfaces}
The derivative of the normal field $DN$ encodes all the curvature information of a surface. To understand it precisely we fall back to the curvature of curves.

A curve lying on a surface has two kinds of curvatures: a curvature induced by the surface, as well as a curvature that only depends on the curve. A curve lying in a plane gets no curvature induced by the plane, a curve on a sphere will always have some curvature. As seen in the section about curves, curvature of a curve is associated with a vector $\kappa n$, describing in what direction the curve curves. If a curve lies on a surface we can split $\kappa n$ in a component lying in the tangential plane and a component aligned with the surface normal.

\begin{definition}[Normal Curvature] The normal curvature $\kappa_n$ of a curve lying on a surface $S$ at a point $p$ is $\langle \kappa n, N(p)\rangle$ \note{Image}
\end{definition}
The normal curvature of a curve at some point only depends on the direction of the curve, i.e. its tangent, which lies in the tangential space of the surface at that point:
let $\phi: U \subset \mathbb R^2 \rightarrow S$ be a local parametrization and $\alpha$ be a parametrization by arc length of a curve. We can express $\alpha$ in local coordinates $\alpha(s) = \phi(u(s), v(s))$.  The functions $u(s)$ and $v(s)$ describe the curve in the local coordinates given by the map $\phi$. The curvature $\kappa n$ of the curve is 
\[\kappa n(s) = \frac{\partial^2 \alpha(s)}{\partial s^2}\]
and denoting the surface normals along the curve $\alpha(s)$ by $N(s)$, the normal curvature at the point $p = \alpha(s)$ is 
\[\kappa_n = \langle\frac{\partial^2 \alpha(s)}{\partial s^2}, N(s)\rangle\]
Using that the tangent $\frac{\partial \alpha}{\partial s}$ is orthogonal to the surface normal $N$ we get
\[\langle\frac{\partial \alpha(s)}{\partial s}, N(s)\rangle = 0 \]
\[ \frac{\partial}{\partial s} \langle\frac{\partial \alpha(s)}{\partial s}, N(s)\rangle = 0\]
\[\langle\frac{\partial^2 \alpha(s)}{\partial s^2}, N(s)\rangle + \langle\frac{\partial \alpha(s)}{\partial s}, \frac{\partial N(s)}{\partial s}\rangle  = 0\]
and therefore 
\[\kappa_n = -\langle\frac{\partial \alpha}{\partial s}, \frac{\partial N(s)}{\partial s}\rangle\]
Stated slightly differently the normal curvature at a point $p$ of a curve in some direction $v \in T_pS$ is given by
\[\kappa_n(v) =  -\langle v, dN \cdot v\rangle \]
and does only depends on the direction $v$. So $dN$ at some point $p$ encodes the normal curvatures in all directions at the point $p$.


The mapping $dN$ is a linear mapping from the linear space $T_pS$ onto itself. It has a very important further property, namely that it is self adjugated which formally means that
\[\langle v, dNw\rangle = \langle dN v, w \rangle\]
and is equivalent to saying that it can be represented by a symmetric matrix if an orthonormal base is selected for $T_pS$. The consequence is that $dN$ has a full set of Eigenvalues and that it has an orthonormal base of Eigenvectors. Say $\kappa_1 > \kappa_2$ are the Eigenvalues and $v_1$, $v_2$ the Eigenvectors.
\begin{enumerate}
\item $\kappa_1$ and $\kappa_2$ are called the principal curvatures and $v_1, v_2$ principle curvature directions. $\kappa_1$ is the largest normal curvature and $\kappa_2$ the smallest and the directions are orthogonal to each other. \note{(Image)}
\item The mean curvature , henceforth denoted by $H$, is given by $H = (\kappa_1 + \kappa_2)/2 = tr(dN)/2$ and equals the average normal curvature at the point.
\item The so called gauss curvature, denoted by $\kappa_G$, is given by $\kappa_G = \kappa_1 \cdot \kappa_2 = det(dN)$ and beyond other important properties captures the local structure of a surface:

\note{(IMAGE)}
 
\begin{enumerate}
\item $p$ is an elliptic point if $\kappa_G >0$
\item $p$ is a hyperbolic point if $\kappa_G <0$
\item $p$ is a parabolic point if $\kappa_G =0$ and $dN \neq 0$
\item $p$ is a planar point if $dN =0$
\end{enumerate}
\end{enumerate}
 Note that while $dN$ is represented as a matrix in some base, the Eigenvalues of this matrix do NOT depend on the base, just as the determinant and the trace don't either. This means all curvatures introduced can be directly calculated from any local representation. And we can associate a determinant $det (dN)$  directly to the map $dN$, as well as a trace $tr(dN)$.
 
\subsubsection{A metric on a surface}
\note{The first fundamental form as riemannian metric. This is not needed in this chapter only in connection with the hodge star. I will write it when i write the hodge star and decide then in what chapter to put it.}

\note{ introduce  maybe the euler characteristic?}

\section{Discrete Curvature, A Discrete Laplacian}
In this part we will translate the notion of curvature to 2D meshes. We will follow the approach given in \note{...} There are quite a few different approaches to define curvature on meshes, the problem is not trivial. \note{Could reference a few papers and approaches here.}



Geometric approach via minimal surfaces to define mean curvature and at the same time also the DEC Laplacian
		
\section{Implementation: discrete $\Delta$ and curvatures}
Split up the Laplacian into star and neighborhood matrices for implementation

\section{Implementation: surface smoothing}
Using the Laplacian flow to smoothe a surface

\section{Implementation: remeshing (?)}
\note{You could use the laplacian also for remeshing to get very even meshes. The implementation is simple. These meshes would probably be nice enough for fluid simulations as described... This section is a 'nice to have' section i maybe do if everything else is done.}