\begin{appendix}

\chapter{The integral of differential forms}
\label{app:integrals}

That the integral definition of Section \ref{sec:integralOfForms} really is independent of the chosen map $\phi$ follows directly from the common transformation formula. Say we use a different map $\psi: V \rightarrow \Omega \subset M$, then $\psi = \phi \circ h$ for some mapping $h:U \subset \mathbb R^k \rightarrow V \subset \mathbb R^k$ \note {image}. 
\[\int_{\psi(V)} \omega^k = \int_V \omega_{\psi(x_1,...,x_k)}(\frac{\partial \psi}{\partial x_1},...,\frac{\partial \psi}{\partial x_k}) d x_1...d x_k\]
\[= \int_V \omega_{\psi(x_1,...,x_k)}(\frac{\partial \phi \circ h}{\partial x_1},...,\frac{\partial \phi \circ h}{\partial x_k}) d x_1...d x_k\]
\[= \int_V \omega_{\phi\circ h(x_1,...,x_k)}(D\phi \cdot Dh) d x_1...d x_k\]
\[= \int_{V} det(Dh) \omega_{\phi\circ h(x_1,...,x_k)}(\frac{\partial \phi}{\partial x_1},...,\frac{\partial \phi}{\partial x_k}) d x_1...d x_k\]
and using the transformation formula
\[= \int_U \omega_{\phi(x_1,...,x_n)} (\frac{\partial \phi}{\partial x_1},...,\frac{\partial \phi}{\partial x_k}) d x_1...d x_k \]

\subsubsection{Example}
The integral of a $k$-form $\omega^k = f(x) \cdot dVol $ over a $k$-dimensional set $\phi(U)$ parametrized by the orientation preserving map $\phi$ is the following:
\[\int_{\phi(U)} f dVol = \int_{U} f(\phi(x)) dVol(\frac{\partial \phi}{\partial x_1},...,\frac{\partial \phi}{\partial x_k})dx_1...dx_k\]
And as $dVol(\frac{\partial \phi}{\partial x_1},...,\frac{\partial \phi}{\partial x_k}) = \sqrt{det((D\phi)^TD\phi)}$ this is
\[\int_{\phi(U)} f dVol = \int_{U} f(\phi(x)) \sqrt{det((D\phi)^TD\phi)} dx_1...dx_k\]

\chapter{Primal and Dual Discrete Border Operator}

\note{What follows now comes in the Appendix - this is already a much clearer formulation than anything i found anywhere, but still a bit technical. Could put a short note on orientation definition using forms in Appendix too} 
[Inspired by Discrete Exterior Calculus by Desbrun et al.]
The setting is the following: we are on an $n$-dimensional (discrete) manifold and the $n$-simplex $\sigma^n$ is oriented according to the orientation of the manifold. $\sigma^k$ is some face on $\sigma^n$ with an arbitrary orientation and $\sigma^{k-1}$ is a face of $\sigma^k$. On the dual-mesh side we have the cells $\star \sigma^k$ that lies on the border of $\star \sigma^{k-1}$; both these dual have an orientation induced by the primary cells. Our goal is to relate the primary border matrix $\delta_k$ which stores the relative orientation of $\sigma^k$ and $\sigma^{k-1}$ 
and the dual border matrix $\delta_{n-k+1}^{dual}$ that stores the relative orientation of the (n-k+1)-cell $\star \sigma^{k-1}$ and $\star \sigma^{k}$.

\begin{figure}[h]
\begin{center}
\includegraphics[height= 5cm]{imgs/6_4_dualPrimalBorder.eps}
\end{center}
\caption{$v_1,v_2$ defines the orientation of $\sigma^k$ (here $k=2$) and $v_1,v_2,v_3$ defines the orientation of the 3-manifold we are on; $v_2$ ($v_k$) is an inside pointing border normal for $\sigma^{k-1}$ and outside normal for $\star \sigma^k$. $\sigma^{k+1}$ is one border simplex, oriented according to $\omega^k$. The duals $\star \omega^k$ and $\star \omega^{k-1}$ have orientations induced by the primal cells (red). As you can see, while the primals are oriented consistently, the induced orientations on the dual are opposite to each other.}
\end{figure}

First of all notice that orientation can be treated very neatly by using forms, as volume forms measure signed volume.
We saw in Section \note{...}, that a $k$ simplex $\sigma^k$ encodes orientation by the ordering of its vertices, and that induces an oriented basis $v_1,...,v_k$ that spans its volume. In terms of forms the ordering defines a volume form
\[dv_1\wedge...\wedge dv_k\]
to the $k$-simplex. The $k$-simplex lies in an $n$-simplex $\sigma^n$, oriented according to the orientation of the discrete manifold; it induces a volume form
\[dv_1\wedge ... \wedge dv_n\]
on its volume. We can select the vector $v_k$ (for some fixed $j<=k$) to be a normal on the face $\sigma^{k-1}$ pointing inside $\sigma^k$, and adapt $v_1,...,v_{k-1}$ such that $dv_1\wedge...\wedge dv_k$ remains the same. Then the orientation / volume form  of $\sigma^{k-1}$ induced by $\sigma^k$ (via its border relation) is
\[dv_1\wedge... \wedge dv_{k-1} (-1)^{k}\]
consistent with the way we defined border orientation ( preappending the 'outside normal' $-dv_k$ leads to the form $dv_1\wedge ... \wedge dv_k$, i.e. the orientation of $\sigma^k$). Furthermore we ask $v_{k+1},...,v_{n}$ to be orthogonal on $v_{1},...,v_{k}$; i.e. they should be aligned with the dual of the cell $\sigma^k$. Then the induced orientation of the dual simplex $\star \sigma^k$ is given by
\[\star (dv_1\wedge...\wedge dv_k) = dv_{k+1}\wedge...\wedge dv_{n}\]
The orientation of the $j$th border component $\sigma_j^{k-1}$ of $\sigma^k$ induced by $\sigma^k$ is
\[\delta (dv_1\wedge...\wedge dv_k) \Rightarrow\]
\[(-1)^{k}dv_1\wedge...\wedge dv_{k-1}\]
its dual $\star \sigma^{k-1}$ has the orientation
\[\star ((-1)^{k} v_1\wedge...\wedge dv_{k-1}) = (-1)^{k} dv_k \wedge dv_{k+1}\wedge ... \wedge dv_n\]
Both, $\star \sigma^k$ and $\star \sigma^{k-1}$, have orientations induced from the primary simplices as seen. But by its border relation $\star \sigma^{k-1}$ also induces an orientation to $\star \sigma^k$. Note that $v_k$ acts as outside pointing normal for $\star \sigma^k$ as border of the dual $\star \sigma_j^{k-1}$ and therefore
\[\delta (-1)^{k} dv_k \wedge dv_{k+1}\wedge ... \wedge dv_n = (-1)^{k} dv_{k+1}\wedge ... \wedge dv_n\]
This means that the relative orientation of the pair $(\sigma^k, \sigma^{k-1})$ and the relative orientation of the pair $(\star \sigma^{k-1}, \star \sigma^k)$ is related by
\[orient(\sigma^k, \sigma^{k-1}) = (-1)^{k} orient(\star \sigma^{k-1}, \star \sigma^k)\]
Therefore, if the border matrix
\[\delta_k\]
stores the relative orientations of $k$ and $k-1$ simplices $\sigma^k$ and $\sigma^{k-1}$, then the relative orientation of $n-k-1$ and $n-k$-dimensional dual cells $\star \sigma^{k-1}$  and $\star \sigma^k$, with orientations induced by the primary simplices, is given by the dual border matrix
\[\delta_{n-k+1}^{dual} = (-1)^{k} (\delta_k)^T\]

\note{End of Appendix part}

\end{appendix}