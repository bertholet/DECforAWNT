\chapter{Introduction}
	
	\section{Goal of this Thesis}
	
	The goal of this thesis is twofold. One goal is to introduce the reader to exterior calculus, which provides a very elegant way to formulate many things from standard calculus and leads to deeper insights about various differential operators. The second goal is to introduce discrete exterior calculus, which allows to reformulate and approximate equations from exterior calculus with matrices, such that the equations can be solved computationally (at least approximately). 
	
	After reading this thesis you should have a working knowledge of both exterior calculus and discrete exterior calculus (DEC), such that you can apply DEC to problems and reason (or at least follow reasonings) using exterior calculus. 
	
	

		
	\section{A Glimpse of DEC}
	
	\begin{figure}[h]
	\begin{center}
	\includegraphics[height = 7cm]{imgs/1_1_SolvingProcessUsingDEC.eps}
	\end{center}
	\caption{General solution process using DEC}
	\end{figure}
	
	
		\begin{itemize}
			\item Derivative as a simple example 
			\item vs FEM 
			\item show a picture of the deRahm complex saying that this diagram helps you translate Differential Operators into Matrices.
		\end{itemize}
				\begin{itemize}
			\item From equations to Meshes.
		\end{itemize}

	\section{A Tour to this Thesis}
		\begin{itemize}
		\item	Structure of this Thesis: Goal: acquire founded knowledge about DEC. The Thesis as three pillars: smooth Theory, discrete theory and application/implementation of example problems. The Smooth Theory we try to discretize is introduced alongside the corresponding discrete version of the Theory, such that you understand  the reasons and assumptions behind it. With an emphasis to get an intuitive understanding of the smooth theory rather than giving a rigorous mathematical introduction. Every chapter gets a table describing what is introduced and where it belongs to.
		
		\item Short trailer like chapter by chapter overview, one image per chapter.
		\end{itemize}
		
	\section{What this thesis is and what it is not}
		Disclaimer, no error analysis, no FEM. Not a rigorous introduction to EC
	