\chapter{Exterior Calculus \& Discrete Exterior Calculus}
\label{chap:EC}

This chapter constitutes the core of the theory of this thesis. We introduce the exterior derivative, describe its geometry and use its geometry to build discrete exterior calculus on discrete manifolds. Yet, the main argument of this chapter can be summed up in the following three lines:
\begin{equation} \int_{\delta \Omega} \omega = \int_{\Omega} d \omega \label{eq:stokes}\end{equation}
therefore
\[d_{discrete} := \delta_{discrete}^T.\]
These three lines are rather compact and use most of the concepts introduced up to now. While manifolds ($\Omega$), the border operators ($\delta, \delta_{discrete}$), forms ($\omega$) and their integrals have already been treated, the discrete exterior derivative $d$ has yet to be introduced. The three line argument above says, that according to Stokes theorem (Equation \ref{eq:stokes}), the discrete exterior derivative should be defined as the transposed of the discrete border operator. 
Discretizing the differential operator $d$ is interesting, because $d$ generalizes various operators like the divergence and curl. By discretizing $d$ we directly get discrete versions of these operators. 

This chapter begins with an introduction of the exterior derivative in Section \ref{sec:EC_exteriorDerivative} and an explanation of Stokes theorem and the geometry of $d$ in Section \ref{sec:EC_stokes}. Then follows a description of
how $d$ is discretized. In the fourth section, Section \ref{sec:hodgeStar}, we introduce the Hodge star operator $\star$ and its discrete version, the last missing elements needed to formulate various operators known from classical calculus.

%\begin{longtable}{|p{4.5cm}|p{4.5cm}|p{4.5cm}|}
%\hline
%Smooth Theory& Discrete Theory& Implementation (Notes)\\
%\hline
%	External Calculus
%	\begin{packed_enum}
%		\item[-] Gradient, Curl and Divergence
%		\item[-] d
%		\item[-] Stokes Theorem
%		\item[-] Star and DeRham Complex
%	\end{packed_enum}
%	&
%	Discrete External Calculus
%	\begin{packed_enum}
%		\item[-] Discrete d
%		\item[-] Dual Mesh
%		\item[-] Show also intuitive match to curl etc
%	\end{packed_enum}
%	 & 
%	 A look at the Laplacian from chapter 2
%	 \begin{packed_enum}
%		\item[-] The DEC matrices (and tests)
%	\end{packed_enum}
%	 \\		
%\hline
%\end{longtable}

\section{The Exterior Derivative $d$}
\label{sec:EC_exteriorDerivative}
The exterior derivative $d$ is a generalization of the usual derivative. But it is better to look at $d$ as something completely new and unknown, because thinking of it as the 'derivative for differential forms' can lead to wrong associations and expectations. For example applying the differential operator multiple times to get an $n$th derivative does not make sense. But the exterior derivative generalizes the usual derivative in the sense that the exterior derivative is the counterpart to the integral, by Stokes theorem.
%An example of the exterior derivative is the gradient. The gradient takes a function $f:\mathbb R^n\to \mathbb R$ and returns a vector field $grad(f)$. In terms of forms the gradient takes a $0$-form $f$ and returns a $1$-form $grad(f)$. This is the general behavior of the differential operator $d$: it takes $k$-forms and returns $k+1$-forms.

The exterior derivative allows a rich theory and is of great practical relevance because  many differential operators from classical calculus can be expressed using it. We give examples in Section \ref{subsec:EDeuclidean} and also coordinate free examples in Section \ref{subsec:geometryCDO}.

\subsection{Definition} 
The differential operator $d$ is easy to define on $\mathbb R^n$, its relevance is not obvious on first sight, but is explained in the following sections.
\begin{definition}[Exterior Derivative on $\mathbb R^n$]
The exterior derivative $d$ maps differential $k$-forms to differential $k+1$-forms. If $\omega^k$ is given in standard coordinates $x_1,..,x_n$ as 
\[\omega^k_p= \sum_{i_1<...<i_k} \omega_{i_1,..,i_k}(x) dx_{i_1} \wedge ... \wedge dx_{i_k}\]
the exterior derivative is given by
given by 
\[d\omega^k = \sum_{i_1<...<i_k}\sum_{\alpha = 1}^{n}\frac{\partial \omega_{i_1,..,i_k}(x)}{\partial x_\alpha} dx_{\alpha} \wedge dx_{i_1} \wedge ... \wedge dx_{i_k}\]
\end{definition}

To define $d$ on manifolds in general we use pullbacks, as introduced in Section \ref{sec:pullbacks}, namely: if $M$ is a manifold, $h$ a local map and $\omega^k$ a $k$-form, then
\[d\omega^k := (h^*)^{-1}d(h^*\omega^k).\]
The first pullback $h^*$ transforms $\omega^k$ to a $k$-form on $\mathbb R^n$, where the exterior derivative is already defined and can be used. Then the result is pulled back to the manifold.\footnote{There are some technicalities about $(h^*)^{-1}$ that are omitted, see e.g. \cite{globalAnalysis}. Also note that this definition does not depend on the map $h$.}

\subsection{Properties}
The exterior derivative has the following properties that are more or less straight forward to check by plugging in the definitions; you can find detials e.g. in \cite{globalAnalysis}.

\begin{enumerate}
\item $d(\omega^k + \psi^k) = d\omega^k + d\psi^k$
\item $d(\omega^k \wedge \psi^l) =( d\omega^k) \wedge \psi^l + (-1)^k \omega^k \wedge(d \nu^l)$
\item $d(d\omega^k) = 0$
\item $f^*(d\omega^k) = d(f^* \omega^k)$
\end{enumerate}
 The third and fourth property are the most noteworthy.  Applying $d$ two times in a row always leads to zero (as you can check by simply writing out $dd$). And the exterior derivative commutes with pullbacks. This means that you can freely chose where and in what map you want to work and calculate derivatives; just pull everything to a space where you want to have it.


\subsection{The Exterior Derivative in Euclidean Coordinates}
\label{subsec:EDeuclidean}
Let's look at the exterior derivative on $\mathbb R^n$ with the standard basis. The differential forms are interpreted as functions and vectorfields, as described in the last chapter in Section \ref{subsec:diffformsare}.

\subsubsection{0-Forms}
If we have a $0$-form on $\mathbb R^n$  given by $f:\mathbb R^n \to \mathbb R$, then the exterior derivative is by definition the 1-form
\[df = \sum_{\alpha = 1}^n \frac{\partial f}{\partial x_\alpha} dx_\alpha .\]
Therefore, applying $df$ at any point $p$ to a vector $v$ is
\[df_p(v) = \langle \nabla f, v \rangle,\]
where $\nabla f = (\frac{\partial f}{\partial x_1},...,\frac{\partial f}{\partial x_n})$ is the gradient in euclidean coordinates. This means that $d$ acts on $0$-forms like the gradient operator.


\begin{figure}
\begin{center}
\includegraphics[height=3cm]{imgs/6_1_SCvsEC_2d.eps}
\caption{The differential forms arising on 2-manifolds and the exterior derivative (EC) and the corresponding objects and operators in standard calculus (SC).}
\label{fig::6_1_SC2d}
\end{center}

\end{figure}

%\note{Auskommentiert: kommentar ueber die notation $dx_i$ / $du_i$... Braeuchtes das...?}
%On a side note: this also motivates the notation $dx_\alpha$ for the special one forms we use as a basis of the space of $1$-forms; $x_\alpha$ is the $\alpha$th coordinate of a point $x$, i.e. short for the function $f(x) = f(x_1,...,x_n) = x_\alpha$. Therefore if we interpret $d$ as the exterior derivative 
%\[(dx_\alpha)_p(v) = \langle e_\alpha,v \rangle \]
%is exactly what we defined it \note{in sec...} to be. The same is true for an arbitrary map $\phi(u_1,...,u_k)$ that locally assigns the coordinates $u_1,..,u_k$ to a manifold $M$:
%\[du_i = (\phi^*)^{-1}(d (\phi^*)(u_i)\]
%\[=(\phi^*)^{-1}(d e_i)\]
%\[= \langle D\phi \cdot e_i, D \phi v \rangle\]
%\note{or similar}
\subsubsection{1-Forms}
A vector field 
\[\mathcal V : \mathbb R^n \to \mathbb R^n\]
\[\mathcal V(x) = (v_1(x),...,v_n(x))\]
 interpreted as a one form is
\[\omega^1_p = \sum_{i = 1}^n v_i(p) d x_i .\]
To apply the exterior derivative to $\omega^1$ amounts to
\[d\omega^1 = \sum_{i=1}^n \sum_{j = 1} ^n \frac{\partial v_i(p)}{\partial x_j} dx_j \wedge d x_i.\]
If we reorder these terms we get
\[= \sum_{1\leq i < j \leq n } (\frac{\partial v_j(p)}{\partial x_i} - \frac{\partial v_i(p)}{\partial x_j}) dx_i \wedge d x_j\]
and on $\mathbb R^3$ this is exactly the $rot$ operator: if we represent the arising 2 form as a vector we get
\[d \begin{pmatrix}
v_1(x) \\ v_2(x) \\ v_3(x)
\end{pmatrix} = \begin{pmatrix}
\frac{v_3(x)}{\partial x_2} -\frac{v_2(x)}{\partial x_3}\\
\frac{v_1(x)}{\partial x_3} -\frac{v_3(x)}{\partial x_1}\\
\frac{v_2(x)}{\partial x_1} -\frac{v_1(x)}{\partial x_2}\\
\end{pmatrix}\]
This means that $d$ applied to 1-forms can be interpreted as the curl operator.

\subsubsection{$2$-Forms on $\mathbb R^3$}
A differential 2-form can be represented as a vector field $\mathcal V = (v_1,v_2,v_3)$ as
\[\omega^2 = v_1 dx_2 \wedge dx_3 + v_2 dx_3 \wedge dx_1 + v_3 dx_1 \wedge dx_2.\]
The exterior derivative is then
\[d \omega^2 = (\frac{\partial v_1}{\partial x_1} + \frac{\partial v_2}{\partial x_2} + \frac{\partial v_3}{\partial x_3})dx_1\wedge dx_2 \wedge dx_3\]
which means that $d$ is exactly the divergence operator.

\subsubsection{Summary}
The Figures \ref{fig::6_1_SC2d} and  \ref{fig::6_1_SC3d} summarize the relation between differential forms and standard calculus on 2-manifolds and  on 3-manifolds.


\begin{figure}
\begin{center}
\includegraphics[height=3cm]{imgs/6_1_SCvsEC_3d.eps}
\end{center}
\caption{The differential forms arising on 3-manifolds and the exterior derivative (EC) and the corresponding objects and operators in standard calculus (SC)}
\label{fig::6_1_SC3d}
\end{figure}


\section{Stokes Theorem}
\label{sec:EC_stokes}

Now we finally explain Stokes theorem
\[\int_{\delta\Omega} \omega = \int_{\Omega} d \omega.\]
Stokes theorem is a generalization of and follows from the fundamental theorem of calculus: if $f:\mathbb R \to \mathbb R $ has an antiderivative $F$ i.e. $F' = f$, then
\[\int_a^b f(x) dx = F(b) - F(a).\]
The fundamental theorem of calculus can be rewritten in differential form notation: say $\Omega = [a,b]$ is an oriented line with border $\delta \Omega = -\{a\} + \{b\}$. If we identify $F$ with a 0-form $\omega^0 = F$, then $d F = F' =f$ is a 1-form. As $dF$ is a 1-form defined on a one manifold it can also be represented as a function. The fundamental theorem then becomes:
\[\int_{[a,b]} d\omega^0 = \int_{\delta [a,b]} \omega^0 = \int_{-\{a\}}\omega^0 + \int_{\{b\}} \omega^0 = -\omega^0(a) + \omega^0(b)\]
So the fundamental theorem is Stokes theorem applied to $0$-forms. Note how it is important that $\omega^0$ respects the orientation of the points it is applied to.

We will only sketch a proof for Stokes theorem. 
There are a few technical difficulties that are omitted. A clean proof can for example be found in \cite{globalAnalysis} or \cite{FRANKEL11}.

\subsection{Proof Sketch}

\begin{figure}%
\begin{center}
\includegraphics[height = 4cm]{imgs/6_singularcube.eps}%	
\end{center}
\vspace{-0.5cm}
\caption{A singular cube: a $k$-dimensional patch that can be parametrized by the cube $[0,1]^k$ using a single map.}%
\label{fig:6_singularCube}%
\end{figure}
This sketch is following strongly the reasoning made in \cite{globalAnalysis}.
It is enough to show the theorem for so called singular cubes. A $k$-dimensional singular cube $c^k$ is a manifold $C^k$ together with a global parametrisation (see Figure \ref{fig:6_singularCube})
\[c^k: [0,1]^k \to C^k \subset \mathbb R^n.\] 
We show
\[\int_{\delta C^{k+1}} \omega^{k} = \int_{C^{k+1}}d\omega^k.\] 
To extend the theorem to arbitrary manifolds consider the Figure \ref{fig::6_1_singularCubes}; the basic idea is that the theorem holds if a manifold is the union of a set of disjoint cubes $\Omega = \bigcup c^k_i$, as
\[\int_{\Omega} d\omega = \sum_i \int_{c^k_i} d\omega = \sum_i \int_{\delta c^k_i} \omega = \int_{\delta \Omega} \omega ,\]
where the second step is the Stokes theorem for singular cubes and the last step holds because internal boundaries cancel out, as motivated in Figure \ref{fig::6_1_singularCubes}.

\begin{figure}[t]
\begin{center}
\includegraphics[height=3cm]{imgs/6_1_singularCubeChain.eps}
\end{center}
\caption{Some random bordered manifold that can be made out of 3 singular cubes. Some integral $\int_O$ over this object amounts to the sum of integrals over the singular cubes. For every cube the Stokes theorem is true. So the integral $\int_O$ is given by the sum of border integrals of the cubes. But inner edges cancel out because of their opposite orientations. Therefore $\int_O d\omega= \int_{\delta O}\omega$} 
\label{fig::6_1_singularCubes}
\end{figure}

\subsubsection{Proof (sketch)}

Given a singular cube $C^{k+1}$ with parametrisation $c^{k+1}$ we can pull the whole problem back to $[0,1]^{k+1}$. If the theorem is proven for $[0,1]^{k+1}$ it also holds for singular cubes $C^{k+1}$:
\[\int_{c([0,1]^{k+1})}d \omega = \int_{[0,1]^{k+1}} c^*(d\omega) = \int_{[0,1]^{k+1}} d(c^*(\omega)) = \int_{\delta [0,1]^{k+1}} c^*(\omega) = \int_{\delta(c([0,1]^{k+1}))} \omega\] 
We write an arbitrary $k$-Form $\omega^k$ on $[0,1]^{k+1} \subset \mathbb R^{k+1}$ as 
\[\omega^k = \sum_{i=1}^{k+1} f_i dx_1 \wedge...dx_{i-1} \wedge dx_{i+1} ...\wedge dx_{k+1}\]
where in each term the $i$th $dx_i$ is omitted. Then
\[d\omega^k = \sum_{i=1}^{k+1}(-1)^{i-1}\frac{\partial f}{\partial x_i}dx^1\wedge ... \wedge dx^k\]
and
\[\int_{I^{k+1}} d \omega^k = \sum_{i=1}^{k+1}(-1)^{i-1} \int_{[0,1]^{k+1}} \frac{\partial f}{\partial x_i} dx_1 \wedge...\wedge dx_{k+1}\]


Now we can simply use the known fundamental theorem to integrate the single terms in the sum relative to $x_i$
\[\int_{0}^1 \frac{\partial f}{\partial x_i} (x_1,...,x_{i-1},t,x_{i+1},...) dt = f(x_1,...x_{i-1},1,x_{i+1},...) - f(x_1,...x_{i-1},0,x_{i+1},...)\]
getting 
\[\int_{[0,1]^{k+1}} d \omega^k = \sum_{i=1}^{k+1}(-1)^{i-1} \underbrace{\int_{0}^1...\int_{0}^1}_{\textit{k+1 integrals}} \frac{\partial f}{\partial x_i} dx_1...dx_{k+1}\]
\[= \sum_{i=1}^{k+1}(-1)^{i-1} \left(\underbrace{\int_{0}^1...\int_{0}^1}_{\textit{k integrals}} f(x_1,...,x_{i-1},1,x_{i+1}...,x_{k+1}) dx \right.\]
\[- \left. \underbrace{\int_{0}^1...\int_{0}^1}_{\textit{k integrals}} f(x_1,...,x_{i-1},0,x_{i+1}...,x_{k+1}) dx \right)\]
where the integrals used are the 'common' integrals. Every term in the last sum integrates $f$ over one side of the cube $[0,1]^{k+1}$, because plugging in a $1$ or a $0$ for one parameter and integrating over the others has exactly that effect. The factors $(-1)^{i-1}$ together with the minus from the fundamental theorem, result exactly in assigning to every term a sign matching the orientation of the respective cube side induced by the border operator applied to $[0,1]^{k+1}$, see Figure \ref{fig:6_stokesproof}. Therefore
\[\int_{[0,1]^{k+1}} d \omega^k = \int_{\delta [0,1]^{k+1}}\omega^k\]
\qed

\begin{figure}[t]%
\includegraphics[width=\columnwidth]{imgs/6_stokesproof.eps}%
\caption{On the left the orientations of the borders when computing the integrals $\int_0^1\int_0^1$ are depicted, on the right are the orientations induced by the border operator. The additional signs $(-1)^{i-1}$ for the faces $x_i=1$ and $(-1)^{i}$ for the faces $x_i=0$ lead to the induced border orientation.}%
\label{fig:6_stokesproof}%
\end{figure}

\subsection{Geometry of $d$}

Stokes theorem is more then a  valuable tool for calculations and for reformulations. It unmasks the geometry of the exterior derivative $d$. Stokes theorem binds the exterior derivative to the border operator, which is a purely geometric operation. Both operations are equivalent, as you can choose either to apply the border operator to a region or to apply the exterior derivative to the differential form at hand.
This can be made even clearer by using a bracket notation for the integral; the first argument is the manifold, the second argument is the differential form:
\[[\Omega, \omega] := \int_{\Omega} \omega\]
Then Stokes theorem can be formulated as
\[[\Omega, d\omega] = [\delta \Omega, \omega].\]
The border operator and the exterior operator play an equivalent role.

\subsection{The Geometry of Classical Differential Operators}
\label{subsec:geometryCDO}
That Stokes' theorem captures the geometry of $d$ can be highlighted by looking at the standard calculus operators $div$ and $curl$ on arbitrary 3-manifolds.  The geometric, coordinate free definition of the curl $\nabla \times v$ of a vector field $v$ is 

\[(\nabla \times v) \cdot n = \lim_{diam(A)\rightarrow 0 } \frac{\int_{\partial A} \langle v , t \rangle dx}{Area(A)} \]
where $n$ is a normal vector, $A$ is an area patch orthogonal to $n$ and $t$ is a tangent to the border of $A$, see Figure \ref{fig:6_curlDiv}. So the curl operator takes a vector field and returns a linear mapping that can be used to measure how much the vector field curls around a point. This geometric property is exactly captured by Stokes theorem applied to 1-forms, when forms are interpreted as in \ref{subsec:diffformsare}:
\[\lim_{diam(A)\rightarrow 0 }\frac{\int_{\partial A} \omega^1}{Area(A)} = \lim_{diam(A)\rightarrow 0 }\frac{\int_{A} d \omega^1}{Area(A)} = d \omega^1 (a,b)\]
where $a,b$ are vectors with $a\times b = n$ and $n$ normal to $A$.
The same is true for the geometric definition of divergence, where the net flow in and out of a volume is measured
\[div(v) = \lim_{diam(V)\rightarrow 0 } \frac{\int_{\partial V} \langle v , n \rangle dx}{Vol(V)}\]
where $n$ is the normal field on the border of the volume $V$, see Figure \ref{fig:6_curlDiv}. And again, this is exactly captured by Stokes theorem and the exterior derivative for $2$-forms
\[\lim_{diam(V)\rightarrow 0 }\frac{\int_{\partial V} \omega^2}{Vol(V)} = \lim_{diam(V)\rightarrow 0 }\frac{\int_{V} d \omega^2}{Vol(V)} = d \omega^2\]

\begin{figure}%
\begin{center}
\def\svgwidth{8cm}
\input{imgs/6_curlDiv.eps_tex}%	
\end{center}
\caption{The geometric definition of curl and divergence. Curl measures the flow around a infinitesimal patch orthogonal to a normal $n$. Divergence, here displayed in a 2D setting, measures the local netflow. }%
\label{fig:6_curlDiv}%
\end{figure}
\newpage
\section{The Discrete Exterior Derivative}

Stokes theorem captures the geometry of the exterior derivative. Both smooth and discrete manifolds have common geometric operations, namely the border operator. The common geometry of smooth and discrete manifolds can now be used to define a discrete exterior derivative on discrete manifolds, which preserves the geometry of the exterior derivative. The discrete exterior derivative forms the pillar of DEC.

To conserve the geometry of $d$ means to conserve Stokes theorem.  We can directly translate Stokes theorem to the discrete setting- In the discrete setting the integral of a discrete form $\discrete{w}$ over a set of simplices $\sigma$ is the scalar product $\langle \sigma, \discrete{w} \rangle$ and Stokes theorem
\[[\Omega,d\omega] = [\delta \Omega, \omega]\]
becomes
\[\langle \sigma, d_{discrete}^k\textbf{w}^k \rangle = \langle \delta_{k+1} \sigma, \textbf{w}^k \rangle,\]
where $d_{discrete}$ is yet unknown. But this relation \emph{defines} the yet unknown $d_{discrete}$; it has to be the transposed of the border-operator matrix:
\[d_{discrete} = \delta^T\]
By discretizing the exterior derivative like this we get at once consistent discretizations of all its special cases like gradient, divergence and curl. 

\subsection{Examples}
For example we know that $d$ applied to $0$-forms is the gradient. Our discrete realization of the exterior derivative for $0$-forms is the matrix
\[d_{discrete}^0 = \delta_1^T\]
which has size $(\# edges \times \#vertices)$. Applying this matrix to a discrete $0$-form yields a vector of dimension $\# edges$, a discrete $1$-form. As $\delta_1^T$ is the incidence matrix of the edges, it assigns the value $\textbf{w}^0(v_1) - \textbf{w}^0(v_0)$ to an edge $(v_0,v_1)$. The gradient is simply realized as a difference.

Another example is the curl operator, depicted in Figure \ref{fig::6_1_curl}. The curl operator is a realization of $d$ applied to differential $1$-forms. In the discrete setting curl therefore is realized as
\[d_{discrete}^1 = \delta_2^T\]
\begin{figure}[t]
\begin{center}
\includegraphics[height = 4cm]{imgs/6_4_curl.eps}
\end{center}
\caption{Curl is realized as the incidence matrix of the faces. By applying this matrix the values on the border edges of a face are summed up according to the orientation of the face (thus $-g_1$)}
\label{fig::6_1_curl}
\end{figure}

As this is the incidence matrix of faces, applying $d_1$ to a 1-form sums up the values of the discrete $1$-form on edges along a face and assigns the sum to the face. 

\subsection{Correctness}
The discrete exterior derivative $d_{discrete}$ does not introduce any new errors. It is consistent with the way we interpret and sample discrete differential forms- you can take the exterior derivative before sampling or the discrete exterior derivative after sampling, it does not matter.

Suppose that $\textbf{w}^0$ samples $\omega^0$. Then $d_{discrete}^0 \textbf{w}^0$ exactly samples $d\omega^0$:
\[\int_{[v_0,v_1]} d\omega^0 = \omega^0(v_1) - \omega^0(v_0)= \textbf{w}^0(v_1) - \textbf{w}^0(v_0)\] 
But, by design, this is true for all $d_{discrete}^k$: suppose $\textbf{w}^k$ samples $\omega^k$, then
\[\int_{\sigma^{k+1}} d\omega^k  = \int_{\delta \sigma^{k+1}} \omega^k = \langle \delta_{k+1}(\sigma^{k+1}), \textbf{w}^k\rangle = (d_{discrete}^{k} \textbf{w}^k) (\sigma^{k+1}).\]


\section{Duality: The Hodge Star}
\label{sec:hodgeStar}
By having a discrete exterior derivative, we have a discretization of all the operators that arise as special cases of the exterior derivative (see Figures \ref{fig::6_1_SC2d} and \ref{fig::6_1_SC3d}). But there is still an important element missing. For example we would like to apply the divergence operator to the gradient of a function $div(grad(f))$ to get the Laplacian, as it is done in standard calculus:
\[\Delta = div\circ grad\]
Looking at Figure $\ref{fig::6_1_SC3d}$ tells us that on 3-dimensional manifolds $grad$ is $d_0$ and $div$ is $d_2$.   This is a problem: $d_0$ takes a $0$-form and maps it to a $1$-Form and we can not apply $d_2$ to $1$-forms!
The key to this 'problem' lies in the duality of forms. 

We have already seen that different forms have the same representation. In three dimensions differential $0$-forms and differential $3$-forms can be represented as real valued functions, $1$-forms and $2$-forms as vectorfields. So it should not come as a surprise that you can generically make  a $0$-form out of a $3$-form, a $2$-form out of a $1$-form, and vice versa. 

\subsection{Intuition}
\label{subsec:hodgeintuition}
The idea is to treat a differential $k$-form like a differential $(n-k)$-form. But while $k$-forms measure $k$-dimensional volumes, $(n-k)$-forms measure $n-k$ dimensional volumes.
We can use a trick: given a small $k$-dimensional cube $c$ we can associate with $c$ the $n-k$-dimensional cube $c^\perp$ that is perpendicular to $c$ and has the same volume as $c$ i.e.
\[c\perp c^\perp \]
\[vol_k(c) = vol_{n-k}(c^\perp)\] 
Then, if we want to calculate an approximate integral of a $k$-form over a set of $n-k$-dimensional cubes $\{c_1,...,c_l\}$, thereby treating the $k$-form like an $n-k$-form, the sum
\[\sum_{j} \text{''}\omega^k(c_j)\text{``}\]
is instead calculated as 
\[\sum_{j} \omega^k(c_j^{\perp}),\]
see the sketch in Figure \ref{fig::6_1_dualIntegral}. To keep the notion of $k$-forms clean we associate a \emph{dual} $n-k$-form $\star\omega^k$ to a $k$-form $\omega^k$, that behave like described, i.e. in a very dirty notation
\[\star\omega^k(c) \approx  \omega^k(c^\perp)\]
\begin{figure}[t]
\begin{center}
\includegraphics[height = 4cm]{Imgs/6_4_dualiIntegral.eps}
\end{center}
\caption{Intuition for duality in $\mathbb R^3$: We have a one form $\omega^1$ that can be evaluated on lines and would like a $2$-Form $\star \omega^1$ that can be evaluated on 2-dimensional regions. The dual form $\star \omega$ evaluated on a small square should be the value of $\omega^1$ evaluated on the orthogonal complement of the small square with equal volume, in this case an orthogonal line with length $area(square)$. }
\label{fig::6_1_dualIntegral}
\end{figure}

\subsection{Dual Forms}
We will now formally define the dual of a form and then examine how it fits the 'intuition' developed above. 

\subsubsection{Definition of $\star$}
The spaces of forms $\Lambda^k(\mathbb R^n)$ and $\Lambda^{n-k}(\mathbb R^n)$ have the same dimension:
\[\dim(\Lambda^k(\mathbb R^n)) = \begin{pmatrix}
n \\ k
\end{pmatrix}= \begin{pmatrix}
n \\ n-k
\end{pmatrix} = \dim(\Lambda^{n-k}(\mathbb R^n))\]
As they have the same dimension we can find a bijective linear mapping between these two spaces. Usually this mapping is defined by first defining  a scalarproduct on the space of forms, and then using the scalar product to define the mapping. 

In order to define a scalar product on the space $\Lambda^k$ we select an orthogonal basis $e_1,...,e_n$ of $\mathbb R^n$ with respect to the euclidean scalar product\footnote{Note that the Hodge duality is usually defined with respect to an arbitrary scalar product, but for us this will do} and define for two $k$-forms $\omega^k, \nu^k$
\[\langle \omega^k, \nu^k \rangle = \sum_{i_1 <...<i_k} \omega^k(e_{i_1},...,e_{i_k})\cdot \nu^k(e_{i_1},...,e_{i_k}).\]
This definition does \emph{not} depend on the choice of the basis. If both forms are written in the base given by the $de_i$'s, the scalar product is
\[\omega^k = \sum_{i_1<...<i_k}w_{i_1,..,i_k} de_{i_1}\wedge ...\wedge de_{i_k}\]
\[\nu^k = \sum_{i_1<...<i_k}v_{i_1,..,i_k} de_{i_1}\wedge ...\wedge de_{i_k}\]
\[\langle \omega^k, \nu^k\rangle = \sum_{i_1<...<i_k} v_{i_1,..,i_k} \cdot w_{i_1,..,i_k}\]
i.e. the euclidean scalar product of the vectors describing $v$ and $w$. This really is a scalar product: it is linear in both parameters and symmetric.

Duality is then defined using the volume form $dVol$. The dual $\star\omega^k$ of a $k$-form $\omega^k$ is defined to be the $n-k$-form that fulfills
\[\langle\star\omega^k, \nu^{n-k}\rangle dVol  = \omega^k \wedge \nu^{n-k} \text{ for all $(n-k)$-forms $\nu^{n-k}$}\] 
The operator $\star$ is called the Hodge star. The Hodge star describes a linear mapping from $\Lambda^k$ to $\Lambda^{n-k}$:
 \[\star(\omega_1^k + \lambda \omega_2^k) = \star\omega_1^k + \lambda \star\omega_2^k \]

\subsubsection{Understanding $\star$}
We can understand the Hodge operator by looking at how it acts on basis elements $de_{i_1}\wedge...\wedge de_{i_k}$ for a positively oriented orthonormal basis $e_1,...,e_n$. From the definition of $\star$ follows:
\[\star (de_{i_1}\wedge...\wedge de_{i_k}) = sign \cdot de_{j_1}\wedge...\wedge de_{j_{n-k}} \]
where $j_1,...,j_{n-k}$ is the complement of the indices $i_1,...,i_k$ in the set $\{1,...n\}$ and $sign$ comes from reordering the indices, i.e. is the sign in the equation
\[de_{i_1}\wedge...\wedge de_{i_k} \wedge de_{j_1}\wedge...\wedge de_{j_{n-k}} = sign \cdot de_1\wedge de_2 \wedge ... \wedge de_n .\]
For example in a three dimensional setting with an orthonormal basis $e_1, e_2, e_3$
\[\star de_1 =  de_2 \wedge de_3\]
\[\star de_3 =  de_1 \wedge de_2\]
and 
\[\star de_2 =  - de_1 \wedge de_3 =  de_3 \wedge de_1\]
where we have a minus because
\[de_2 \wedge de_1 \wedge de_3 = - de_1 \wedge de_2 \wedge de_3\]
This shows that the $\star$ is close to the intuitive description in Section \ref{subsec:hodgeintuition}:  suppose $v = a \times b$ for vectors $a$,$b$ and $v$ in $\mathbb R^3$. The vectors $a,b$ span a patch perpendicular to $v$ with an area equal to the length of $v$. Furthermore 
\[de_1(v) = v_1 = (a_2b_3 -a_3b_2)\] 
\[\star de_1 (a,b)= (de_2 \wedge de_3)(a,b) = a_2b_3 -a_3b_2 \]
therefore
\[\star de_1 (a,b) = de_1(v)\]
so $\star de_1$ing the patch spanned by $a,b$ is like $de_1$ing the 'line' $v$. The same is true for $de_2$ and $de_3$. And as the $\star$ operator is linear and any one form is a sum $\omega^1 = \lambda_1de_1 + \lambda_2de_2 + \lambda_3de_3$ we get
\[\star\omega^1(a,b) = \omega^1(v)\]

\vspace{0.5cm}
\begin{center}
\includegraphics[height = 3cm]{imgs/6_4_staromega1.eps}
\end{center}
\newpage
\subsection{Realization of $\star$ in Standard Calculus}
Differential forms relate to objects from standard calculus. We now determine how the $\star$-operator acts on these presentations.

\subsubsection{0-forms and volume forms}
The dual of a 0-form $f$ is simply $f dVol$ and the dual of an $n$-form $f dVol$ is simply $f$- so the $\star$ operator denotes only a change of interpretation of a function $f$ as a $0$-form or an $n$-form and is the identity in standard calculus.
\[\star f = f\; dVol\]
\[\star f\;dVol = f\]
\subsubsection{1-forms and 2-forms in $\mathbb R^3$}
In three dimensions applying the hodge operator to the one form associated to a vectorfield leads to the two form associated to the same vectorfield. The star operator again denotes only a change of interpretation.
\[\star(v_1de_1 + v_2 de_2 + v_3 de_3) = v_1(de_2\wedge de_3) + v_2 (de_3\wedge de_1) + v_3(de_1\wedge de_2)\]
\[\star(v_1(de_2\wedge de_3) + v_2 (de_3\wedge de_1) + v_3(de_1\wedge de_2)) = v_1de_1 + v_2 de_2 + v_3 de_3\]

\subsubsection{1-forms in $\mathbb R^2$}
For 1-forms on two dimensional manifolds something happens. The dual of a one form is again a one form, and
\[\star de_1 = de_2\]
\[\star de_2 = -de_1\]
\[\star(a\;de_1 + b\;de_2) = b\;de_1 -a\;de_2\]
The operation equivalent to the Hodge $\star$ for a two dimensional vector field is a rotation by $90^\circ$, in orthonormal coordinates simply
\[\begin{pmatrix}
0 & 1 \\
-1 & 0
\end{pmatrix}\]
It also follows that $\star \star \omega^1 = -\omega^1$ as rotating a vector twice by $90^\circ$ changes its orientation. Note that this is directly related to the two sampling schemes for $1$-forms described in Section \note{...}. \note{TODO}.
\subsubsection{The dual of the dual k-form}
Differential $1$-forms on $2$-dimensional manifolds are not special, considering their behavior under the Hodge star. One can  show with little effort that
\[\star\star \omega ^k = (-1)^{k(n-k)}\omega^k.\]
Applying $\star$ twice will switch the orientation of $\omega^k$ exactly when $k(n-k)$ is not even. But for $n= 2,3$, the pair $n=2$ and $k=1$ is the only one where $k(n-k)$ is not even. 


\section{Exterior Calculus}

\begin{figure}[t]
\begin{center}
\includegraphics[height = 5cm]{imgs/6_4_exteriorCalculus}

\includegraphics[height = 5cm]{imgs/6_4_standardCalculus}
\end{center}
\caption{Top: the de Rham complex for an $n$-dimensional manifold. Bottom: the realization of the deRham complex in standard calculus.  The dashed arrows represent Laplacians defined by concatenating operators.}
\label{fig::deRhamComplex}
\end{figure}

Translated in standard calculus terms, it the $\star$ operator turns out to be often nothing more than a change of interpretation. But for exterior calculus the Hodge $\star$ plays an important role, as without it you could not higher order operators: applying $d$ once has only simple partial derivatives; applying $d$ twice is zero $dd=0$. With the $\star$ operator you can circumvent this : $d\star d$ is a second order operator.

With $\star$, $d$, $\wedge$ and also $^\#$ and $^\flat$, who formalize the relation between 1-forms and vectorfields (see Section \ref{subsec:diffformsare}), we have a very elegant and powerful language at hand.
The $\star$ and $d$ operator build the so-called de Rham complex, depicted in Figure
\ref{fig::deRhamComplex}. The de Rham complex is a so called chain complex which has all kinds of interesting properties, but goes beyond the scopes of this thesis. %But it is interesting to know that the border operator together with manifolds, the exterior derivative together with differential forms, simplicial complexes with the discrete border operator and discrete forms with the discrete exterior derivative each form closely related chain complexes. That the discrete exterior calculus operators form a chain complex is partially a reason for the strength of DEC.
This complex summarizes the operators of exterior calculus and we will meet quite a few important relations on this conglomerate of differential-form spaces and the operators $d$ and $\star$.

The realization of the de Rham complex in 2 dimensions is depicted in Figure \ref{fig:deRham2d} where $\nabla \times$ in $\mathbb R^2$ is
\[\nabla \times := \frac{\partial}{\partial x_2} - \frac{\partial}{\partial x_1}.\] 
The realization of $\star^{-1} d \star$ for 1-forms is minus the divergence operator $\nabla\cdot$, as you can easily check.

\begin{figure}
\begin{center}
\includegraphics[height = 5cm]{imgs/6_4_standardCalculusn2}
\end{center}
\caption{The realization of the de Rham complex in two dimensions.}
\label{fig:deRham2d}
\end{figure}

\subsection{The Coderivative}
\label{subsec::coderivativ}
It is useful to define the so called coderivative $\partial_k$, which is given by
\[\partial_k = (-1)^{n-k}\star^{-1}_{k+1}d_{n-k}\star_k,\]
takes a differential $k$-form and returns a differential $k-1$-form. In two dimensions $\partial_1$ is the divergence operator. Besides being a handy abbreviation, it is the adjoint to the exterior derivative on borderless compact manifolds $M$ or if one of the differential forms is zero on the border $\delta M$\footnote{The adjointness follows directly from Stokes Theorem and the behavior of $d$ and $\wedge$: $0=\int_{\delta M} \omega^k\wedge \star \nu^{k+1} = \int_M (d\omega\wedge\star \nu + (-1)^k\omega\wedge d\star \nu)$ and then setting $\omega\wedge d\star\nu = (-1)^{n-k}\omega\wedge \star \star d\star\nu$ we end $0= \int_M \langle d\omega,\nu\rangle dVol  - \int_M \langle \omega,\delta\nu \rangle dVol$}
\begin{equation}\langle d\omega, \nu\rangle = \langle \omega , \partial \nu \rangle \end{equation}
where the scalar for forms is extended to differential forms via
\[\langle \omega, \nu \rangle = \int_{M} \langle\omega^k,\nu^k\rangle dVol = \int_{M} \omega \wedge \star \nu.\]

%Not all operators can be built yet. Introduce Star, duality. The star could also be motivated with the obvious relation between k and n-k forms.

\section{Discrete Exterior Calculus}
What is left is finding a discrete star operator and finding a way to represent discrete dual forms $\star \textbf{w}$. If $\textbf{w}^k$ is a discrete $k$-form associated to $k$-simplices i.e. discrete objects of $k$-dimensional volume, the dual should be a discrete $n-k$-form, therefore associated to $n-k$-dimensional objects. For this we use a \textbf{dual mesh}. We will use Voronoi duality, because it facilitates the definition of a discrete $\star$.

\subsection{Dual Mesh}
Suppose we have a discrete $n$-manifold. Then the Voronoi dual of a $n$-simplex is its circumcenter and the dual of a $k$-simplex is the $n-k$ dimensional cell spanned by the circumcenters of the incident $n$-simplices, as depicted for $n=2$ in the inlined image. From now on we will always make the difference the primal mesh consisting of primal vertices, edges, faces etc and the dual mesh consisting of dual vertices, edges and so on. If $\sigma$ is a simplex, we will denote its dual cell by $\star \sigma$.

\begin{center}
\includegraphics[height=4cm]{imgs/6_4_dualMesh}
\end{center}

\subsubsection{Border Operator and Orientation of the dual mesh}

The border-relation on a dual complex is simply the incidence relation of the primary mesh; a dual $j$ cell $\star \sigma^{n-j}$ is on the border of the dual $j+1$-cell $\star \sigma^{n-j-1}$ exactly if $\sigma^{n-j-1}$ is on the border of $\sigma^{n-j}$.

But the dual border-matrix is not directly given by the transposed primary incidence matrix $\delta^T$; we have to take care of orientations of the dual cells, as a dual cell $\star \sigma$ on the dual mesh gets an orientation induced by the primary simplex $\sigma$, as depicted in Figure \ref{fig:6_inducedorientations}.
\begin{figure}%
\includegraphics[height=3cm]{imgs/6_inducedorientations.eps}
\caption{The dual cells (red) with the orientations induced by the primary simplices. The first three sketches are in $\mathbb R^2$, the last two in $\mathbb R^3$}%
\label{fig:6_inducedorientations}%
\end{figure}

Bottom line is that
\[\delta_{k}^{dual} = (-1)^{n-k+1} (\delta_{n-k+1}^{primal})^T\]
We give a proof of this in the Appendix \ref{app:dualborder}. An example for $n=2$ is given in Figure \ref{fig:6_dualborder}.

\begin{figure}%
\begin{center}
\includegraphics[height= 4cm]{imgs/6_7_dualborder.eps}%
\end{center}
\caption{The central vertex is negatively oriented with respect to the incident edges (left). The dual edges and the dual face have an induced orientation, but the dual edges are oriented positively with respect to the dual face. Therefore $\delta_{2}^{dual} = - (\delta_1^{primal})^T$ in the $n=2$ dimensional setting.}%
\label{fig:6_dualborder}%
\end{figure}



\subsection{Discrete Dual Forms and Star Operator}
While in the continuous case the dual of a differential form is again a differential form, we make a strict distinction between discrete primary forms and discrete dual forms. While the discrete primary forms are defined on the simplices of the primary mesh, the discrete dual forms are defined on the dual mesh.
We 'sample' dual forms on the dual mesh. The value of the discrete dual form $\star \textbf{w}$ sampling $\star \omega$ on the dual simplex $\star \sigma$ can be interpreted as
\[\star \textbf{w}(\star \sigma) = \int_{\star \sigma^k}  \star \omega^k.\]
The discrete exterior derivative  on the dual mesh is given by
\[d^{dual}_{n-k} = (\delta^{dual}_{n-k+1})^T \]
just as we defined the discrete exterior derivative on the primary mesh, to preserve the geometry of $d$ revealed by Stokes Theorem. Expressed with primal matrices this is
\[d^{dual}_{n-k}= (-1)^k(\delta_k^{primal}) = (-1)^k(d^{primal}_{k-1})^T \]

What is left is the question how to get form a discrete form $\textbf{w}^k$ to its dual $\star \textbf{w}^k$, i.e. how the two integrals
\[\int_{\star \sigma^k} \star \omega^k ,\;\;\; \int_{\sigma^k} \omega^k\]
relate. As we use the Voronoi duality, $\star \sigma$ is orthogonal to $\sigma$. If $\omega^k$ is constant on $\sigma^k$ and $\star \sigma^k$, we have, because of the way the Hodge star is defined
\[\int_{\star\sigma} \star\omega^k =  \frac{Vol_{n-k}(\star \sigma^k)}{Vol_k(\sigma^k)}\int_{\sigma} \omega^k\]
This motivates the use of the diagonal matrix
\[\star^{discrete}_k = \begin{pmatrix}
\frac{Vol_{n-k}(\star \sigma^k_1)}{Vol_k(\sigma_1^k)} \\
&\frac{Vol_{n-k}(\star \sigma_2^k)}{Vol_k(\sigma_2^k)} \\
& & \ddots \\
& & & \frac{Vol_{n-k}(\star \sigma_m^k)}{Vol_k(\sigma_m^k)}
\end{pmatrix}\]
as a discrete version of the $\star$ operator to relate the discrete dual and primary forms 
\[\star\textbf{w}^k  = (\star\textbf{w}^k(\star\sigma^k_1),...,\star\textbf{w}^k(\star\sigma^k_m))\]
\[\textbf{w}^k = (\textbf{w}^k(\sigma^k_1),...,\textbf{w}^k(\sigma^k_m))\]
i.e.
\[\star \textbf{w}^k = \star^{discrete}_k \textbf{w}^k.\]
This discrete $\star$ operator is \emph{not} compatible with the sampling scheme; the dual of a discrete form $\textbf{w}^k$ sampling $\omega^k$ only approximates a correctly sampled $\star \omega^k$. But if the size of the simplices gets smaller, the error of $\star^{discrete}$ goes to zero, as $\omega^k$ will be close to constant in small regions.

This simple $\star^{discrete}$ operator will still prove to be quite good. From a numerical point of view it is great that it is a diagonal matrix. And by associating dual forms to dual Voronoi cells, the geometry of the Hodge star is captured quite well by this discrete Hodge star. 

\subsubsection{Drawbacks of Voronoi Duality}
One drawback when choosing the dual mesh and discrete star as Voronoi cells is that Voronoi cells 'degenerate' in the presence of obtuse simplices. The circumcenter of a simplex can lie arbitrarily far away from the simplex such that $\star_{discrete}\textbf{w}^k$ is not a good estimation for a sampled dual form. Adapting the dual mesh and star matrices as with the Laplacian in Section \note{...} or using a different dual mesh and star operator might be beneficial. But this was not investigated in this thesis. 
 
\subsection{This is Discrete Exterior Calculus}
With the discrete star, forms, dual forms and discrete exterior derivative for dual and primal forms we can build a 'discrete de Rham' complex. \note{mention of more general chain complexes} The discrete de Rham complex keeps many properties from the continuous one. \note{Maybe list them here... PoinCarr� and Hodge ... TODO}

\note{Dual forms implemented as vectors. While related to a different geometric object it is a vector of same dimension}

\note{\textbf{Scalar product for discrete $k$-Forms}\newline
Adjointness of the covariant derivative with the discrete scalar product. Define the discrete co-derivative such that adjointness holds. Or the scalar product such that adjointness holds.
Example scalar product of $0$-Forms needs to be divided by area to sum up correctly. $1$-Forms: diamond shaped area...
}

\subsubsection{Discrete Scalarproduct and Adjointness}
\label{subsec:5_discreteScalarprod}
We define the scalar product for discrete differential forms such that the discrete coderivative is adjoint to the discrete exterior derivative.
\[\langle d\discrete{v}, \discrete{w} \rangle_{discrete} = \langle \discrete{v}, \partial \discrete{w}\rangle_{discrete}\]
But a scalar product on a finite dimensional vector space can be  described by a symmetric matrix $S$:
\[\langle \discrete{v}, \discrete{w} \rangle_{discrete} = \discrete{v}^TS\discrete{w}\]
And the first equation is fulfilled when $S = \star_{discrete}$ is used as then
\[\langle d\discrete{v}, \discrete{w} \rangle_{discrete} = (d\discrete{v})^T\star \discrete{w} =  
(\discrete{v})^T d^T\star \discrete{w} =  (\discrete{v})^T \star (\star^{-1}d^T\star \discrete{w}) =  \langle \discrete{v}, \partial \discrete{w}\rangle_{discrete}\]
%\[\langle \omega^k , \nu^k \rangle = \int_{M}  \omega^k\wedge \star \nu^k\]

To use the $\star$ matrix as scalar product also makes intuitively sense: for 0-forms
\[\int_{M}\langle \omega^0, \nu^0 \rangle = \sum \int_{\star \sigma^0} \omega^0 \cdot \nu^0 dVol \approx \sum \discrete{v}(\sigma^0) \cdot \discrete{w}(\sigma^0) \cdot Vol(\star\sigma^0) \]
\[= \discrete{v}^T \star^0_{discrete} \discrete{w}\]
as the matrix $\star^0$ stores the volumes of the dual cells of vertices $\sigma^0$. The scalar product for discrete 0-forms gets weighted by dual areas. Generally the values of the discrete form are scaled such that they represent values scaled to the diamond shaped areas spanned by a simplex and its dual, see Figure \note{Todo} for discrete 1-forms on a discrete 2-manifold
\[\langle \omega^k, \nu^k\rangle = \int_{M}\langle \omega^k, \nu^k \rangle = \sum \int_{hull(\sigma ,\star \sigma)} \omega^k \wedge \star \nu^k \approx \discrete{v}^T \star^k_{discrete} \discrete{w}\]
\note{
\textbf{The 0 Form Laplacian from the beginning}\newline
Now easy to write that guy down.}