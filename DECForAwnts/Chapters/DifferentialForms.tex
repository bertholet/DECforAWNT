\chapter{Differential Forms}
	\begin{longtable}{|p{4.5cm}|p{4.5cm}|p{4.5cm}|}
		\hline
		Smooth Theory& Discrete Theory& Implementation (Notes)\\
		\hline
			Differential Forms: \begin{itemize}
			  \setlength{\itemsep}{1pt}
			  \setlength{\parskip}{0pt}
				\setlength{\parsep}{0pt}
				\item[-]Diff Form motivation
				\item[-]Forms (multilinear mappings) and the dimension of k Form space 
				\item[-]Differential Forms 
				\item[-]Riemann Integral of Diff Forms 
				\item[-]Interpretation of Diff Forms in $\mathbb R^3$ 
			\end{itemize}
			&
			\begin{packed_enum}
				\item[-] Discrete Form
				\item[-] Sampling Forms
			\end{packed_enum}
			 & - none
			 \\		
		\hline
	\end{longtable}
	\subsection{Motivation: The perfect thing to integrate}
		Have a look at what you need when taking an integral: an either symmetric or antisymmetric form taking
		vectors from the tangential space
	\subsection{Forms and Differential Forms}
		Concretize The space of multilinear antisymmetric functions: bases, dimension, wedge. Define Differentialforms
	\subsection{What Forms are in $\mathbb{R}^3$}
		Using a metric (or simply by saying we look only at embedded manifolds with induced metric) we can associate
		Vector fields etc to Forms.
	\subsection{Integrating Forms}
		Now again lets have a look at integration of Diffforms.
	\subsection{Discrete Forms}
		What discrete forms are and where they live
	\subsubsection{Sampling Forms}
		How you sample vectorfields etc / what the sampled values mean. Here the duality of forms already emerges.
	