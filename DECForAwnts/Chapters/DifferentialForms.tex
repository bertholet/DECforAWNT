\chapter{Differential Forms}
\label{chap:diffforms}

\begin{figure}[h]%
\begin{center}
\includegraphics[height=3.5cm]{imgs/4_differentialform.eps}%	
\end{center}

\caption{Here a differential form $\omega$ is represented. A single differential form provides linear mappings $\omega_p$ at all points $p$ on a manifold; the mappings $\omega_p$ map the tangential spaces $T_pM$ to $\mathbb R$.}%
\label{fig:4_differentialform}%
\end{figure}
As seen in the last chapter, you can do differential calculus on smooth manifolds. To be able to define discrete calculus on discrete manifolds we first need a geometric understanding of calculus. We get to this understanding by generalizing functions to 'objects` that can be integrated over subsets of manifolds. These objects are \emph{differential forms}; one is schematically depicted in Figure \ref{fig:4_differentialform}. In a next step we will then define an exterior derivative $d$ for differential forms, whose geometry is easier to understand and can be used to define a discrete exterior derivative on discrete manifolds.
 The point why we are interested in differential forms and the exterior derivative $d$ is that $d$ unifies many differential operators, like divergence, gradient and curl.

%In the next two chapters we introduce forms, differential forms and exterior calculus. Differential forms are mathematical objects that allow us to treat many things in a unified way, such as real valued functions and vector fields. Differential forms can be integrated and differentiated, actually they are designed to behave well under an integral. The point why we are interested in differential forms is that they allow the definition of a differential operator $d$ that unifies many differential operators, like divergence, gradient and curl. Exterior calculus will unmask common properties of these differential operators and also their geometry, via Stokes theorem
%\[\int_{\delta \Omega} \omega = \int_{\Omega} d\omega,\]
%which directly connects the differential operator $d$ to the border operator $\delta$. It will be this relation that is exploited to define the differential operator $d$ for simplicial complexes, and by doing so preserve many important properties of these differential operators.

The basic theory for differential forms is split in two chapters. This chapter motivates differential forms, captures them more formally, covers integration of differential forms and relates them to standard calculus objects like real valued functions  and vector fields. It ends with the introduction of discrete differential forms.

In the next chapter, Chapter \ref{chap:EC}, we will then introduce the most important elements of exterior calculus and discrete exterior calculus, like the exterior derivative $d$, the operators $\partial$ and $\star$ and Stokes theorem, which describes the geometry of the exterior derivative.
%\begin{table}[h]
%	\begin{longtable}{|p{4.5cm}|p{4.5cm}|p{4.5cm}|}
%		\hline
%		Smooth Theory& Discrete Theory& Implementation (Notes)\\
%		\hline
%			Differential forms: \begin{itemize}
%			  \setlength{\itemsep}{1pt}
%			  \setlength{\parskip}{0pt}
%				\setlength{\parsep}{0pt}
%				\item[-]Diff form motivation
%				\item[-]Forms (multilinear mappings) and the dimension of $k$-form space 
%				\item[-]Differential forms 
%				\item[-]Riemann Integral of Diff forms 
%				\item[-]Interpretation of Diff forms in $\mathbb R^3$ 
%			\end{itemize}
%			&
%			\begin{packed_enum}
%				\item[-] Discrete forms
%				\item[-] Sampling forms
%			\end{packed_enum}
%			 & - none
%			 \\		
%		\hline
%	\end{longtable}
%	\end{table}

\section{Smooth Differential Forms}

We start introducing differential forms by motivating them as objects that fulfill all requirements to be useful under an integral in Section \ref{sec:dfmotivation}. In Section \ref{subsec::forms} we describe the  local structure of differential forms. At any point a differential form $\omega$ provides a \emph{form} $\omega_p$, a multilinear, antisymmetric mapping. These multilinear mappings form a vector space of a finite dimension. Using the so called wedge product $\wedge$, we describe simple bases for them in Section \ref{subsec:wedge}. Having understood the local structure of differential forms we give a clean definition of them in Section \ref{subsec:defDiffform} and finally use everything to relate the differential forms to the more intuitive well known objects from standard calculus in Section \ref{subsec:diffformsare}.

\subsection{The perfect thing to integrate}
\label{sec:dfmotivation}
\begin{figure}
	\begin{center}
	\includegraphics[width = 14cm]{imgs/5_1_riemann.eps}
	\end{center}
	\vspace{-1cm}
	\caption{To calculate the Riemann integral over a surface we select a grid and refine it. In the sums $f$ is evaluated at arbitrary positions in the patches.}
	\label{fig:5_1_riemsum}
\end{figure}

Differential forms arise very naturally when considering integrals. Suppose that we have a two dimensional surface $M$ and a function $f$ defined on the surface. We want to integrate $f$ over $M$ i.e. calculate the Riemann integral
\[\int_{M} f dA.\]
To compute the integral by brute force we can use Riemann sums as depicted in Figure \ref{fig:5_1_riemsum}: we choose a grid, sum up the areas of the parallelograms $s$ weighted by the  function value $f(s)$, and take the limit under grid refinement:
\[\lim_{diam(s\in grid)\rightarrow 0} \sum_{s \in grid} f(s)\cdot area(s) \]
We take a step back and consider what is essential for an 'object` to be integrated in this way.
Basically we can integrate anything that assigns values to areas $s$. Say $\omega_p$ assigns the value $\omega_p(s)$ to an area $s$ located at some point $p$, then its integral can be computed as:
\[\int_M \omega = \lim_{diam(s\in grid)\rightarrow 0} \sum_{s \in grid} \omega_{p\in s}(s)\]
Obviously $\omega_p$ has to follow some rules to be useful for integration. For one it should scale with the area of $s$. Assume a grid segment $s$ is spanned by two vectors $a_s$,$b_s$, then we can write $\omega_p(s)$ as $\omega_p(a_s,b_s)$, see Figure \ref{fig:5_linear}. For $\omega_p$ to be proportional to the area of $s$ we need it to be linear in both $a_s$ and $b_s$
\[\omega_p(\lambda a_s, b_s) =\lambda \omega_p(a_s,b_s)\]
\[\omega_p(a_s , \lambda b_s) =\lambda \omega_p(a_s,b_s)\]
Furthermore, $\omega$ should behave well when the parameters are swapped, as the vector pairs $(a_s,b_s)$ and $(b_s,a_s)$ span the same area, but for orientation. There are two possibilities that make sense: we can choose $\omega$ to be symmetric or to be antisymmetric
\[\omega(a_s,b_s) = \omega(b_s,a_s)\]
\[\omega(a_s,b_s) = - \omega(b_s,a_s).\]
Symmetry would mean that $\omega$ only depends on the absolute area of $s$. Antisymmetry means that $\omega$ respects the orientation of $s$. We choose antisymmetry, $\omega$ then is a \emph{differential form}. The first variant would lead to so called \emph{pseudo forms}. 

\begin{figure}%
\begin{center}
	\includegraphics[height = 3.5cm]{imgs/4_difformscaling.eps}%
\end{center}
\caption{Locally at some point $p$, a differential form 'measures` volumes and therefore has to be proportional to the volume measured. If the volumes are described by the vectors that span the volume, here $a_s$ and $b_s$, this means that $\omega$ should be linear in both $a_s$ and $b_s$.}%
\label{fig:5_linear}%
\end{figure}

Lastly we have to clarify what $a_s$ and $b_s$ are. The vectors $a_s$ and $b_s$ are bound to some position $p$. Also, if you look at the grid in Figure \ref{fig:5_1_riemsum}, you see that the grid elements nearly lie in the tangential spaces of the surface $M$. At least they do so in the limit. This gets us to the full definition of a differential form on a two dimensional surface: a differential form $\omega$ provides at any point $p$ on the surface $M$ a mapping $\omega_p$ that takes two vectors from the tangential space $T_pM$, is linear in both arguments and antisymmetric. I.e. for all $p \in M$
\[\omega_p: T_p M \times T_p M \to \mathbb R\]
\begin{align*}&\omega_p(\lambda a,b) = \lambda \omega_p(a,b) = \omega_p(a,\lambda b) &\text{bilinearity} \\
&\omega_p(a,b) = -\omega_p(b,a)  &\text{antisymmetry}\end{align*}
A differential form $\omega$ should also  change smoothly between neighboring $p$'s.


\subsection{Forms}
\label{subsec::forms}
We motivated that differential forms have a position $p$ and some number of tangential vectors from $T_pM$ as input variables. And for a fixed $p$ the differential form should be multilinear and antisymmetric.  Before we give a full definition of differential forms, we have a look at how they behave at single points. A multilinear antisymmetric mapping is called a form (without the word 'differential`). Forms form a finite dimensional vector space and allow the definition of a wedge product. This is important for us because this will let us describe differential forms with greater ease and let us relate differential forms to standard calculus objects.
\subsubsection{Definition}
A $k$-form on $\mathbb R^l$ is a multilinear antisymmetric mapping $\mathbb R^l \times ... \times \mathbb R^l \to \mathbb R$ which depends on $k$ vectors from $\mathbb R^l$: 
\begin{definition}[$k$-form]
 A $k$-form (not a differential $k$ form, mind you) on a $l$ dimensional space $\mathbb R^l$ is a mapping $\omega : \mathbb R^l \times \mathbb R^l\times \cdots \times  \mathbb R^l \rightarrow \mathbb R$ with the following properties:
\begin{enumerate}
\item $\omega(x_1,...,x_k)$ is linear in all $k$ parameters, meaning that \[\omega(x_1,..,\lambda a + b,..., x_k) = \lambda \omega(x_1,..,a,..., x_k) + \omega(x_1,.., b,..., x_k).\]
\item $\omega(x_1,...,x_k)$ is skew symmetric or antisymmetric, meaning that switching any two variables leads to a change of sign:
\[\omega(x_1,...,x_i,...,x_j,...,x_k) = - \omega(x_1,...,x_j,...,x_i,...,x_k)\]
\item In particular:
\[\omega(x_1,...,x_k) = 0 \;\text{ if }\;x_1,...,x_k \text{ are linearly dependent}\]
\end{enumerate}
\end{definition}

The first property makes sure that $\omega$ is proportional to the volume spanned by the input vectors, while the second ensures that $\omega$ respects the orientation of the input.
The third property follows from the first two.\footnote{From $\omega$'s antisymmetry follows $\omega(...,v,...,v,...) = -\omega(...,v,...,v,...)$, i.e. $\omega(...,v,...,v,...) = 0$. Then from the linearity directly follows  $\omega(x_1,...,x_{k-1}, \sum_{j=1}^{k-1} a_jx_j) = 0$} The space of all $k$-Forms on $\mathbb R^l$ is denoted by $\Lambda^k (\mathbb R^l)$ and is a vector space: if $\omega$ and $\nu$ are $k$ forms so are $\omega + \nu$ and any multiples $\lambda \omega$. A natural question is what dimension $\Lambda^k(\mathbb R^l)$ has and to find a suitable basis of this space.

\subsubsection{Basis and Dimension of $\Lambda^k(\mathbb R^l)$}
Given a $k$-form $\omega$ on $\mathbb R^l$ and a basis $e_1,...,e_l$ of $\mathbb R^l$, then $\omega(a_1,...,a_k)$ can be rewritten the following way: we express the $a_j$ explicitly as a sum of basis vectors
\[a_j = \sum_{i}a_j^ie_i\]
and using the linearity of forms we get
\[\omega(a_1,...,a_l) = \omega(\sum_{i}a_1^ie_i,...,\sum_{i}a_l^ie_i)\]
\[= \sum_{i_1,...,i_k \in\{1,...,l\}}a_1^{i_1}\cdot ... \cdot a_k^{i_k} \omega(e_{i_1},...,e_{i_k}).\]
We can reorder this sum such that all terms treating the same set of basis vectors are grouped together
\begin{equation}=\sum_{i_1<...<i_k}\left(\sum_{\sigma \in S^k} a_1^{i_{\sigma(1)}}\cdot ... \cdot a_k^{i_{\sigma(k)}} \omega(e_{i_{\sigma(1)}},...,e_{i_{\sigma(k)}})\right),\label{eq:diffformsum}\end{equation}
where the permutation group $S^k$ is used to express that the inner sum goes over all orderings of basis vectors. Because of the antisymmetry, reordering $e_{i_{\sigma(1)}},...e_{i_{\sigma(k)}}$ to $e_{i_1},...,e_{i_k}$ such that $i_1<...<i_k$, affects only the sign of $\omega(e_{i_1},...,e_{i_k})$:
\[\omega(e_{i_1},...,e_{i_k}) = sgn(\sigma)\omega(e_{i_{\sigma(1)}},...e_{i_{\sigma(k)}}).\]
Rewriting the sum \ref{eq:diffformsum} yields
\[\sum_{i_1<...<i_k}\left(\sum_{\sigma \in S^k} sgn(\sigma) a_1^{i_{\sigma(1)}}\cdot ... \cdot a_k^{i_{\sigma(k)}}\right) \omega(e_{i_1},...,e_{i_k})\]
\[ = \sum_{i_1<...<i_k} det_{i_1,...,i_k}(a_{i_1},...,a_{i_{k}}) \omega(e_{i_1},...,e_{i_k}),\]
where $det_{i_1,...,i_k}(a_{i_1},...,a_{i_{k}})$ is a sub determinant of the matrix formed by the vectors $a_1,...,a_k$ restricted to the lines $i_1,...,i_k$:
\[det_{i_1,...,i_k}(a_1,...,a_k) = det \begin{pmatrix}
a_1^{i_1} &a_2^{i_1} &...&a_k^{i_1} \\
\vdots & & & \vdots \\
a_1^{i_k} &a_2^{i_k} &...&a_k^{i_k} 
\end{pmatrix}\]
Put on one line we get

%\fbox{\parbox{\textwidth}{
\[\omega(a_1,...,a_k)= \sum_{i_1<...<i_k} \omega(e_{i_1},...,e_{i_k}) \cdot det_{i_1,...,i_k}(a_{i_1},...,a_{i_{k}})\]
%}}
We can read a few things out of this. For one, the $k$-form $\omega$ is determined uniquely by the values it assumes on $k$-tuples of basis vectors $e_{i_1},...,e_{i_k}$ with $i_1 <...< i_k$. And the $k$-forms
\[det_{i_1,...,i_k}(a_{1},...,a_{k})\]
which calculate $k$-subdeterminants of the input vectors, form a basis of $\Lambda^k(\mathbb R^l)$. From this follows directly that that the dimension of the  space of $k$-forms on $\mathbb R^l$ equals the number of ordered tuples $i_1<...<i_k$ of integers $i_1,...,i_k \in \{1,...,l\} $ i.e.
\[\dim (\Lambda^k(\mathbb R^l)) = \begin{pmatrix}
l \\
k
\end{pmatrix}\] 
In particular, the space of $k$ forms on $\mathbb R^l$ with $k>l$ is 0-dimensional, which means that there are no $k>l$-forms.

\subsection{The Wedge Product}
\label{subsec:wedge}
The wedge product for forms is a way to create higher order forms out of lower order forms, for example out of a $j$ form $\omega^j$ and a $k$ form $\nu^k$ you can make a $j+k$ form $\omega^j\wedge \nu^k$. The important points to understand in this section are that the wedge product can be used to create higher order forms and to simply describe a base of the space of $k$-Forms $\Lambda^k(\mathbb R^l)$. Furthermore, the wedge product is associative, distributive and has some symmetry. 

\subsubsection{Definition of the Wedge Product}
The wedge product is easy to define but not very intuitive. You directly define the wedge product as
\[\omega^j\wedge \nu^k (v_1,...,v_{l+k}) = \frac{1}{k!l!}\sum_{\sigma \in S^{k+l}} sgn(\sigma) \omega^j(v_{\sigma(1)},...,v_{sigma(j)})\nu^l(v_{\sigma(k+1)},...,v_{\sigma(k+l)}).\]
The wedge product has the following properties, as is easy to show and is done e.g. in \cite{globalAnalysis}. These algebraic rules are handy for calculations.
\emph{
\begin{enumerate}
\item Linearity in both arguments, i.e. $(\lambda\omega_1^k + \omega_2^k)\wedge\nu^l = \lambda(\omega_1^k \wedge\nu^l) + \omega_2^k \wedge \nu^l$ and the same for $\nu^l$
\item Associativity, i.e. $ (\omega^j \wedge \nu^k) \wedge \mu^l = \omega^j \wedge (\nu^k \wedge \mu^l)$
\item Symmetry: $\omega^k\wedge \nu^l = (-1)^{kl} \nu^l \wedge \omega^k$
\end{enumerate}
}

The wedge product is closely connected to determinants. For two arbitrary 1-forms $\omega^1$, $\nu^1$ we get
\[\omega^1\wedge\nu^1(a,b)= det \begin{pmatrix}
\omega(a) & \omega(b) \\
\nu(a) & \nu(b)
\end{pmatrix}\]
and wedging $k$ one forms $\omega_1^1,...\omega_k^1$ leads to
\[\omega_1^1\wedge\omega_2^1 \wedge...\wedge\omega^1_k(a_1,...,a_k):= det \begin{pmatrix}
\omega_1(a_1) &  ... & \omega_1(a_k) \\
\vdots & & \vdots \\
\omega_k(a_1) &... & \omega_k(a_k)
\end{pmatrix}\]

\subsubsection{A Basis with the Wedge Product}
The wedge product allows to elegantly describe a basis for the space of $k$-forms $\Lambda^k(\mathbb R^l)$, using a basis $e_1,...,e_l$ of $\mathbb R^l$. The space of 1-forms $\Lambda^1(\mathbb R^l)$ has dimension $l$ and is spanned by the special set of basis forms
\[de_i(a) := det_i(a) = a^i\]
i.e. the projection to the $i$th coordinate of $a$ with respect to the chosen base $e_1,...,e_l$.  If we apply the wedge product to the 'standard' basis 1-forms $de_1,..., de_l$ we get
\begin{eqnarray*}de_{i1}\wedge de_{i2} \wedge ... \wedge de_{ik}(a_1,...,a_k) &= &det \begin{pmatrix}
de_{i1}(a_1) &  ... & de_{i1}(a_k) \\
\vdots & & \vdots \\
de_{ik}(a_1) &... & de_{ik}(a_k)
\end{pmatrix} \end{eqnarray*}
which is $det_{i_1,...,i_k}(a_1,...,a_k)$ when $i_1 <...<i_k$. These are exactly the 'standard' basis forms for the space of $k$-forms from the Section \ref{subsec::forms}. This means that a basis of $\mathbb R^l$ induces a basis to the space of forms and any $k$-form can be written as a linear combination 
\[\omega^k = \sum_{i_1<...<i_k} w_{i_1,..,i_k} de_{i_1}\wedge...\wedge de_{i_k} \]
%Finally for a $k$ and an $l$ form $\omega^j$ and $\nu^k$ we define the wedge product by writing them in such a basis
%\[\omega^j = \sum_{i_1<...<i_j} w_{i_1,..,i_j} de_{i_1}\wedge...\wedge de_{i_j} \]
%\[\nu^k = \sum_{h_1<...<h_k} v_{h_1,..,h_k} de_{h_1}\wedge...\wedge de_{h_k} \]
%and then using the associativity and distributivity of the wedge product you can compute
%\begin{eqnarray*}
%\omega^j\wedge \nu^k &=& \sum_{i_1<...<i_j} w_{i_1,..,i_j} de_{i_1}\wedge...\wedge de_{i_j} \\
%& & \wedge \sum_{h_1<...<h_k} v_{h_1,..,h_k} de_{h_1}\wedge...\wedge de_{h_k} \\
%& =& \sum_{\begin{subarray}
%\{\{ i_1,...,i_j\}\cap \{h_1,...,h_k\} = \emptyset \\
%i_1 < ... < i_j,\;\; h_1 <...<h_k
%\end{subarray}} w_{i_1,..,i_j} \cdot v_{h_1,..,h_k} de_{i_1}\wedge...\wedge de_{i_j} \wedge de_{h_1}\wedge...\wedge de_{h_k}
%\end{eqnarray*}
Note that often $x_1,...,x_l$ or $x,y,z$ or similar is chosen to denote the base of $\mathbb R^l$ and the basis forms consequently are denoted by $dx_1,..., dx_l$ or $dx,dy,dz$, $dx \wedge dy$ and so on. 

\subsection{Differential Forms}
\label{subsec:defDiffform}
Now we can correctly define differential forms. They are exactly as motivated in the beginning of this chapter; a differential form assigns to each point $p$ of a manifold $M$ a form $\omega_p$ defined on the tangential space $T_pM$. With the wedge product we can now formulate that the differential form should vary smoothly between points. A local map 
\[\phi:  U \subset\mathbb R^l \to M\] 
induces a basis to all tangential spaces of points in $\phi(U)$, namely $\frac{\partial \phi}{\partial u_i}$. We can directly use this basis to induce a basis to the space of $k$-forms, as done in the last section. Henceforth we will use the following notation for the basis forms induced by a map $\phi$:
\[d u_i:= d\frac{\partial \phi}{\partial u_i}.\]

\begin{definition}[Differential Form]
A differential $k$-form $\omega^k$ is a mapping that assigns a $k$-form $\omega_p \in \Lambda^k(T_pM)$ to every point $p\in M$.

Given a local map $\phi: U \rightarrow M$ all $k$-forms $\omega_p$ with $p\in \phi(M)$ can be expressed in the  coordinates induced by $\phi$:
\[\omega_p = \sum_{i_1<...<i_k}\omega_{i_1,...,i_k}(p) du_{i_1}\wedge...\wedge du_{i_k}\]
for some realvalued functions $\omega_{i_1,...,i_k}(p)$. We then say that the differential form $\omega$ is $k$ times differentiable if expressed in local coordinates, the $\omega_{i_1,...,i_k}(p)$ are $k$ times differentiable. For simplicity sake we will always assume that $\omega$ is infinitely often differentiable, i.e. smooth.

\end{definition}

We will see examples and relate differential forms to more common things like vector fields in the next section. 
%Note that any operation defined for forms can point-wisely be defined for differential forms. For example the wedge product $\wedge$ for two differential forms $\omega^k$ and $\nu^l$ would be
%\[(\omega^k\wedge \nu^l)_p := \omega^k_p\wedge \nu^l_p\]


\subsection{Interpretation of Differential Forms in $\mathbb{R}^3$}


\label{subsec:diffformsare}
Differential forms are of high practical relevance because standard calculus objects like vector fields are just realizations of differential forms. Therefore, the theory about differential forms can be applied directly to a wide range of standard problems. In the following we focus on $\mathbb R^3$, or equivalently, 3-dimensional manifolds, and relate the differential forms to standard calculus objects. The relation is depicted in Figure \ref{fig:4_difformsAre}.

\subsubsection{Differential 0-Forms}
Differential 0-forms are real valued functions. By definition a differential 0-form assigns a 0-form, i.e. a constant, to every point $p$ on a manifold. This means a differential 0-form is simply a smooth function $\omega: M \to \mathbb R$. 

\subsubsection*{Differential 1-Forms}
Differential $1$-forms are equivalent to tangential vector fields. A 1-form is a linear mapping $\omega: \mathbb R^k \rightarrow \mathbb R$. Linear mappings to $\mathbb R$ can be represented as the scalar product of some vector $\omega^{\#} \in \mathbb R ^k$ with the input vector:
\[\omega(v) = \langle \omega^{\#}, v \rangle\]
This works just as well on manifolds with tangential spaces. Yet, there is an additional technical difficulty to mention: we need a scalar product on the tangential spaces. As we look only at manifolds embedded in a higher dimensional space $M \subset \mathbb R^n$, the most natural choice of a scalar product on the tangential spaces is the scalar product induced by the surrounding space. A scalar product that is consistently defined for all tangential spaces is called a \emph{Riemannian metric}.

Note that in principle you could use a different metric i.e. different scalar products. In that case $\omega$ would be described by a different vector $\omega^{\#}$. This is why the sharp operator is \emph{depending on a metric}. The reverse operation of making a 1-form out of a vector $v$ is usually denoted by the 'flat' operator $\flat$, i.e. $v^\flat$.

\subsubsection*{Differential n-1-Forms}
The space of 2-forms on tangential spaces $\Lambda^2(T_pM)$ on three dimensional manifolds $M$ has dimension \[\dim\Lambda^2(T_pM) =\begin{pmatrix}
3\\
2
\end{pmatrix} =3.\] Therefore a differential $2$-form can again be represented as a vector field. In the $\mathbb R^3$ with the standard basis and euclidean scalar product a basis of $\Lambda^2(\mathbb R^3)$ is given by $dy \wedge dz$, $dz \wedge dx$, $dx \wedge dy$. If $\widehat{w} = (w_1,w_2,w_3)$ is a vector field we define the related $2$-form as
\[\omega^2 = w_1 dy \wedge dz + w_2 dz \wedge dx + w_3 dx \wedge dy \]
 As $dy \wedge dz (a,b) = a_yb_z -a_zb_y$,  the 2-form $\omega^2(a,b)$ can be written as
\begin{align*}
\omega^2(a,b) &= (w_1 dy \wedge dz + w_2 dz \wedge dx + w_3 dx \wedge dy)(a,b) \\
&= w_1(a_2b_3 - b_2 a_3) + w_2(a_3b_1-b_3a_1) + w_3(a_1b_2 - a_2b_1) \\
&= \langle \widehat{w}, a \times b \rangle\end{align*} 
This means if we want a vector $\widehat{w} \in \mathbb R^3$ to act like a two form on two input vectors $a,b$ in $\mathbb R^3$, it amounts to taking the scalar product of $\widehat{\omega}$ and a vector normal to $a,b$ scaled by the area spanned by $a,b$. 
This can be done for $(n$-$1)$-forms on $n$-manifolds in general.

\begin{figure}%
\def\svgwidth{\columnwidth}
%\includegraphics[width=\columnwidth]{imgs/4_differentialforms.eps}%
\input{imgs/4_differentialforms.eps_tex}
\caption{Interpretation of differential forms as standard calculus objects. On a $n$-dimensional manifold $0$-forms and $n$-forms can be identified with real valued functions, $1$-forms and $(n$-$1)$-forms as tangential vector fields.}%
\label{fig:4_difformsAre}%
\end{figure}

\subsubsection*{Volume Forms}
Differential $3$-forms on $\mathbb R^3$ can again be represented as real valued functions. As can differential $n$-forms on $n$-dimensional manifolds in general.
There is a special differential $n$-form on an $n$ dimensional manifold $M^n$: the 'volume form'. The volume form measures the signed (orientation dependent) volume spanned by the input vectors. In $\mathbb R^n$ with the standard basis $e_1,...,e_n$ and the euclidean scalar product this is exactly the determinant:
\[d\mathbb R^n (v_1,..., v_n) = det(v_1,...,v_n) = de_1\wedge ...\wedge d e_n\]
The volume on some space $V$ is sometimes denoted as $dV$ or $dVol$.
 As the dimension of the space of $n$-forms on $T_pM^n$ is
\[\dim(\Lambda^n(\mathbb R^n)) = \begin{pmatrix}
	n\\n
\end{pmatrix}= 1,\]
every $n$-form is simply a multiple of the volume form $dV$. Therefore any differential $n$-form can be represented with a real valued function $f:M \rightarrow \mathbb R$ as
\[\omega^n = f \cdot dVol\]

%\subsubsection{Example: Volume Forms in Local Coordinates}
%As an example we express the volume form on a $k$-manifold $M^k$ in local coordinates given by a map $\phi$.
%The $k$-dimensional volume spanned by the vectors $\frac{\partial \phi}{\partial u_i}$ is $\sqrt{det((D\phi)^TD\phi)}$, therefore
%\[dT_pM (\frac{\partial \phi}{\partial u_1},...,\frac{\partial \phi}{\partial u_k}) = \sqrt{det((D\phi)^TD\phi)}.\]
%As any $k$-form is a multiple of the volume form, the $k$-form given by $\phi$ must be
%\[du_1\wedge...\wedge du_k = c \cdot dT_pM\]
%for some $c$. The $c$ can be computed by
%\[c = \frac{dT_pM(\frac{\partial \phi}{\partial u_1},...,\frac{\partial \phi}{\partial u_k})}{du_1\wedge...\wedge du_l (\frac{\partial \phi}{\partial u_1},...,\frac{\partial \phi}{\partial u_k})} = \frac{\sqrt{det((D\phi)^TD\phi)}}{det((D\phi)^TD\phi)}\]
%and therefore the volume form in local coordinates is
%\[dT_pM =\frac{1}{\sqrt{det((D\phi)^TD\phi)}}  du_1\wedge...\wedge du_k.\]

%\subsubsection{Differential $n$-Forms on $\mathbb R^n$}
%The one-dimensionality of the space of $n$-forms on $\mathbb R^n$ can also be used to get a very simple description of what happens to a $n$ differential form under a base change.
%If $A$ is a linear map $\mathbb R^n \to \mathbb R^n$ 
%\[det(Av_1,Av_2,...,Av_n) = det(A)\cdot det(v_1,...,v_n)\]
%and therefore if $\omega^n$ is a $n$ differential form on $\mathbb R^n$
%\[\omega^n(Av_1,...,Av_n) = det(A)\cdot \omega^n(v_1,...,v_n)\]
%as any $n$ form is simply a multiple of the volume form $\omega^n = c \cdot det$. We can play around with this a bit more such that for fixed $v_1,...,v_n$ building a matrix $V$ we get
%\[\omega^n(Av_1,...,Av_n) = det(A)\cdot \omega^n(v_1,...,v_n)= det(V) \cdot \omega^n(a_1,...,a_n)\]

%\subsubsection*{Summary}
%Images or tables that summarise what differential forms are on 2 dimensional and 3 dimensional manifolds. \note{Todo, as image, consistent with later images...}

%\begin{table}[h]
%\begin{longtable}{lccc}
%& & 2-Manifolds &\\
%Forms & $\Lambda^0(T_pM)$ & $\Lambda^1(T_pM)$ & $\Lambda^2(T_pM)$\\
%Dimension & 1 & 2 & 1 \\
%Differential Form  & $f:M\rightarrow \mathbb R$ & $\omega^{\sharp}: M \rightarrow TM$  & $f:M\rightarrow \mathbb R$ \\
% & $\omega_p =f(p)$ & $\omega_p(v) = \langle \omega^{\sharp} ,v\rangle$ & $\omega_p(v_1,v_2)=   f(p) \cdot dArea(v_1,v_2)$
%\end{longtable}
%\end{table}
%
%\begin{table}[h]
%\begin{longtable}{lcccccr}
% Forms & $\dim$ & Diff. Forms & Representation & & \\
% $\Lambda^0(T_pM)$ & 1 & 0-Forms & Function & $f:M\rightarrow \mathbb R$ \\%& $\omega_p =f(p)$ \\
% $\Lambda^1(T_pM)$ & 3 & 1-Forms & VField & $\omega^{\sharp}: M \rightarrow TM$  \\%& $\omega_p(v) = \langle \omega^{\sharp} ,v\rangle$\\
% $\Lambda^2(T_pM)$ & 3 & 2-Forms & VField & $\widehat{w}: M \rightarrow TM$ \\%& $\omega_p(v_1,v_2) = \langle \widehat{w} ,v_1\times v_2\rangle$ \\
% $\Lambda^3(T_pM)$ & 1 & 3-Forms & Function & $f:M\rightarrow \mathbb R$ \\%& $\omega_p(v_1,v_2, v_3)=f(p) \cdot dVol( v_1,v_2,v_3)$
%\end{longtable}
%\end{table}

\subsection{Integration of Forms}
\label{sec:integralOfForms}
We started this chapter saying that we wanted to design objects that are well suited for integration. We ended up with differential forms that in 3-dimensional spaces turn out to be either vector fields or functions. Let's now look at how differential forms are integrated. We omit various technicalities-for a clean introduction of the integral see e.g. \cite{globalAnalysis}; for a very understandable introduction see \cite{bachman2006geometric}.

A $k$-form can be integrated over $k$-dimensional regions. If $\phi : U \subset \mathbb R^k \to M$ is a map, $M$ a $k$-dimensional manifold and $\omega^k$ a differential $k$-form on $M$ and a region $\Omega = \phi(U)$ parametrized with $\phi$. The integral 
\[\int_{\phi(U)} \omega^k \]
is defined and calculated by pulling everything back to $\mathbb R^l$:
\[\int_{\phi(U)} \omega^k = \int_{U\subset\mathbb R^k} \omega_{\phi(x_1,...,x_k)}(\frac{\partial \phi}{\partial x_1},...,\frac{\partial \phi}{\partial x_k}) d x_1...d x_k\]
The integral on the right integrates a function depending on $k$ variables over a region $U$ in $\mathbb R^k$ as usual. But note: the integral on the left  lacks any '$d x_i$`s. It is independent of local coordinates, and can be computed with any set of coordinates, see Appendix \ref{app:integrals}. As intended $\omega$ automatically scales according to the volume spanned by the vectors $\frac{\partial \phi}{\partial x_i}$ for any coordinates $\phi$, thereby canceling out the choice of parametrization.

The way differential forms are designed, they can be integrated only over sets of a fixed dimensionality- a 1-form over one dimensional smooth sets, a $2$-form over 2-dimensional sets etc. A $k$-form defined on a $n$-dimensional manifold $M$ can only be integrated over any $k$-dimensional subset of $M$. This restriction is somewhat softened by the Hodge star operator $\star$, introduced in Section \ref{sec:hodgeStar}. The $\star$ will allow to make a $n-k$-form out of a $k$-form. Like that for example a $0$-form can indirectly be integrated over a $n$-dimensional set too.

\subsection{Pull-Backs}
\label{sec:pullbacks}

\begin{figure}%
\begin{center}
\includegraphics[height= 5cm]{imgs/5_pullback.eps}%	
\end{center}
\caption{A mapping $h$ can be used to pullback a differential form $\omega$ from one manifold to another, as $h$ provides a mapping between the manifolds and their tangential spaces. }%
\label{fig:5_pullback}%
\end{figure}

The 'pulling back` used to define an integral can be done more generally. Suppose we have a mapping between two manifolds $N$ and $M$ $h: N\to M$, which has $det(Dh) \neq 0$ and for simplicity  is smooth. As seen in Section \ref{sec:derivativeBetweenMfs}, $Dh$ is a mapping between the tangential spaces of $N$ and $M$. Therefore if we have a differential $k$-form $\omega$ on $M$ we can 'pull it back' to $N$ via
\[(h^*\omega)_p (v_1,...,v_k) := \omega_{h(p)}(Dh v_1,...,Dh v_k). \]
The pullback is also depicted in Figure \ref{fig:5_pullback}. The action of pulling back $\omega$ using $h$ is denoted by $h^* \omega$. This mapping preserves the integral (check it by using the definitions!)
\[\int_{h(U)\subset M} \omega = \int_{U \subset N} h^*\omega \]
This means that we can integrate either $\omega$ over a subset of $M$ or the pulled back mapping over a subset of $N$. Usually, as in the definition of the integral, you pull back forms to $\mathbb R^k$, using one of the local maps.
The pullback is very powerful as it conserves properties under the integral. 

\section{Discrete Differential Forms}

Differential forms are defined on Manifolds and as it is to be expected, discrete forms are defined on discrete manifolds. But there are no tangential spaces on discrete manifolds. While differential forms are spatially varying multilinear mappings, a discrete differential is something much simpler - it simply is a set of averaged values, as depicted in Figure \ref{fig:5_discreteForms}

\begin{definition}[Discrete Form]
A discrete $j$-form on a discrete $k$-manifold assigns a real number to every $j$-simplex contained in the discrete manifold. This vector of values is also sometimes called a $j$ co-chain. 
\end{definition}
The question to be answered is how this set of values relates to a non-discrete differential form.

\subsection{Sampling Forms}
\label{subsec:samplingForms}

\begin{figure}%
\begin{center}
\def\svgwidth{0.8\columnwidth}
\input{imgs/5_discreteForms.eps_tex}
\end{center}
\caption{A discrete differential $k$-form is a set of values associated to the $k$-simplices of a discrete manifold. A value represents the integral of the sampled differential $k$-form over the associated simplex.}%
\label{fig:5_discreteForms}%
\end{figure}

To relate discrete forms with differential forms, the discrete manifold $K$ needs to be related somehow to a non-discrete manifold $M$. We will just assume that the discrete manifold $K$ approximates the manifold $M$ and for any simplex $\sigma \in K$ there is a continuous analogue $\sigma \subset M$ on the manifold $M$, as in Figure \ref{fig:5_simplexVsMF}. We denote both the discrete simplex and the continuous counterpart by the same symbol. Which is meant should be clear from the context.

The relation then is simple: given a differential $k$-form $\omega^k$ on $M$ the value of the discrete $k$ form $\textbf{w}$ on a $k$-simplex $\sigma$ is value of $\omega^k$ accumulated over $\sigma$.

 \[\textbf{w}(\sigma) = \int_\sigma \omega^k\]
In the following we look at some examples in $\mathbb R^2$ and $\mathbb R^3$.

\subsubsection{Sampling 0-Forms}
As seen in Section \ref{subsec:diffformsare}, $0$-forms can simply be represented as functions $f: M\to \mathbb R$. A $0$-form has to be integrated over 0-dimensional sets, i.e. points. The discrete 0-form is a set of values associated to vertices. The value at a vertex position $v$ or 0-simplex $v$
 is
\[\textbf{w}^0(v) = f(v)\]
i.e. $f$ evaluated at $v$.

\subsubsection{Sampling 1-Forms}
A $1$-form $\omega^1$ on a manifold $M$ can be represented by a tangential vector field $\nu:M\to TM$ via 
\[\omega^1_p(v) = \langle\nu(p),v\rangle.\] 
A $1$-form can be integrated over $1$-dimensional curves. A discrete $1$-form is therefore a set of values associated to the $1$-simplices i.e. edges of the discrete manifold. The value on an edge $e$ is
\[\textbf{w}^1(e) = \int_{e} \omega^1 = \int_{0}^1 \langle\nu(e(t)),\frac{\partial}{\partial t}e(t)\rangle dt\]
where in the last integral $e(t)$ is a parametrization of the curve on the manifold $M$ associated to the edge $e$. A discrete $1$-form samples a vectorfield by projecting the field on the edge and accumulating these values along the edge. The resulting value can be thought of as measuring how much the vectorfield 'flows' along the edge. If the edge $e$ is a straight line and the vectorfield a constant vector  $\textbf{w}^1(e)$ is simply the projection of the vector to the edge
\[\textbf{w}^1(e) = \langle \nu, e \rangle.\]

On a $2$-manifold there is a second way of how to sample the vectorfield by measuring the flow \emph{through} the edge instead of \emph{along} the edge; we will come back to this in a while when talking about the Hodge star $\star$ operator in Section \ref{sec:hodgeStar}. 
\[\textbf{w}^1(e) = \int_{0}^1 \langle\nu(e(t)),(\frac{\partial}{\partial t}e(t))^\perp \rangle dt\]
Here $^\perp$ denotes the vector rotated by $90^\circ$ according the orientation of the surface. The two sampling schemes are depicted in Figure \ref{fig:5_samplingForms}.


\begin{figure}%
\begin{center}
\includegraphics[height= 3.5cm]{imgs/5_simplexVsMF.eps}%	
\end{center}
\vspace{-0.5cm}
\caption{We assume that for a simplicial complex that samples a manifold every simplex has related smooth region on the manifold.}%
\label{fig:5_simplexVsMF}%
\end{figure}

\subsubsection{Sampling 2-Forms}
A $2$-form can be integrated over $2$D patches and the discrete 2-form associates values to the 2-simplices i.e. the triangles of the discrete manifold.

On a 2-manifold: here a differential $2$-form $\omega^2$ is represented by a function $f$. The value $\textbf{w}^2(t)$ on a triangle $t$ is the integral of $f$ over $t$:
\[\textbf{w}^2(t)= \int_{t} f\, dVol\]

On a 3-manifold: here a differential $2$-form $\omega^2$ is represented by a vectorfield $\nu: M\to TM$, but evaluating it on two vectors amounts to
\[\omega_p(a,b) = \langle \nu(p) , a \times b \rangle\]
such that the value of the discrete $2$-form associated to a triangle $t$ is
\[\textbf{w}^2(t)= \int_{t} \langle \nu(p), n(p) \rangle \, dp\]
where $n(p)$ denotes the normal on the surface $t$ at the point $p$. This measures the flow of the vectorfield \emph{through} the surface $t$, see Figure \ref{fig:5_samplingForms}.


%\subsubsection{Some Observations}
%\note{is this needed?? Is this the right place? Should Duality be introduced here for real? Decide this later} This is the right place to make some observations. As we have seen on 3 manifolds vectorfields can either be interpreted as $1$ OR as $2$ forms, which decides on how they are integrated. The same is true for functions $f$ that can be interpreted as $0$ forms or $3$ forms, which again decides on how to integrate them. 
%
%On 2D surfaces $0$ and $2$ forms are represented the same way and for $1$-forms there were 2 ways to integrate them. There is a principle behind this: on an $n$-dimensional manifold there is a strong relation between differential $(n-k)$-forms and differential $k$-forms. We can make a $(n-k)$-form out of a $k$-form and vice versa. This is mirrored in the representation of differential forms above: a $1$-form can be interpreted as a $(3-1) =2$ form in an $(n =3)$D spaces and so on. The 'related' form will be called 'dual' form and you will get from the $k$ form $\omega^k$ to its dual $\nu^{n-k}$ using the so called Hodge operator $\star$:
%\[\nu^{k} = \star \omega^{n-k}\]
%In a 2D setting the dual of a $1$-forms is again a $1$-form, which explains (or at least motivates) why we gave two ways to sample $1$ forms.

%Functions $f$ are special anyway as, restricted to any $k$-manifold they can be interpreted as a $k$-form and integrated over it.

\subsection{Integration of Discrete Forms}

\begin{figure}%
\includegraphics[width=\columnwidth]{imgs/5_samplingForms.eps}%
\caption{Possibilities to sample 1-forms on 2-manifolds: you can either measure the flow through or along the edge. 2-forms on 3-manifolds are sampled by measuring the flow through a face.}%
\label{fig:5_samplingForms}%
\end{figure}
A discrete form can very easily be integrated over a set of simplices. Integrating a discrete $k$-Form $\textbf{w}^k$ over a set of $k$-simplices $\{\sigma_1,...,\sigma_l\}$ can be done simply by summing up the values on those simplices. If $\textbf{w}^k$ is the sampled version of $\omega^k$ this sum is exactly the integral of $\omega^k$  over the k dimensional set $\{\sigma_1,...,\sigma_l\}$
\[\int_{\{\sigma_1,...,\sigma_l\}} \omega^k = \sum_{i=1}^l \textbf{w}^k(\sigma_i)\]
as
\[\textbf{w}^k(\sigma_i) = \int_{\sigma_i} \omega^k\]
As we have seen we can describe a set of simplices in a discrete manifold as a vector $\sigma$ of dimension $\# k$-$simplices$ consisting of plus/minus ones and zeros. The discrete form $\textbf{w}^k$ is a vector of the same dimension and the discrete integral over $\sigma$ is the scalar product of those two vectors
$$\langle \sigma , \textbf{w}^k \rangle$$
The discrete analogon of the integral is in our setting the scalar product!